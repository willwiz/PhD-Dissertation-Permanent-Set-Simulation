%%%%%%%%%%%%%%%%%%%%%%%%%%%%%%%%%%%%%%%%%%%%%%%%%%%%%%%%%%%%%%%%%%%%%
%%  08 April 2002	Version 4				    					%
%%%%%%%%%%%%%%%%%%%%%%%%%%%%%%%%%%%%%%%%%%%%%%%%%%%%%%%%%%%%%%%%%%%%%
%%%%%%%%%%%%%%%%%%%%%%%%%%%%%%%%%%%%%%%%%%%%%%%%%%%%%%%%%%%%%%%%%%%%%
%%								    								%
%%  Doctoral Dissertation By Will Zhang at							%
%%	The University of Texas at Austin			     				%
%%%%%%%%%%%%%%%%%%%%%%%%%%%%%%%%%%%%%%%%%%%%%%%%%%%%%%%%%%%%%%%%%%%%%

\documentclass[12pt]{report}

\usepackage{utdiss2}	% The dissertation style as required by UT.

%%%%%%%%%%%%%%%%%%%%%%%%%%%%%%%%%%%%%%%%%%%%%%%%%%%%%%%%%%%%%%%%%%%%%
% Optional packages used for this dissertation. 					%
%%%%%%%%%%%%%%%%%%%%%%%%%%%%%%%%%%%%%%%%%%%%%%%%%%%%%%%%%%%%%%%%%%%%%

\usepackage{amsmath,amsthm,amsfonts,amscd,commath,mathtools} % Mathematic fonts
%%. and styles used in this dissertation
\DeclareMathAlphabet\mathbfcal{OMS}{cmsy}{b}{n}	% Bolded caligraphy font

\usepackage{verbatim}      	% Allows quoting source with commands.
\usepackage[utf8]{inputenc}
\usepackage{graphicx}
\graphicspath{ {images/} }

\usepackage{array,multirow,graphicx}
\newcolumntype{L}[1]{>{\raggedright\let\newline\\\arraybackslash\hspace{0pt}}m{#1}}
\newcolumntype{C}[1]{>{\centering\let\newline\\\arraybackslash\hspace{0pt}}m{#1}}
\newcolumntype{R}[1]{>{\raggedleft\let\newline\\\arraybackslash\hspace{0pt}}m{#1}}


\usepackage{etoolbox}
\apptocmd{\sloppy}{\hbadness 10000\relax}{}{}
\apptocmd{\thebibliography}{\raggedright}{}{}

\usepackage{enumitem}

\usepackage[table]{xcolor}
\definecolor{Gray}{gray}{0.9}

% Used to create the nomenclatures					%
\makeatletter

\newlength\tdima
\newcommand\tabfill[1]{%
      \setlength\tdima{\linewidth}%
      \addtolength\tdima{\@totalleftmargin}%
      \addtolength\tdima{-\dimen\@curtab}%
      \parbox[t]{\tdima}{#1\ifhmode\strut\fi}}

      
\newcommand\mytabs{\hspace*{0.5in}\=\hspace*{0.25in}\=\hspace*{0.25in}\=\hspace*{0.25in}\=\hspace*{0.25in}\=\hspace*{0.25in}\=\hspace*{0.25in}\=\hspace*{0.25in}\=\hspace*{0.25in}\=\hspace*{0.25in}\=\hspace*{0.25in}\=\hspace*{0.25in}}
\newenvironment{mynom}[1][\mytabs]
  {\begin{tabbing}#1\kill\ignorespaces}
  {\end{tabbing}}
% Used to create the nomenclatures					%
  
  
\usepackage{rotating}

\usepackage{eucal} 	 	% Euler fonts
\usepackage{makeidx}       	% Package to make an index.
\usepackage{url}		% Allows good typesetting of web URLs.

\usepackage{extdash}

\usepackage[square,numbers,sectionbib]{natbib}
\usepackage{chapterbib}

\usepackage{bibentry}


%%%%%%%%%%%%%%%%%%%%%%%%%%%%%%%%%%%%%%%%%%%%%%%%%%%%%%%%%%%%%
%%  About the Authors										%
%%%%%%%%%%%%%%%%%%%%%%%%%%%%%%%%%%%%%%%%%%%%%%%%%%%%%%%%%%%%%

\author{Will Zhang}  	% Author

\previousdegrees{B.S.}

\address{13926 Winding Cypress Brook Dr. \\ Cypress, Texas 77429}  % Address

%%%%%%%%%%%%%%%%%%%%%%%%%%%%%%%%%%%%%%%%%%%%%%%%%%%%%%%%%%%%%
%%  About the dissertation									%
%%%%%%%%%%%%%%%%%%%%%%%%%%%%%%%%%%%%%%%%%%%%%%%%%%%%%%%%%%%%%

\title{Modeling and simulating time-dependent changes in soft-tissue fabricated bioprosthetic heart valve in response to cyclic loading}

\supervisor
	{Michael S. Sacks}
    
\committeemembers
	[Aaron Baker]
	[Christopher G. Rylander]
	[Krishnaswamy Ravi-Chandar]
	{Naren Vyavahare}
    

\oneandonehalfspacequote

\topmargin 0.1in


%%%%%%%%%%%%%%%%%%%%%%%%%%%%%%%%%%%%%%%%%%%%%%%%%%%%%%%%%%%%%
%%  Macros that are needed									%
%%%%%%%%%%%%%%%%%%%%%%%%%%%%%%%%%%%%%%%%%%%%%%%%%%%%%%%%%%%%%

\newcommand{\latexe}{{\LaTeX\kern.125em2%
                      \lower.5ex\hbox{$\varepsilon$}}}

\newcommand{\amslatex}{\AmS-\LaTeX{}}

\chardef\bslash=`\\	% \bslash makes a backslash (in tt fonts)
			%	p. 424, TeXbook

\newcommand{\cn}[1]{\texttt{\bslash #1}}

\makeatletter		% Starts section where @ is considered a letter
			% and thus may be used in commands.
\def\square{\RIfM@\bgroup\else$\bgroup\aftergroup$\fi
  \vcenter{\hrule\hbox{\vrule\@height.6em\kern.6em\vrule}%
                                              \hrule}\egroup}
\makeatother		% Ends sections where @ is considered a letter.
			% Now @ cannot be used in commands.

\makeindex    % Make the index
 
 
%%%%%%%%%%%%%%%%%%%%%%%%%%%%%%%%%%%%%%%%%%%%%%%%%%%%%%%%%%%%%%%%%%%%%%
% Beginning of the document										     %
%%%%%%%%%%%%%%%%%%%%%%%%%%%%%%%%%%%%%%%%%%%%%%%%%%%%%%%%%%%%%%%%%%%%%% 
 
\begin{document}

\copyrightpage          % Produces the copyright page.

% NOTE: In a doctoral dissertation, the Committee Certification page
%		(with signatures) is BEFORE the Title page.
%	In a masters thesis or report, the Signature page
%		(with signatures) is AFTER the Title page.
%
%	If you are writing a masters thesis or report, you MUST REVERSE
%	the order of the \commcertpage and \titlepage commands below.
%
\commcertpage           % Produces the Committee Certification
			%   of Approved Version page (doctoral)
			%   or Signature page (masters).
			%		20 Mar 2002	cwm

\titlepage              % Produces the title page.

%%%%%%%%%%%%%%%%%%%%%%%%%%%%%%%%%%%%%%%%%%%%%%%%%%%%%%%%%%%%%%%%%%%%%%
% Dedication and/or epigraph are optional, but must occur here.      %
%%%%%%%%%%%%%%%%%%%%%%%%%%%%%%%%%%%%%%%%%%%%%%%%%%%%%%%%%%%%%%%%%%%%%%
%
% \begin{dedication}
% \index{Dedication@\emph{Dedication}}%
% Dedicated to my wife Shirley.
% \end{dedication}

\begin{acknowledgments}		% Optional
\index{Acknowledgments@\emph{Acknowledgments}}%
    Firstly, I would like to thank my advisor, Dr. Michael Sacks for the introducing to me to the world of biomechanics, the opportunity to work with his research group, and his patience and guidance over the years to prepare me for an academic career. Also, thank you to my committee members andthe Department of Biomedical Engineering at the University of Texas at Austin for their support throughout this process. I would also like to thank. It has been an honor to represent this department in my time here.
    

    I would like to thank my wonderful lab mates for giving me strength and support throughout this long process, lifting me up when I needed it the most. I also like to thank the post-doctoral fellows in our lab in particular, specifically Rong Fan, Chung-Hao Lee, Ankush Aggarwal and Joao Soares for their mentorship and guidance both in research and academically.
    

    Lastly, and most importantly I would like to thank my family, my parents, for raising me to the way I am, and all of the support they have given me throughout my life. I love you all.
\end{acknowledgments}

\utabstract
\index{Abstract}%
\indent
Soft-tissue-derived exogenously cross-linked (EXL) biomaterials continue to the best choice for the fabrication of bioprosthetic heart valves (BHV). Despite the ongoing research, our understanding of these materials and of the mechanisms leading to their failure remain at an empirical level. The need for advancements in modeling their behavior is further underscored by the development of percutaneously-delivered BHV devices. While these devices offer reduced surgical risk, they also present additional challenges for the design of the leaflets due to limitations in thickness and folding during delivery, resulting in a 2-year mortality rate of 33.9\% in general[2]. Thus, we seek to develop a framework for modeling and simulating soft-tissue-derived EXL biomaterials, accounting for the effects of exogenous cross-linking in permanent set and mechanical fatigue. Such approaches can significantly improve the accuracy and reliability of long-term predictions of durability and mechanical function. Firstly, we will establish the form of a nonlinear hyperelastic meso-scale structural constitutive model (MSSCM) for fibrous soft tissues, that can accurately capturing the mechanical response of common soft tissues used as a basis for EXL biomaterial. Secondly, we will study the effect of exogenous crosslinks on these tissues using glutaraldehyde (GLUT) EXL bovine pericardium. GLUT-EXLs form polymeric chains through the cross-linking process which more tightly bonds the fibers to the matrix, increasing the non-fibrous matrix stiffness and fiber-fiber interactions.  However, GLUT EXLs undergo Schiff-base reactions that lead to scission-healing behaviors that changes the geometry of BHVs. We model this effect based on first order kinetics of the scission healing reaction and validate it using static strain, cyclic strain and stress control experiments. Next, we will develop a full 3D finite element implementation of the MSSCM with the modifications for EXL for real device applications. We then parametrically examine the changes in geometry and stress distribution of BHVs overtime, exploring initial geometries and material properties which may minimize the risks of the former effects. With this, we aim to develop a better understanding of the underlying process that occurs during long-term cyclic loading through our constitutive modeling approach and device level applications, and translate the insights gained to improve BHV design and durability. 
 
\tableofcontents{}

\listoftables      % List of Tables and List of Figures will be placed
\listoffigures     % here, if applicable.

\chapter{Introduction} \index{Introduction@\emph{Introduction}}%

\section*{Preface}
\addcontentsline{toc}{section}{Preface}%

    Heart valves play a critical role of preventing back flow during cardiac operations. They are designed to withstand the demanding mechanical environment of the heart while maintaining optimal state for over 3 billion cycles. When they can no longer function properly due to fatigue, diseases or trauma, and surgical repair is not an option, surgical replacement using bioprosthetic heart valves is often the best choice. This introductory chapter focus on the current state of heart valve replacements and design, and methods for improving the durability of these devices. First, an overview of the structure and functions of heart valves is provided. Then the current state of heart valve repair and replacement is summarized, followed by the current state of bioprosthetic valve design and fabrication as well as they durability and limitations. Finally, the causes and mechanisms of bioprosthetic heart valve failure is discussed, and we outline the motivation, rationale and aims of this dissertation, on how constitutive modeling and simulations can be used to better understand bioprosthetic heart valve failure and facilitate the improvement of bioprosthetic heart valve design. 

\section{Introduction and Background}

    
    
    Heart valves are complex multi-layered structures that prevent backflow by opening and closing depending on the direction of flow. There are four valves in the heart, consisting of two atrioventricular preventing backflow between the atriums and ventricles, the mitral (MV) and tricuspid valves (TV), and two semilunar preventing backflows from the aorta and to the vena cava, the aortic (AV) and pulmonary valves (PV) (Fig. \ref{fig:heartdiagram}). The coordinated movement of the four heart valves enables them to maintain unidirectional blood flow during the cardiac cycle. When healthy, heart valves are incredibly resilient, opening and closing approximately 3 - 4 billion times throughout an average life-span \cite{sacks_biomechanics_2009}. The pressure changes during the cardiac cycle expose the heart valves to constant changes in forces and hemodynamics. This physiological demand is especially harsh on the mitral and aortic valves, needing to withstand average pressures of 80mmHg for the aortic and 120mmHg for the mitral valve to sustain circulation throughout the rest of the body. The biomechanical properties of heart valves must be able to withstand and function efficiently in this complex mechanical environment. Thus, the heart valve leaflets develop and maintain an intricate, highly organized, and multi-scale connective tissue system that allows them to do so \cite{tao_heart_2012}. 

    
    More than five million people are diagnosed with heart valve disease in the United States (US) each year [5, 7], with approximately 95,000 annual valve replacement surgeries, and 20,000 deaths per year [8]. Although valve disease can occur in any of the four valves, diseases of the 
    

%-------------------	begin FIGURE 	-------------------%
\begin{figure}
\centering
\includegraphics[width=5.0in]{Images/chapter1/heartdiagram.jpeg}
\caption{Artist rendition of the human heart depicting the location of the four heart valves and major vessels (obtained from U.S. National Library of Science website)}
\label{fig:heartdiagram}
\end{figure}
%-------------------	 end FIGURE 	-------------------%



\section{Structure and function of heart valves}
\section{Structure and function of heart valves}
    
    Heart valves are complex multi-layered structures that prevent backflow by opening and closing depending on the direction of flow. There are four valves in the heart, consisting of two atrioventricular preventing backflow between the atriums and ventricles, the mitral (MV) and tricuspid valves (TV), and two semilunar preventing backflows from the aorta and to the vena cava, the aortic (AV) and pulmonary valves (PV) (Fig. \ref{fig:heartdiagram}). The coordinated movement of the four heart valves enables them to maintain unidirectional blood flow during the cardiac cycle. When healthy, heart valves are incredibly resilient, opening and closing approximately 3 - 4 billion times throughout an average life-span \cite{sacks_biomechanics_2009}. The pressure changes during the cardiac cycle expose the heart valves to constant changes in forces and hemodynamics. This physiological demand is especially harsh on the mitral and aortic valves, needing to withstand average pressures of 80mmHg for the aortic and 120mmHg for the mitral valve to sustain circulation throughout the rest of the body. The biomechanical properties of heart valves must be able to withstand and function efficiently in this complex mechanical environment. Thus, the heart valve leaflets develop and maintain an intricate, highly organized, and multi-scale connective tissue system that allows them to do so \cite{tao_heart_2012}. 

%-------------------	begin FIGURE 	-------------------%
\begin{figure}
\centering
\includegraphics[width=5.0in]{Images/chapter1/heartdiagram.jpeg}
\caption{Artist rendition of the human heart depicting the location of the four heart valves and major vessels (obtained from U.S. National Library of Science website)}
\label{fig:heartdiagram}
\end{figure}
%-------------------	 end FIGURE 	-------------------%

\subsection{Multi-scale structure of heart valves}

    To fully understand the functional properties of heart valves, multi-scale approaches are needed (Fig. \ref{fig:multiscalevalve}) \cite{salma_heart_2016}. This complex hierarchical structure is what lends to seamless heart valve performance under highly dynamic loading conditions. Heart valves have evolved to have multi-layered leaflet structures. The aortic valve, for example, consists of three histologically distinct layers, whereas the mitral valve has four (Fig. \ref{fig:valvelayers}). The fibrosa layer, which is located on the ventricular side of atrioventricular valves and the atrial side of semilunar valves, is composed of circumferentially aligned collagen fibers that provide the leaflets with the necessary tensile strength to open and transmit forces during coaptation while closed. The spongiosa layer is situated adjacent to the fibrosa and though it contains some collagen, its main constituents are the hydrophilic glycosaminoglycans and proteoglycans, which give the valve its compressive properties and allow it to absorb high forces during coaptation. The ventricularis and atrialis are the layers that are adjacent to blood flow in atrioventricular valvesand semilunar valves, respectively. These layers are rich in radially oriented elastin fibers and facilitate the closure movement by extending the valve leaflet as it opens and recoils when it closes. The annulus and chordae tendineae of the atrioventricular valves and the connection between the leaflets and the surrounding myocardium in the semilunar valves provide additional support. 
    


%-------------------	begin FIGURE 	-------------------%
\begin{figure}
\centering
\includegraphics[width=\textwidth]{Images/chapter1/multiscalevalve.png}
\caption{The multiscale nature of heart valve biomechanics: a representation of the mitral valve at the organ-, tissue-, and cell-levels. At the tissue-level: a circumferentially oriented cross-section of the mitral valve anterior leaflet stained with Movat pentachrome, which colors collagen yellow, elastic fibers black, and hydrated PGs and GAGs blue. At the cell-level: a transmission electron micrograph of a mitral VIC from the fibrosa layer. (Adapted from \cite{salma_heart_2016})}
\label{fig:multiscalevalve}
\end{figure}
%-------------------	 end FIGURE 	-------------------%


%-------------------	begin FIGURE 	-------------------%
\begin{figure}
\centering
\includegraphics[width=\textwidth]{Images/chapter1/valvelayers.jpg}
\caption{Scanning electron micrograph of the multilayered microenvironment of the MV anterior leaflet. Individual micrographs of each layer are also presented: elastin-rich ventricularis and atrialis, highly collagenous fibrosa, and proteoglycan-rich spongiosa. The collagen fibrils and elastic fibers closely surround the interstitial cells and highlight the long cellular extensions. In the fibrosa, collagen fibrils are aligned in the circumferential direction of the leaflet, which is responsible for the observed anisotropy in leaflet mechanical behavior. (T: transmural, C: circumferential). (Adapted from \cite{salma_heart_2016}.)}
\label{fig:valvelayers}
\end{figure}
%-------------------	 end FIGURE 	-------------------%
   




    The collagen dense fibrosa layer is the predominant stress-bearing layer. Collagen fibers have low torsional and flexural stiffness but can withstand high tensile forces. As such, fiber orientation can be used as an identifier for the material axes of the tissue, or in other words the directions in which the tissue is able to withstand the greatest tensile stresses. To measure the orientation distribution of the collagen fibers, small angle light scattering (SALS) is a popular choice \cite{sacks_small_1997}. In this technique, a beam of laser, 500-600 nm in wavelength, is passed through a tissue specimen and is then scattered. The spatial intensity distribution of the resulting scattered light depends on the orientation of the collagen fibers (which is at a similar length scale as the light particles), which can in turn be used to obtain structural information of the tissue. As an example, Sacks et al. used SALS to quantify the changes that occur in AV leaflet structure with increasing transvalvular pressure (TVP) \cite{sacks_aortic_1998}.  Fresh porcine AVs were fixed at TVPs ranging from 0 to 90 mmHg and imaged using small angle light scattering (SALS). Overall, increasing TVP induced the greatest changes in fiber alignment between 0 and 1 mmHg, and past 4 mmHg there was no detectable improvement in fiber alignment (Fig. \ref{c1:fig:salsaortic}b-d). 
    
    
%-------------------	begin FIGURE 	-------------------%
\begin{figure}
\centering
\includegraphics[width=5.5in]{Images/chapter1/salsaortic.pdf}
\caption{(a) Diagram of the AV cusp highlighting the belly, commissures, nodulus, and regions of coaptation. Small angle light scattering results with the orientation index at (b) 0mmHg, (c) 4mmHg, and (d) 90mmHg transvalvular pressure. No further changes in fiber alignment were observed past 4mmHg. These results are consistent with histological-based data that quantifies the percent area of tissue displaying collagen fiber crimp (e). (Adapted from Sacks et al. \cite{sacks_aortic_1998})}
\label{c1:fig:salsaortic}
\end{figure}
%-------------------	 end FIGURE 	-------------------%

    
    The other most important structural information about collagen fibers is the amount to which the collagen fibers are crimped. As collagen fibers have low torsional and flexural stiffness, they bear minimal stresses before they are fully straightened. Thus, the stretch needed to straighten these collagen fibers will determine the compliance or extensibility at the tissue-level. One example for the attempt to quantification of collagen fiber crimp is the methods of Hilbert et al. \cite{hilbert_optical_1986, hilbert_porcine_1990}. They quantified the amount of collagen fiber crimp in the native pulmonary and aortic HVs by identifying the cross-sectional regions that displayed observable crimp \cite{joyce_functional_2009}. It was found that at 0 mmHg, approximately 60\% of the AV transverse cross-sectional area was occupied by crimp structure (Fig. \ref{c1:fig:salsaortic}e). As the TVP increased, the percent crimp decreased rapidly until 20 mmHg, with minimal decreases in percent crimp thereafter. For the AV, much of the observed change in collagen structure is due to the finely tuned straightening of the collagen fibers, which must occur at the right strain level and at the right rate to facilitate coaptation without allowing excessive tissue deformations that could lead to regurgitation. The unique structure of the commissure region, which approximately corresponds to the coaptation region, highlights the adaptive structure of HVs. Instead of undergoing TVP differences, the coaptation region is loaded in a uniaxial-like manner due to tethering forces generated at the attachment of the commissures to the aortic root. Unlike the biaxially loaded belly regions of the valve, the uniaxial loading of the commissures makes them more highly aligned, similar to tendon. Fiber uncrimping with stress occurs very rapidly for this highly aligned fiber network, as demonstrated by the short transition region from low to high stiffness. The highly aligned nature of the commissure region at unloaded state and the more rapid realignment with TVP in the 10 commissure regions are consistent with the pre-transition strain level behavior of tendon- like materials. 

\section{Valvular diseases and prevalence}
\begin{enumerate}
\item Bioprosthetic heart valve replacements cannot grow, remodel or self repair.
\item Current bioprosthetic heart valves are not patient specifically designed. 
\item Surgical success even after only sixth month can be hard to predict. 
\end{enumerate}

\section{Surgical repair and replacement}


\section{Heart valve prosthesis}

\subsection{Brief history of the development of heart valve replacements}

\subsection{Advantages of tissue-engineered valves}

\subsection{Fabrication of bioprosthetic heart valves}

\subsection{Current and future technology on bioprosthetic heart valves}




\section{Bioprosthetic heart valve durability and limitations}

\subsection{Bioprosthetic heart valve life-span}

\subsection{Limitations in bioprosthetic heart valve design and fabrication}


\section{Causes of bioprosthetic heart valve failure}

\subsection{Biological and mechanical aspects of fatigue}

\subsection{Brief summary on the role of calcification}

\subsection{Mechanical fatigue and failure}

\section{Motivation, rationale, and specific aims}

\section{Currently studies on the mechanical fatigue of bioprosthetic heart valves}


\begin{enumerate}
\item Bioprosthetic heart valve replacements cannot grow, remodel or self repair.
\item Current bioprosthetic heart valves are not patient specifically designed. 
\item Surgical success even after only sixth month can be hard to predict. 
\end{enumerate}




\bibliographystyle{plainnat}
\bibliography{phd}





\chapter{Structural constitutive models for planar collagenous soft tissues}


\section*{Preface}
\addcontentsline{toc}{section}{Preface}%

    Fundamental to developing a deeper understanding of soft tissue function and pathology is the development of an accurate tissue-level constitutive model. In the present work, we developed a novel meso-scale (i.e. at the level of the fiber, 10-100 $\mu$m in length scale) structural constitutive model (MSSCM) with application to MV leaflet and ovine pulmonary arterial tissues. This model takes into account the layered structure of these tissue and the contributions from the distinct collagen and elastin fiber networks within each tissue layer. Requisite collagen and elastin fibrous structural information for each layer were quantified using second harmonic generation microscopy and conventional histology. A comprehensive mechanical data set was also used to guide model formulation and parameter estimation. Furthermore, novel to tissue-level structural constitutive modeling approaches, we allowed the collagen fiber recruitment function to vary with orientation. Finally, a novel fibril-level (0.1 to 1 $\mu$m) validation approach was used to compare the predicted collagen fiber/fibril mechanical behavior with extant MV small angle X-ray scattering data. Results demonstrated excellent agreement, indicating that the MSSCM fully captures the tissue-level function. Future utilization of the MSSCM in computational models of the MV will aid in producing highly accurate simulations in non-physiological loading states that can occur in repair situations, as well as guide the form of simplified models for real-time simulation tools.

\textbf{The work contained in this chapter was published as}:  %\bibentry*{}




\section{Major structural bearing components of soft tissues}
\subsection{Collagen}
\subsection{Elastin}
\subsection{Proteoglycans and Glycosaminoglycans}
\subsection{Ground matrix}

\section{Characterizing the mechanical responses of soft tissues}
\subsection{Current techniques for acquiring mechanical responses of soft tissues}
\subsection{Methods for the analysis of mechanical data}
\subsection{Improved method for analyzing the biaxial mechanical response of soft tissues*}

\section{Material models for soft tissues}
\subsection{Phenomenological approaches}
\subsection{Multi-scale and other structural-mechanism approaches}
\subsection{Meso-scale structural approaches for collagenous soft tissues}

\section{Key mechanisms of meso-scale structural constitutive models}
\subsection{Fiber-level response of collagen and elastin fibers}
\subsubsection{Mechanical response of single collagen fibers}
\subsubsection{The elastica effect of collagen fibers}
\begin{itemize}
\item Write of the work on improved elastica model based on Garikipati's paper.
\item Summarizing the elastica portion of "Large strain stimulation promotes extracellular matrix production and stiffness in an elastomeric scaffold model" by Joao. 
\end{itemize}

\subsection{Modeling the tissue-level response}
\subsubsection{Response of collagen fiber ensembles}
\subsubsection{Affine kinematics in dense collagenous soft tissues}
Summarized the results of the affine kinematics paper by Chung-hao Lee here





%\include{Chapters/chapter3-biaxialtesting}

% \chapter{Modeling the response of mitral valves and pulmonary arteries}

\section{Insights into regional adaptations in the growing pulmonary artery using a meso-scale structural model*}

Insert article "Insights into regional adaptations in the growing pulmonary artery using a meso-scale structural model: effects of ascending aorta impingement."

\section{A meso-scale layer-specific structural constitutive model of the mitral heart valve leaflets*}

Insert article "A meso-scale layer-specific structural constitutive model of the mitral heart valve leaflets"



\chapter{Effect of exogenous cross-linking on the mechanical response of soft tissues}


\section*{Preface}
\addcontentsline{toc}{section}{Preface}%

    Exogenous cross-linking of soft collagenous tissues is a common method for biomaterial development and medical therapies. It is an important part of the fabrication process of BHVs and significantly affects their mechanical properties. To enable improved applications through computational methods, physically realistic constitutive models are required. Yet, despite decades of research, development and clinical use, no such model exists. In this study, we develop the first rigorous full structural model (i.e. explicitly incorporating various features of the collagen fiber architecture) for exogenously cross-linked soft tissues. This was made possible, in part, with the use of native to cross-linked matched experimental datasets and an extension to the collagenous structural constitutive model so that the uncross-linked collagen fiber responses could be mapped to the cross-linked configuration. This allowed us to separate the effects of cross-linking from kinematic changes induced in the cross-linking process, which in turn allowed the non-fibrous tissue matrix component and the interaction effects to be identified. It was determined that the matrix could be modeled as an isotropic material using a modified Yeoh model. The most novel findings of this study were that: (i) the effective collagen fiber modulus was unaffected by cross-linking and (ii) fiber-ensemble interactions played a large role in stress development, often dominating the total tissue response (depending on the stress component and loading path considered). An important utility of the present model is its ability to separate the effects of exogenous cross-linking on the fibers from changes due to the matrix. Applications of this approach include the utilization in the design of novel chemical treatments to produce specific mechanical responses and the study of fatigue damage in bioprosthetic heart valve biomaterials.

\textbf{The work contained in this chapter was published as}:  Sacks, M. S.; Zhang, W. \& Wognum, S.
A novel fibre-ensemble level constitutive model for exogenous cross-linked collagenous tissues 
Interface Focus, The Royal Society, 2016, 6, 20150090 


%---    INTRODUCTION
\section{Introduction}

    The application of exogenous-cross-links (EXLs) to native or biologically derived soft collagenous tissues finds its way into a wide range of medical therapies and device applications, such as surgical biomaterials, modification of corneal tissues (using riboflavin/UVA) and vascular grafts. Perhaps the most mechanically demanding application is the so-called bioprosthetic heart valve (BHV), which is fabricated from several types of biologically derived soft collagenous tissue membranes. From a clinical perspective, BHVs have important advantages in that they do not require permanent anticoagulation therapy, operate noiselessly, and have blood flow characteristics similar to the native valve, and thus have become the dominant heart valve therapy worldwide \cite{bini_noncollagenous_1999,schoen_cardiac_2005,schoen_founders_1999}. However, BHV durability continues to remain limited to the range of 10–15 years, resulting from leaflet structural deterioration mediated by fatigue and/or tissue mineralization \cite{vesely_tissue_2001,sacks_calcification_1999}. In general, structural damage is a critical factor in BHV degeneration, and clearly implicates coupled material and design factors as major limiters to long-term durability \cite{schoen_calcification_2005,schoen_pathology_2001}. However, a major reason why advances in the use of cross-linked tissues in BHVs and other biomedical applications is a dearth of knowledge on how EXLs affect the underlying tissue structure and macroscopic mechanical behaviour. Moreover, integration of such information into truly predictive modelling cannot proceed without accurate constitutive models of the EXL tissues and the subsequent fatigue processes \cite{sacks_incorporation_2003,sun_finite_2005}.


    The most common medical applications of cross-linked biologically-derived tissues are for dense collagenous tissues. Such tissues are typically composed of a dense, highly-organized network of type I collagen fibres, along with elastin, proteoglycans, glycosaminoglycans, cellular materials and a small amount of other fibrillar proteins. Type I collagen is the major determinant of its mechanical behaviour \cite{parry_molecular_1988,gelse_collagens_2003} and is the major tissue component affected by EXLs. At the molecular level, tropocollagen molecules are composed of a triple helix of three alpha chains \cite{parry_molecular_1988,gelse_collagens_2003} that arrange themselves into a quarter stacking array to form the collagen fibril \cite{parry_molecular_1988,gelse_collagens_2003}. Collagen fibrils form the functional subunits of the collagen fibres, as described by the Hodge–Petruska model \cite{parry_molecular_1988,petruska_subunit_1964}, and then exhibit distinct large-scale structures (figure \ref{c3:fig:1}). Like the fibrils from which their functional properties are derived, collagen fibres exhibit high tensile but low flexural stiffness \cite{sacks_biomechanics_2009}. We note that there is no standard definition for a fibre and its relation to the fibril. They are very dependent on the specific tissues involved, and thus caution should always be exercised in the terminology used. In the mainstream tissue biomechanics literature, the fibre/fibril definition is often taken from the well-known work of Kastelic et al. \cite{nimni_collagen_2018}, which focused on tendon. The pericardial tissues considered in this study clearly show fibre and fibril structures, including the fibre undulations commonly observed optically \cite{shen_stress_2008} (figure \ref{c3:fig:1}).
    
    
%%%%%%%%%%%%%%%%%%%%	begin FIGURE 	%%%%%%%%%%%%%%%%%%%%
\begin{figure}
\centering
\includegraphics[width=\textwidth]{Images/chapter3/F1large.jpg}
\caption{(a) Photomicrograph of native bovine pericardium showing the undulated collagen fibres (adapted from \cite{sacks_incorporation_2003}). (b) TEM image also of native bovine pericardium clearly showing interrelationships between the undulated collagen fibres and the underlying fibril structures (magnification 4150$\times$; adapted from \cite{nimni_collagen_2018}).}
\label{c3:fig:1}
\end{figure}
%%%%%%%%%%%%%%%%%%%%	 end FIGURE 	%%%%%%%%%%%%%%%%%%%%

    
    The fibre longitudinal/axial direction is the primary determinant of the stiffness of the tissue composite. Interestingly, collagen fibres typically have a stiffness of 1 GPa \cite{shen_stress_2008,gentleman_mechanical_2003,eppell_nano_2006,yang_mechanical_2008} and extend by no more than 4–5\%. 
    To increase the tissue-level compliance, collagen fibres at the macroscopic scale are sinusoidally crimped \cite{parry_molecular_1988}. Tissue-level stress will not occur until the fibre-level crimp has been straightened. Moreover, the distribution of fibre straightening strains is the mechanism of tissue nonlinearity at large strains \cite{lanir_constitutive_1983,sacks_multiaxial_2003}. Thus, as in many other fields, the connection to the underlying structure can greatly inform our understanding of how collagenous tissues work and guide the development of mathematical models of their mechanical function.


    While there is a wide range of application-specific chemical agents (e.g. riboflavin/UVA cross-linking for corneas), for mechanically-demanding and blood\Hyphdash contacting applications (e.g. BHVs) using collagenous tissues, an aqueous solution of glutaraldehyde (GLUT) is used. GLUT application is necessary to both biochemically and mechanically stabilize the tissue for in vivo use. During the cross-linking process GLUT rapidly permeates the tissue, with the cross-linking process largely complete within an hour and essentially stabilized by 24 hrs. Much of what we know about how GLUT EXLs alter the structures of collagenous tissues was reported by Nimni, Cheung and co-workers \cite{cheung_mechanism_1990,nimni_chemically_1987,cheung_mechanism_1985,gendler_toxic_1984,cheung_presence_1983,cheung_mechanism_1982,cheung_mechanism_1982II}. Briefly, GLUT reacts primarily with e-amino groups of lysyl residues in proteins, with Michael-addition reaction products of Schiff bases usually the final stable products. Based on the spectral characteristics and the molecular weights of the reaction products, it has been predicted that GLUT reacts with free amines to form an intermediate with a molecular weight of about 200 Da. The GLUT–polymer amine complex is self-limiting in size and can undergo internal rearrangement to become chemically inert. An increased molecular length of GLUT polymers from the initial glutaraldehyde and lysyl-residue reaction is more likely than an increased number of cross-linked sites. Following free GLUT depletion by binding to reactive groups, additional GLUT molecules attached to already reacted molecules can give rise to larger GLUT polymers that are able to generate ‘long-range cross-links’ between further removed reactive sites (figure \ref{c3:fig:2}). It is apparent from a mechanical behaviour perspective that GLUT-associated chemical cross-linking of the collagen structure and biochemistry can produce complex changes from the native state at the molecular, fibril, fibre and tissue levels.
    
    
%%%%%%%%%%%%%%%%%%%%	begin FIGURE 	%%%%%%%%%%%%%%%%%%%%
\begin{figure}
\centering
\includegraphics[width=\textwidth]{Images/chapter3/F2large.jpg}
\caption{A diagram showing the interaction of tropocollagen molecules with glutaraldehyde and how cross-links can form. As the concentration of GLUT increases, the number of activation sites and chain length increases, and a limited number of cross-links will form between such molecules (magnification 4150$\times$; adapted from \cite{nimni_collagen_2018}).}
\label{c3:fig:2}
\end{figure}
%%%%%%%%%%%%%%%%%%%%	 end FIGURE 	%%%%%%%%%%%%%%%%%%%%

    
    To enable improved application of the use of EXL tissues in in situ treatment and prosthesis design, the development of physically realistic constitutive models is clearly required. In a previous work, a structural approach was used that incorporated experimentally measured angular distribution of collagen fibres and an assumed isotropic form for the EXL matrix \cite{sacks_structural_2000}. Good agreement with the experimental data was observed, supporting the basic approach. An important utility of that early model was its ability to separate the effects of the fibres and matrix. However, it was only a first step; other factors such as bending rigidity of EXL fibres, fibre–fibre interactions, and fibre–matrix interactions were not considered. Moreover, the experimental data were reduced under the assumption of an isotropic Fung model; no rigorous investigation of the most appropriate form was undertaken.
    
    
    The focus of the present work is to more fully investigate the underlying characteristics of the effects of EXLs on soft collagenous tissues, and to use this information to develop a meso-scale (i.e. at the level of the fibre) structural constitutive model. In particular, we explored the following questions of the effects of EXLs on native collagenous tissues: (i) what are the effects on individual collagen fibres, (ii) what are the effects on single collagen fibre ensembles, (iii) are there interactions between fibre ensembles, and (iv) what is the functional form of the effective matrix response? This was done by exploiting experimental data from \cite{sun_biaxial_2003}, wherein structurally controlled pericardial specimens were tested in the native state and then the EXL state. From these results, a comprehensive structural constitutive model was developed for EXL collagenous tissues and its predictive capability was evaluated. We note that, while we ultimately seek the micro-mechanical basis for macro-scale function, the present work is focused on a fibre-ensemble level approach.
    
    
    



%---    Experimental methods and data post-processing
\section{Experimental methods and data post-processing}

\subsection{Tissue sources and experimental methods} \label{c3:sec:21}

    Details of the tissue source, preparation and mechanical evaluation have been previously presented \cite{sun_biaxial_2003}. Briefly, large sections of native bovine pericardium were stored in phosphate-buffered saline (pH 7.4) at $4^\circ C$, then optically cleared using a hyperosmotic solution and the collagen fibre architecture (CFA) quantified. From the resulting CFA information, $25\times25$ mm test specimens exhibiting a high degree of structural uniformity suitable for biaxial testing were selected. The collagen fibre preferred and cross-preferred directions were aligned to the $X_1$-$X_2$ axes (figure \ref{c3:fig:3}). A total of five specimens were prepared in the native state.
    
%%%%%%%%%%%%%%%%%%%%	begin FIGURE 	%%%%%%%%%%%%%%%%%%%%
\begin{figure}
\centering
\includegraphics[width=\textwidth]{Images/chapter3/F3large.jpg}
\caption{a)(i) Pericardial test specimen showing a high degree of fibre orientation and uniformity in preferred fibre directions, with the PD = $X_1$ and XD = $X_2$ axes defined, and (ii) a typical biaxial test specimen mounted on the device. (b) A schematic of the biaxial test specimen geometry changes with cross-linking and the corresponding mean deformation gradient tensor components in native state $\beta_0$ and EXL state state $\beta_1$. Here, cross-linking induced a 6\% contraction in the PD and and 7\% direction in the XD, with some small shearing.}
\label{c3:fig:3}
\end{figure}
%%%%%%%%%%%%%%%%%%%%	 end FIGURE 	%%%%%%%%%%%%%%%%%%%%

    Biaxial mechanical testing methods have been previously described in detail \cite{sacks_orthotropic_1998,sacks_method_1999}. Briefly, testing was performed with the specimen immersed in phosphate-buffered normal saline (pH 7.4) at room temperature. First the Piola Kirchhoff stress $\mathbf{P}$ controlled test protocol was used, wherein the ratio of the normal stress components $P_{11}$:$P_{22}$ was kept constant, with $P_{12} = P_{21} = 0$ and a maximum stress level of 1 MPa was used. Tissue deformations were quantified from the motion of four markers placed in the central third of the specimen, from which the deformation gradient tensor $\mathbf{F}$ was determined. For the first testing phase, an equi-biaxial stress protocol (i.e. $P_{11}$:$P_{22}$ = 1 : 1) was used for both preconditioning and data acquisition. A total of 15 contiguous cycles were run with an approximate strain rate of 0.01 $s^{-1}$. Next, seven successive protocols were performed using ratios $P_{11}$:$P_{22}$ = 1 : 0.1, 1 : 0.5, 1 : 0.75, 1 : 1, 0.75 : 1, 0.5 : 1 and 0.1 : 1. This range was chosen for extensive coverage of in-plane strain state. After testing, each native specimen was allowed to mechanically re-equilibrate by storing them in a stress-free state at $4^\circ C$ for 24 h. Next, each specimen was chemically treated with 0.625\% GLUT for a minimum of 72 h, with the tissue marker dimensions monitored throughout the cross-linking procedure, and then stored in phosphate-buffered normal saline at $4^\circ C$. As a final step, the above biaxial testing sequence was repeated. Data post-processing included computation of the second Piola-Kirchhof tensor $\mathbf{S}$ and deformation gradient tensor $\mathbf{F}$ using established methods \cite{zhang_generalized_2015}. This test design allowed a comprehensive planar mechanical behaviour dataset to be collected on matched native and EXL specimens, compensating for inter-specimen variations.
    
    
    
    
\subsection{Kinematic considerations and mechanical data post-processing}

    As observed in our other studies \cite{sacks_biaxial_2000,zhang_generalized_2015}, the chemical fixation process will affect the specimen dimensions, and any analysis must carefully account for these effects on the collagen fibre kinematics. We thus defined the following configurations: $\beta_0$-native, $\beta_1$-EXL (figure \ref{c3:fig:3}b), used as the referential configurations for the native and EXL states, respectively. We represented all deformations using the notation for the deformation gradient tensor $\prescript{j}{i}{\mathbf{F}}$ where $i$ and $j$ represent the initial and final configurations, respectively (Nomenclature). Values for the components of $\prescript{j}{i}{\mathbf{F}}$ were determined using the same method from section \ref{c3:sec:21} for the displacements of the four markers pre- and post-cross-linking for each specimen. Next, as first described by Lanir \cite{lanir_constitutive_1983}, we defined a fibre ensemble as a group of fibres with a common orientation. It has been shown that the ensemble stress–strain relation can be obtained from the interpolated equi-biaxial strain path, where $\mathbf{F} = \operatorname{diag}[\lambda, \lambda, 1/\lambda^2]$ using $S_{ens} = S_{11} + S_{22}$ \cite{sacks_incorporation_2003}. To derive the equi-biaxial strain path $\lambda_1 = \lambda_2$ from the stress-controlled experimental data, all mechanical data were combined and interpolated using cubic Hermite patches \cite{fata_insights_2014}. A strain path with $\lambda_1 = \lambda_2$ was interpolated within the range of the data as defined by the convex hull of ($\lambda_1$, $\lambda_2$), and was implemented separately for each stress component (figure \ref{c3:fig:4}). To reliably overcome regions of sparse data we enforced the surface to be strictly convex everywhere. Finally, since the $S_{12}$ component was negligible in all specimens, it was ignored in the subsequent analyses.
    
    
%%%%%%%%%%%%%%%%%%%%	begin FIGURE 	%%%%%%%%%%%%%%%%%%%%
\begin{figure}
\centering
\includegraphics[width=5in]{Images/chapter3/F4large.jpg}
\caption{An example of the bicubic Hermite surface interpolation of the $S_{22}$ biaxial test responses to allow interpolation of an equi-biaxial strain path, shown here in red. The blue path defines the span of the strain.}
\label{c3:fig:4}
\end{figure}
%%%%%%%%%%%%%%%%%%%%	 end FIGURE 	%%%%%%%%%%%%%%%%%%%%
    


\subsection{Establishing and modelling the mechanical behaviour of the native collagen fibre ensemble}
    
    Based on our previous tissue model findings \cite{sacks_incorporation_2003,fata_insights_2014,fan_simulation_2014,lee_presence_2015}, the dominant cause of the nonlinearity of the tissue-level mechanical behaviour of collagenous tissues is the gradual recruitment of collagen fibres \cite{lanir_constitutive_1983}. The collagen fibres themselves behave linearly under typical fibre strains experienced under physiological stresses in tissues (2–5\%). 
    Once all fibres are fully straightened, the summed response should appear linear in the ensemble response. When applied to the equi-biaxial strain derived fibre-ensemble data, the upper bound can be directly determined as the transition point between the nonlinear and linear regions, with the slope of the linear region establishing the maximum tangent modulus (MTM) of the collagen fibre ensemble (e.g. \cite{fata_insights_2014}). The matrix response can also be determined from the pre-recruitment region where no collagen fibre have contributed. To determine the recruitment upper bound in the native tissue, we started at the largest measured strain and decreased the strain level until the region above was no longer linear. Linearity was defined from the mean squared error (MSE) of the linear regression to be less than 0.005\% of the total MSE of all data, where $\mathrm{MSE} = \sum_i^n (\mathrm{S}^i_{ens} - \bar{\mathrm{S}}_{ens})/n$. Similarly, we determined the lower bound by starting at zero strain and increasing the strain until a deviation from linearity was determined.
    
    
    Next, we note that in some previous structural models a fibre stress–strain relation linear in the second Piola-Kirchhoff stress and Green Lagrange strain has been used \cite{sacks_incorporation_2003,fan_simulation_2014,lanir_structural_1979}. However, SAXS studies have demonstrated a linear force\Hyphdash displacement relation for collagen fibrils in the tendon \cite{sasaki_elongation_1996,sasaki_stress_1996} and MV tissue \cite{liao_relation_2007}. This is further corroborated by the atomistic modelling results by Buehler \cite{buehler_atomistic_2006}, where the force\Hyphdash displacement relation is essentially linear at strains lower than 0.35. We have recently determined that for the mitral valve leaflet the tissue level-derived collagen fibre mechanical behaviour is actually quite linear, with an effective modulus of approximately 160 MPa. Based on these considerations, we assumed for the native collagen fibres that
        \begin{enumerate}
            \item they exhibit a linear $P-\lambda$ response
            \item slack stretch of the collagen fibres does not vary with orientation.
        \end{enumerate}
    From these two basic considerations, we used the following effective native collagen fibre model \cite{fata_insights_2014,fan_simulation_2014}. We start by defining the native collagen fibre strain energy as
        %-------------------	begin EQUATION 	-------------------%
        \begin{equation}\label{c3:eqn:21}
        \begin{aligned}
        \Psi_f(\lambda_t) = 
            \begin{cases} 
                \frac{\eta_c}{2} \left(\lambda_t -1\right)^2 & \text{for } \lambda_t >1\\
                0 & \text{for } \lambda_t < 1  
            \end{cases}
        \end{aligned}
        \end{equation}
        %-------------------	 end EQUATION 	-------------------%
    where $\lambda_t = \lambda_f/\lambda_s$ is the true stretch of the fibre. This leads to the following $P-\lambda$ form using $P_f = \partial \Psi_f(\lambda_t)/\partial \lambda_f = \partial \Psi_f(\lambda_t)/\partial \lambda_t \cdot \partial \lambda_t/\partial \lambda_f$
        %-------------------	begin EQUATION 	-------------------%
        \begin{equation}\label{c3:eqn:22}
        \begin{aligned}
        P_f = 
            \begin{cases} 
                \frac{\eta_c}{\lambda_s} \left(\lambda_t -1\right) & \text{for } \lambda_t >1\\
                0 & \text{for } \lambda_t < 1  
            \end{cases}
        \end{aligned}
        \end{equation}
        %-------------------	 end EQUATION 	-------------------%
    where $P_f$ is the first Piola-Kirchhoff stress of the fibre, $\eta_c$ is the modulus of the fibre, $\lambda_f$ is the fibre stretch and $\lambda_s$ is the fibre slack stretch. Next, we use this fibre model in the expression for the native collagen fibre ensemble using
        %-------------------	begin EQUATION 	-------------------%
        \begin{equation}\label{c3:eqn:23}
        \begin{aligned}
        P_c^{ens} =& \phi_c\eta_c\int_1^{\lambda_\theta}\frac{D(x)}{x} \left(\frac{\lambda_\theta}{x} - 1\right) \dif x    \\
        \text{and} \quad \dpd{P_c^{ens}}{\lambda_\theta} =&\mathrm{TM}_c^{ens} = \phi_c\eta_c\int_1^{\lambda_\theta}\frac{D(x)}{x^2}\dif x,
        \end{aligned}
        \end{equation}
        %-------------------	 end EQUATION 	-------------------%
    where $\phi_c$ is the collagen fibre mass fraction, $\lambda_\theta$ is the fibre-ensemble stretch along the direction defined by $\theta$ (computed from the tissue-level deformation using $\lambda_\theta = \mathbf{F}\cdot\mathbf{n}(\theta)$), and $D(\lambda_s)$ is the probability distribution function describing the distribution of collagen fibre slack length within the ensemble. We assumed $D(\lambda_s)$ is Beta distributed, so that
        %-------------------	begin EQUATION 	-------------------%
        \begin{equation}\label{c3:eqn:24}
        \begin{aligned}
        D(\alpha, \beta, \lambda_{lb}, \lambda_{ub}, \lambda_s) =& 
            \begin{cases}
            \frac{y^{\alpha-1}(1-y)^{\beta-1}}{B(\alpha,\beta)(\lambda_{ub}-\lambda_{lb})} & \text{for } y \in [0,1] \\
            0 & \text{otherwise}
            \end{cases} \\
        y=&\frac{\lambda_s-\lambda_{lb}}{\lambda_{ub}- \lambda_{lb}}, \quad \bar{\mu} =\frac{\mu - \lambda_{lb}}{\lambda_{ub}-\lambda_{lb}}   \\
        and \quad \bar{\sigma} =& \frac{\sigma}{\lambda_{ub} - \lambda_{lb}}, \quad \alpha = \frac{\bar{\mu}^2 - \bar{\mu}^3 - \bar{\sigma}^2\bar{\mu}}{\bar{\sigma}^2},   \\
        \beta =& \alpha \frac{1-\bar{\mu}}{\bar{\mu}}
        \end{aligned}
        \end{equation}
        %-------------------	 end EQUATION 	-------------------%
    where $\lambda_{lb}$ and $\lambda_{ub}$ are the lower and upper bound stretch of the collagen fibre recruitment, respectively. Note that in preliminary examinations of the data we found that all specimens exhibited distinct pre- and post-transition locations (figure \ref{c3:fig:5}), allowing $\lambda_{lb}$ and $\lambda_{ub}$ to be determined directly from the collagen fibre ensemble data. Thus, the complete initial ensemble model (equation \ref{c3:eqn:23}) has three parameters $\{\eta_c, \alpha, \beta\}$ to fit to the data using standard techniques \cite{fata_insights_2014}.
    
    
    
    
    
%%%%%%%%%%%%%%%%%%%%	begin FIGURE 	%%%%%%%%%%%%%%%%%%%%
\begin{figure}
\centering
\includegraphics[width=\textwidth]{Images/chapter3/F5large.jpg}
\caption{A representative fibre-ensemble stress–strain (in $P_\mathrm{ens}-\lambda_\mathrm{ens}$) response for (a) native and (b) cross-linked bovine pericardium illustrating a well-defined post-transition fibre recruitment point wherein the response becomes linear. While the native pericardium demonstrated a very low initial modulus (approx. 75 kPa; table \ref{c3:tab:2}), the EXLs demonstrate a significantly stiffer modulus.}
\label{c3:fig:5}
\end{figure}
%%%%%%%%%%%%%%%%%%%%	 end FIGURE 	%%%%%%%%%%%%%%%%%%%%

%---    Delineation and modelling of the tissue-level mechanical effects of exogenous cross-links
\section{Delineation and modelling of the tissue-level mechanical effects of exogenous cross-links} \label{c3:sec:3}

\subsection{Rationale}

    While extensive work has been done on the characterization of the biomechanical effects of EXL formation on soft tissues \cite{sun_biaxial_2003,sacks_bioprosthetic_2006,sun_simulated_2005,sellaro_effects_2007,alferiev_prevention_2003,wells_cyclic_2005,wells_effects_2002,wells_effects_2000,wells_thermomechanical_1998,lee_high_1994,naimark_correlation_1992,lee_effect_2001,barber_mechanics_1999,mirnajafi_effects_2005}, there has been surprisingly little work done known to the authors on the development of formal constitutive models (other than \cite{sacks_structural_2000}). Related work on proteoglycan and related collagen fibril sub\Hyphdash forms have revealed complex micromechanical interactions (e.g. \cite{cavalcante_mechanical_2005,coughlin_dynamic_1996}), but micromechanical interactions modified by EXLs on the macroscale tissue responses remain largely unknown. Thus, prior to developing the constitutive model form, we first carefully examined the effects of EXL formation on the measured tissue\Hyphdash level biomechanical behaviours in the present data set.


\begin{table}
\centering
\caption{Equibiaxial strain testing results.}\label{c3:tab:2}
\begin{tabular}{L{.5in}R{.48in}R{.48in}R{.48in}R{.48in}R{.48in}R{.48in}R{.48in}R{.48in}}
\hline
\multicolumn{1}{l}{\textbf{ID}} &
\multicolumn{1}{c}{\textbf{12}} &
\multicolumn{1}{c}{\textbf{21}} &
\multicolumn{1}{c}{\textbf{41}} &
\multicolumn{1}{c}{\textbf{61}} &
\multicolumn{1}{c}{\textbf{68}} &
\multicolumn{1}{c}{\textbf{mean}} &
\multicolumn{1}{c}{\textbf{s.e.m.}} &
\multicolumn{1}{c}{\textbf{p-value}}\\
\hline
\multicolumn{9}{c}{initial tangent modulus (kPa)} \\
\hline
native & 73.9 & 65.6 & 25.5 & 110.6 & 107.3 & 76.6 & 17.4 & \\
\hline
GLUT & 1059.3 & 248.4 & 152.4 & 746.0 & 272.1 & 495.6 & 195.2 & 0.037   \\
\hline
\multicolumn{9}{c}{MTM (kPa)} \\
\hline
native & 54208 & 49014 & 53091 & 72460 & 77121 & 61179 & 6341.3 &   \\
\hline
GLUT & 67771 & 28615 & 58025 & 62319 & 70718 & 57490 & 8433.8 & 0.567 \\
\hline
\multicolumn{9}{c}{upper bound stress (kPa)} \\
\hline
native & 1164.0 & 885.4 & 956.2 & 1048.4 & 1289.4 & 1068.7 & 80.80 &  \\
\hline
GLUT & 1081.7 & 925.8 & 1204.2 & 1138.2 & 1353.8 & 1140.8 & 78.69 & 0.245   \\
\hline
\end{tabular}
\end{table}





\subsection{Effects of exogenous cross-link formation on collagen fibre ensembles}

    All specimens exhibited anisotropic dimensional changes due to preconditioning and cross-linking (figure 3b). Interestingly, we found that about 6\% shrinkage occurred in the preferred direction, and approximately 7\% expansion in the cross-preferred directions. Such changes can alter both the angular dependence on collagen recruitment and the collagen fibre orientation distribution. We determined basic characteristics of the collagen fibre stress–strain relations directly from the data (no modelling), including the lower and upper bound-associated stresses, the initial tangent modulus, and the MTM from a running 15-point window. From this analysis, we were able to determine a number of important mechanical characteristics (table 2 and figure 5). These include:
        \begin{enumerate}
            \item All specimens exhibited an approximately 6.5-fold increase in the initial tangent modulus (table 2), very similar to values and native/EXL ratios reported in \cite{mirnajafi_effects_2005}, which were conducted under flex conditions.
            \item The upper bound stress and MTM were found to be unaffected by EXL formation (table 2).
            \item EXL formation induced a reduction in achieved strain levels compared with the native state (figure 5a).
            \item The effective collagen modulus was unaffected by EXL formation (figure 5c).
        \end{enumerate}
    Collectively, these results reveal some important features of the effects of EXLs on collagen tissues. First, as noted in our previous studies \cite{sacks_structural_2000,mirnajafi_effects_2005} EXLs produce a substantial increase in the low-strain modulus. Next, equation \ref{c3:eqn:23} indicates that the MTM is proportional to the collagen fibre modulus and the recruitment function D. Use of equation \ref{c3:eqn:23} compensates for the effect of the changes in tissue dimensions due to cross-linking on the fibre recruitment, allowing separation of changes in fibre architecture from the modulus on the ensemble stress–strain curve. Thus, the lack of changes in effective modulus are independent of any effects of changes resulting from tissue dimensions and represent an accurate modulus estimate.
    
    
    

%---    Initial model formulation
\section{Initial model formulation}

\subsection{General approach}

    The above findings provide sufficient information to develop a new model. In the present study, we assume EXLs induce fibre–fibre and fibre–matrix interactions that are mechanically significant. We ignore any time-dependent effects, as we have found that native and cross-linked valvular tissues exhibit minimal time-dependent effects \cite{grashow_planar_2006,grashow_biaxial_2006,stella_time_2007,eckert_biomechanical_2013}. Next, we assume that the pericardial tissues considered are only composed of collagen fibres and a matrix constituent that represents non-cross-linked and cross-linked components, and water. The contributions from elastin or other tissue components are ignored, because they have either negligible mass or stiffness. In all previous structural models of soft tissues, interactions between components (fibres, matrix) have been ignored. As we cannot assume this in the present investigation, we use the following hyperelastic general form:
        %-------------------	begin EQUATION 	-------------------%
        \begin{equation}\label{c3:eqn:41}
        \begin{aligned}
        \Psi(\mathbf{C}) = \phi_c[\Psi_c(\mathbf{C} +\Psi_\mathrm{int}(\mathbf{C}] + (1-\phi_c)\Psi_m(\mathbf{C} + p(J-1)
        \end{aligned}
        \end{equation}
        %-------------------	 end EQUATION 	-------------------%
    where $\phi_c$ is the mass fraction of the collagen fibres, $\phi_c$, $\phi_m$ and $\phi_\mathrm{int}$ are the strain energy density functions of the collagen, matrix and interaction terms, respectively, $J=\operatorname{det}(\mathbf{F})$, and $p$ is the Lagrange multiplier to enforce incompressibility. The resulting tissue-level response in terms of the second Piola-Kirchhoff stress tensor $\mathbf{S}$ is given by
        %-------------------	begin EQUATION 	-------------------%
        \begin{equation}\label{c3:eqn:42}
        \begin{aligned}
        \mathbf{S} =& 2\dpd{\psi}{\mathbf{C}} - p\mathbf{C}^{-1} \\
            =& 2\left[\phi_c\dpd{\Psi_c}{\mathbf{C}} + (1-\phi_c) \dpd{\Psi_m}{\mathbf{C}} + \phi_c \dpd{\Psi_\mathrm{int}}{\mathbf{C}}\right] - p\mathbf{C}^{-1}.
        \end{aligned}
        \end{equation}
        %-------------------	 end EQUATION 	-------------------%
        



\subsection{Accounting for changes in tissue dimensions for the collagen phase} \label{c3:sec:42}

    Results from section \ref{c3:sec:3} suggest that the native collagen fibre modulus is unaffected by cross-linking. However, the observed changes in tissue dimensions can also induce changes in tissue-level mechanical behaviour by altering the structure due to tissue shrinkage. This essentially results in a different reference configuration. There is thus a need to reformulate structural models to account for these effects directly. The formulation described in the following allows handling of changes in tissue reference state geometry. The key assumptions are:
        \begin{enumerate}
            \item Changes are due to alterations in the initial geometric configuration only, so that
            \begin{enumerate}
                \item Mass fractions of each phase remain unchanged.
                \item The internal mechanical energy remains zero. Thus, all changes in internal component configurations are not associated with any change in internal energy (which remains zero)—just initial configuration (e.g. fibre orientation, degree of undulation, thickness and length).
            \end{enumerate}
            \item Tissue dimensions and internal architecture change under the \textit{affine} kinematic assumption. Thus, the configuration of all constituent fibres in the altered reference state (after all changes in initial specimen geometry have taken place) can be predicted. Moreover, the configurational change is homogeneous, and can be thus described by a deformation gradient tensor with constant components.
            \item To be consistent with the fibre recruitment mechanisms (e.g. \cite{sacks_incorporation_2003,fata_insights_2014}), all fibres remain undulated in the new reference state.
            \item The matrix phase is unaffected by the geometric configuration changes and is referenced to $\beta_1$ (figure \ref{c3:fig:3}b) for all subsequent stress calculations.
        \end{enumerate}
        
        
    As a first step, we recast the recruitment function parameters determined in $\beta_0$ but mapped to $\beta_1$ using
        %-------------------	begin EQUATION 	-------------------%
        \begin{equation}\label{c3:eqn:43}
        \begin{aligned}
        &D_1(\mu_0, \sigma_0, \prescript{}{0}{\lambda}_{lb}, \prescript{}{0}{\lambda}_{ub}, \prescript{1}{0}{\lambda}, \prescript{}{1}{\lambda}_s) = 
            \begin{cases}
            \frac{y^{\alpha-1}(1-y)^{\beta-1}}{B(\alpha,\beta)(\prescript{}{1}{\lambda}_{ub}-\prescript{}{1}{\lambda}_{lb})} & \text{for } y \in [0,1] \\
            0 & \text{otherwise}
            \end{cases} \\
        &y=\frac{\prescript{t}{1}{\lambda}_s-\prescript{}{1}{\lambda}_{lb}}{\prescript{}{1}{\lambda}_{ub}- \prescript{}{1}{\lambda}_{lb}}, 
        \quad \prescript{}{1}{\lambda}_{ub} = \frac{\prescript{}{0}{\lambda}_{ub}}{\prescript{1}{0}{\lambda}}, 
        \quad \prescript{}{1}{\lambda}_{lb} = \frac{\prescript{}{0}{\lambda}_{lb}}{\prescript{1}{0}{\lambda}}   \\
        &\bar{\mu} =\frac{\mu - \prescript{}{0}{\lambda}_{lb}}{\prescript{}{0}{\lambda}_{ub}-\prescript{}{0}{\lambda}_{lb}},
        \quad \bar{\sigma} = \frac{\sigma}{\prescript{}{0}{\lambda}_{ub} - \prescript{}{0}{\lambda}_{lb}}, \\
        &\alpha = \frac{\bar{\mu}^2 - \bar{\mu}^3 - \bar{\sigma}^2\bar{\mu}}{\bar{\sigma}^2},  
        \quad \beta = \alpha \frac{1-\bar{\mu}}{\bar{\mu}}
        \end{aligned}
        \end{equation}
        %-------------------	 end EQUATION 	-------------------%
    where the left subscript denotes the configuration state. Note carefully that $\prescript{1}{0}{\lambda}$ will in general be a function of $\theta_1$, so that even though the collagen fibre recruitment is orientation-independent in $\beta_0$, it will have angular dependence in $\beta_1$ unless $\prescript{1}{0}{F}_{11} = \prescript{1}{0}{F}_{22}$ and $\prescript{1}{0}{F}_{12} = \prescript{1}{0}{F}_{21} = 0$. The resulting expression for the ensemble stress is
        %-------------------	begin EQUATION 	-------------------%
        \begin{equation}\label{c3:eqn:44}
        \begin{aligned}
        \prescript{t}{1}{\mathbf{S}}_c^{ens} = \phi_c\frac{\eta_c}{\prescript{t}{1}{\lambda}}
            \int_1^{\prescript{t}{1}{\lambda}} \frac{D_1(x)}{x}\left(\frac{\prescript{t}{1}{\lambda}}{x} - 1\right) \dif x
        \end{aligned}
        \end{equation}
        %-------------------	 end EQUATION 	-------------------%
    To complete the tissue-level formulation, we use the affine transformation assumption and the formulation described in \cite{fan_simulation_2014} to obtain the collagen fibre orientation distribution function $\Gamma_0$ in the native state to that in $\beta_1$ using
        %-------------------	begin EQUATION 	-------------------%
        \begin{equation}\label{c3:eqn:45}
        \begin{aligned}
        \Gamma_1[\mu_\Gamma,\sigma_\Gamma, \theta_1 = \Gamma_0[\mu_\Gamma, \sigma_\Gamma, \theta_0(\prescript{1}{0}{\mathbf{F}},\theta_1)\frac{\prescript{1}{0}{\lambda}_{\theta_0}^2}{\prescript{1}{0}{J_\mathrm{2D}}}].
        \end{aligned}
        \end{equation}
        %-------------------	 end EQUATION 	-------------------%
    Note that the angle $\theta_1$ of a fibre originally oriented at $\theta_0$ can be determined using
        %-------------------	begin EQUATION 	-------------------%
        \begin{equation}\label{c3:eqn:46}
        \begin{aligned}
        \theta_1(\prescript{1}{0}{\mathbf{F}},\theta_0) = \tan^{-1}\left(\frac{\prescript{1}{0}{F}_{21}\cos{\theta_0} + \prescript{1}{0}{F}_{22}\sin{\theta_0}}{\prescript{1}{0}{F}_{11}\cos{\theta_0} + \prescript{1}{0}{F}_{12}\sin{\theta_0}}\right)
        \end{aligned}
        \end{equation}
        %-------------------	 end EQUATION 	-------------------%
    The final form of the native collagen fibre phase expressed in the EXL state is thus
        %-------------------	begin EQUATION 	-------------------%
        \begin{equation}\label{c3:eqn:47}
        \begin{aligned}
        \prescript{t}{1}{\mathbf{S}}(\eta_c, \mu_\Gamma, \sigma_\Gamma, \mu_0, \sigma_0)& = \phi_c\frac{\eta_c}{\prescript{t}{1}{\lambda}} \int_{\theta_1} \Gamma_1(\mu_\Gamma, \sigma_\Gamma,\theta_1))
        \\
        &\left\{ \frac{D_1(\mu_0, \sigma_0, \prescript{}{0}{\lambda}_{lb}, \prescript{}{0}{\lambda}_{ub},\prescript{1}{0}{\lambda}[\theta_0(\theta_1)],x)}{x}\left(\frac{\prescript{1}{0}{\lambda}}{x} - 1\right)\dif x\right\}\mathbf{n}_1\otimes\mathbf{n}_1 \dif \theta
        \end{aligned}
        \end{equation}
        %-------------------	 end EQUATION 	-------------------%
    The above equation has a total of five fitted model parameters $\eta_c$, $\mu_\Gamma$, $\sigma_\Gamma$, $\mu_0$, and $\sigma_0$ ($\lambda_{lb}$ and $\lambda_{lb}$ are determined from the experimental data), all with a physical meaning and all referenced to $\beta_0$.

        
\subsection{The matrix phase}

    The matrix response can be estimated from the low stress region where collagen does not contribute any stress (figures \ref{c3:fig:5} and \ref{c3:fig:6}). A careful examination revealed that the toe region is a convex curve in $S-\lambda$, which is inconsistent with a neo-Hookean material model which is concave in $S-\lambda$ (figure \ref{c3:fig:6}). While we have used an exponential isotropic function for the matrix before \cite{sacks_structural_2000}, we considered the Yeoh model but found it unable to fit the data. Therefore, we developed a modified Yeoh model as
        %-------------------	begin EQUATION 	-------------------%
        \begin{equation}\label{c3:eqn:48}
        \begin{aligned}
        \Psi_m(\mathbf{C}) =& \frac{1}{2}\frac{\mu_a}{a}(I_1 - 3)^a + \frac{1}{2}\frac{\mu_b}{b}(I_1 - 3)^b  \\
        \mathbf{S}_m =& \left(\mu_a(I_1 - 3)^{a-1} + \mu_b(I_1 - 3)^{b-1}\right)(\mathbf{I} - C_{33}\mathbf{C}^{-1}), \\
        &\quad\quad \text{with } 1<a<b, \quad a\times b < 2
        \end{aligned}
        \end{equation}
        %-------------------	 end EQUATION 	-------------------%
    When applied to all pre-transition mechanical data (i.e. from all protocols and where there is no collagen fibre contribution), this form was found to fit the low-stress data quite well (figure \ref{c3:fig:6}).
    

%%%%%%%%%%%%%%%%%%%%	begin FIGURE 	%%%%%%%%%%%%%%%%%%%%
\begin{figure}
\centering
\includegraphics[width=\textwidth]{Images/chapter3/F6large.jpg}
\caption{A representative fibre-ensemble stress–strain (in $P_\mathrm{ens}-\lambda_\mathrm{ens}$)  for EXL treated bovine periardium, and (b) a close-up of the low stress region. A careful examination revealed that the toe region suggests that a modified Yeoh model was necessary to accurately capture its response (equation \ref{c3:eqn:48}) due to the convexity of the response.}
\label{c3:fig:6}
\end{figure}
%%%%%%%%%%%%%%%%%%%%	 end FIGURE 	%%%%%%%%%%%%%%%%%%%%
    
    
\subsection{Interactions}
        
    Our key working assumption is that all fibre–fibre and fibre–matrix interactions can be represented at the fibre-ensemble level. This is done to simplify the model formulation, and further since the exact micromechanical mechanisms of cross-linking have yet to be determined. Based on the form assumed in equation \ref{c3:sec:42}, we now have the ability to estimate the form and magnitude of the fibre-ensemble interactions from the fibre-ensemble data using $\mathbf{S}_\mathrm{int} \approx \mathbf{S}_\mathrm{ens} - 1/\phi_c(\phi_c \mathbf{S}_c + (1 - \phi_c)\mathbf{S}_m)$, where $\mathbf{S}_\mathrm{ens}$ is the ensemble stress (figure \ref{c3:fig:5}). Note here that the collagen stress Embedded Image and is thus the contribution of the collagen fibres expressed in the EXL configuration $\beta_1$ using equation \ref{c3:eqn:47}. This approach allowed us to exploit the matched native–EXL mechanical data by fitting the native responses then mapping them to the EXL state, so that they are a known quantity rather than one that required data fitting. Results of this analysis indicated some intriguing results. First, while the collagen phase contributed substantially to the total ensemble stress, it was only about 50\%, and the matrix only about 20\%. This revealed that the remaining approximately 30\% portion of the total ensemble stress must be a result of the interaction mechanisms.
    
    
    To model the interactions, we first consider two fibre ensembles with orientation vectors $\mathbf{n}_0(\alpha)$ and $\mathbf{m}_0(\beta)$ in the reference configuration (figure \ref{c3:fig:7}). These two ensembles can mechanically interact by elongation and relative rotation. Kinematically, these mechanisms can be captured using the pseudo-invariant $I_8$ \cite{holzapfel_nonlinear_2000,merodio_influence_2006} 
        %-------------------	begin EQUATION 	-------------------%
        \begin{equation}\label{c3:eqn:49}
        \begin{aligned}
        I_8 &= \mathbf{n}_0\cdot\mathbf{C}\mathbf{m}_0 = \cos(\theta)\lambda_\alpha\lambda_\beta \\
        \cos(\theta) &= \frac{\mathbf{n}_0\cdot\mathbf{m}_0}{\lambda_\alpha\lambda_\beta}, \quad \lambda_\alpha=\sqrt{\mathbf{n}_0\cdot\mathbf{C}\mathbf{n}_0}, \quad \lambda_\alpha=\sqrt{\mathbf{m}_0\cdot\mathbf{C}\mathbf{m}_0} 
        \end{aligned}
        \end{equation}
        %-------------------	 end EQUATION 	-------------------%
    Note that we can also use $I_8^\prime = I_8  - I_8^0$ \cite{merodio_influence_2006} to account for the relative change in fibre rotations if necessary. Thus, $i_8$ can be considered the product of an extensional term $\lambda_\alpha \lambda_\beta$ and a rotational term $\cos(\theta)$. We consider two sub-aspects of ensemble-level effects: intra- and inter-ensemble levels. The intra-ensemble incorporates all fibre–fibre interactions that occur within a single ensemble and are limited to extensional effects only. By contrast, inter-ensemble effects can include both extensional and rotational effects.
    
    
%%%%%%%%%%%%%%%%%%%%	begin FIGURE 	%%%%%%%%%%%%%%%%%%%%
\begin{figure}
\centering
\includegraphics[width=3in]{Images/chapter3/F7large.jpg}
\caption{A schematic of two collagen fibre ensembles with respective orientations $\alpha$ and $\beta$ (not restricted to be symmetric about the $X_1$-axis) with associated orientation vector $\mathbf{n}_0$ and $\mathbf{m}_0$ in the reference configuration.}
\label{c3:fig:7}
\end{figure}
%%%%%%%%%%%%%%%%%%%%	 end FIGURE 	%%%%%%%%%%%%%%%%%%%%
    
    
    To best capture these phenomena, we do not use $I_8$ directly but rather the following forms. The results for the ensemble stress suggest that the interaction terms are exponential in character (figure \ref{c3:fig:8}). Thus, for the extensional intra-ensemble effects we use
        %-------------------	begin EQUATION 	-------------------%
        \begin{equation}\label{c3:eqn:410}
        \begin{aligned}
        \Psi_\mathrm{int}^e(\mathbf{C}) = \frac{c_0}{4}\int_\theta\Gamma(\theta)\left[e^{c_1(\lambda-1)^2} - 1\right]\dif \theta
        \end{aligned}
        \end{equation}
        %-------------------	 end EQUATION 	-------------------%
    where $\lambda = \sqrt{\mathbf{n}_0\cdot\mathbf{C}\mathbf{n}_0} = \sqrt{I_4}$ and $c_0$, $c_1$ are constants, and the associated single ensemble stress extensional interaction is
        %-------------------	begin EQUATION 	-------------------%
        \begin{equation}\label{c3:eqn:411}
        \begin{aligned}
        \mathbf{S}_\mathrm{int}^e =& 2\dpd{\Psi(\mathbf{C})}{\mathbf{C}} = 2 \dpd{\Psi(\lambda)}{\lambda}\dpd{\lambda}{\mathbf{C}}
        = \int_\theta \Gamma(\theta)\left[\frac{c_0c_1(\lambda-1)e^{c_1(\lambda-1)^2}}{\lambda}\mathbf{n}_0\otimes\mathbf{n}_0\right]\dif \theta
        \end{aligned}
        \end{equation}
        %-------------------	 end EQUATION 	-------------------%
    Next, for the extensional inter-ensemble interactions, we use a similar form
        %-------------------	begin EQUATION 	-------------------%
        \begin{equation}\label{c3:eqn:412}
        \begin{aligned}
        \Psi_\mathrm{int}^{ee}(\mathbf{C}) = \frac{d_0}{4}\int_\alpha\int_\beta\Gamma(\alpha)\Gamma(\beta)\left[e^{d_1(\lambda_\alpha\lambda_\beta-1)^2} - 1\right]\dif \alpha \dif\beta
        \end{aligned}
        \end{equation}
        %-------------------	 end EQUATION 	-------------------%
    where $d_0$ and $d_1$ are parameters, with associated stresses
        %-------------------	begin EQUATION 	-------------------%
        \begin{equation}\label{c3:eqn:413}
        \begin{aligned}
        \mathbf{S}_\mathrm{int}^e =& \int_\alpha\int_\beta \Gamma(\alpha)\Gamma(\beta) \\
        &\times\left[d_0d_1(\lambda_\alpha\lambda_\beta-1)e^{d_1(\lambda_\alpha\lambda_\beta-1)^2}\left(\frac{\lambda_\beta}{\lambda_\alpha}\mathbf{n}_0\otimes\mathbf{n}_0 + \frac{\lambda_\alpha}{\lambda_\beta}\mathbf{m}_0\otimes\mathbf{m}_0\right)\right]\dif\alpha\dif\beta
        \end{aligned}
        \end{equation}
        %-------------------	 end EQUATION 	-------------------%


%%%%%%%%%%%%%%%%%%%%	begin FIGURE 	%%%%%%%%%%%%%%%%%%%%
\begin{figure}
\centering
\includegraphics[width=\textwidth]{Images/chapter3/F8large.jpg}
\caption{Representative final model results (equation \ref{c3:eqn:53}) for a single fibre-ensemble stress–strain (in $S_\mathrm{ens}-\lambda_\mathrm{ens}$) response for EXL treated bovine pericardium. All model components contributed signficantly to the total stress. Surprisingly, while the collagen phase produced the greatest contribution, the interaction term was of comparable magnitude. }
\label{c3:fig:8}
\end{figure}
%%%%%%%%%%%%%%%%%%%%	 end FIGURE 	%%%%%%%%%%%%%%%%%%%%
    
    Note that the exponential term was set to zero if $\lambda_\alpha\lambda\beta<1$ In this approach, we integrate over all ensembles, weighted by their respective orientation distribution functions, to obtain the total contribution.
    
    
    Next, we developed a rotational pseudo-invariant defined as the change in the cosine between the two ensemble fibre directions, which is simply
        %-------------------	begin EQUATION 	-------------------%
        \begin{equation}\label{c3:eqn:414}
        \begin{aligned}
        I_\mathrm{int}^r(\alpha,\beta) = \frac{I_8}{\lambda_\alpha\lambda\beta} - \mathbf{n}_0\cdot\mathbf{m}_0 = \cos{\theta}-\cos{\theta_0}\approx \Delta\theta.
        \end{aligned}
        \end{equation}
        %-------------------	 end EQUATION 	-------------------%
    Using $\Psi_\mathrm{int}^r(\mathbf{C}) = \frac{\eta^r}{4}(I_\mathrm{int}^r)^2$ it can be shown that for planar distributed fibre ensembles the stress tensor is
        %-------------------	begin EQUATION 	-------------------%
        \begin{equation}\label{c3:eqn:415}
        \begin{aligned}
        \mathbf{S}_\mathrm{int}^r =& \eta^r \int_\alpha\int_\beta \Gamma(\alpha)\Gamma(\beta) \frac{I_\mathrm{int}^r(\alpha,\beta)}{\lambda_\alpha\lambda_\beta}  \\
        &\left[\left(\mathbf{m}_0\otimes\mathbf{n}_0 + \mathbf{n}_0\otimes\mathbf{m}_0\right)
        - I_8
        \left(\frac{\mathbf{n}_0\otimes\mathbf{n}_0}{\lambda_\alpha^2} + \frac{\mathbf{m}_0\otimes\mathbf{m}_0}{\lambda_\beta^2}\right)\right]\dif\alpha\dif\beta
        \end{aligned}
        \end{equation}
        %-------------------	 end EQUATION 	-------------------%
    Combining equations \ref{c3:eqn:47}\ref{c3:eqn:48}\ref{c3:eqn:411}\ref{c3:eqn:413}\ref{c3:eqn:415}, we obtain the final form of the full model stress as
        %-------------------	begin EQUATION 	-------------------%
        \begin{equation}\label{c3:eqn:416}
        \begin{aligned}
        \mathbf{S} =& \phi_c\mathbf{S}_c(\eta_c, \mu_\Gamma, \sigma_\Gamma, \mu_0, \sigma_0) + (1-\phi_c)\mathbf{S}_m(\mu_a, \mu_b, a, b) \\
        &+ \phi_c[\mathbf{S}_\mathrm{int}^e(c_0,c_1) + \mathbf{S}_\mathrm{int}^{ee}(d_0,d_1) + \mathbf{S}_\mathrm{int}^r(\eta^r)] - p\mathbf{C}^{-1}
        \end{aligned}
        \end{equation}
        %-------------------	 end EQUATION 	-------------------%
    
        
        
        

%---    Model simplifications and parameter estimation
\section{Model simplifications and parameter estimation}

\subsection{Final form}

    While in principle equation \ref{c3:eqn:416} can be implemented within a robust parameter estimation procedure, simplifications are clearly in order given its complexity (14 parameters). We first consider an equi-biaxial test wherein all fibre rotations are zero, so that $I_\mathrm{int}^r = 0$ The total interaction stress is thus given by just the following extensional contributions:
        %-------------------	begin EQUATION 	-------------------%
        \begin{equation}\label{c3:eqn:51}
        \begin{aligned}
        \mathbf{S}_\mathrm{int} =&\int_\theta \Gamma(\theta)\left[\frac{c_0c_1(\lambda-1)e^{c_1(\lambda-1)^2}}{\lambda}\mathbf{n}_0\otimes\mathbf{n}_0\right]\dif \theta  \\
        &+ \int_\alpha\int_\beta \Gamma(\alpha)\Gamma(\beta)\left[d_0d_1(\lambda^2-1)e^{d_1(\lambda^2-1)^2}\left(\mathbf{n}_0\otimes\mathbf{n}_0 + \mathbf{m}_0\otimes\mathbf{m}_0\right)\right]\dif\alpha\dif\beta
        \end{aligned}
        \end{equation}
        %-------------------	 end EQUATION 	-------------------%
    In practice, we found that while the intra-ensemble form (first r.h.s. term) was used alone it was able to capture the equi-biaxial strain behaviour (figure \ref{c3:fig:5}), it was unable to capture the response from all test protocols. Moreover, given the similarity in form, the two components on the r.h.s. of equation (5.1) can capture similar responses. We thus chose to ignore the intra-ensemble stress contribution $\mathbf{S}_\mathrm{int}^e(c_0,c_1)$ removing two parameters. Next, while it is intuitive that fibre-ensemble rotations produce important contributions to the total tissue stress, closer analysis of equation \ref{c3:eqn:415} indicated that it will produce compressive stresses in the direction of lesser stretch. These characteristics were not consistent with any of the observed experimental data. Even when choosing various forms of $\Psi_\mathrm{int}^r$ we could not match the experimentally observed responses. Interestingly, only the $S_\mathrm{int}^{ee}$ contribution of equation \ref{c3:eqn:416} was found to model the interaction stresses well.
    
    Thus, we are left with the following interaction stresses:
        %-------------------	begin EQUATION 	-------------------%
        \begin{equation}\label{c3:eqn:52}
        \begin{aligned}
        \mathbf{S}_\mathrm{int}^e =& \int_\alpha\int_\beta \Gamma(\alpha)\Gamma(\beta) \\
        &\times\left[d_0d_1(\lambda_\alpha\lambda_\beta-1)e^{d_1(\lambda_\alpha\lambda_\beta-1)^2}\left(\frac{\lambda_\beta}{\lambda_\alpha}\mathbf{n}_0\otimes\mathbf{n}_0 + \frac{\lambda_\alpha}{\lambda_\beta}\mathbf{m}_0\otimes\mathbf{m}_0\right)\right]\dif\alpha\dif\beta
        \end{aligned}
        \end{equation}
        %-------------------	 end EQUATION 	-------------------%
    leading to the following final form of the constitutive model:
        %-------------------	begin EQUATION 	-------------------%
        \begin{equation}\label{c3:eqn:53}
        \begin{aligned}
        \mathbf{S} =& \mathbf{S}_c + \mathbf{S}_{int} + \mathbf{S}_m  \\
        =& \phi_c\frac{\eta_c}{\prescript{t}{1}{\lambda}} \int_{\theta_1} \Gamma_1(\mu_\Gamma, \sigma_\Gamma,\theta_1))
        \frac{D_1(x)}{x}\left(\frac{\prescript{1}{0}{\lambda}}{x} - 1\right)\mathbf{n}_1\otimes\mathbf{n}_1 \dif x \dif \theta \\
        +& \phi_c\int_\alpha\int_\beta \Gamma(\alpha)\Gamma(\beta) d_0d_1(\lambda_\alpha\lambda_\beta-1)e^{d_1(\lambda_\alpha\lambda_\beta-1)^2}\left(\frac{\lambda_\beta}{\lambda_\alpha}\mathbf{n}_0\otimes\mathbf{n}_0 + \frac{\lambda_\alpha}{\lambda_\beta}\mathbf{m}_0\otimes\mathbf{m}_0\right)\dif\alpha\dif\beta    \\
        +&(1-\phi_c) \left(\mu_a(I_1 - 3)^{a-1} + \mu_b(I_1 - 3)^{b-1}\right)(\mathbf{I} - C_{33}\mathbf{C}^{-1}), \\
        \end{aligned}
        \end{equation}
        %-------------------	 end EQUATION 	-------------------%
        It is understood that $\mathbf{n}_0$ and $\mathbf{m}_0$ are referred to $\beta_1$ and that we merged the Lagrange multiplier with the matrix by assuming a planar tissue to simplify the formulation. This final model has 11 independent fitted parameters \{$\eta_c$, $\mu_\Gamma$, $\sigma_\Gamma$, $\mu_0$, $\sigma_0$, $d_0$, $d_1$, $\mu_a$, $\mu_b$, $a$, $b$\} and three directly determined parameters $\phi_c, \prescript{}{0}{\lambda}_{lb}, \prescript{}{0}{\lambda}_{ub}$ all with a physical meaning.
        
        
        
        
\subsection{Parameter estimation procedures}

    While at first glance this appears to be a major nonlinear optimization undertaking with all the usual pitfalls, we can use the following sequence to make actual parameter estimation quite tractable:
        \begin{enumerate}
            \item From the native tissue mechanical data, predict the collagen phase parameters $\{\eta_c, \mu_\Gamma, \sigma_\Gamma, \mu_0, \sigma_0\}$ using standard procedures \cite{fata_insights_2014,zhang_meso_2016}.
            \item From the pre-transition collagen recruitment portion of all of the EXL tissue mechanical data, determine the matrix parameters $\{\mu_a, \mu_b, a, b\}$.
            \item Taking the collagen and matrix responses, determine the interaction stress responses for all test protocols using $\mathbf{S}_\mathrm{int} = \mathbf{S} - \frac{1}{\phi_c}\left(\phi_c \prescript{1}{0}{S}_c + (1-\phi_c)\mathbf{S}_m\right)$
            \item Using the results of step 3, determine the final two parameters ($d_0$ and $d_1$) by fitting equation \ref{c3:eqn:53} but only allowing them to vary while keeping the other terms to their above-fitted values.
        \end{enumerate}
    We found that this basic sequence ensured a robust parameter is obtained, because the entire model is never fitted at once. Moreover, this approach allowed us to separate the contributions to the stress of each of these mechanisms. As in our previous studies \cite{fata_insights_2014,zhang_meso_2016}, we employed the genetic based Differential Evolution algorithm in Mathematica to perform the optimization. All parameter estimation was performed using a custom program written in Mathematica (Wolfram Research Corp.).

        
        

%---    primary results
\section{Primary results}
    
    From the five specimens used, the model was able to successfully fit all data quite well (total fit $r^2 > 0.97$). Moreover, the final parameter values were quite consistent, with generally low standard errors (table 3). The mean collagen fibre modulus (approx. 279 MPa) and fibre splay (approx. $38^\circ$) were comparable to previous studies \cite{fan_simulation_2014}. Interestingly, the lower bound stretch was small (1.01 or approx. 1\% strain), 
    so it is likely that it could be set to 1 (i.e. zero strain). The native collagen fibre recruitment parameters were also consistent (table 3), and indicated a very rapid recruitment at stretch of approximately 1.18–1.2 (figure 9). This is a more complete picture of the entire fibre recruitment than in our previous work \cite{sun_finite_2005}\cite{fan_simulation_2014}, and suggests that the collagen fibres are effectively well ordered with a small deviation in crimp amplitude and wavelength.
    

\begin{table}
\centering
\caption{Equibiaxial strain testing results.}\label{c3:tab:3}
\begin{tabular}{L{.5in}R{.7in}R{0.5in}R{0.5in}R{0.6in}R{0.5in}R{0.6in}R{0.5in}}
\hline
& \multicolumn{1}{c}{\textbf{modulus}} & \multicolumn{2}{c}{\textbf{ODF}} & \multicolumn{4}{c}{\textbf{recruitment}}\\
\cline{2-8}
& $\eta_c(MPa)$ & $\mu_\Gamma({}^\circ)$ & $\sigma_\Gamma({}^\circ)$ & $\mu_0$ & $\sigma_0$ & $\prescript{}{0}{\lambda}_{lb}$
& $\prescript{}{0}{\lambda}_{ub}$\\
mean & 278.94 & 6.513 & 38.430 & 1.185 & 0.014 & 1.011 & 1.197  \\
s.e.m. & 22.38 & 1.645 & 0.922 & 0.032 & 0.001 & 0.007 & 0.035  \\
\hline
\hline
& \multicolumn{2}{c}{\textbf{Interactions}} & & \multicolumn{4}{c}{\textbf{Matrix}} \\
\cline{2-8}
& $d_0(kPa)$ & $d_1$ & & $\mu_a(kPa)$ & $a$ & $\mu_b(kPa)$ & $b$\\
mean & 1.040 & 42.267 & & 56.74 & 1.068 & 1294.38 & 1.873  \\
s.e.m. & 0.255 & 9.772 & & 10.69 & 0.004 & 340.71 & 0.007 \\
\hline
\end{tabular}
\end{table}


    One advantage of our approach is that the various contributions to the total stress can be separated (figure 8). To better reveal the present findings, it is useful to examine the effects on the individual stress components under various loading paths. Following the trends of the ensemble results (figure 9), we noted that the interactions produced substantial contributions to the total stress (figure 10). Interestingly, for $S_{11}$ the interactions actually produced the largest contributions, followed by the matrix and collagen fibres. By contrast, for $S_{22}$ the contributions were much more dependent on the particular loading path, with the collagen phase dominating when $\lambda_2>\lambda_1$. When $\lambda_1>\lambda_2$, the matrix phase dominated $S_22$. We further note here that the contribution of the matrix was much less loading path sensitive, due to its near-linear, isotropic behaviour.

%---    discussion
\section{Discussion}

\subsection{Major findings}

    This study represents the first rigorous full structural model (i.e. explicitly incorporating various features of the CFA) for cross-linked soft tissues, and also includes a specific interaction term. An important utility of this model is its ability to separate the effects of EXLs on the fibres and matrix, so that the matrix, collagen and interaction effects could be clearly identified. This was made possible, in-part, with the use of the native–EXL matched experimental dataset and a modification to the structural model so that the uncross-linked collagen fibre responses could be mapped to the EXL configuration. As in our previous study, we found that the matrix could be well modelled as an isotropic material. However, we found that a much more linear-like response was necessary. Perhaps the most novel findings of this study were that (i) the effective collagen fibre modulus was unaffected by cross-linking and (ii) the interaction term played such a large role in stress development, often dominating the response (depending on the component and loading path being considered).
    
    
    The lack of change in the effective collagen fibre modulus has been corroborated by experimental results from Gentleman et al. \cite{gentleman_mechanical_2003}. In that study, they found a modulus range of $269.7\pm11.9$ to $ 484.7\pm76.3$ for cross-linked collagen fibres in the bovine Achilles tendon, which corresponds to the same range as another study for native collagen fibres from various sources \cite{eppell_nano_2006}. Yang et al. \cite{yang_mechanical_2008,yang_micromechanical_2007} determined that for the mechanical properties of hydrated native and cross-linked type I collagen fibrils that cross-linking the collagen fibrils with a water-soluble carbodiimide did not significantly affect the bending modulus. The work by Shen et al. \cite{shen_stress_2008} noted a modulus of $0.86\pm0.45$ GPa (range, 0.36–1.60 GPa; n = 13), in reasonable agreement with our results. Further, six of the 13 fibrils showed linear behaviour. At the tissue level, our findings are also consistent with the findings of Lee et al. \cite{lee_bovine_1989I,lee_bovine_1989II,lee_bovine_1989III}, who found no change in MTM in cross-linked pericardial tissues as in our study (table \ref{c3:tab:2}). It is interesting to note that, when using the native tissue fibre-ensemble model (equation \ref{c3:eqn:44}) on both the native and cross-linked data (figure \ref{c3:fig:5}), one can increase the fibre modulus determined from the native state to match the post-EXL data (figure \ref{c3:fig:11}a). However, this will induce a parallel increase in the MTM of approximately 75\% (figure \ref{c3:fig:11}b), which is inconsistent with the experimental findings (table \ref{c3:tab:2}). This is the case even when compensating for the effects of tissue contraction. This simple simulation lends support to the collagen modulus results (figure \ref{c3:fig:5}c).
    
    
%%%%%%%%%%%%%%%%%%%%	begin FIGURE 	%%%%%%%%%%%%%%%%%%%%
\begin{figure}
\centering
\includegraphics[width=\textwidth]{Images/chapter3/F11large.jpg}
\caption{Simulation results using the unmodified native tissue fibre-ensemble model (equation \ref{c3:eqn:44}) on both the native and cross-linked data from figure \ref{c3:fig:5}, showing that one can increase the fibre modulus determined from the native state to match the post-EXL data (a). However, this will induce a parallel increase in the MTM of approximately 75\% 
(b), which is inconsistent with the experimental findings (table \ref{c3:tab:2}). This is the case even when compensating for the effects of tissue contraction. }
\label{c3:fig:11}
\end{figure}
%%%%%%%%%%%%%%%%%%%%	 end FIGURE 	%%%%%%%%%%%%%%%%%%%%


    
    Our model results suggest that the major effect of EXL formation was not the formation of a mechanically substantial matrix or stiffening of the collagen fibres, but rather a dramatically enhanced bonding both within and between fibre ensembles. Note that our specific interaction term (equation \ref{c3:eqn:413}) captured the effects of both individual ensemble stretch and relative stretching between ensembles. This is consistent with what is known about GLUT bond formations (figure \ref{c3:fig:1}a) \cite{nimni_chemically_1987,cheung_mechanism_1985,cheung_presence_1983,cheung_mechanism_1982,cheung_mechanism_1982II}. Yet, we found that relative rotations between fibre ensembles as modelled by equation \ref{c3:eqn:415} could not capture the observed responses. The underlying structural mechanisms for this behaviour remain largely uncharacterized. One possibility is that the protein core of the proteoglycans that bind collagen fibrils become strongly cross-linked and are thus substantially stiffened, acting to more cohesively bind the collagen fibres. This is supported by findings of Liao \& Vesely \cite{liao_relationship_2004}, who observed substantial deformations of proteoglycans in mitral valve chordae under stretch. Moreover, the present model suggests that such mechanisms are strongly associated with fibre-ensemble orientation distributions. While not the final word, our results suggest that EXLs produce a stiffening effect via an isotropic matrix, with the interaction effects being the dominant effect.
    
    


\subsection{Modelling approach}

    The present approach was a direct extension of the stochastic, tissue-level meso-structural models first pioneered by Lanir \cite{lanir_constitutive_1983} and used in various applications and extensions by our group \cite{sacks_incorporation_2003,fata_insights_2014,zhang_meso_2016}. By meso-scale, we refer to the fibre-ensemble scale, which is fibre features down to approximately 100 $\mu$m. Moreover, recent evidence has demonstrated that the underlying affine kinematic assumption is valid \cite{fata_insights_2014,fan_simulation_2014}, so that the strain energy of each fibre ensemble can be kinematically connected with the macroscopic strain tensor. We took the approach to develop a more general model first, then show what components could be modified or removed entirely. Given the lack of knowledge and modelling efforts in this area, we felt this was appropriate and helped to illustrate what underlying mechanisms should be incorporated. We also considered a more extensive approach based on elastica-based theory for sinusoidal fibres based on \cite{garikipati_elastica_2008} under the assumption that EXL formation dramatically increased the fibre modulus, which ultimately proved to be unnecessary. An inter-fibre sliding model was also developed based on the relative sliding between fibres due to differences between their respective slack lengths, and used as a means to model the intra-ensemble EXL effects. However, this approach was found to be unable to capture the individual fibre-ensemble responses, suggesting that relative sliding between fibres at the ensemble level was not a major mechanism. A final question that may impact physical plausibility of the current model is convexity and physical plausibility. Lanir \cite{lanir_plausibility_1994} demonstrated that the native tissue structural model is compatible with a physically plausible response. Based on both this and the experimental observations, we focused on monotonically increasing functions of strain for all model terms to ensure physical plausibility and that convexity was maintained.
    



\subsection{Limitations and future directions}

    The current model is limited in the fact that it is quasi-static and does not account for permanent set phenomena we have observed \cite{sun_response_2004}. The homogenization method used in section \ref{c3:sec:3} is fairly standard and has been used by others and the author for some time (e.g. \cite{sacks_incorporation_2003,fata_insights_2014,fan_simulation_2014}). We have observed that for soft tissues, whose composition is dominated by a dense network of distinct fibres, especially collagen type I and elastin, these structures can be mechanically treated as fibre ensembles; that is, groups of fibres with a common orientation. The scale of the representative volume element is about 100 $\mu$m, which is sufficient to capture the salient mechanical features of the fibre ensemble. We emphasize that this is not meant to be a universal model of all types of soft tissue structures, which are very diverse for a single theoretical treatment. Rather, we focus on a sufficiently generalized approach for exogenously cross-linked dense collagenous tissues, such as pericardium, heart valves and sclera. These structures fit our approach well and also have important biomedical therapeutic applications.


    A complete understanding of the current phenomena must be based on well-characterized micro-scale events. For example, Kojic et al. \cite{kojic_numerical_1998} developed a model for fibre–fibre kinetics that uses Coulomb friction, which results in a simple and robust approach for tissue simulations. However, our knowledge of even native tissues at the micro-level remains limited. The situation remains more complicated by the fact that our knowledge of the interrelationships between the physical chemistry of EXLs and the macro-scale mechanics of collagenous tissues remains limited at the present time for more sophisticated models to be reliably attempted. To date, no material model is able to fully account for such complex observed microstructural and biological behaviour. The next steps include incorporation of the permanent set effect commonly observed in cross-linked tissues into the present model, and exploration of how alternative cross-linking methods affect macro-scale tissue behaviours.

\newpage
%%%%%%%%%%%%%%%%%%%%%%%%%%%%%%%%%%%%%%%%%%%%%%%%%%%%%%%%%%%%%
%%  nomenclature											%
%%%%%%%%%%%%%%%%%%%%%%%%%%%%%%%%%%%%%%%%%%%%%%%%%%%%%%%%%%%%%
\section*{Nomenclature} \label{c3:sec:nomenclature}
\begin{mynom}
{int}\>\tabfill{Interaction term} \\
{c}\>\tabfill{Collagen} \\
{m}\>\tabfill{Matrix} \\
{EB}\>\tabfill{Equibiaxial} \\
{ODF}\>\tabfill{Orientation distribution functio z} \\
{$\mathbf{F}$}\>\tabfill{Deformation gradient tensor} \\
{$\mathbf{E}$}\>\tabfill{Green Lagrange strain} \\
{$\mathbf{P}$}\>\tabfill{First Piola Kirchhoff stress tensor} \\
{$\mathbf{S}$}\>\tabfill{Second Piola Kirchhoff tensor} \\
{$\mathbf{C}$}\>\tabfill{Right Cauchy-Green strain tensor} \\
{$\mathbf{m}_0$,$\mathbf{n}_0$}\>\tabfill{fibre-ensemble orientation vectors} \\
{$_ens$}\>\tabfill{ensemble} \\
{$\lambda$}\>\tabfill{Stretch} \\
{$\lambda_s$}\>\tabfill{The slack stretch, the stretch needed to straighten the collagen fiber crimp} \\
{$\lambda_t$}\>\tabfill{The true stretch after the collagen fibers are straightened} \\
{$E_s$}\>\tabfill{The slack strain} \\
{$\Psi$}\>\tabfill{The strain energy} \\
{$\phi$}\>\>\tabfill{The mass fractions} \\
{$D$}\>\tabfill{Distribution of slack strains for fibre recruitment} \\
{$\Gamma$}\>\tabfill{Fiber orientation distribution function} \\
{$\eta$}\>\tabfill{modulus, subscript for elastin collagen and matrix} \\
{$\mu_\theta$}\>\tabfill{mean circumferential orientation} \\
{$\sigma$}\>\tabfill{standard deviation of the fibre splay} \\
{$\mu_r$}\>\tabfill{mean of the recruitment distribution} \\
{$\sigma_r$}\>\tabfill{standard deviation of the recruitment distribution} \\
{$\lambda_{lb}$}\>\tabfill{lower bound of the recruitment distribution} \\
{$\lambda_{ub}$}\>\tabfill{upper bound of the recruitment distribution} \\
{$c_0$, $c_1$}\>\tabfill{exponent for intra-fibre-ensemble interaction terms} \\
{$d_0$, $d_1$}\>\tabfill{exponent for inter-fibre-ensemble interaction terms}
\end{mynom}

\newpage
%---    Bioliography
\bibliographystyle{plainnat}
\bibliography{phd}

\chapter{Modeling the effects of cyclic loading on exogenously cross\Hyphdash linked tissues}

\section*{Preface}
\addcontentsline{toc}{section}{Preface}%

    Bioprosthetic heart valves (BHVs), fabricated from exogenously crosslinked collagenous tissues, remain the most popular heart valve replacement design. However, the lifespan of BHVs remains limited to 10–15 years, in part because the mechanisms that underlie BHV failure remain poorly understood. Experimental evidence indicates that BHVs undergo significant changes in geometry with in vivo operation, which lead to stress concentrations that can have a significant impact on structural damage. These changes do not appear to be due to plastic deformation, as the leaflets only deform in the elastic regime. Moreover, structural damage was not detected by the 65 million cycle time point. Instead, we found that this nonrecoverable deformation is similar to the permanent set effect observed in elastomers, which allows the reference configuration of the material to evolve over time. We hypothesize that the scission-healing reaction of glutaraldehyde is the underlying mechanism responsible for permanent set in exogenously crosslinked soft tissues. The continuous scission-healing process of glutaraldehyde allows a portion of the exogenously crosslinked matrix, which is considered to be the non-fibrous part of the extracellular matrix, to be re-crosslinked in the loaded state. Thus, this mechanism for permanent set can be used to explain the time evolving mechanical response and geometry of BHVs in the early stage. To model the permanent set effect, we assume that the exogenously crosslinked matrix undergoes changes in reference configurations over time. The changes in the collagen fiber architecture due to dimensional changes allow us to predict subsequent changes in mechanical response. Results show that permanent set alone can explain and, more importantly, predict how the mechanical response of the biomaterial change with time. Furthermore, we found is no difference in permanent set rate constants between the strain controlled and the stress-controlled cyclic loading studies. An important finding we have is that the collagen fiber architecture has a limiting effect on the maximum changes in geometry that the permanent set effect can induce. This is due to the recruitment of collagen fibers as the changes in geometry due to permanent set increase. This means we can potentially optimize the BHV geometry based on the predicted the final BHV geometry after permanent set has largely ceased. Thus, we have developed the first structural constitutive model for the permanent set effect in exogenously crosslinked soft tissue, which can help to simulate BHV designs and reduce changes in BHV geometry during cyclic loading and thus potentially increasing BHV durability.
    
    
    
    
\textbf{The work contained in this chapter was published as}: Zhang, W. \& Sacks, M. S.
Modeling the response of exogenously crosslinked tissue to cyclic loading: The effects of permanent set. 
Journal of the mechanical behavior of biomedical materials, 2017, 75, 336-350 




%---    INTRODUCTION
\section{Introduction}
\subsection{Background and clinical significance}

	Heart valve treatment is an expensive cardiovascular surgerical procedure with over 100,800 done annually in the U.S. alone \cite{mozaffarian_heart_2016} and 275,000 to 370,000 in developed nations \cite{manji_future_2012}. 
	Almost all contemporary heart valve replacement designs use exogenously crosslinked (EXL) collagenous soft tissues (bovine pericardium) to manufacture leaflets for bioprosthetic valves (BHVs) \cite{starr_artificial_2007, soares_biomechanical_2016}. 
	BHVs have advantages in immunogenicity and hemodynamics over other designs, but also have a limited life span of 10-15 years. 
	As a recent development in BHV technology, transcatheter valve interventions \cite{bonow_accaha_2006, guidoin_marvel_2010} reduce surgical risk and make valve replacement more feasible for those who cannot tolerate full surgical interventions. 
	However, this new technology also presents additional design challenges, including complex folding and compression during delivery. 
	As a result, the leaflets are required to be significantly thinner than in traditional BHVs, which increases the leaflet stress and potentially the rate of failure. 
	Existing data on transcatheter aortic valve interventions suggest a 2-year mortality rate of 33.9\% overall \cite{mozaffarian_heart_2016} and only 68\% when specifically replacing stenotic aortic valves \cite{makkar_transcatheter_2012}. 
As such, this further accentuates the need to develop an approach to improve BHV durability. 
	
	 
\subsection{Mechanisms of BHV failure}

	The causes of BHV failure can be divided into two broad categories, mineralization and structural damage, with both processes occurring in parallel or independently \cite{sacks_collagen_2002}. 
	Mineralization is the accumulation of mineral deposits, mainly calcium phosphate, within the BHV leaflets \cite{schoen_calcification_2005}. 
	This disrupts the underlying microstructure preventing the proper mechanical function of BHVs, increasing the likelihood of tearing, and reducing flexibility (preventing normal opening and closing motions, and induce valve stenosis). 
	This process is exacerbated by the presence of exogenous crosslinkers, such as glutaraldehyde(GLUT), where the phosphates from the devitalized endothelial cells bind with the calcium in blood to form deposits. 
	The causes of calcification and associated anti-calcification treatments are extensively studied in literature \cite{park_novel_1997, isenburg_tannic_2005, vyavahare_prevention_1997}.
	On the other hand, structural damage includes the collagen fiber damage and breakdown of the non-fibrous part of extracellular matrix (ECM), \emph{which we will refer as simply the matrix}.
%causing a reduction in the matrix modulus and fiber-fiber interactions.
	Fourier transform infrared spectroscopy(FITR) results have shown changes in the collagen fiber molecular structure after 50 million cycles \cite{sun_response_2004}, which suggests that some collagen fiber damage has occurred during this period. 
	However, its effect on the mechanical response of BHVs is not detectable. 
	Nevertheless, it is important to maintain the structural integrity of the BHV, as this will help to improve BHV durability. 
	

	
\subsection{Response to long-term cyclic loading}

	The current process for evaluating BHV designs is an expensive and time consuming three-stage process: 1) accelerated wear testing(AWT), 2)large animal studies, and 3) clinical trials.
	AWT is performed by cycling BHVs in sterile saline at 10 to 20 times the normal heart rate. 
	It is currently the only way to evaluate BHV durability in a feasible amount of time (months instead of years). 
	However, the loading conditions and environment during AWT are not physiological and the only durability information currently used is the presence of visible damage. 
	Designs which show promise are then put through costly large animal studies, which still do not fully duplicate the human native environment, followed by clinical evaluations. 
	Only clinical trials in this process can provide true indications of the \textit{in vivo} performance of BHV designs, but this is the last, most difficult, most expensive and most time consuming stage. 
	Clearly, current methods of evaluating BHV designs are not feasible for advancing the BHV technology in a timely manner. 
	Computational simulations have been presented as an effective approach to this problem\cite{soares_biomechanical_2016}. 
	To be effective, we need better understanding of the underlying mechanisms of structural damage. 
	

	However, the mechanisms behind how the mechanical response of the exogenously crosslinked tissues used to construct the BHV leaflets changes with cyclic loading are poorly understood. 
	The most significant change due to cyclic loading is the geometry of BHVs. 
	In the study on the porcine aortic BHVs \cite{smith_high_1997},  Smith \textit{et al}. found that the unloaded geometry of the BHV changes permanently over time with AWT, especially in the belly region of the leaflet (Fig. \ref{fig:PSeffects}A). 
	By changing the configuration in the unloaded state, the shape of the leaflets and their properties will change as well, increasing the stress in those regions. 
	Further analysis on their results shown that significant structure damage occurred within this belly region as compared to other regions of the leaflet \cite{smith_fatigue_1999}.  
	Interestingly, Smith et al. also found most change in BHV leaflet geometry to occur within first 50 million cycles \cite{smith_high_1997}. 
	Moreover, Sacks and Smith \cite{sacks_effects_1998} also found minimal structural damage to occur in this early stage (Fig. \ref{fig:PSeffects}B). 
This is further supported by the study of Wells \textit{et al} \cite{wells_cyclic_2005}, where only minimal structural changes occurred within the pressure fixed BHVs during the first 50 million cycles, with most change in the first million cycles.  
	Clearly, there is a non-damage based mechanism at play that changes the geometry of the material with significant impact on the early stage of cycling and the rate of fatigue in later stages. 
	
\begin{figure}[hbt]
\centering
\includegraphics[width=0.55\paperwidth]{Images/chapter4/figure1.jpg}
\caption{A) The 3D unloaded geometry a BHV leaflet before and after cyclic loading. The color shows the root mean squared curvature of the leaflet. The most significant change in geometry is in the belly region. B) Small angle light scattering imaging of the flattened BHV leaflet used to examine the collagen fiber architecture. The collagen fiber architecture are convected by the dimensional changes but the corresponding regions remain mostly the same.}
\label{fig:PSeffects}
\end{figure}





\subsection{The effect of exogenous crosslinking}

	Bovine pericardium (BP) is a dense collagenous tissue composed mostly of collagen type I fibers with some elastin, GAGs, cells and vasculature. 
	Collagen fibers are complex protein structures at the 2 - 10 $\mu$m scale, made from tightly bundled collagen fibrils. 
	Much of we know about the use of GLUT to exogenously crosslink collagenous tissue is from the previous works of Nimni \textit{et al.}\cite{cheung_mechanism_1990, nimni_chemically_1987, cheung_mechanism_1985, gendler_toxic_1984, cheung_presence_1983, cheung_mechanism_1982, cheung_mechanism_1982II}. 
	GLUT, which is an aldehyde based crosslinker, aggressively crosslinks free amines in proteins. 
	This suppresses immunogenicity by crosslinking all cell membrane protein remnants, but also form polymeric networks (at the nm scale) which allows for long range crosslinks to the nearby tissue microstructures. 
	We previously found that exogenous crosslinking increases the bending stiffness of the matrix by four time the original stiffness \cite{mirnajafi_effects_2010}.
	We also tested the mechanical response exogenously crosslinked BP before and after crosslinking, and developed the first constitutive model for exogenously crosslinked soft tissue \cite{sacks_novel_2016}. 
	Interestingly, we also found that exogenously crosslinking does not increase the collagen fiber modulus, but significantly increases the interactions between collagen fibers \cite{sacks_novel_2016}, which is responsible for up to 30\% of the stress in the fully loaded state. 


	GLUT crosslinking is a Schiff-base reaction, which are known for their unstable molecular bonds \cite{migneault_glutaraldehyde_2004, damink_glutaraldehyde_1995}. 
	Although highly reactive and readily form crosslinks, the crosslinks formed by GLUT also readily undergo hydrolysis \cite{migneault_glutaraldehyde_2004}, allowing for the scission-healing of the GLUT polymer network to occur continuously at body temperature. 
	Under cyclic loading, this re-crosslinking occurs continuously and allow for the exogenously crosslinked matrix to gradually change in reference configuration to the current loaded state. 
	\emph{We hypothesize that this scission-healing behavior could be linked to the  mechanisms that underlie the changes in BHV geometry over time}. 
	
	
This non-damage related change in the BHV geometry has many similarities to the permanent set (PS) mechanisms observed in elastomers. 
Permanent set is an irreversible deformation that remains in a structure or material after it has been subjected to stress. 
It does not induce any damage to the constituents of the material, but instead changes the referential configuration. 
Although the outcomes may be similar, permanent set is not a plastic deformation as deformations are entirely in the elastic regime of the material. 
In particular, it is used to model the scission-healing reaction of elastomers that occurs when heated. 
Here, elastomers are stretched, heated, cooled, then unloaded, allowing some of the materials to change in referential configuration to the loaded state. 
In this process, permanent set does not damage the polymeric fibers but is a result of the change in configuration of the underlying polymeric fiber network. 
Some works on constitutive models of this process include the works of Rajagopal and Wineman\cite{rajagopal_constitutive_1992}, Andrews and Tobolsky\cite{andrews_theory_1946}, and Rottach \textit{et al}.\cite{rottach_effect_2004, rottach_molecular_2007, rottach_permanent_2006}. 
However, there are no known studies on the permanent set effect in soft tissues or soft tissue derived biomaterials. 
	

	\emph{Based on known considerations, we hypothesize that the initial time evolving BHV geometry and mechanical response can be predicted by permanent set mechanisms.} 
	Rather than heating and cooling, we assume that permanent set in exogenously crosslinked tissue occurs continuously at body temperature due to the GLUT, allowing the reference configuration of the exogenously crosslinked matrix to evolve over time. 
	As a result, the reference configuration of BHVs will change while not affecting the intrinsic material properties of their constituents such as collagen fibers and the exogenously crosslinked matrix. 
	Since the BHVs deform into a new configuration (Fig. \ref{fig:PSeffects}), there may develop regions of stress concentrations. 
	The high stresses will increase the rate of structural damage and thus accelerate the rate of BHV failure. 
	Thus, \emph{by compensating for the permanent set effects, it may be possible to reduce the BHV leaflet stresses and thus reduce structural damage to BHVs during \textit{in vivo} operation after implant.} 
	Therefore, constitutive models that can predict the effects of permanent set are crucial for the simulation of BHVs. 
		

\subsection{Constitutive models for the permanent set effect for soft tissues}

	There has been very little research in the constitutive modeling and simulation of the permanent set effect in native or exogenously crosslinked soft tissues. 
	The only work in this area is the pioneering study of Martin and Sun \cite{martin_modeling_2013}, who developed a time dependent constitutive model for uniaxial cyclic loading of GLUT exogenously crosslinked BP strips using a damage analog. 
	Briefly, to model the uncycled mechanical response, they used the Fung hyperelastic model for the strain energy density function, $\Psi_0$. 
	The time dependent stress softening was then described using a modification to the strain energy function as $\Psi_\mathrm{PS} = (1-D_s)\Psi_0 - \Psi_0(\mathbf{E}_\mathrm{PS})$. 
	Here the strain energy of the material after permanent set ($\Psi_\mathrm{PS}$) is reduced from $\Psi_0$ in accordance with a scaled damage function $D_s$ and the same strain energy function evaluated at the current amount of permanent set ($\mathbf{E}_\mathrm{PS}$). 
	Martin and Sun then measured the maximum permanent set deformation that occurs for each specimen, and used that to set the maximum bound of $\mathbf{E}_\mathrm{PS}$. 
	Additionally, Martin and Sun set two parameters, a lowerbound below which no permanent set occurs, and an upperbound above which the tissue fails. 
	In the intermediate region, $D_s$ and $\mathbf{E}_\mathrm{PS}$ simply increases linearly with the number of cycles. 
	The number of cycles needed to reach the maximum permanent set deformation was then described using an inverse exponential like function of the maximum strain applied, which is an analogous to the stress versus cycles (S–N) curve used to describe the fatigue behavior of traditional engineering materials.
	However, such approaches have the following limitations: 
	1) they are not predictive as there are no underlying mechanisms in the model. 
	A damage-like model can mimic the results of permanent set but cannot explain the mechanisms responsible in exogenously crosslinked tissue, and 2) there is no way to extend it for predicting into unmeasured regimes using the Fung hyperelastic model. 
	Therefore, there is a need for a more predictive constitutive model that utilizes the underlying mechanisms.

	

\subsection{A new approach for modeling the permanent set effect in exogenously crosslinked soft tissues}

	We utilize structural constitutive models, which integrate information on tissue composition and structure for material characterization \cite{sacks_structural_2000}. 
	In principal, using structural models, we only need to do parameter estimation for the innate material properties such as collagen fiber modulus and matrix modulus, and then use the microstructure of the tissue to predict the mechanical response \cite{zhang_meso_2016, fata_insights_2014}. 
	As a result, structural models have the potential to predict the mechanical response when extrapolating past the range of the experiment data.
	Thus, by using a structural modeling approach, we can use 1) the uncycled mechanical response of the BHV leaflet tissue and 2) the mechanisms responsible for the evolving leaflet microstructure due to permanent set to predict the evolving mechanical response of the BHV leaflet. The key aspects of our approach include:
\begin{enumerate}
\item Model the change in mechanical response entirely as a kinematic change in the underlying microstructure; no actual change in mechanical properties (damage)
\item Predict the change in geometry from the loading history and the permanent set mechanism
\item Validate the permanent set mechanism under both strain and stress controlled loading conditions
\item Develop a time dependent implementation with an evolving loaded configuration under stress control
\end{enumerate}

%---    Experimental methods and data post-processing
\section{Overall modeling approach} \label{sec:modelapproach}

	 We hypothesize that permanent set and structural damage are two separate and largely sequential mechanisms that underlie the mechanical response of BHVs. 
	 Thus, the life span of BHVs can by separated into three stages: early (up to 2-5 years),  intermediate (up to 10 years), and late stage(up to failure) (Fig. \ref{fig:hypothesis}). 
	 In the early stage, permanent set induces significant changes in BHV geometry while structural damage does not play a detectable role. 
	 This leads to increased stress and structural damage in the intermediate term which leads to failure in the late term. 
	 Thus, by compensating for the effect of permanent set on the geometry, we hypothesize that we can extend the life span of BHVs. 
	 We focus our model on the early stage of cyclic loading and build a solid foundation for modeling and simulating the later stages.
	 
\begin{figure}[hbt]
\centering
\includegraphics[width=0.4\paperwidth]{Images/chapter4/figure2}
\caption{We speculate that the progression of structural damage and permanent set of BHV leaflets progress during long term cyclic loading can be separated into three stages: early, intermediate, and late.}
\label{fig:hypothesis}
\end{figure}

	We assume that permanent set is driven by the scission-healing behaviors of the GLUT polymer network in the non-fibrous exogenously crosslinked matrix, allowing for fractions of it to change in reference state (Fig. \ref{fig:PS}A). 
	The formation of aldehyde bonds during crosslinking, as well as the occasional hydrolysis of the aldehyde bonds follows \emph{first order molecular kinetics} \cite{migneault_glutaraldehyde_2004}. 
	Since this process drives the scission-healing of GLUT polymers and thus permanent set, we hypothesize that permanent set also follows first order kinetics. 
	In addition, we note that the length scale of crosslinks formed by GLUT polymers ($\mathrm{nm}$) in the exogenously crosslinked matrix is several orders of magnitude smaller than that of the collagen fibers ($\mu \mathrm{m}$). 
	Thus, we assume that the collagen fiber architecture is \emph{not} unraveled in the scission-healing process. 
	Instead, the collagen fiber architecture remains intact during cyclic loading. 
	Rather, it is convected by the changes in geometry that occurs with permanent set in the exogenously crosslinked matrix (Fig. \ref{fig:PS}B).
	\emph{We refer to this mechanism as structural convection, which is defined as applying a permanent deformation (elongation and rotation of collagen fibers, Fig. \ref{fig:PS}C\&D) to the collagen fiber architecture based on the change in the reference configuration}. 
	Previously, we have shown that dense collagenous tissues deform under affine kinematics \cite{lee_presence_2015}. 
	Since deformations during cyclic loading are in the elastic regime and cyclic loading (second) is on time scale much shorter than that of permanent set (weeks), the only change in collagen fiber architecture on a cycle to cycle period is also under affine kinematics. 
	Therefore, we also assume that the convection of collagen fiber architecture due to permanent set is under affine kinematics as well. 
	Thus, our approach is based on using the permanent set effect to model the evolving properties of the exogenously crosslinked matrix, which is used to determine the change in reference configuration and convect collagen fiber architecture, which then allows us to determine the change in mechanical response of the collagen fibers using structural models. 

\begin{figure}[hbt]
\centering
\includegraphics[width=0.5\paperwidth]{Images/chapter4/figure3}
\caption{Illustration of the permanent set effect under cyclic uniaxial loading. A) There is a transfer of mass fraction of the exogenously crosslinked matrix to the loaded configuration $\Omega(s)$ from the original state $\Omega_0$. B) This results in a change in the unloaded geometry of the tissue. The collagen fiber embedded within the exogenously crosslinked matrix is convected by the geometry change and goes under C) rotation and D) straightening.}
\label{fig:PS}
\end{figure}
	
	To develop the constitutive model form, we start by modeling the exogenously crosslinked tissue under cyclic loading as parts of a mixture of materials with the same properties but evolving reference states. 
	The reference state of each part of the exogenously crosslinked matrix is updated according the strain history (Fig. \ref{fig:PS}A). 
	The response of each part are then summed together for the bulk-level mechanical response of the matrix.
	The new bulk level mechanical response is then used to determine the new unloaded state (Fig. \ref{fig:PS}B).
	The change in the matrix from its uncycled state is then used to convect the collagen fiber architecture(Fig. \ref{fig:PS}C\&D). 
The new mechanical response of the exogenously crosslinked matrix is summed with the fiber ensemble interactions and collagen fiber response predicted from the new collagen fiber architecture to determine the full tissue-level mechanical response. 
Thus, how the \emph{collagen fiber architecture (CFA) is convected by the exogenously crosslinked matrix} is crucial to our model.
We start developing our constitutive model by 1) modeling and parameter estimation for the native uncycled mechanical response of the exogenously crosslinked tissue. 
Then 2) develop the model form for the time evolving mechanical response based on the permanent set mechanism and cyclic loading data. The basic assumptions of our models are:
\begin{enumerate}
\item There is no structural damage
\item Permanent set follows first order kinetics
\item We are only modeling permanent set under physiological conditions and the process is thus isothermal
\item Permanent set occurs in the isotropic matrix, so that its rate constant is directionally independent
\item The permanent set rate constant is strain level independent in the physiological range
\item Permanent set occurs over a time scale much longer than a single cardiac cycle
\item The tissue is functionally elastic, and thus ignore viscoelastic effects
\item The convection of the collagen fiber architecture follows affine kinematics \cite{lee_presence_2015}
\end{enumerate}

%---    Delineation and modelling of the tissue-level mechanical effects of exogenous cross-links
%%%%%%%%%%%%%%%%%%%%%%%%%%%%%%%%%%%%%%%%%%%%%%%%%%%%%%%%%%%%%%%%%%%%%%%%%%%%%%%%
%%  Experimental Data
%%%%%%%%%%%%%%%%%%%%%%%%%%%%%%%%%%%%%%%%%%%%%%%%%%%%%%%%%%%%%%%%%%%%%%%%%%%%%%%%


\section{Extant experimental data}
\label{sec:database}

	We utilized data from the following two extant studies, which differ by their loading conditions and specimen orientations. In the study by Sun \textit{et al}.\cite{sun_response_2004}, exogenously crosslinked BP patches were cycled along the preferred collagen fiber direction (PD) at a peak strain of 16\% and maintained for up to 65 million cycles (Fig. \ref{fig:database}A\&B). The reference configurations of the exogenously crosslinked BP were tracked using fiducial markers attached to the center of the specimens\cite{sacks_biaxial_2000}. The mechanical cycling was stopped at 30 and 65 million cycles for mechanical testing with multiple protocols and different loading paths. It was found that there were significant extensions in the direction of loading (7.1\%), PD, and contractions in the cross direction (-7.7\%). In addition, the collagen fiber crimp period increased from 40.6$\mu$m to 45.24$\mu$m which is consistent with the convection of the collagen fiber architecture that would occur as collagen fiber straighten (Fig. \ref{fig:structuralconvection}). The mechanical response also changed accordingly. The compliance in the PD decreased over time while the compliance in the cross direction of the tissue increased. 
	

\begin{figure}[hbt]
\centering
\includegraphics[width=\textwidth]{Images/chapter4/figure5}
\caption{We utilize two extant databases in this study. The first is strain controlled, with A) constant strain level and is B) sorted for orientation along the preferred direction then put through cyclic loading and mechanical testing. The second is with C) constant strain but evolving strain level. The specimens are then D) sorted for orientation along both the preferred direction and cross-preferred direction, then put through cyclic loading and mechanical testing.}
\label{fig:database}
\end{figure}



	In the study by Sellaro \textit{et al}.\cite{sellaro_effects_2007}, the same exogenously crosslinked BP patches were cycled at 500kPa for up to 50 million cycles (Fig. \ref{fig:database}C\&D). In this case, the specimens were separated into two groups: with stress controlled loading along the 1) PD and 2) orthogonal to the PD (XD). Similarly, the reference states of the exogenously crosslinked BP specimens were tracked, with cycling stopped at 20 and 50 million cycles for mechanical testing. Analysis of the results showed interesting differences between the PD and XD cycled specimens. For both the PD and XD cycled specimens, we observed significant elongation in the direction of loading and contraction in the orthogonal direction with cycling. Additionally, the effective stiffness in the direction of loading increased over time, whereas the orthogonal direction decreased over time. In addition, we also studied how the collagen fiber architecture of the tissue changed with cyclic loading. For the PD cycled specimens, it was observed that the collagen fiber orientation distribution became more aligned, and the collagen fiber crimp period increased from 24 $\mu$m to 28$\mu$m. On the other hand, for the XD cycled specimens, it was observed that the collagen fiber orientation distribution became more spread, and the collagen fiber crimp period remained unchanged. This findings on the structural changes with direction of cyclic loading are consistent with the structural convection that we hypothesizes with permanent set and lend further support to our model (Fig. \ref{fig:structuralconvection}). 

%---    Initial model formulation
\section{Constitutive model for uncycled exogenously crosslinked tissue}
\subsection{Constitutive model for exogenous crosslinking}

	We have previously developed the first constitutive model for exogenously crosslinked collagenous tissues \cite{sacks_novel_2016} and determined the following three contributors to the mechanical response: collagen fibers, exogenously crosslinked matrix and fiber-fiber interactions. 
	In particular, we found the fiber ensemble interaction term to be especially important in modeling exogenously crosslinked tissues, accounting for approximately 30\% of the stress in the fully loaded state (Fig. \ref{fig:EXLforms}A). 
	To determine the model form of the interaction component, we used the remaining stress after subtracting the collagen fiber response and exogenously crosslinked matrix response from the mechanical response of the tissue after exogenous crosslinking (Fig. \ref{fig:EXLforms}A). 
	In that study, we consider three possible forms of interactions: intra-fiber \textbf{ensemble} (Fig. \ref{fig:EXLforms}B) (an ensemble being a family of fiber sharing a common orientation), ensemble-ensemble rotations, and ensemble-ensemble relative extensions (Fig. \ref{fig:EXLforms}C\&D). 
	We found that intra-fiber ensemble and ensemble-ensemble rotations were not consistent with the experimental data, whereas ensemble-ensemble extensional interactions were able to explain all of the remaining stress. 
	
	We developed the fiber ensemble interaction model form utilizing the pseudo invariant $I_8$, and separated it into its rotational and extensional components\cite{sacks_novel_2016},
\begin{equation}
\begin{gathered}
I_8 = \mathbf{n}_0\left( \alpha\right)\cdot\mathbf{C}\cdot\mathbf{n}_0\left( \beta\right) = \lambda_\alpha \lambda_\beta \cos(\alpha - \beta), \\
I_8^{\mathrm{ext}} = \lambda_\alpha \lambda_\beta, \qquad I_8^{\mathrm{rot}} = \cos(\alpha - \beta) = \frac{I_8}{\lambda_\alpha \lambda_\beta},
\end{gathered}
\end{equation}
where $\mathbf{n}_0$ is an unit vector pointing along $\theta$, $I_8^{ext}$ is the pseudo invariant for ensemble-ensemble relative extensions, and $I_8^{rot}$ is the ensemble-ensemble rotations. 
From this, we established model form for the interactions to be
\begin{equation}
\Psi_{\mathrm{int}} = \frac{d_0}{4}\int\displaylimits_\alpha \int\displaylimits_\beta \Gamma\left(\alpha\right)\Gamma \left( \beta \right)\left[ e^{d_1(\lambda_\alpha \lambda_\beta - 1)^2}-1 \right] \mathrm{d}\alpha\, \mathrm{d}\beta,
\end{equation}
where $\Gamma$ is the fiber orientation distribution (ODF), and $d_0$ and $d_1$ are material constants. 
	However, this model form is still essentially phenomenological. 
	Specifically, while it is sufficient to model the mechanical response in the range of the acquired experimental data, we have no method for predicting how it will change with changes in dimensions with cyclic loading. Thus, an extension to this model component is necessary.

\begin{figure}[hbt]
\centering
\includegraphics[width=0.5\paperwidth]{Images/chapter4/figure5}
\caption{A) The mechanical response of exogenously crosslinked BP, which is composed of 3 parts: (C)ollagen(Red), (M)atrix(Blue), and the fiber ensemble (I)nteractions (Green). B) Illustration for intra-ensemble interactions due to crosslinking is shown. C) Ensemble-ensemble interactions could be separated into rotational effects and extensional effects. }
\label{fig:EXLforms}
\end{figure}

\subsection{Extension of the structural derivation of the fiber-fiber interactions term}

The key to our approach in the constitutive model for the permanent set effect is to use the change in the collagen fiber architecture to predict the new mechanical response. 
Thus, having a full structural model, including a full structural derivation of the fiber-fiber interactions, is crucial. 
As in the previous model form \cite{sacks_novel_2016}, we will only keep the extensional component $I_8^{ext}$. In addition, since collagen fibers do not bear stress until fully straightened \cite{soares_biomechanical_2016}, we also assume that \emph{collagen fibers do not play a role in the interactions of the fiber ensembles until they are straightened}. Given the true fiber stretch ($\lambda_t$) as defined in Zhang et al. \cite{zhang_meso_2016}, $\lambda_t = \lambda_{ens}/\lambda_s$, we incorporated the collagen slack stretch, $\lambda_s$, into the invariant, with
\begin{equation}
I_8^{\mathrm{ext}} = \frac{\lambda_\alpha \lambda_\beta}{\lambda^\alpha_s \lambda_s^\beta}.
\end{equation}
To develop the form for the interactions, we start from the strain energy. 
At the ensemble level, we integrate over the slack stretch $(D)$, as define in Zhang \textit{et al.} \cite{zhang_meso_2016}, of fiber ensembles orienting along $\alpha$ and $\beta$, 
\begin{equation}
\Psi_{\mathrm{int}}^{\mathrm{ens}} = \frac{\eta_I}{2} \int\displaylimits_1^{\lambda_\alpha} \int\displaylimits_1^{\lambda_\beta} D\left( x_\alpha \right) D\left( x_\beta \right) \left( \frac{\lambda_\alpha \lambda_\beta}{x_\alpha x_\beta} - 1\right)^2 \,\mathrm{d}x_\alpha \,\mathrm{d}x_\beta.
\end{equation}
This is then integrated over the fiber ODF, $\Gamma$, for the tissue-level model
\begin{equation}
\Psi_{\mathrm{int}} = \frac{\eta_I}{2} \int\displaylimits_\alpha \int\displaylimits_\beta \Gamma(\alpha) \Gamma(\beta) \int\displaylimits_1^{\lambda_\alpha} \int\displaylimits_1^{\lambda_\beta} D\left( x_\alpha \right) D\left( x_\beta \right) \left( \frac{\lambda_\alpha \lambda_\beta}{x_\alpha x_\beta} - 1\right)^2 \,\mathrm{d}x_\alpha \,\mathrm{d}x_\beta \,\mathrm{d}\alpha \,\mathrm{d}\beta.
\end{equation}
The second Piola Kirchhoff stress, using $\mathbf{S}=2\frac{\partial\Psi}{\partial\mathbf{C}}$, is 
\begin{equation} \label{eq:interaction}
\begin{split}
\mathbf{S}_{\mathrm{int}} = \phi_C \eta_I \int\displaylimits_\alpha \int\displaylimits_\beta \Gamma \left(\alpha \right) \Gamma \left( \beta \right) 
\left[ \left\lbrace 
\int\displaylimits_1^{\lambda_\alpha} \int\displaylimits_1^{\lambda_\beta} 
\frac{2 \lambda_\beta D(x_\alpha) D(x_\beta)}{x_\alpha x_\beta} 
\left( \frac{\lambda_\alpha}{x_\alpha} \frac{\lambda_\beta}{x_\beta} - 1\right) \mathrm{d}x_\alpha \, \mathrm{d}x_\beta \right.\right. +&\\
\left. \left. \int\displaylimits_1^{\lambda_\beta} D(x_\beta) \left( \frac{\lambda_\beta}{x_\beta} -1  \right)^2 \mathrm{d}x_\beta \right\rbrace \right.  \frac{\mathbf{n}_\alpha \otimes \mathbf{n}_\alpha}{\lambda_\alpha} +& \\
\left. \left\lbrace
\int\displaylimits_1^{\lambda_\alpha} \int\displaylimits_1^{\lambda_\alpha} 
\frac{2 \lambda_\beta D(x_\alpha) D(x_\beta)}{x_\alpha x_\beta} 
\left( \frac{\lambda_\alpha}{x_\alpha} \frac{\lambda_\beta}{x_\beta} - 1\right) \mathrm{d}x_\alpha \, \mathrm{d}x_\beta 
\right. \right. +&\\
\left. \left. \int\displaylimits_1^{\lambda_\alpha} D(x_\alpha) \left( \frac{\lambda_\alpha}{x_\alpha} -1  \right)^2 \mathrm{d}x_\alpha \right\rbrace \frac{\mathbf{n}_\beta \otimes \mathbf{n}_\beta}{\lambda_\beta}  \right]& \mathrm{d}\alpha \, \mathrm{d}\beta.
\end{split}
\end{equation}
This model form has only one constant $\eta_I$ to account for all interactions. The remaining mechanisms are all structurally based, which is determined through a convection of the collagen fiber architecture. 


%---    Model simplifications and parameter estimation
%%%%%%%%%%%%%%%%%%%%%%%%%%%%%%%%%%%%%%%%%%%%%%%%%%%%%%%%%%%%%%%%%%%%%%%%%%%%%%%%
%%  Permanent Set Model
%%%%%%%%%%%%%%%%%%%%%%%%%%%%%%%%%%%%%%%%%%%%%%%%%%%%%%%%%%%%%%%%%%%%%%%%%%%%%%%%


\section{Permanent set model}

%%%%%%%%%%%%%%%%%%%%%%%%%%%%%%%%%%%%%%%%%%%%%%%%%%%%%%%%%%%%%%%%%%%%%%%%%%%%%%%%
%%%%    Kinematics

\subsection{Kinematics}

	For the bulk tissue level mechanical response, consisting of the EXL matrix, the collagen fibers, and fiber ensemble interactions, we first consider the EXL matrix alone. Permanent set occurs in the EXL matrix, and is the main driver for the evolving mechanical response and structural changes in the tissue. As such the permanent set model is our starting point (section \ref{sec:modelapproach}), which then can be used to convect collagen fiber architecture and predict the remaining collagen fibers and fiber ensemble interactions components of the mechanical response. To model the EXL matrix under permanent set, many reference states need to be considered. Due to scission-healing, the reference configuration of the EXL matrix is the configuration at its time of formation. For this we first introduce the following definition for the for the evolution of the configurations involved (Fig. \ref{fig:stateevolution}):

\begin{enumerate}

\item The original unloaded configuration $\Omega_0$ 

\item The evolving loaded configuration is $\Omega(s)$, where $s$ is the current time. 

\item We also make the distinction between $\hat{s}$, the intermediate time for which the EXL matrix is formed, and $s$. As the reference configuration of the EXL matrix evolves, there needs to be a distinction between the configuration for which the EXL matrix formed by the scission-healing reaction at time $\hat{s}$, referenced to $\Omega(\hat{s})$, and the current loaded configuration $\Omega(s)$. In this way, $\hat{s}$ is also suitable as an variable of integration.

\item The strain history, $\mathbf{A}(s)$, which is a deformation gradient tensor as a function of time $s$, that maps between the original configuration $\Omega_0$ to the loaded configuration $\Omega(s)$ at which the EXL matrix is formed

\item We define $\tilde{\mathbf{B}}(s) = \mathbf{A}\mathbf{A}^\mathsf{T}$ to be left Cauchy Green tensor of the strain history $\mathbf{A}(s)$, which should not be confused with the left Cauchy Green tensor of the deformation applied to the tissue

\item We define the deformation gradient tensor from the intermediate loaded state $\Omega(\hat{s})$ to the current loaded state $\Omega(s)$ to be $\mathbf{\bar{F}}(\hat{s})$, where the following relation is assumed (Fig. \ref{fig:stateevolution})
\begin{equation} \label{eq:strainhistory}
\begin{split}
&\mathbf{F} = \mathbf{\bar{F}}(\hat{s})\mathbf{A}(\hat{s}), \\
&\mathbf{\bar{F}}(\hat{s}) = \mathbf{F} \cdot \mathbf{A}(\hat{s})^{-1}.
\end{split}
\end{equation}

\item The evolving unloaded reference configuration after permanent set is $\Omega_\mathrm{PS}(s)$ with $\mathbf{F}_\mathrm{PS}(s)$ mapping from the original configuration $\Omega_0$ to $\Omega_\mathrm{PS}(s)$

\end{enumerate}

\begin{figure}[hbt]
\centering
\includegraphics[width=3.5in]{Images/chapter4/figure7}
\caption{The relation between the reference configurations during cyclic loading.}
\label{fig:stateevolution}
\end{figure}

    We also note the following important considerations:
\begin{itemize}
\item The original reference configuration $\Omega_0$ is important as it is the configuration for which the mechanical response of the original material as well as the collagen fiber architecture is referenced. Similarly, we track the loaded configuration ($\Omega(s)$) using $\mathbf{A}(s)$ and the unloaded configuration ($\Omega_\mathrm{PS}(s)$) using $\mathbf{F}_\mathrm{PS}$ from $\Omega_0$. 

\item We also note that under cyclic loading, the exogenously crosslinked tissue is not held at the constant loaded configuration, it is put through a range of deformation each cycle. Because permanent set happens at a time scale much longer than a single cycle, we define the loaded configuration $\Omega(s)$ as the root mean squared configuration, which is the time averaged deformation for a single loading cycle. 

\item We note that the tissue is never fully unloaded during cyclic loading. Thus, to simulate cyclic loading, the stresses are still referenced to the origin configuration $\Omega_0$. The unloaded geometry is determined \textit{a posteriori} from $\Omega_\mathrm{PS}(s)$ for when cyclic loading is stopped for mechanical testing. 
\end{itemize}



    Thus the right Cauchy Green tensor when referenced to the strain history $\Omega(\hat{s})$ is given by $\mathbf{\bar{C}}(\hat{s}) = \mathbf{\bar{F}}(\hat{s})^\mathsf{T} \mathbf{\bar{F}}(\hat{s})$. This also has the following relation to the applied deformation $\mathbf{F}$,
\begin{equation} \label{eq:rightcauchy}
\begin{split}
\mathbf{\bar{C}}(\hat{s}) &= \left( \mathbf{F}\cdot\mathbf{A}(\hat{s})^{-1} \right)^\mathsf{T} \left(\mathbf{F} \cdot \mathbf{A}(\hat{s})^{-1} \right)\\
	&= \mathbf{A}(\hat{s})^{-\mathsf{T}} \left( \mathbf{F}^\mathsf{T} \mathbf{F} \right) \mathbf{A}(\hat{s})^{-1} \\
	&= \mathbf{A}(\hat{s})^{-\mathsf{T}} \cdot \mathbf{C} \cdot \mathbf{A}(\hat{s})^{-1}.
\end{split}
\end{equation}
    Our material model for the EXL matrix is an isotropic model of $\Psi_m = \Psi_m(I_1)$, where $I_1$ is the first invariant of $\mathbf{C}$. For this, we can define the first invariant $\bar{I}_1(\hat{s})$ for the right Cauchy Green tensor $\bar{\mathbf{C}}(\hat{s})$ as 
\begin{equation}\label{eq:pseudo1stinv}
\bar{I}_1(\hat{s}) = \operatorname{Trace}(\mathbf{\bar{C}}(\hat{s})) = \operatorname{Trace}\left(\mathbf{A}^{-\mathsf{T}}(\hat{s}) \cdot \mathbf{C} \cdot \mathbf{A}^{-1}(\hat{s})\right).
\end{equation}
    With this, the stress of the EXL matrix is given by 

\begin{equation}
\mathbf{\bar{S}}_m = 2 \frac{\partial \bar{\Psi}}{\partial \mathbf{C}} - p\mathbf{I} = 2 \frac{\partial \bar{\Psi}}{\partial \bar{I}_1(\hat{s})} \frac{\partial \bar{I}_1(\hat{s})}{\partial \mathbf{C}} - p\mathbf{I},
\end{equation}
    where $p$ is the Lagrange multiplier enforcing incompressibility. The partial derivative of $\bar{I}_1(\hat{s})$ is
\begin{equation}
\frac{\partial\bar{I}_1}{\partial\mathbf{C}} = \frac{\partial\bar{I}_1}{\partial\bar{\mathbf{C}}(\hat{s})} \frac{\partial\bar{\mathbf{C}}(\hat{s})}{\partial \mathbf{C}}= \frac{\partial\mathbf{A}(\hat{s})^{-\mathsf{T}} \cdot \mathbf{C} \cdot \mathbf{A}(\hat{s})^{-1}}{\partial \mathbf{C}} = \mathbf{A}(\hat{s})^{-\mathsf{T}} \mathbf{A}(\hat{s})^{-1} = \mathbf{\tilde{B}}(\mathbf{A}(\hat{s}))^{-1},
\end{equation}
    which is the inverse of the left Cauchy Green tensor of the strain history $\mathbf{A(s)}$.


%%%%%%%%%%%%%%%%%%%%%%%%%%%%%%%%%%%%%%%%%%%%%%%%%%%%%%%%%%%%
%%%%%%      Constitutive Model for crosslinking

\subsubsection{Kinematics for updating the reference configuration}

    As the unloaded configuration changes due to permanent set, we need to be able to express the stresses with the new reference configuration. Since all configurations are referenced to $\Omega_0$, the deformation from the new reference configuration $\Omega_\mathrm{PS}$ to the current loaded state $\Omega(s)$ is given by (Fig. \ref{fig:PSevolution})
\begin{equation}
\mathbf{F} = \mathbf{\bar{F}}(\hat{s}) \cdot \mathbf{A}(\hat{s}) \cdot \mathbf{F}_\mathrm{PS}^{-1}.
\end{equation}
    Following the same derivation as equations \ref{eq:strainhistory}--\ref{eq:pseudo1stinv}, we have
\begin{equation} \label{eq:newhistory}
\begin{split}
\mathbf{\bar{F}}(\hat{s}) &= \mathbf{F} \cdot \mathbf{F}_\mathrm{PS} \cdot \mathbf{A}(\hat{s})^{-1},\\
\mathbf{\bar{C}}(\mathbf{F}_\mathrm{PS}, \mathbf{A}(\hat{s})) &= \mathbf{A}(\hat{s})^{-\mathsf{T}} \cdot \mathbf{F}_\mathrm{PS}^\mathsf{T} \cdot \mathbf{C} \cdot \mathbf{F}_\mathrm{PS} \cdot \mathbf{A}(\hat{s})^{-1}, \\
\bar{I}_1\left(\mathbf{F}_\mathrm{PS}, \mathbf{A}(s)\right) &= \operatorname{Trace}\left(\mathbf{A}(\hat{s})^{-\mathsf{T}} \cdot \mathbf{F}_\mathrm{PS}^\mathsf{T} \cdot \mathbf{C} \cdot \mathbf{F}_\mathrm{PS} \cdot \mathbf{A}(\hat{s})^{-1}\right). 
\end{split}
\end{equation}
    and the new inverse of the left Cauchy Green tensor of the strain history is given by
\begin{equation}
\begin{split}
\mathbf{\tilde{B}}(\mathbf{F}_\mathrm{PS}, \mathbf{A}(\hat{s}))^{-1} &= \frac{\partial \mathbf{\bar{C}}(\mathbf{F}_\mathrm{PS}, \mathbf{A}(\hat{s}))}{\partial \mathbf{C}} = \frac{\mathbf{A}(\hat{s})^{-\mathsf{T}} \cdot \mathbf{F}_\mathrm{PS}^\mathsf{T} \cdot \mathbf{C} \cdot \mathbf{F}_\mathrm{PS} \cdot \mathbf{A}(\hat{s})^{-1}}{\partial \mathbf{C}} \\
 &= \mathbf{A}(\hat{s})^{-\mathsf{T}} \cdot \mathbf{F}_\mathrm{PS}^\mathsf{T} \cdot \mathbf{F}_\mathrm{PS} \cdot \mathbf{A}(\hat{s})^{-1}. 
\end{split}
\end{equation}

\begin{figure}[hbt]
\centering
\includegraphics[width=4in]{Images/chapter4/figure8}
\caption{Modification of the relation between the different reference configurations when the unloaded configuration changes.}
\label{fig:PSevolution}
\end{figure}


%%%%%%%%%%%%%%%%%%%%%%%%%%%%%%%%%%%%%%%%%%%%%%%%%%%%%%%%%%%%%%%%%%%%%%%%%%%%%%%%
%%%%    Change in reference configuration

\subsection{Extension to the EXL matrix model for changes in reference configuration}

	We start by extending the EXL matrix model from our previous model \cite{sacks_novel_2015}. The strain energy is a modified form of the Yeoh model, which exhibits the following stress-strain relation,
\begin{equation}
\begin{split}
\Psi_m = &\frac{\eta_m}{2} \left( \frac{1}{\alpha}\left( I_1 -3\right)^{\alpha} + \frac{r}{\beta} \left( I_1 -3\right)^{\beta} \right), \\
&\text{with } 1<\alpha<\beta, \alpha\beta <2, 0 \leq r.
\end{split}
\end{equation}
    Here $\eta_m$ is the EXL matrix modulus, $\alpha,\beta$ are the exponent and $r$ is the relative weight between the two terms which is typically between 10 to 20. To modify the model form to reference the configuration at the time of formation $\Omega(\hat{s})$, we make the substitution for $\bar{I}_1(\mathbf{F}_\mathrm{PS}, \mathbf{A}(\hat{s}))$ (Eqn. \ref{eq:pseudo1stinv}),
\begin{equation} \label{eq:matrixenergyform}
\bar{\Psi}_m\left( \mathbf{C}, \mathbf{A}(\hat{s})\right) = \frac{\eta_m}{2} \left(\frac{1}{\alpha} \left( \bar{I}_1\left(\mathbf{F}_\mathrm{PS}, \mathbf{A}(s)\right) -3\right)^\alpha +\frac{r}{\beta} \left( \bar{I}_1\left(\mathbf{F}_\mathrm{PS}, \mathbf{A}(s)\right) -3\right)^\beta \right),
\end{equation}
    The derivative of the strain energy with respect to the invariant $\bar{I}_1$ is 
\begin{equation}
\frac{\partial \bar{\Psi}}{\partial \bar{I}_1} =	\frac{\eta_m}{2} \left(\left( \bar{I}_1\left(\mathbf{F}_\mathrm{PS}, \mathbf{A}(s)\right)- 3\right)^{\alpha - 1} + r \left( \bar{I}_1\left(\mathbf{F}_\mathrm{PS}, \mathbf{A}(s)\right) - 3\right)^{\beta - 1}\right).
\end{equation}
    After solving for the Lagrange multiplier $p$, with $\bar{S}_{m,33} = 0$, we have 
\begin{equation}\label{eq:matrixfinal}
\begin{split}
\mathbf{\bar{S}}_m \left( \mathbf{F}_\mathrm{PS},\mathbf{A}(\hat{s}),\mathbf{C}\right) &= \mu_m \left(\left( \bar{I}_1\left(\mathbf{F}_\mathrm{PS}, \mathbf{A}(s)\right) - 3\right)^{\alpha - 1} + r \left( \bar{I}_1\left(\mathbf{F}_\mathrm{PS}, \mathbf{A}(s)\right) - 3\right)^{\beta - 1}\right) \\
&\times \left( \mathbf{\tilde{B}}(\mathbf{F}_\mathrm{PS}, \mathbf{A}(\hat{s}))^{-1} - \tilde{B}_{33}^{-1}(\mathbf{F}_\mathrm{PS}, \mathbf{A}(\hat{s}))C_{33}\mathbf{C}^{-1}\right).
\end{split}
\end{equation}


%%%%%%%%%%%%%%%%%%%%%%%%%%%%%%%%%%%%%%%%%%%%%%%%%%%%%%%%%%%%
%%%%%%      Extension for permanent set

\subsubsection{Extension of the EXL matrix for the permanent set effect}

	Next, we developed the model form for the EXL matrix after permanent set. This approach is based on the work by Rajagopal and Wineman \cite{rajagopal_constitutive_1992}, where we assume that the response of the full EXL matrix to be 
\begin{equation} \label{eq:wineman}
\phi_m \mathbf{S}_m = b(s)\mathbf{\bar{S}}_m^\mathrm{existing} + \int\displaylimits_0^s a(s,\hat{s})\mathbf{\bar{S}}_m^\mathrm{new} \mathrm{d}\hat{s},
\end{equation}
    where $b(s)$ is the remaining amount of the existing material, $a(s,\hat{s})$ is the remaining amount of the new material formed during the strain history ($\mathbf{A}(s)$) at time $\hat{s}$, $\mathbf{\bar{S}}_m^\mathrm{existing}$ is the stress of the existing material and $\mathbf{\bar{S}}_m^\mathrm{new}$ is the stress of the new material. Assuming first order kinetics with the permanent set rate constant $k $, we have for the material remaining (Fig. \ref{fig:masstransfer})
\begin{equation}
\frac{\partial b(s)}{\partial s} = - k\cdot b(s), \qquad \frac{\partial a(s,\hat{s})}{\partial s} = - k \cdot a(s,\hat{s}).
\end{equation}
    Given that the total amount of EXL matrix ($\phi_m$) has not changed, $b(s) + \int_0^s a(s, \hat{s}) \mathrm{d}\hat{s} = \phi_m$, and we have
\begin{equation}
b(s) = \phi_m \mathrm{Exp}\left[-k  \cdot s\right], \qquad a(s,\hat{s}) = \phi_m k  \mathrm{Exp}\left[-k (s - \hat{s})\right].
\end{equation}
    Substitution into equation \ref{eq:wineman}, we have the mechanical response of the EXL matrix after permanent set,
\begin{equation}
\begin{aligned}
\phi_m \mathbf{S}_m =& \phi_m \left[\mathrm{Exp}\left[-k  \cdot s\right]\mathbf{\bar{S}}_m \left(\mathbf{F}_\mathrm{PS},\mathbf{A}(0),\mathbf{C}\right) \phantom{\int\displaylimits_0^s} \right. \\
&+ \left. \int\displaylimits_0^s k \cdot \mathrm{Exp}\left[-k (s - \hat{s})\right] \mathbf{\bar{S}}_m \left(\mathbf{F}_\mathrm{PS},  \mathbf{A}(\hat{s}),\mathbf{C}\right) \mathrm{d}\hat{s} \right].
\end{aligned}
\end{equation}
    The final form for the EXL matrix component of the permanent set model is thus,
\begin{equation}
\begin{split}
\phi_m \mathbf{S}_m &\left(\mathbf{F}_\mathrm{PS}, \mathbf{A}(\hat{s}),\mathbf{C}\right) \\
=& \phi_m \mu_m \left[ \vphantom{\int\displaylimits_0^s} \mathrm{Exp}\left[-k  \cdot s\right]  \left(\left( \bar{I_1} (\mathbf{F}_\mathrm{PS}, \mathbf{A}(0)) - 3\right)^{\alpha - 1} + r \left( \bar{I_1} (\mathbf{F}_\mathrm{PS}, \mathbf{A}(0)) - 3\right)^{\beta - 1}\right)  \right.\\
&\times \left( \mathbf{\tilde{B}}(\mathbf{F}_\mathrm{PS}, \mathbf{A}(0))^{-1} - \tilde{B}_{33}(\mathbf{F}_\mathrm{PS}, \mathbf{A}(0))^{-1}C_{33}\mathbf{C}^{-1}\right) \\
&+ \int\displaylimits_0^s k \cdot \mathrm{Exp}\left[-k (s - \hat{s})\right] \left(\left( \bar{I_1} (\mathbf{F}_\mathrm{PS}, \mathbf{A}(\hat{s})) - 3\right)^{\alpha - 1} + r \left( \bar{I_1} (\mathbf{F}_\mathrm{PS}, \mathbf{A}(\hat{s})) - 3\right)^{\beta - 1}\right) \\
&\times \left. \vphantom{\int\displaylimits_-^s} \left( \mathbf{\tilde{B}}(\mathbf{F}_\mathrm{PS}, \mathbf{A}(\hat{s}))^{-1} - \tilde{B}_{33}(\mathbf{F}_\mathrm{PS}, \mathbf{A}(\hat{s}))^{-1}C_{33}\mathbf{C}^{-1}\right) \mathrm{d}\hat{s}\right].
\end{split}
\end{equation}

\begin{figure}[hbt]
\centering
\includegraphics[width=4.5in]{Images/chapter4/figure9}
\caption{The mass fractions of the EXL matrix change over time, assuming first order kinetics. }
\label{fig:masstransfer}
\end{figure}

%%%%%%%%%%%%%%%%%%%%%%%%%%%%%%%%%%%%%%%%%%%%%%%%%%%%%%%%%%%%%%%%%%%%%%%%%%%%%%%%
%%%%    Structural Convection

\subsection{Convection of the collagen fiber architecture} \label{sec:convection}

    Now that the model form for the EXL matrix under permanent set is established, we need to consider how the collagen fiber architecture is convected by the change in reference configuration. The convection of the collagen fiber architecture is done through two parts: the ODF and the recruitment function. This is done by assuming that the collagen fiber architecture is convected under the affine assumption\cite{lee_presence_2015}. The form of the ODF and the recruitment distribution was previously described in Zhang \textit{et al.} \cite{zhang_meso_2016}, and the operation used to convect the collagen fiber architecture is given in Sacks \textit{et al}. \cite{sacks_novel_2015}. Briefly, the convected ODF $\Gamma_1$ is determined from the ODF in the 0-cycle state $\Gamma_0$ and the deformation $\prescript{1}{0}{\mathbf{F}}$ by the conservation of the number of fiber $\Gamma(\theta_0) \mathrm{d}\theta_0 = \Gamma(\theta_1) \mathrm{d}\theta_1$ (Fig. \ref{fig:effectsofconvection}A). This is given by
\begin{equation} \label{eq:pfODF}
\begin{gathered}
\Gamma_1\left( \prescript{1}{0}{\mathbf{F}},\theta_1 \right) = \Gamma_0\left( \theta_0\left( \prescript{1}{0}{\mathbf{F}},\theta_1\right)\right)\frac{\prescript{1}{0}{\lambda(\theta_0)}^2}{\prescript{1}{0}{J_\mathrm{2D}}}, \\
\prescript{1}{0}{\lambda(\theta_0)} = \sqrt{\mathbf{n}_{\theta_0}\cdot  \prescript{1}{0}{\mathbf{F}^\mathsf{T}}  \prescript{1}{0}{\mathbf{F}} \cdot \mathbf{n}_{\theta_0}}, \qquad \prescript{1}{0}{J_\mathrm{2D}} =\det (\prescript{1}{0}{\mathbf{F}}),
\end{gathered}
\end{equation}
    where $\mathbf{n}_\theta$ is a unit vector for the orientation $\theta$. The distribution of slack stretch needed to straighten collagen fiber crimp after convection, in other words the recruitment distribution $D_1(\lambda_s)$ (Fig. \ref{fig:effectsofconvection}B), is given by
\begin{equation} \label{eq:pfrecruitment}
\begin{gathered}
D_1(\prescript{1}{0}{\mathbf{F}}, \lambda_s) = \begin{cases} \frac{\operatorname{B}[\gamma_0,\gamma_1](y)}{\prescript{}{1}{\lambda}_\mathrm{ub}-\prescript{}{1}{\lambda}_\mathrm{lb}} & \prescript{}{1}{\lambda}_\mathrm{ub} < y < \prescript{}{1}{\lambda}_\mathrm{lb} \\ 0 & \text{else}\end{cases}, \\
\qquad \prescript{}{1}{\lambda}_s = \frac{\lambda_s}{\prescript{1}{0}{\lambda}(\theta)}, \qquad y = \frac{\prescript{}{1}{\lambda}_s - \prescript{}{1}{\lambda}_\mathrm{lb}}{\prescript{}{1}{\lambda}_\mathrm{ub}-\prescript{}{1}{\lambda}_\mathrm{lb}}.
\end{gathered}
\end{equation}
where $\prescript{1}{0}{\lambda(\theta_0)}$ is defined in equation \ref{eq:pfODF}, $(\prescript{}{1}{\lambda}_\mathrm{lb},\prescript{}{1}{\lambda}_\mathrm{ub})$ are the new bounds of the distribution after being convected, and $(\gamma_0,\gamma_1)$ are the shape parameters of the beta distribution function $B$ .
The model form for collagen fibers in exogenously crosslinked BP after being convected by a change in geometry is previously presented in Sacks et al. \cite{sacks_novel_2015} and can also be used for the convection of the collagen fiber architecture due to the permanent set effect. 
The model form for the fiber ensemble interactions is given by equation \ref{eq:interaction}, where equations \ref{eq:pfODF} and \ref{eq:pfrecruitment} are substituted in for the ODF and recruitment distribution function for the mechanical response after permanent set.


\begin{figure}[hbt]
\centering
\centerline{\includegraphics[width=\textwidth]{Images/chapter4/figure10}}
\caption{The effects of convecting the collagen fiber architecture show A) gradual alignment of collagen fibers, B) changes in the probability distribution of collagen fiber lack stretches and C) its effect on the mechanical response.}
\label{fig:effectsofconvection}
\end{figure}



%%%%%%%%%%%%%%%%%%%%%%%%%%%%%%%%%%%%%%%%%%%%%%%%%%%%%%%%%%%%%%%%%%%%%%%%%%%%%%%%
%%%%    Further considerations

\subsection{Further considerations of the permanent set mechanism.}

	In addition to the above considerations, we explored the the following as a means to validate the mechanisms we proposed for permanent set. This was done by performing an analysis on predicting the mechanical response after cyclic loading using only the convection of the collagen fiber architecture by the measured $\textbf{F}_{PS}$. The analysis takes the following steps: 1) Material model parameter estimation for the 0-cycle data for the strain controlled data. This is done using the approach from Sacks \textit{et al}.\cite{sacks_novel_2015}, with the constitutive model for collagen fibers and the EXL matrix from the same study, and the interaction model from equation \ref{eq:interaction}. 
	2) Use equations \ref{eq:pfODF} and \ref{eq:pfrecruitment} (Sec.\ref{sec:convection}) to convect the collagen fiber architecture (Fig. \ref{fig:structuralconvection} and Fig. \ref{fig:effectsofconvection}). 
	The deformation used to convect the collagen fiber architecture is determined from the fiducial markers. 
	3) We compare the new convected exogenously crosslinked mechanical response to the experimental data.

%\subsection{Mechanical response predicted from convected collagen fiber architecture vs experimental data after cycling}

	We used this approach to examine both the strain (n = 3) and stress controlled specimens (n = 5), and tested the hypothesis that \emph{there is a significant different between the mechanical response predicted from structural convection and the experimentally measure data}, in other words, our model is not sufficient to explain the change in mechanical response. We take the maximum stress of each specimen under equibiaxial strain, and compared the difference between the model and the experimental data using student t-test.	For the strain control specimens, we found that there is no statistical significant difference between the convected mechanical response and the experimental data, where the average p-value of the PD and XD for both after 30 million cycle and after 65 million cycle is $p = 0.37$ (minimum $p > 0.07$). 
	Likewise, we also found no statistical significant differences between the convected response and the experimental data for the stress controlled specimens, where the average p-value is 0.57 (Fig. \ref{fig:mechconvec}).
	These results suggest that 1) the underlying collagen fiber architecture, including collagen fiber crimp and orientation, is convected affinely according to the permanent set in the EXL matrix. This indicates that our approach to model permanent set is realistic, and the underlying mechanism for permanent set is likely to be correct. 
	2) Structural damage to the collagen fiber architecture was not detectable at this stage (up to 65 million cycles). 
	These important results indicate that permanent set alone is sufficient to capture the response to cyclic loading up to 50-65 million cycles. 


\begin{figure}[hbt]
\centering
\includegraphics[width=3.5in]{Images/chapter4/figure11}
\caption{The equibiaxial mechanical response of an exogenously crosslinked BP specimen after 50 million cycles and the mechanical response determined from structural convection (Red) using the measured $\textbf{F}_{PS}$}
\label{fig:mechconvec}
\end{figure}


%%%%%%%%%%%%%%%%%%%%%%%%%%%%%%%%%%%%%%%%%%%%%%%%%%%%%%%%%%%%%%%%%%%%%%%%%%%%%%%%
%%%%    Full model form

\subsection{Full model form}
	Combining all three components, we have the final model form as a function of the permanent set rate constant $k $, the permanent set deformation $\mathbf{F}_\mathrm{PS}$, the strain history $\mathbf{A}(s)$, and the material parameters of the constitutive model in the uncycled state. The input of the model is the applied deformation $\mathbf{C}$ referenced to the current unloaded state $\Omega_\mathrm{PS}$, given by the deformation $\mathbf{F}_\mathrm{PS}$ from $\Omega_0$. The full form is
\begin{equation}\label{eq:fullEXLmodel}
\mathbf{S} = \mathbf{S}\left(k , \mathbf{F}_\mathrm{PS}, \mathbf{A}(\hat{s}), \mathbf{C}\right) = \phi_\mathrm{col} \left[ \mathbf{S}_\mathrm{col} + \mathbf{S}_\mathrm{int}\right] + \phi_m \mathbf{S}_\mathrm{m},
\end{equation}
where the collagen contribution is 
\begin{equation} \label{eq:fullcollagen}
\begin{split}
\phi_\mathrm{col}\mathbf{S}_\mathrm{col}&\left(k , \mathbf{F}_\mathrm{PS}, \mathbf{A}(\hat{s}), \mathbf{C}\right) \\
&= \phi_\mathrm{col} \eta_C \int\displaylimits_\theta \Gamma_1(\mathbf{F}_{\mathrm{PS}}, \theta)\left\lbrace 
\int\displaylimits_1^{\lambda_\theta} \frac{D_1\left( \mathbf{F}_{\mathrm{PS}}, x \right)}{x} \left( \frac{1}{x}- \frac{1}{\lambda_\theta}\right) \mathrm{d}x \right\rbrace \mathbf{n}_\theta\otimes\mathbf{n}_\theta \mathrm{d}\theta,
\end{split}
\end{equation}
where $\lambda_\theta = \sqrt{\mathbf{n}_\theta \cdot \mathbf{C}\mathbf{n}_\theta}$ is the stretch of the fiber ensemble oriented along $\theta$, the fiber ensemble interactions is 
\begin{equation} \label{eq:fullinteractions}
\begin{split}
\phi_\mathrm{int}\mathbf{S}_\mathrm{int}&\left(k , \mathbf{F}_\mathrm{PS}, \mathbf{A}(\hat{s}), \mathbf{C}\right) \\
=& \phi_\mathrm{col} \eta_\mathrm{int} \int\displaylimits_\alpha \int\displaylimits_\beta \Gamma_1 \left(\mathbf{F}_\mathrm{PS}, \alpha \right) \Gamma_1 \left(\mathbf{F}_\mathrm{PS},  \beta \right) \\
&\times\left[ \left\lbrace 
\int\displaylimits_1^{\lambda_\alpha} \int\displaylimits_1^{\lambda_\beta} 
\frac{2 \lambda_\beta D_1(\mathbf{F}_\mathrm{PS}, x_\alpha) D_1(\mathbf{F}_\mathrm{PS}, x_\beta)}{x_\alpha x_\beta} 
\left( \frac{\lambda_\alpha}{x_\alpha} \frac{\lambda_\beta}{x_\beta} - 1\right) \mathrm{d}x_\alpha \, \mathrm{d}x_\beta \right.\right. \\
&+ \left. \left. \int\displaylimits_1^{\lambda_\beta} D_1(\mathbf{F}_\mathrm{PS}, x_\beta) \left( \frac{\lambda_\beta}{x_\beta} -1  \right)^2 \mathrm{d}x_\beta \right\rbrace \right.  \frac{\mathbf{n}_\alpha \otimes \mathbf{n}_\alpha}{\lambda_\alpha}  \\
&+ \left. \left\lbrace
\int\displaylimits_1^{\lambda_\alpha} \int\displaylimits_1^{\lambda_\alpha} 
\frac{2 \lambda_\beta D_1(\mathbf{F}_\mathrm{PS}, x_\alpha) D_1(\mathbf{F}_\mathrm{PS}, x_\beta)}{x_\alpha x_\beta} 
\left( \frac{\lambda_\alpha}{x_\alpha} \frac{\lambda_\beta}{x_\beta} - 1\right) \mathrm{d}x_\alpha \, \mathrm{d}x_\beta 
\right. \right. \\
&+\left. \left. \int\displaylimits_1^{\lambda_\alpha} D_1(\mathbf{F}_\mathrm{PS}, x_\alpha) \left( \frac{\lambda_\alpha}{x_\alpha} -1  \right)^2 \mathrm{d}x_\alpha \right\rbrace \frac{\mathbf{n}_\beta \otimes \mathbf{n}_\beta}{\lambda_\beta}  \right] \mathrm{d}\alpha \, \mathrm{d}\beta,
\end{split}
\end{equation}
and the EXL matrix is
\begin{equation} \label{eq:fullmatrix}
\begin{split}
\phi_m \mathbf{S}_\mathrm{m}&\left(k , \mathbf{F}_\mathrm{PS}, \mathbf{A}(\hat{s}), \mathbf{C}\right) \\
&= \phi_m \eta_m \left[ \vphantom{\int\displaylimits_0^s} \mathrm{Exp}\left[-k  \cdot s\right]  \left(\left( \bar{I_1} (\mathbf{F}_\mathrm{PS}, \mathbf{A}(0)) - 3\right)^{\alpha - 1} + r \left( \bar{I_1} (\mathbf{F}_\mathrm{PS}, \mathbf{A}(0)) - 3\right)^{\beta - 1}\right)  \right.\\
&\times \left( \mathbf{\tilde{B}}(\mathbf{F}_\mathrm{PS}, \mathbf{A}(0))^{-1} - \tilde{B}_{33}^{-1}(\mathbf{F}_\mathrm{PS}, \mathbf{A}(0))C_{33}\mathbf{C}^{-1}\right) \\
&+ \int\displaylimits_0^s k \cdot \mathrm{Exp}\left[-k (s - \hat{s})\right] \left(\left( \bar{I_1} (\mathbf{F}_\mathrm{PS}, \mathbf{A}(\hat{s})) - 3\right)^{\alpha - 1} + r \left( \bar{I_1} (\mathbf{F}_\mathrm{PS}, \mathbf{A}(\hat{s})) - 3\right)^{\beta - 1}\right) \\
&\times \left. \vphantom{\int\displaylimits_-^s} \left( \mathbf{\tilde{B}}(\mathbf{F}_\mathrm{PS}, \mathbf{A}(\hat{s}))^{-1} - \tilde{B}_{33}^{-1}(\mathbf{F}_\mathrm{PS}, \mathbf{A}(\hat{s}))C_{33}\mathbf{C}^{-1}\right) \mathrm{d}\hat{s}\right].
\end{split}
\end{equation}

%%%%%%%%%%%%%%%%%%%%%%%%%%%%%%%%%%%%%%%%%%%%%%%%%%%%%%%%%%%%%%%%%%%%%%%%%%%%%%%%
%%%%    Finding permanent set deformations

\subsection{Determining permanent set deformation and loaded state}
To solve for the permanent set deformation ($\mathbf{F}_\mathrm{PS}$), and for the deformation ($\mathbf{A}(s)$) to the new loaded state after permanent set ($\Omega(s)$) when under an applied stress of $\mathbf{\hat{S}}$, we need to use optimization as the permanent set constitutive model has no analytical inverse form. Specifically,
\begin{equation}\label{eq:optimization}
\begin{gathered}
\mathbf{F}_\mathrm{PS} = \operatorname*{arg\,min}_\mathbf{F} \left\Vert \mathbf{S}\left(k , \mathbf{I}, \mathbf{A}(\hat{s}), \mathbf{C}=\mathbf{F}^\mathsf{T}\mathbf{F}\right) - 0 \right\Vert, \\
\mathbf{A}(s) = \operatorname*{arg\,min}_\mathbf{F} \left\Vert \mathbf{S}\left(k , \mathbf{I}, \mathbf{A}(\hat{s}), \mathbf{C}=\mathbf{F}^\mathsf{T}\mathbf{F}\right) - \mathbf{\hat{S}} \right\Vert.
\end{gathered}
\end{equation}

%%%%%%%%%%%%%%%%%%%%%%%%%%%%%%%%%%%%%%%%%%%%%%%%%%%%%%%%%%%%%%%%%%%%%%%%%%%%%%%%
%%%%    Modeling and parameter estimation

\subsection{Modeling approach and parameter estimation}

%%%%%%%%%%%%%%%%%%%%%%%%%%%%%%%%%%%%%%%%%%%%%%%%%%%%%%%%%%%%
%%%%    Strain controlled cycling

\subsubsection{Strain controlled cycling}
	Our goal using the strain controlled data is to validate the constitutive model form and perform parameter estimation. Since the loaded state never changes, the permanent set model can be simplified into a two-state model, with the loaded state ($\Omega(s)$) being the root mean square strain. Specifically, the EXL matrix can be simplified to
\begin{equation} 
\begin{split}
\phi_m \mathbf{S}_m &= \phi_m \left[ \mathrm{Exp}\left[-k  \cdot s\right] \mathbf{\bar{S}}_m \left(\mathbf{F}_\mathrm{PS}, \mathbf{I},\mathbf{C}\right) + \int\displaylimits_0^s k \cdot  \mathrm{Exp}\left[-k (s - \hat{s})\right] \mathbf{\bar{S}}_m \left(\mathbf{F}_\mathrm{PS}, \mathbf{A},\mathbf{C}\right) \mathrm{d}\hat{s} \right] \\
&= \phi_m \left[ \mathrm{Exp}\left[-k  \cdot s\right] \mathbf{\bar{S}}_m \left(\mathbf{F}_\mathrm{PS}, \mathbf{I},\mathbf{C}\right) + \left(1 - \mathrm{Exp}\left[-k  \cdot s\right] \right) \mathbf{\bar{S}}_m \left(\mathbf{F}_\mathrm{PS}, \mathbf{A},\mathbf{C}\right) \right].
\end{split}
\end{equation}
Of the 3 time points (0, 30, and 65 million cycles), we fit the first two time points (0 and 30 million cycles) and use the results to predict the response to cycling at 65 million cycles as a way to validate our model (Fig. \ref{fig:datamethods}A). We note that the mounted configuration of the specimen for mechanical testing and cyclic loading are not the same, thus there is a small rigid body rotation of the specimen between the two testing configurations. We will compensate for this during the parameter estimation, which is done as followed
\begin{enumerate}
\item Determine the 0-cycled mechanical response using the methods of Sacks \textit{et al}.\cite{sacks_novel_2015}
\item Fit the permanent set deformation $\mathbf{F}_\mathrm{PS}$ and the mechanical response at the same time for the 30 million cycles time point
	\begin{enumerate}
	\item choose a $k $
	\item choose a mounting direction $\theta_\mathrm{mount}$
	\item compute $err_\mathrm{PS} = \mathbf{F}_\mathrm{PS} - \mathbf{F}_\mathrm{PS}^\mathrm{data}$ error at 30 million cycles
	\item compute $err_\mathrm{\mathbf{S}} = \mathbf{S}^\mathrm{max} - \mathbf{S}_\mathrm{data}^\mathrm{max}$
	\item compute the weighted error $err_\mathrm{PS} + W_\mathbf{S} err_\mathbf{S}$, where the weight $W_\mathbf{S} = $max strain in the direction of loading/$\mathbf{S}_\mathrm{data}^\mathrm{max}$
	\item update $k $ and $\theta_\mathrm{mount}$ using the Quasi-Newton method \cite{king_dlib_2009}
	\end{enumerate}
\item Predict $\mathbf{F}_\mathrm{PS}$ and the mechanical response at 65 million cycles
\end{enumerate}


\begin{figure}[hbt]
\centering
\centerline{\includegraphics[width=\textwidth]{Images/chapter4/figure12}}
\caption{Our parameter estimation approach for the A) strain controlled cyclic loading data, B) stress controlled data cycled in the cross-preferred direction and C) in the preferred direction}
\label{fig:datamethods}
\end{figure}


%%%%%%%%%%%%%%%%%%%%%%%%%%%%%%%%%%%%%%%%%%%%%%%%%%%%%%%%%%%%
%%%%    Stress controlled cycling

\subsubsection{Stress controlled cycling}
The parameter estimation for the stress controlled specimens is more complicated than the strain controlled specimens as we do not know the strain history \textit{a priori}. This becomes a dynamic simulation, and we need to use optimization to determine the strain history (Eqn. \ref{eq:optimization}) at each time point. Thus, this data set is well suited to validate the full model using time dependent simulations. Since we have data for both PD-loading and XD-loading, we can fit the XD-loading data and used the resulting rate constant $k $ to predict the PD-loading data. First, we discretized the problem as followed
\begin{equation} 
\begin{aligned}
\phi_m \mathbf{S}_m =& \phi_m \left[\mathrm{Exp} \left[-k  \cdot n \cdot \Delta s \right] \mathbf{\bar{S}}_m \left(\mathbf{F}_\mathrm{PS}, \mathbf{I},\mathbf{C}\right)\vphantom{\sum_{i = 1}^n}\right. \\
&+ \left.\sum_{i = 1}^n  (k \Delta s) \mathrm{Exp}\left[-k (n\Delta s - i \Delta s)\right] \mathbf{\bar{S}}_m \left(\mathbf{F}_\mathrm{PS}, \mathbf{A}(i\Delta s),\mathbf{C}\right)\right].
\end{aligned}
\end{equation}
After each time step $\Delta s$, we compute the new loaded state $\mathbf{A}(i\Delta s)$ using optimization(Eqn. \ref{eq:optimization}, Fig. \ref{fig:implementation}). Through preliminary trials, the most optimal resolution in time is $\Delta s = 1$ million cycles when considering both time to run the simulations and accuracy of the results. We note that since an additional optimization is added to the parameter estimation process, we can no longer fit both the permanent set deformation $\mathbf{F}_\mathrm{PS}$ and the mechanical data at the same time. Our attempts at fitting both at the same time were not able to converge. Thus, we choose to predict the mechanical data as a way to validate our results. 
The parameter estimation process for the XD-loading data is
\begin{enumerate}
\item Determine the 0-cycled mechanical response
\item Fit the permanent set deformation $\mathbf{F}_\mathrm{PS}$ 
	\begin{enumerate}
	\item choose a $k $
	\item choose a mounting direction $\theta_\mathrm{mount}^{20}$ for cycling up to 20 million cycles 
	\item compute $err_\mathrm{PS} = \mathbf{F}_\mathrm{PS} - \mathbf{F}_\mathrm{PS}^\mathrm{data}$ error at 20 million cycles
	\item choose a mounting direction $\theta_\mathrm{mount}^{50}$ for cycling from 20 to 50 million cycles 
	\item compute $err_\mathrm{PS} = \mathbf{F}_\mathrm{PS} - \mathbf{F}_\mathrm{PS}^\mathrm{data}$ error at 50 million cycles
	\item update $k$, $\theta_\mathrm{mount}^{20}$, and $\theta_\mathrm{mount}^{50}$ using Quasi-Newton
	\end{enumerate}
\item Predict the mechanical response 
\end{enumerate}
Next we, used $k $ from the XD-loading data to predict the PD-loading data. 
\begin{enumerate}
\item Determine the 0-cycled mechanical response
\item Compute Fit the permanent set deformation $\mathbf{F}_\mathrm{PS}$
	\begin{enumerate}
	\item set $k $ from XD-loading data
	\item choose a mounting direction $\theta_\mathrm{mount}^{20}$ for cycling up to 20 million cycles 
	\item compute $err_\mathrm{PS} = \mathbf{F}_\mathrm{PS} - \mathbf{F}_\mathrm{PS}^\mathrm{data}$ error at 20 million cycles
	\item choose a mounting direction $\theta_\mathrm{mount}^{50}$ for cycling from 20 to 50 million cycles 
	\item compute $err_\mathrm{PS} = \mathbf{F}_\mathrm{PS} - \mathbf{F}_\mathrm{PS}^\mathrm{data}$ error at 50 million cycles
	\item update $\theta_\mathrm{mount}^{20}$ and $\theta_\mathrm{mount}^{50}$ using Quasi-Newton
	\end{enumerate}
\item Predict the mechanical response 
\end{enumerate}

%%%%%%%%%%%%%%%%%%%%%%%%%%%%%%%%%%%%%%%%%%%%%%%%%%%%%%%%%%%%%%%%%%%%%%%%%%%%%%%%
%%%%    Parametric studies


\begin{figure}[hbt]
\centering
\includegraphics[width=5in]{Images/chapter4/figure13}
\caption{Implementation of the full model with updates in time.}
\label{fig:implementation}
\end{figure}


\subsection{Parametric studies}
Next, we performed an parametric study using the stress-controlled PD data. The same material parameters and rate constant, $k$, from parameter estimation results above was used. We simulated one specimen by extending the cycle duration to 100 million cycles to examine how the changes in geometry due to permanent set respond to an extended cycling period. 

%---    primary results
%%%%%%%%%%%%%%%%%%%%%%%%%%%%%%%%%%%%%%%%%%%%%%%%%%%%%%%%%%%%%%%%%%%%%%%%%%%%%%%%
%%  Results
%%%%%%%%%%%%%%%%%%%%%%%%%%%%%%%%%%%%%%%%%%%%%%%%%%%%%%%%%%%%%%%%%%%%%%%%%%%%%%%%


\section{Permanent set model results}
\subsection{Model fit and predictive capabilities}


	For the strain control specimens, we found that we were able to fit the permanent set deformations very well ($R^2 = 0.96$) (Fig. \ref{fig:strainresults}A). We were able to predict the mechanical response of the strain controlled specimens at 65 million cycles ($R^2 = 0.83$) (Fig. \ref{fig:strainresults}B\&C). 
	For the stress controlled specimens, we found we were able to fit the permanent set deformations for the XD specimens ($R^2 = 0.93$)(Fig. \ref{fig:stressXDdef}B), as well as predict the PD specimens very well ($R^2 = 0.97$) (Fig. \ref{fig:stressXDdef}C). 
	The resulting rate constant from both data sets shown no statistical difference ($p > 0.98$)(Fig. \ref{fig:stressXDdef}A). 
The mechanical response for the stress controlled XD specimens did not match as well in terms of the $R^2$ value($R^2 = 0.72$). 
	However, given none of the mechanical data was involved in the parameter estimate this was nevertheless a very good prediction. 
	For example, when comparing the model to the experimental data by extrapolating the loading path of the equibiaxial protocol and finding the peak strain at 1MPa, the $R^2$ value increases to 0.93. 
	One the other hand, the predicted mechanical response for the PD controlled data were very good ($R^2 = 0.95$) (Fig. \ref{fig:stressPDmech}), suggesting that our model was able to capture the underlying mechanisms. 
	These results agree with our hypothesis that the initial changes in the mechanical response (in the first 50 million cycles) can be predicted by the change in collagen fiber architecture alone, and that structural damage is low at this stage. 


\begin{figure}[hbt]
\centering
\centerline{\includegraphics[width=\textwidth]{Images/chapter4/figure14}}
\caption{Results of the strain controlled cycling data, shown how the A) model fits the permanent set deformation at 30 million cycles and predicts the 65 million cycles time point (hollow points). The dotted line shows the model prediction with $0.5k$ and the dashed line shows the model prediction with $2k$ and C) Shows how the model predicts the mechanical response at 65 million cycles using material parameters from the B) 0 cycle time point.}
\label{fig:strainresults}
\end{figure}

\begin{figure}[hbt]
\centering
\centerline{\includegraphics[width=\textwidth]{Images/chapter4/figure15}}
\caption{) Comparison of the permanent set rate constant between the strain controlled and stress controlled specimens. B) The model fit for the permanent set deformation at both 20 and 50 million cycles for the stress controlled XD cycled specimens.  C) The predicted permanent set deformation for the PD cycled specimens using the rate constant from fitting the XD cycled specimens. }
\label{fig:stressXDdef}
\end{figure}

\begin{figure}[hbt]
\centering
\centerline{\includegraphics[width=\textwidth]{Images/chapter4/figure16}}
\caption{A) Best fit of the mechanical response at 0 cycle for the material parameters ($r^2 = 0.98$). The predicted mechanical response for the representative PD cycled specimen at B) 20 ($r^2 = 0.87$) and C) 50 million cycles. ($r^2 = 0.82$)}
\label{fig:stressPDmech}
\end{figure}


\subsection{Parametric study results}
	By extending the cycling duration in the parametric study, we found that the permanent set deformation reaches an assymptote after approximately 70 million cycles when loading along the PD. 
	This threshold slightly exceeds the lower bound for collagen recruitment. 
	We estimate that around 2.6\% of collagen fiber are recruited when the permanent set deformation reaches this threshold (Fig. \ref{fig:parametric}). This suggests that collagen fibers are limiting the maximum change in geometry that can occur, and that the lower bound of the collagen fiber recruitment can serve as an estimated bound for the changes in geometry due to the permanent set effect in BHVs. Once this bound is exceed, some collagen fibers can exist perennially in an extended state. This could be a potential mechanism in exacerbating the rate of damage to the collagen fiber architecture. This can have significant implications in optimizing BHV design. By optimizing the BHV geometry for the post permanent set state, we can minimize the stresses in for the BHV for the majority of the BHV life span, minimizing structural damage and potentially increasing BHV durability. 


\begin{figure}[hbt]
\centering
\includegraphics[width=4in]{Images/chapter4/figure17}
\caption{The results of the parametric study. The red solid line shows the lower bound for the collagen fiber slack stretches and the red dashed line show the approximate cycle when the permanent set stretches reach an assymptote.}
\label{fig:parametric}
\end{figure}

%---    discussion
%%%%%%%%%%%%%%%%%%%%%%%%%%%%%%%%%%%%%%%%%%%%%%%%%%%%%%%%%%%%%%%%%%%%%%%%%%%%%%%%
%%  Discussions
%%%%%%%%%%%%%%%%%%%%%%%%%%%%%%%%%%%%%%%%%%%%%%%%%%%%%%%%%%%%%%%%%%%%%%%%%%%%%%%%


\section{Discussion}

\subsection{Permanent set is sufficient to describe early stages of BHV cycling}
	The most important result from this study is that a permanent set mechanism in the EXL matrix alone is sufficient to explain the responses due to cyclic loading in the range of 0 to 65 million cycles. 
	This further suggests that there is no detectable damage to the collagen fiber architecture, and that the overall collagen fiber architecture stays intact and convected under affine kinematics. 
	This is not unexpected, as we previously found dense collagenous tissues to behave affinely when deforming in the physiological range \cite{lee_presence_2015}. 
	Although the permanent set effect is very noticeable in the early stages of the cyclic loading, it still takes millions of cycles; in other words, years. 
	Due to the time scale difference between the opening and closing of heart valves versus permanent set, BHVs always deforms quasi-statically, which means it always follows affine kinematics. 
	It follows that any structural changes in the collagen fiber architecture is affine as well. 
	This is all extremely important, as this allows us to completely separate permanent set from other cyclic loading effects and independently determine the permanent set rate constant just from the cyclic loading data in the early stage. 

\subsection{Lack of detectable structural damage}
	We have observed that there are molecular conformation changes in the collagen fiber during this early stage \cite{sun_response_2004, sellaro_effects_2007}. 	This suggests that while effects of collagen fiber damage are not detectable at the bulk level, it remains an ongoing but much slower process in comparison to permanent set. 
	Significant tearing and delamination have been observed after 500 million cycles \cite{sacks_effects_1998}, but this corresponds to the late stage (Fig. \ref{fig:hypothesis}), for which we do not have extant mechanical data. 
	There are no other existing experimental data quantifying the intrinsic structural damage in BHVs during cyclic loading. 
	This is not surprising as actual structural damage is difficult to distinguish from other processes such as permanent set. 
	The relation between molecular changes and mechanical response is not well understood.
	The most promising way of quantifying structural damage is through constitutive modeling and simulations. 
	Structural models can separate structural damage and permanent set, but we do not have sufficient data at high cycle numbers where structural damage is detectable. This remains an important extension for the model in the future. 

\subsection{Permanent set is driven by the scission-healing of the EXL matrix}
	One important assumption in our model is that permanent set only occurs in the EXL matrix due to scission-healing. 
	Based on our theory for permanent set, the process is driven entirely by the first order kinetics of the crosslinking reactions of GLUT leading to scission-healing, as well as the kinematics involved in the reference state evolution. 
	Our results indicate that these mechanisms and permanent set can indeed explain the response to cyclic loading.
	This highlights the importance of understanding the effects of GLUT crosslinks and their role in the cyclic loading response of BHVs. 
	The use of GLUT was originally intended for suppressing immunogenicity by crosslinking antigen within BP xenographs, but also has the fortunate consequence of stiffening the mechanical response of the BHV. 
	Unfortunately, the scission-healing behavior of GLUT also plays a major role in the cyclic loading of BHVs. 
	By severely changing the geometry of the BHV in the first 1-2 years, it strongly influences the cyclic loading response of BHVs at latter stages. 
	It may be possible to design BHVs to accommodate permanent set deformations caused by GLUT's scission-healing. 
	Alternative exogenous crosslinking chemistry \cite{tam_fixation_2017, tam_novel_2015} may be an important area for technological advancement of BHVs. 
	Protecting the optimal mechanical response of the BHVs by reducing the impact of permanent set can significantly limit the peak stress on BHVs and protect the underlying tissue microstructure. 

\subsection{Collagen fiber recruitment can limit the maximum change in geometry due to permanent set}
	One of the most important finding from our permanent set model is that collagen fibers may play a significant role in limiting the changes in BHV geometry due to the permanent set effect. Our parametric study results shows that the permanent set deformation eventually reaches a threshold asymptotically (Fig. \ref{fig:parametric}). This is due to the different parts of the EXL matrix as well as the collagen fibers separating into different reference configurations due to the permanent set effect. Although the bulk tissue as a whole is at a stress free equilibrium, there exists some internal stress between the different parts of the EXL matrix and the collagen fibers as a result of the different reference states. If the changes in geometry is sufficiently large, some collagen fibers will be recruited and exert compressive stress on the EXL matrix. However, since the stiffness of collagen fibers is over three magnitudes higher than the EXL matrix, the deformation of collagen fibers due to the internal stress is insignificant. As such, the recruitment of collagen fibers can resist against further changes in the geometry of the tissue due to permanent set. In addition, we generally found significant collagen fiber structural reserve in collagenous tissue \cite{zhang_meso_2016}. From the parametric study, no more than 2-4\% of collagen fibers are straightened under physiological loading levels (up to 1 MPa) due to permanent set. Coupled with the exponentially increased cumulative stiffness of the collagen fibers with strain, significant structural reserved implies that loading stresses several orders of magnitudes higher than the physiological loading level is necessary to further deform the collagen fibers. Also, taking into account the rapid recruitment of native collagen fibers (collagen fibers do not extend by more than 5-8\% strain before breaking \cite{buehler_atomistic_2006}), the collagen fiber architecture serves as a barrier in limiting the changes in geometry due to permanent set. This potentially gives us a way of predicting the final stress free BHV geometry after permanent set has largely ceased. This can have significant impact on the design BHVs as we can optimize the BHV geometry based on these results to minimize the leaflet stresses. Since the permanent effect is most significant during the first 40 to 50 million cycles (Fig. \ref{fig:parametric}), which approximates to the first 8-9 month after implant, BHVs will operate in the post permanent set geometry during the majority of its life span (Fig. \ref{fig:hypothesis}, when structural damage the accumulation of structural damage is most significant. Thus predicting the post permanent set geometry can have significant implications on the durability of BHV designs.
	
\subsection{Effects on the durability of BHVs}
    Our findings suggests that this that the permanent set mechanism has great potential in predicting the non-biological driven changes in the BHVs in the first 2-5 years. As such, the permanent set mechanism can greatly aid the use of computational simulations in exploring BHV designs and the impact of permanent set at the valve-level. Specifically, computational models can be used to adjust the initial geometry to tailor for optimal stress distribution in the valve leaflets after permanent set. Thus, the permanent set mechanism can help us explain why stress concentrations develop, which will accelerate structural damage, and help to create designs which can mitigate this effect. High stress regions have been linked to regions with high structural damage which have significant impact on the long-term behavior of BHVs. By quantifying factors such as how the peak stress and stress distribution change due to permanent, we can predict the increased likelihood of structural damage in the leaflet due to permanent set and how it impact BHV durability.
    

\subsection{Limitations} 
	Our available cyclic loading data for exogenously crosslinked BP is limited in terms of cycled duration, multitude of strain levels, the rate of cyclic loading and number of specimens. 
	However, it is important to note that our goal was to develop the constitutive model form for exogenously crosslinked tissue under cyclic loading, not to obtain a population of material parameters for statistical testing and simulations. 
	The extant experimental data we used for parameter estimation was sufficient for this purpose, which have specimens under both constant strain level and time evolving strain level, and with loading along the preferred collagen fiber direction and more rigorous loading orthogonal to preferred collagen fiber direction.
	We note that some modification may be necessary, such as extending the model for the rate constant as a function of strain level and rate of cycling. 
	However, the permanent set mechanism described was able to explain the effects of cyclic loading in the early stage. 
	The lack of cyclic loading data beyond 65 million cycles, especially up to 100 or 200 million cycles, means that we are not able to incorporate structural damage into our constitutive model at this point. 
	Although, based on what we observed, structural damage does not appear to play an observable ~65 million cycles. 

%---    discussion
%%%%%%%%%%%%%%%%%%%%%%%%%%%%%%%%%%%%%%%%%%%%%%%%%%%%%%%%%%%%%%%%%%%%%%%%%%%%%%%%
%%  Conclusions
%%%%%%%%%%%%%%%%%%%%%%%%%%%%%%%%%%%%%%%%%%%%%%%%%%%%%%%%%%%%%%%%%%%%%%%%%%%%%%%%


\section{Summary and Future Directions}
	We have developed the first structural-based constitutive model for the time evolving properties of exogenously crosslinked collagenous soft tissues under cyclic loading. We focused on permanent set as the mechanism for the geometry changes in the early stage of cycling and developed our constitutive model based on the underlying scission-healing reaction of the GLUT crosslinked matrix. Permanent set allows the reference configuration of the exogenously crosslinked matrix to evolve over time and convect the collagen fiber architecture through the change in geometry. The results show that permanent set alone is sufficient to explain all changes in the early stage of BHV cycling, and more importantly predict how the shape and reference configuration evolve during this stage. 	Moreover, structural damage does not play a detectable role up to 65 million cycles. Our model also indicates that the collagen fiber architecture can play a role in limiting the permanent set effect, where the straightening of collagen fibers prevents further changes in geometry. Thus, accounting for the permanent set effect is especially important in the design of BHVs to better improve their performance and durability. 
	
	In addition to the exogenously crosslinked tissue applications addressed herein, we have observed permanent set like phenomenon in mitral valve tissue during pregnancy \cite{rego_mitral_2016}. In that study, our results suggested that much of the growth and remodeling in the MV leaflet does not begin immediately, but rather undergoes mostly passive leaflet enlargement until these parameters reach a critically low level, at which point growth and remodeling are triggered. This initial tissue distension process is very similar in behavior to the permanent set mechanism outlined in the present work. Thus, the current approach could be applied to these types of the early phases soft tissue remodeling, where non-failure mechanisms occur before the onset of growth of tissue growth and remodeling. In addition, although the permanent set model we described only include the remodeling of the matrix due to scission-healing, the same concept can be extended by separating the rate constant into growth and resorption to simulate growth and remodeling of the matrix. Furthermore, the frame work outlined in section 5 can also be extended for the remodeling of the collagen fiber architecture, given further studies on mechanisms for how the collagen fiber architecture grows once the critical level observed in Rego \textit{et al} \cite{rego_mitral_2016} is exceeded. This is the advantage for the structural-based approach to modeling permanent set, which allows us to describe the mechanical response based on real physically measure-able quantities. We can further extend the more toward effects such as structural damage are the fiber-level, proteolytic degradation, and growth based on how this effects affect the components of the permanent set model layed out herein.

\newpage
%%%%%%%%%%%%%%%%%%%%%%%%%%%%%%%%%%%%%%%%%%%%%%%%%%%%%%%%%%%%%
%%  nomenclature											%
%%%%%%%%%%%%%%%%%%%%%%%%%%%%%%%%%%%%%%%%%%%%%%%%%%%%%%%%%%%%%
\section*{Nomenclature} \label{c4:sec:nomenclature}
\begin{mynom}
\textbf{Key Terms}  \\
{EXL}\>\tabfill{Exogenously crosslinked} \\
{PS}\>\tabfill{Permanent set, an irreversible deformation that remains in a structure or material after it has been subjected to stress.} \\
{Damage}\>\>\tabfill{Loss of mechanical properties} \\
{Fatigue}\>\>\tabfill{Weakening of a material caused by repeated loading} \\
{Plastic deformation}\>\>\>\>\>\tabfill{Deformation of a material undergoing irreversible change in shape in response to applied forces} \\
{Structural convection}\>\>\>\>\>\>\tabfill{A permanent deformation of the collagen fiber architecture based on the change in the reference configuration} \\
{BP}\>\tabfill{Bovine pericardium} \\
{CFA}\>\tabfill{Collagen fiber architecture} \\
{BHV}\>\tabfill{Bioprosthetic heart valve} \\
{AWT}\>\tabfill{Accelerated wear testing} \\
{GLUT}\>\tabfill{Glutaraldehyde} \\
{TVI}\>\tabfill{Transcatheter aortic intervention } \\
{PD}\>\tabfill{Preferred direction} \\
{XD}\>\tabfill{Cross-preferred direction} \\
{ODF}\>\tabfill{Orientation distribution function} \\
{Recruitment}\>\>\>\>\tabfill{Probability distribution function describing the strain at which a collagen fiber's crimp is straightened } \\ 
{Fiber ensemble}\>\>\>\>\tabfill{A group of fibers which share a common orientation} \\
\textbf{Symbols}    \\
{$\Psi$}\>\tabfill{Strain energy} \\
{$D$}\>\tabfill{Collagen fiber recruitment distribution function} \\
{$\Gamma$}\>\tabfill{Collagen fiber orientation distribution function} \\
{$\phi$}\>\tabfill{Mass fraction} \\
{$\eta_C$, $\eta_m$, $\eta_I$}\>\>\>\tabfill{The modulus of collagen, exogenously crosslinked matrix and fiber-fiber interactions} \\
{$\mathbf{I}$}\>\tabfill{Identity tensor} \\
{$\mathbf{F}$}\>\tabfill{An arbitrary deformation gradient tensor applied to the tissue, referenced to the original uncycled stress free state} \\
{$\mathbf{C}$}\>\tabfill{Right Cauchy-Green strain tensor, referenced to the original uncycled stress free state} \\
{$\mathbf{E}$}\>\tabfill{Green Lagrange strain, referenced to the original uncycled stress free state} \\
{$\lambda$}\>\tabfill{Stretch} \\
{$\mathbf{S}$}\>\tabfill{Second Piola Kirchhoff tensor, referenced to the original uncycled stress free state} \\
{$I_1$}\>\tabfill{First invariant} \\
{$I_8$}\>\tabfill{Eighth pseudo invariant} \\
{$k$}\>\tabfill{Rate constant for permanent set} \\
{$s$}\>\tabfill{The current time in seconds } \\
{$\hat{s}$}\>\tabfill{The intermediate time when the exogenously crosslinked matrix is formed by the scission-healing process } \\
{$\Omega_0$}\>\tabfill{The original unloaded configuration of the tissue before any cyclic loading} \\
{$\Omega(s)$}\>\tabfill{The current loaded configuration of the tissue during cyclic loading} \\
{$\mathbf{A}(s)$}\>\tabfill{Strain history, which is the root mean square strain of each cycle as a function of time} \\
{$\mathbf{\tilde{B}}(s) = \mathbf{A}\mathbf{A}^\mathsf{T}$}\>\>\>\tabfill{Left Cauchy Green tensor of the strain history} \\
{$\mathbf{\bar{F}}(\hat{s})$}\>\tabfill{Deformation gradient tensor applied to the tissue, referenced to the strain history $\mathbf{A}(\hat{s})$ at time $\hat{s}$} \\
{$\mathbf{\bar{C}}(\hat{s})$}\>\tabfill{The right Cauchy Green strain tensor applied to the tissue, referenced to the strain history $\mathbf{A}(\hat{s})$ at time $\hat{s}$} \\
{$\bar{I}_1$}\>\tabfill{Modified first invariant, referenced to the strain history $\mathbf{A}(\hat{s})$ at time $\hat{s}$} \\
{$\Omega_\mathrm{PS}(s)$}\>\tabfill{The current unloaded configuration due to changes in geometry caused by permanent set} \\
{$\mathbf{F}_\mathrm{PS}(s)$}\>\tabfill{The deformation from $\Omega_0$ to the current unloaded configuration ($\Omega_\mathrm{PS}$) due to permanent set}
\end{mynom}

\newpage
%---    Bioliography
\bibliographystyle{plainnat}
\bibliography{phd}



\chapter{Effective model representation of structural and multi-scale models using effective constitutive models for facilitating numerical simulations}

\section*{Preface}
\addcontentsline{toc}{section}{Preface}%

One of the most crucial aspects of biomechanical simulations of organs and systems that seek to predict the outcomes of disease, injury, and surgical interventions is the underlying constitutive model. Current soft tissue constitutive modeling approaches have become increasingly complex, often utilizing meso- and multi-scale methods for greater predictive capability and linking to the underlying mechanisms. However, such modeling approaches are associated with substantial computational costs. One solution is to use effective constitutive models, which only reproduces the essential responses but not the underlying mechanisms. Effective constitutive models can be implemented in place of meso- and multi-scale models in numerical simulations, but derive their responses by homogenizing the responses of the underlying meso- or multi-scale models. A robust effective constitutive model can thus drastically increase the speed of simulations for a wide range of meso- and multi-scale models. However, there is no general consensus on how to develop a single effective constitutive model for a wide range of soft tissue responses. In the present study, we developed an effective constitutive model, which can fully reproduce the response of a wide range of planar soft tissue responses, along with methods for fast-convergent parameter estimation. We evaluated this approach and demonstrated that it is able to handle materials of widely varying degrees of anisotropy, such as exogenously cross-linked bovine pericardium and aortic valve leaflet. This effective constitutive model approach has shown significant potential for improving the computational efficiency and numerical robustness of multi-scale and meso-scale modeling approaches, facilitating the application of inverse modeling and simulations of growth and remodeling of soft tissues and organs.

% \textbf{The work contained in this chapter was published as}: Zhang, W., Zakerzadeh, Z., Zhang, W. \& Sacks, M. S.
% A Material Modeling Approach for the Effective Response of Planar Soft Tissues for Efficient Computational Simulations. 
% Journal of the mechanical behavior of biomedical materials. under review.



%---    INTRODUCTION
%%%%%%%%%%%%%%%%%%%%%%%%%%%%%%%%%%%%%%%%%%%%%%%%%%%%%%%%%%%%%
%%  Introduction											%
%%%%%%%%%%%%%%%%%%%%%%%%%%%%%%%%%%%%%%%%%%%%%%%%%%%%%%%%%%%%%

\section{Introduction}

%-------	Computational approaches	-------%
	Computational studies of organs and bioprosthetic devices have become increasingly popular for predicting the outcomes of diseases, injuries, and surgical interventions. Such applications include the simulation of aneurysm growth \cite{rissland_abdominal_2009,ramault_comparison_2011,hoi_effects_2004,volokh_model_2008}, blood flow \cite{olufsen_numerical_2000,perktold_computer_1995,pries_blood_1990,oshima_finite_2001,bagchi_mesoscale_2007}, and natural or bioprosthetic heart valves \cite{zakerzadeh_computational_2017, soares_biomechanical_2016, kamensky_immersogeometric_2015, aggarwal_vivo_2016, nobili_numerical_2008, cheng_three_2004}. One of the most important components of predictive simulations is an accurate constitutive model that can predict the mechanical behavior of the soft tissues and biomaterials involved. Such tissues highly nonlinear anisotropic behaviors, often resulting in specific forms for different tissue types. In addition, these constitutive models are often extended to model growth, remodeling, pathology, fatigue, trauma and other time evolving processes. As such, constitutive models are often developed to take advantage of the structure to function relationship to predict how the response of the materials will evolve, utilizing physical models, multi-scale approaches, and/or molecular dynamics. Not surprisingly, constitutive models are becoming exceeding complex, and the computational costs are becoming a hindrance to more complex numerical simulations. 


%-------	structural modeling	-------%
	Examples of such approaches are meso-scale structural approaches for soft tissue modeling \cite{lanir_constitutive_1983}. This class of soft tissue models homogenizes the tissue response at the meso-scale, where the simplified models of the mechanical response of collagen, elastin and other fibers are integrated with tissue microstructure \cite{kassab_structure_2016}. This type of constitutive model has been shown to be able to accurately represent the mechanical behaviors of many soft tissues including valvular tissues \cite{zhang_meso_2016, rego_mitral_2016}, pericardium \cite{zhang_modeling_2017}, myocardium \cite{avazmohammadi_novel_2017}, and elastomeric scaffolds \cite{d.amore_large_2016}. Recently, we have extended these models to include interaction terms \cite{zhang_modeling_2017, avazmohammadi_novel_2017}, which require multiple integrals to accurately compute the strain energy of fiber interactions. 




%-------	multi-scale modeling	-------%
    In a broader context, multi-scale approaches utilize fundamental mechanisms at the micro-scale to derive the response of materials at the macro-scale. Often, multi-scale modeling begins at the molecular level, where the molecular structure of the constituents and the physical laws governing their interactions are well known and well-studied in chemistry and physics. At this level, molecular dynamics can be used to determine the mechanical response of constituent proteins. This can be upscaled to quaternary protein structures using coarse grain methods. This response is then integrated with higher level structures of the tissue to determine the response at even larger scales. Homogenization is thus critical in multi-scale modeling to simplify the response of the downscale models to improve the efficiency of simulations at higher scales. This process is repeated until the macro- or tissue-level. Examples using this approach are the modeling of collagenous tissues by Buehler \textit{et al.} \cite{buehler_atomistic_2006, buehler_nanomechanics_2008} and intermediate filaments of cells by Qin \textit{et al.} \cite{qin_multi_2010}. 
    
    
    However, numerical simulations using these approaches are also quite costly. They can only be used for simulating the response of the materials, but cannot be easily incorporated into inverse modeling and time-dependent frameworks. Even the computational cost of meso-scale structural approaches is five magnitudes higher than conventional phenomenological approaches. For more detailed cell or molecular level information, which are important for better understanding cellular environments and growth and remodeling, the exceedingly high computational cost of multi-scale approaches makes it difficult for them to be directly implemented in computational simulations. As such, the multi-scale models used in simulations need to be simplified.

%-------	modeling material behavior	-------%
    It is for this reason that many types of phenomenological models with computationally efficient forms, such as the generalized Fung type \cite{fung_biomechanics_1993}, Holzapfel-Gasser-Ogden \cite{holzapfel_new_2000}, generalized Ogden \cite{ogden_large_1972}, generalized Rivlin \cite{rivlin_large_1951}, and Humphrey models \cite{may-newman_constitutive_1998}, are popular for numerical simulations. These models utilize constitutive modeling approaches which do not take into account the underlying mechanisms, thus can only reproduce the mechanical response in the limited range of the experimental data utilized for parameter estimation. Furthermore, the mechanical data used and parameter estimation are not done in an optimal manner. This makes finding the optimal parameters inconsistent due to high covariance between parameters. This is demonstrated in Sun and Sacks \cite{sun_biaxial_2003} for modeling pericardium under high in-plane shear. Here, it was shown that fitting only a subset of the loading paths acquired from biaxial mechanical testing cannot predict the remaining unfitted loading paths. Yet for \textit{in vivo} simulations, these constitutive models are often derived from incomplete data while being asked to predict the mechanical response under non-physiological and often unpredictable ranges of deformations. As the parameters of such models have no physical meaning, it is often difficult to extend them for time dependent processes such as growth and remodeling, or to average and produce a population representative. As such, accurate simulations often still require detailed mechanism based models. 
    

%-------	effective model	-------%
    
    Thus, we hereby develop an effective constitutive model for planar soft tissues, and a parameter estimation approach for rapidly determining the model parameters from the micro-model. Planar constitutive models is a good starting point, whihch are applicable to a wide range of soft tissues such as arteries, skin, heart valve, cells, vocal folds, bladder wall, synovial membrane, cornea, and cranial membrane. For the form of the effective constitutive model, we require the following characteristics:
\begin{enumerate}
    \item Widely applicable in that it is able to faithfully reproduce a wide range of tissue responses
    \item Computationally efficient and numerically robust
    \item Allows for fast and accurate convergence during parameter estimation for upscaling micro-models, thus having the minimal number of and minimally covariant model parameters
    \item Easy to implement, no integrations or functions without closed-form expressions
\end{enumerate}
    Using meso-scale structural models as an example, we will examine the ability of the effective constitutive model to fit the mechanical response of micro-models for a wide range of deformations, examine the speed and convergence of parameter estimation, and demonstrate the use of the effective constitutive model to facilitate the simulation of heart valves with a wide range of material properties.
    
%%%%%%%%%%%%%%%%%%%%%%%%%%%%%%%%%%%%%%%%%%%%%%%%%%%%%%%%%%%%
%-------------------	begin FIGURE 	-------------------%
\begin{figure}
\centering
\includegraphics[width=\textwidth]{Images/chapter5/simulationframework}
\caption{Proposed frame work for using an effective constitutive model to improve the efficiency of using complex meso- or multi-scale models (micro\Hyphdash models) in numerical simulations. Here, A) effective constitutive models act as an intermediate step between micro-models and numerical simulations, where micro-models inform the changes to the effective constitutive model while the effective constitutive model for the simulation. B) An example of how this may be implemented for time\Hyphdash evolving is shown.}
\label{fig:simulationframework}
\end{figure}
%-------------------	 end FIGURE 	-------------------%
%%%%%%%%%%%%%%%%%%%%%%%%%%%%%%%%%%%%%%%%%%%%%%%%%%%%%%%%%%%%
    
    
    
    
    
    
    
    


%---    METHODS
%%%%%%%%%%%%%%%%%%%%%%%%%%%%%%%%%%%%%%%%%%%%%%%%%%%%%%%%%%%%%
%%	Constitutive model form									%
%%%%%%%%%%%%%%%%%%%%%%%%%%%%%%%%%%%%%%%%%%%%%%%%%%%%%%%%%%%%%

\section{Effective constitutive model formulation}

%-----------------------------------------------------------
%	Kinematics
%-----------------------------------------------------------
\subsection{Kinematic considerations}
	
	The choice of kinematic basis is the first step to formulate constitutive models. Very often, it is useful to limit the choice of kinematic basis based on the available mechanical data or how well each basis match the response of the soft tissues, thus simplifying the form of the constitutive model and reducing parameter covariance during parameter estimation. However, for our approach, our aim is a highly generalized constitutive model that can match a wide range of possible micro-model responses using the same form, not restricting itself to the response and physics of specific soft tissue types. There is also no limitations to having sufficient mechanical data for parameter estimation, as this will be generated from the micro-models. As such, our considerations for choosing the kinematic basis are mainly:
\begin{enumerate}
    \item Most generalized form for reproducing a wide range of mechanical responses
    \item Simplest form for implementation and computational cost in numerical simulations
    \item Minimal number of parameters
    \item Minimal parameter covariance for parameter estimation
\end{enumerate}
    The smallest set of kinematic variables that can describe a wide range of soft tissues and deformations using the same simple form is ideal. 

    The invariants and pseudo-invariants of the right or left Cauchy Green tensor is very popular for the constitutive models of soft tissues. Indeed, we use them often with our structural models \cite{fata_insights_2014, zhang_meso_2016, avazmohammadi_novel_2016, sacks_novel_2015, zhang_modeling_2017}. There is a large number of invariants, each describes a facet of deformation: isotropic, volumetric strain, anisotropic, or interactions between them. The breadth of choices allow for more freedom in selecting a best combination when modeling specific soft tissues. However, there are simply too many invariants, and many of which are highly covariant. This does not lend itself for minimizing the number of parameters and the parameter covariance in a single fully generalized form. 
    
    
    It's here that using the components of the strain tensors is more practical for our approach. Although all strains are equivalent for constitutive modeling because they can be expressed with respect to each other, different strain tensors can have different effects when used directly in place of each other in the same form. For us, with the purpose of keeping the constitutive model form simple for implementation, the Green Lagrange strains are the most practical. Based on preliminary testing (Appendix \ref{sec:greenvshencky}), the Green Lagrange strains result in the simplest 2nd Piola Kirchhoff stress and elasticity tensor forms, and have behaviors that closely resemble the response of collagen fibers in soft tissues when under compression. We examined these aspects more closely in Appendix \ref{sec:greenvshencky}. 
    
    It is also convenient to express the Green Lagrange strain tensor with respect to the material axis (Fig. \ref{fig:greenkinematics}), $\mathbf{m}_0$, where
\begin{equation} \label{eqn:greenstrain}
E_m = \mathbf{m}_0\cdot\mathbf{E}\mathbf{m}_0, \quad E_n = \mathbf{n}_0\cdot\mathbf{E}\mathbf{n}_0, \quad E_{\phi} = \mathbf{m}_0\cdot\mathbf{E}\mathbf{n}_0,
\end{equation} 
    and $\mathbf{n}_0$ is the direction orthogonal to $\mathbf{m}_0$ (Fig. \ref{fig:greenkinematics}). This symmetry is helpful for further reducing the constitutive model form. 


%%%%%%%%%%%%%%%%%%%%%%%%%%%%%%%%%%%%%%%%%%%%%%%%%%%%%%%%%%%%
%-------------------	begin FIGURE 	-------------------%
\begin{figure}
\centering
\includegraphics[width=5.0in]{Images/chapter5/greenkinematics.pdf}
\caption{By taking the right polar decomposition of the deformation gradient tensor, we can express the components of the Green Lagrange strain tensor with respect to the material axis. This creates a symmetry for the shear component of the Green Lagrange strain tensor, allowing us to further simplify the model form.}
\label{fig:greenkinematics}
\end{figure}
%-------------------	 end FIGURE 	-------------------%
%%%%%%%%%%%%%%%%%%%%%%%%%%%%%%%%%%%%%%%%%%%%%%%%%%%%%%%%%%%%












%-----------------------------------------------------------
%	Model formulation
%-----------------------------------------------------------
\subsection{Effective constitutive model form}
\subsubsection{Possible family of forms for the effective constitutive model}

    Using phenomenological approaches is necessary for minimal computational cost. The form of phenomenogical models for soft tissues generally falls into three families. The first family is composed of a summation of polynomials, 
%==========================================================%
%-------------------	begin EQUATION 	-------------------%
\begin{equation}
\begin{aligned}
\Psi	&= \sum_i\sum_j\sum_k c_{ijk}E_m^i E_n^j E_\phi^k. 
\end{aligned} \label{eqn:polynomialmodelform}
\end{equation}
%-------------------	 end EQUATION 	-------------------%
%==========================================================%
    We will refer to this family as the polynomial series approach. The second family is composed of separated exponential functions of individual or combinations of invariants or strains used, for example by Vito \textit{et al.} \cite{vito_mechanical_1980},
%==========================================================%
%-------------------	begin EQUATION 	-------------------%
\begin{align}\label{eqn:vitomodelforms}
\Psi 	&= \sum_i\sum_j\sum_k c_{ijk} e^{b_{ijk}E_m^i E_n^j E_\phi^k}.
\end{align}
%-------------------	 end EQUATION 	-------------------%
%==========================================================%
    We will refer to this family as the separated exponential approach. The final family is exponential models composed of a single exponential function of the sum of polynomials,
%==========================================================%
%-------------------	begin EQUATION 	-------------------%
\begin{equation}
\begin{aligned}\label{eqn:exponentialmodelform}
\Psi 	&= c_0 \left(e^{Q} - 1\right) \\
Q		&= \sum_i\sum_j\sum_k b_{ijk}E_m^i E_n^j E_\phi^k.
\end{aligned}
\end{equation}
%-------------------	 end EQUATION 	-------------------%
%==========================================================%
    This was first introduce by Fung \cite{fung_pseudoelasticity_1979} and we will refer to this family as the single exponential approach. 


\subsubsection{Generalized effective constitutive model form determination} \label{sec:possibleforms}

	Each approach (Eqn. \ref{eqn:polynomialmodelform}-\ref{eqn:exponentialmodelform}) has its own advantages and disadvantages. Polynomial series approach has the most flexibility. With sufficient number of terms, it can it reproduce the response in a similar manner to Taylor series expansions. In pilot testing, polynomial series approach requires significantly more number of terms than other choices, at least 27 terms in preliminary testing. 21 of the 27 terms are coupling terms. As a result, constraints needed for convexity are both complex and difficult to enforce. In the most general form, convexity cannot be enforce globally or only at the boundaries. It needs to be enforced at separate points within the domain or by integration. These constraints do not only have computational costs that vastly exceed the cost of the model itself, but will also significantly impacts convergence during parameter estimation. The constraints are not convex, with many local minima, often failing to converge even after 200,000 iterations. In additional, extrapolation using this approach is extremely unreliable. This makes constrained optimization often intractable to implement within a simulation framework such as the one proposed (Fig. \ref{fig:simulationframework}).


	The separated exponential approach generally suffers from the same issues as the polynomial series. This model form behaves like polynomial series with variable exponents, i.e. $c_1e^{b_1E_m} = c_1y^{b_1}$, where $y =e^{E_m}$. The advantage of this family of models is that similar and highly covariant terms such as $c_1 E_m + c_2 E_m^2 + c_3 E_m^3 ...$ can be avoided, reducing the number of parameters needed. However, like the polynomial series family, coupling terms such as $c_4 e^{b_4 E_m E_n}$ are not convex or elliptical functions, resulting in the same issues for parameter estimation and enforcing convexity. The number of parameters required to fully reproduce the mechanical response is still quite large. Moreover, for the same number of parameters, the separated exponential form is woefully insufficient at reproducing the mechanical response of soft tissues in comparison to the single exponential approach. As such, the advantages gained for parameter covariance by separating the terms are actually quite minimal.
    
    The single exponential approach has substantial parameter covariance, but is extremely effective at reproducing the response of soft tissues using a small number of parameters, is computationally efficient, and is easy to enforce convexity for. Because the exponential function is monotonically increasing, enforcing convexity and ellipticity only requires the polynomial $Q$ to be convex and elliptical. This is the best balance for our goals, and is thus our choice for the effective constitutive model. The first step is of course to examine the generalized Fung model \cite{fung_pseudoelasticity_1979}
%==========================================================%
%-------------------	begin EQUATION 	-------------------%
\begin{equation}
\begin{aligned}\label{eqn:fungmodel}
\Psi 	&= c_0 \left(e^{Q} - 1\right) \\
Q		&= \sum_i\sum_j\sum_k\sum_l b_{ijkl}E_{ij} E_{kl}.
\end{aligned}
\end{equation}
%-------------------	 end EQUATION 	-------------------%
%==========================================================%
    We find that the generalized Fung model is not able to fully reproduce the mechanical response of pericardium and aortic valve tissues, it can only do so in a limited range. Although this is enough for most numerical simulations, where the deformations are generally limited to the physiologic range, it is not always sufficient for predicting the mechanical response when organs undergo significant changes in geometry, causing the deformations to change drastically. The easiest way to visualize this is through contour plots of the strain energy function (Fig. \ref{fig:strainenergycontours}). The mechanical response of soft tissues general has hyperelliptical contours (Fig. \ref{fig:strainenergycontours}A), whereas the generalized Fung model always has precisely elliptical contours (Fig. \ref{fig:strainenergycontours}B). This attribute of the generalized Fung model makes it easy to enforce ellipticity and convexity, and its elasticity tensor and behavior at small strains is easy to derive. However, when the range of deformation is sufficiently large, the generalized Fung model is essentially limited to stretching and rotating its contours to match that of the soft tissue (Fig. \ref{fig:strainenergycontours}B), but cannot fully reproduce the resulting tissue responses. It can only serve as an approximation.  


%%%%%%%%%%%%%%%%%%%%%%%%%%%%%%%%%%%%%%%%%%%%%%%%%%%%%%%%%%%%
%-------------------	begin FIGURE 	-------------------%
\begin{figure}
\centering
\includegraphics[width=\textwidth]{Images/chapter5/strainenergycontours}
\caption{The contour plots of strain energy (kPa) of A) a bovine pericardial specimen using a meso-scale structural model \cite{zhang_modeling_2017}, B) best fit using the generalized Fung model (Eqn. \ref{eqn:generalizedfungmodel}), and C) best fit using the effective constitutive model we develop from herein showing the necessity of extending existence phenomenological model form.}
\label{fig:strainenergycontours}
\end{figure}
%-------------------	 end FIGURE 	-------------------%
%%%%%%%%%%%%%%%%%%%%%%%%%%%%%%%%%%%%%%%%%%%%%%%%%%%%%%%%%%%%

\subsubsection{Final effective material model form} \label{sec:finalform}

	Thus, for the effective constitutive model, we extended the polynomial $Q$ one step further, allowing hyperellipticity of the strain energy density function (Fig. \ref{fig:strainenergycontours}C). For the additional terms to include, we move up to the next even powers, up to the quartic terms (exponents $i+j+k\leq4$) (Eqn. \ref{eqn:exponentialmodelform}), as odd numbered powers alone do not yield elliptical functions. There are a total of 34 possible terms in Q. Not all terms are required or even admissible. Specifically, the following constraints are enforced on the model:
      
    \underline{Constraint 1}: \underline{The stress must be zero in the reference configuration}. Given that the stress is the gradient of $\Psi$, where is $\Psi^\prime = c_0 Q^\prime e^Q$, all terms in $Q^\prime$ must be zero at zero strain. This corresponds to all $i+j+k = 1$ terms being removed, leaving 31 terms remaining. 
      
    \underline{Constraint 2}: \underline{The response must be elliptic}. That is the shortest line inscribed on the strain energy function surface joining any two points must have a positive curvature. Keeping in mind that the generalized Fung model (Eqn. \ref{eqn:exponentialmodelform}) is already close to being sufficient at reproducing the response of many soft tissues we tested. We only want to extend this to be able to reproduce a wider range of soft tissue responses. Furthermore, in considerations of limiting the number of parameters, reducing parameter correlation, improving the conditioning of the constrained objective function surface, and that the non-elliptical terms must be small, we choose to forgo all $i+j+k = 3$ terms, leaving 21 terms remaining. 
      
    \underline{Constraint 3}: \underline{Response must be independent of the direction of shear}. Since we decompose the Green-Lagrange strain relative to the material axis, this creates a plane of symmetry in the soft tissue response for the direction of shear. Thus, the value of $E_\phi$ can only have even powers, $k = 2,4$. The following terms are thus necessarily zero: $E_m^3E_\phi$, $E_m^2E_nE_\phi$, $E_mE_n^2E_\phi$, $E_n^3E_\phi$, $E_mE_\phi^3$, $E_nE_\phi^3$, $E_mE_\phi$, and $E_nE_\phi$. 
      
The final form of the effective constitutive model is thus
%==========================================================%
%-------------------	begin EQUATION 	-------------------%
\begin{equation}
\begin{aligned}\label{eqn:generalizeexponentialform}
\Psi	=& c_0 \left(e^{Q} - 1\right) \\
Q		=& b_1 E_m^2 + b_2 E_n^2 + b_3 E_\phi^2 + b_4 E_m E_n + b_5 E_m^4 + b_6 E_n^4 + b_7 E_m^3 E_n + b_8 E_m^2 E_n^2 + b_9 E_m E_n^3	\\
	&+ b_{10} E_\phi^4 + b_{11} E_m^2E_\phi^2 + b_{12} E_n^2 E_\phi^2 + b_{13} E_m E_n E_\phi^2.
\end{aligned}
\end{equation}
%-------------------	 end EQUATION 	-------------------%
%==========================================================%








%-----------------------------------------------------------
%	Model convexity
%-----------------------------------------------------------
\subsubsection{Enforcing convexity and ellipticity}

	Perhaps the biggest advantage of the single exponential approach models is the convenience for enforcing ellipticity and convexity. Because ellipticity and convexity are preserved by monotonically increase functions, such as $e^x$, we only have to enforce ellipticity and convexity of $Q$ (Eqn. \ref{eqn:generalizeexponentialform}). For strong ellipticity, the following must be satisfied,
%==========================================================%
%-------------------	begin EQUATION 	-------------------%
\begin{equation}\label{eqn:strongellipticity}
\dmd{\Psi}{2}{F_{ij}}{}{F_{kl}}{}\lambda_i\lambda_k\mu_j\mu_l > 0 \equiv \dmd{Q}{2}{F_{ij}}{}{F_{kl}}{}\lambda_i\lambda_k\mu_j\mu_l > 0,
\end{equation}
%-------------------	 end EQUATION 	-------------------%
%==========================================================%	
	where $F$ is any tensor, and $\lambda$ and $\mu$ are arbitrary non-zero vectors. \emph{This condition is also equivalent of strict convexity \cite{ball_strict_1980}}, so both conditions will be satisfied. Satisfying this constraint requires that the elasticity tensor $C_{ijkl}=\dmd{\Psi}{2}{E_{ij}}{}{E_{kl}}{}$ is positive definite, or rather $\dmd{Q}{2}{E_{ij}}{}{E_{kl}}{}$ is positive definite, satisfying Drucker stability for numerical purposes. Sylvester's criterion \cite{gilbert_positive_1991}, is the most convenient in this scenario, which is given by 
%==========================================================%
%-------------------	begin EQUATION 	-------------------%
\begin{equation}\label{eqn:convexitycriteria}
\begin{aligned}
&\dpd[2]{Q}{E_m} \geq 0, \quad
\det
\begin{bmatrix}
\dpd[2]{Q}{E_m} & \dmd{Q}{2}{E_m}{}{E_n}{}\\
\dmd{Q}{2}{E_m}{}{E_n}{} & \dpd[2]{Q}{E_n}\\
\end{bmatrix} \geq0, \quad  \\
&\det
\begin{bmatrix}
\dpd[2]{Q}{E_m} & \dmd{Q}{2}{E_m}{}{E_n}{} & \dmd{Q}{2}{E_m}{}{E_\phi}{}\\
\dmd{Q}{2}{E_m}{}{E_n}{} & \dpd[2]{Q}{E_n} & \dmd{Q}{2}{E_n}{}{E_\phi}{}\\
\dmd{Q}{2}{E_m}{}{E_\phi}{} & \dmd{Q}{2}{E_n}{}{E_\phi}{} & \dpd[2]{Q}{E_\phi} \\
\end{bmatrix} \geq0.
\end{aligned}
\end{equation}
%-------------------	 end EQUATION 	-------------------%
%==========================================================%
     For the generalized Fung model (Eqn. \ref{eqn:generalizedfungmodel}), $b_1>0$, $b_1b_2-b_4>0$, and $b_3(b_1b_2 - b_4^2) - b_5(b_2b_5 - b_4b_6) - b_6(b_1b_6 - b_4b5)>0$ will enforce convexity everywhere. For equation \ref{eqn:generalizeexponentialform}, this is slightly more complex. The non-convex region starts from a point along the respective axis for each component $E_m$, $E_n$, and $E_\phi$, then spreads out in the shape of a fan as the strain increases depending on which specific coupling terms, such as $E_m^3 E_n$ and $E_m E_n^3$ are present (Fig. \ref{fig:convexitybehavior}). As long as the effective constitutive model is convex on the largest value along the $E_m$, $E_n$, and $E_\phi$ axis respectively, then the effective constitutive model is convex over the entire range. For example, the effective constitutive model is convex if the maximum point on the $E_m$-axis is convex for $E_m^3 E_n$ (Fig. \ref{fig:convexitybehavior}A) or if the maximum point on the $E_n$-axis is convex for $E_m E_n^3$ (Fig. \ref{fig:convexitybehavior}B). Thus, assuming an upper limit of $E_m < 1$, $E_n < 1$, and $E_\phi < 1$, the following constraints on the parameters are sufficient to guarantee convexity and ellipticity,
%==========================================================%
%-------------------	begin EQUATION 	-------------------%
\begin{equation} \label{eqn:effmodelconstraints}
\begin{aligned}
b_1, b_2,b_3,b_5,b_6,b_{10} \geq 0	\\
4(b_1 + 6 b_5) (b_2 + b_8) - (b_4 + 3 b_7)^2 \geq 0		\\
4(b_2 + 6 b_6) (b_1 + b_8) - (b_4 + 3 b_9)^2 \geq 0 	\\
4(b_1 + b_{11}) (b_2 + b_{12}) - (b_{13} + b_4)^2 \geq 0 	\\
b_3+b_{11} \geq 0	\\
b_3+b_{12} \geq 0.	\\
\end{aligned}
\end{equation}
%-------------------	 end EQUATION 	-------------------%
%==========================================================%


%%%%%%%%%%%%%%%%%%%%%%%%%%%%%%%%%%%%%%%%%%%%%%%%%%%%%%%%%%%%
%-------------------	begin FIGURE 	-------------------%
\begin{figure}
\centering
\includegraphics[width=\textwidth]{Images/chapter5/convexitybehavior}
\caption{The criteria for ellipticity (Eqn. \ref{eqn:convexitycriteria}) plotted against the components of the Green-Lagrange strain for A) when including the $E_m^3E_n$ term and B) $E_mE_n^3$ term. The white regions are not convex (N-C), which start at a point along the axes and spreads out as the strain increases.}
\label{fig:convexitybehavior}
\end{figure}
%-------------------	 end FIGURE 	-------------------%
%%%%%%%%%%%%%%%%%%%%%%%%%%%%%%%%%%%%%%%%%%%%%%%%%%%%%%%%%%%%










    
%-----------------------------------------------------------
%	Model scaling for parameter estimation
%-----------------------------------------------------------
\subsection{Model scaling method to improve parameters correlation for parameter estimation} \label{sec:modelscaling}

%-----------------------------------------------------------
%	Computational approaches	
\subsubsection{Parameter correlation for exponential type models}

	One challenging problem with model parameter determination is covariance between the parameters during parameter estimation. Covariance explains how two parameters influence the response of the model and how they will be updated during parameter estimation. High parameter covariance results in both slow convergence and poor reliability and reproducibility of the material parameters. When scaled by the variance, this becomes the correlation between the parameters, with an absolute value between $0$ and $1$. Correlation equal to $1$ implies that two parameters have the exact same effect on the model response, and are thus indistinguishable during parameter estimation. The covariance issue for constitutive models with an exponential function is well described by Aggarwal \cite{aggarwal_inverse_2015, aggarwal_improved_2017}. These constitutive models with exponential functions have a long valley like region in the objective function space. Inside this valley, significantly different parameters produce similar objective function values. This presents several problems. 1) It's difficult to compare model parameters between different specimens, because drastically different parameters can produce similar responses. As such, the average or representative specimen has little real meaning, and each specimen needs to be fitted individually for simulations. 2) The convergence of gradient based optimization algorithms becomes excruciatingly slow due to the small gradients while trapped within this valley. 3) The covariance between parameters being extremely large decreases the accuracy or in other word increases the confidence interval of parameters obtained. 
    
    Aggarwal \textit{et al.} suggested two improvements to alleviate this problem \cite{aggarwal_improved_2017}. These improvements are 1) modifying the modulus parameter $A$ to $e^{a}$, straightening the shape of the valley, and 2) introducing the log-norm for the objective function, improving the gradient along the valley. These modifications have been shown to be effective. However, this is not always ideal. The suggested logarithmic norm faces some issues when fitting stresses or strains, which may be negative and thus becomes undefined. Although this may be alleviated by forgoing data points with negative strains or stresses, but the model may still produce negative values during parameter estimation. Other methods can be used to discard negative values or to take the norm of such values, but these approaches create discontinuities in the gradient of the objective function, causing convergence problems during parameter estimation. Clearly, additional improvements can still be made. 


%-----------------------------------------------------------
%	Scaling method
\subsubsection{Model scaling method}

	We begin by examining the fundamental reason for the high parameter covariance. For this, we will use the 1-D case as an example,
%==========================================================%
%-------------------	begin EQUATION 	-------------------%
\begin{equation}
\begin{aligned}
\Psi &= A \left(e^{B \epsilon} - 1\right) \\
\mathcal{F} &= \sum_i \left(\Psi(\epsilon_i) - \Psi_i \right)^2,
\end{aligned}
\end{equation}
%-------------------	 end EQUATION 	-------------------%
%==========================================================%
where $\Psi$ is the strain energy of our model, $\epsilon$ is some invariant that is a function of the strain, $\epsilon_i$ and $\Psi_i$ are simulated data, and $\mathcal{F}$ is our objective function for parameter estimation. The parameters $A$ and $B$ have different purposes: $A$ is like a modulus, linearly increasing the stiffness of the material, while $B$ modifies the shape of the response, controlling the nonlinearity of the material. However, practically, the two parameters have nearly the same effect on the mechanical response, increasing $A$ increases the stiffness (Fig. \ref{fig:scalingapproach}A) and increasing $B$ also increases the stiffness (Fig. \ref{fig:scalingapproach}B). This is the reason for the high correlation between the parameters (Fig. \ref{fig:scalingapproach}), 0.9979.


%%%%%%%%%%%%%%%%%%%%%%%%%%%%%%%%%%%%%%%%%%%%%%%%%%%%%%%%%%%%
%-------------------	begin FIGURE 	-------------------%
\begin{figure}
\centering
\includegraphics[width=\textwidth]{Images/chapter5/scalingapproach}
\caption{A)The effect of increasing the values of the modulus $A$ on exponential type models. B) The effect of increasing the values of exponent $B$ on exponential type models, which is nearly indistinguishable to the modulus $A$. C) The effect of increasing the values of parameter $B$ after applying the proposed scaling, increasing the values of parameter $A$ remains the same as in A).}
\label{fig:scalingapproach}
\end{figure}
%-------------------	 end FIGURE 	-------------------%
%%%%%%%%%%%%%%%%%%%%%%%%%%%%%%%%%%%%%%%%%%%%%%%%%%%%%%%%%%%%


	To address this problem, we introduce a scaling term to normalize the exponential part of the model. This prevents increasing $B$ from increasing the value of the strain energy as a whole, allowing it to only control the curvature. For this, we will use a value $\epsilon_{max}$, which represents the data point with the maximum strain energy value used for parameter estimation, which is also the point where the strain energy stays constant with changes in $B$. The scaled form is thus given by
%==========================================================%
%-------------------	begin EQUATION 	-------------------%
\begin{equation}
\begin{aligned}
\Psi = \Psi_s = \bar{A} \left[e^{-B\epsilon_{max}} \left( e^{B\epsilon} - 1\right)\right],\label{eqn:scaledmodel1D}
\end{aligned}
\end{equation}
%-------------------	 end EQUATION 	-------------------%
%==========================================================%
    where $\bar{A}$ is the scaled version of the modulus $A$. This scaling keeps the exponential part of the model, $e^{-B\epsilon_{max}} ( e^{B\epsilon} - 1)$, at approximately 1.0 at $\epsilon = \epsilon_{max}$, regardless of the changes in the value of the parameter $B$ (Fig. \ref{fig:scalingapproach}C). This effect is not exact when the value of $B$ is small due to the $-1$ needed to set the strain energy to 0 in the referential configuration, but is nonetheless sufficient for our goal: decoupling the modulus increasing effect of the parameter $A$ from the curvature increasing effect of parameter $B$. Indeed, we found this approach to be successful. We examined the contour plot of the objective function with respect to each of the 4 cases in Aggarwal's work \cite{aggarwal_improved_2017}, with the standard objective function, with $A=e^{a}$, with log-norm, and with $A=e^{a}$ and the log-norm, for both without scaling and with scaling (Fig. \ref{fig:objfunctionsurfaces}). First, the correlation between the parameters does not change with $A=e^{a}$. The log-norm improves the correlation from 0.9979 to 0.9063 (Table \ref{tb:ABcorrelation}), which significantly improves the objective function surface (Fig. \ref{fig:objfunctionsurfaces}). On the other hand, our scaling method improves the correlation from 0.9979 to 0.6186, more significant than using the log-norm. Interestingly, combining scaling and the log-norm has the adverse effect, increasing the correlation back from 0.6186 to 0.8592. This is a result of essentially linearizing the relation between $A$ and $B$, i.e. from $Ae^{B\epsilon}$ to $Log(A)+B\epsilon$, causing the relationship between $A$ and $B$ to go from modulus and nonlinearity to baseline and modulus. Clearly, \emph{the most optimal parameter estimation approach is to use scaling method with no other modifications}. 
    
    
%%%%%%%%%%%%%%%%%%%%%%%%%%%%%%%%%%%%%%%%%%%%%%%%%%%%%%%%%%%%
%-------------------	begin FIGURE 	-------------------%
\begin{figure}
\centering
\includegraphics[width=\textwidth]{Images/chapter5/objfunctionsurfaces}
\caption{(Top) The objective function surface for the traditional unscaled exponential models and (Bottom) objective function surface after scaling. From Left to Right are: the unchanged surface, the surface after changing $A$ to $e^{a}$, using the log-norm for the objective function, and applying both changes. The scaled form with no other changes behaves the best.}
\label{fig:objfunctionsurfaces}
\end{figure} 
%-------------------	 end FIGURE 	-------------------%
%%%%%%%%%%%%%%%%%%%%%%%%%%%%%%%%%%%%%%%%%%%%%%%%%%%%%%%%%%%%


%----------------------------------------------------------%
%-------------------	begin TABLE 	-------------------%
\begin{table}
\caption{The correlation between model parameter when using Hencky strains}
\begin{center}
\label{tb:ABcorrelation}
\begin{tabular}{|l|rrrr|}
\hline
			& No change	& $\log(A)$	& $\log$-norm	& Both \\
\hline
Traditional	& -0.9979	& -0.9979	& -0.9063		& -0.9063 \\
Scaled 		& 0.6186	& 0.6186	& 0.8592		& 0.8592 \\
\hline
\end{tabular}
\end{center}
\end{table}
%-------------------	 end TABLE 		-------------------%
%----------------------------------------------------------%



%-----------------------------------------------------------
%	Relation to unscaled form
% \subsubsection{Relation to unscaled model}

    By design, the value of the exponential parameter $B$ does not change by using the scaling method. Since the scaling term does not depend on the input strain, it acts as a modification to the modulus $A$ while keeping the exponential term the same. This also implies that the relationship between the unscaled modulus $A$ and the scaled modulus $\bar{A}$ is
%==========================================================%
%-------------------	begin EQUATION 	-------------------%
\begin{equation}
\begin{aligned}
A = \bar{A} e^{-B\epsilon_{max}},
\end{aligned}
\end{equation}
%-------------------	 end EQUATION 	-------------------%
%==========================================================%
which makes finding the actual unscaled parameters a simple task. One other benefit of this scaling approach is that the value of $\bar{A}$ is extremely straight forward and intuitive, it is the strain energy of the model at $\epsilon_{max}$. As a result, the value of $\bar{A}$ can be determined \textit{a priori}, or at the very least it is easy to make an initial guess for $\bar{A}$. This will in turn also help to make parameter estimation faster and more accurate, leaving only the parameter $B$ to be determined. 


%-----------------------------------------------------------
%	Extension to multi-variable form
\subsubsection{Extension to multiple variables}

	Extending this method to multiple variables is very simple. For $\Psi_{eff}$ (Eqn. \ref{eqn:generalizeexponentialform}), the input variables become $\mathbf{\epsilon} = \{E_m, E_n, E_\phi\}$, and $\mathbf{\epsilon}_{max} = \{E_m^{max},E_n^{max},E_\phi^{max}\}$. Determining the values for $\mathbf{\epsilon}_{max}$ depends on the form of the objective function. Using the most common case as the example, which is the sum of the squares of the differences in the 2nd Piola Kirchhoff stress,
%==========================================================%
%-------------------	begin EQUATION 	-------------------%
\begin{equation}
\begin{aligned}
\mathcal{F} = \sum_i \left(S_{11}(\epsilon_i) - \hat{S}_{11}^i\right)^2 + \left(S_{12}(\mathbf{\epsilon}_i) - \hat{S}_{12}^i\right)^2 + \left(S_{22}(\epsilon_i) - \hat{S}_{22}^i\right)^2,
\end{aligned}
\end{equation}
%-------------------	 end EQUATION 	-------------------%
%==========================================================%
$\mathbf{\epsilon}_{max}$ is the data point $\mathbf{\epsilon}_i$ which maximizes $\left(\hat{S}_{11}^i\right)^2 + \left(\hat{S}_{12}^i\right)^2 + \left(\hat{S}_{22}^i\right)^2$. 
Thus, we also introduce a $Q_{max}$ such that,
%==========================================================%
%-------------------	begin EQUATION 	-------------------%
\begin{equation} \label{eqn:finalexponentialmodelformscaled}
\begin{aligned}
\Psi_{eff} 	=& c_0 \left(e^{Q} - 1\right) = c_0^\prime e^{-Q_{max}}\left(e^{Q} - 1\right)    \\
Q		=& b_1 E_m^2 + b_2 E_n^2 + b_3 E_\phi^2 + b_4 E_m E_n + b_5 E_m^4 + b_6 E_n^4 + b_7 E_m^3 E_n + b_8 E_m^2 E_n^2 \\ 
&+ b_9 E_m E_n^3 + b_{10} E_\phi^4 + b_{11} E_m^2E_\phi^2 + b_{12} E_n^2 E_\phi^2 + b_{13} E_m E_n E_\phi^2 \\
Q		=& b_1 (E_m^{max})^2 + b_2 (E_n^{max})^2 + b_3 (E_\phi^{max})^2 + b_4 (E_m^{max}) (E_n^{max}) + b_5 (E_m^{max})^4   \\
    &+ b_6 (E_n^{max})^4 + b_7 (E_m^{max})^3 (E_n^{max}) + b_8 (E_m^{max})^2 (E_n^{max})^2 + b_9 (E_m^{max}) (E_n^{max})^3	\\
	&+ b_{10} (E_\phi^{max})^4 + b_{11} (E_m^{max})^2(E_\phi^{max})^2 + b_{12} (E_n^{max})^2 (E_\phi^{max})^2    \\ 
	&+ b_{13} (E_m^{max}) (E_n^{max}) (E_\phi^{max})^2,
\end{aligned}
\end{equation}
%-------------------	 end EQUATION 	-------------------%
%==========================================================%
where parameter estimation will be done for $c_0^\prime$ instead of $c_0$. Computing the response functions and the stresses, or even the elasticity tensor remains very simple, only requiring multiplying each term by $e^{-Q_{max}}$. Thus, this scaling method is a very simple and easy to implement method of improving the speed and convergence for the parameter estimation of exponential type models. 












%-----------------------------------------------------------
%	Optimal loading paths
%-----------------------------------------------------------
\subsection{Optimal \textit{in silico} loading paths for parameter estimation}\label{sec:optimaldesign}

	Another technique for improving the parameter estimation process for determining $\Psi_{eff}$ from respective micro-models is establishing optimal loading paths. An example is the work of Avazmohammadi \cite{avazmohammadi_novel_2016}, where optimal experimental design is used to 1) minimize the amount of data necessary and 2) improve model parameter covariance for parameter estimation. Just like one of the most important question to ask before performing any mechanical testing is how much and what kind of data is necessary, we should also be selective with our choice of sampling points for parameter estimation. The theory for optimal design of experiment is well-studied and documented \cite{lanir_optimal_1996, zhu_d_2014}. Vast majority of the methods for optimal design uses D-optimality as the design variable,
%==========================================================%
%-------------------	begin EQUATION 	-------------------%
\begin{equation}\label{eqn:doptimality}
\begin{aligned}
D = \det(\mathbfcal{I}), \quad \mathbfcal{I} = \mathbf{J}^\mathsf{T}\mathbf{J} \quad \mathrm{or} \quad \mathbfcal{I} = \mathbfcal{H},	\\
\mathrm{where} \ J_{ij} = \dpd{f_i}{\xi_j}, \quad \mathcal{H}_{ij} = \dmd{\mathcal{F}}{2}{\xi_i}{}{\xi_j}{},
\end{aligned}
\end{equation}
%-------------------	 end EQUATION 	-------------------%
%==========================================================% 
    where $\mathbf{\xi}$ is a vector of model parameters and $\mathbfcal{I}$ is the information matrix. $\mathbfcal{I}$ can be computed from the derivatives of the objective function $\mathcal{F}$, where $f$ is the model evaluated at each data point, or it can be computed from the Hessian of the objective function, $\mathbfcal{H}$ (Appendix \ref{sec:parametercorrelation}). D-optimality is the determinant of the information or the hessian matrix at the best fit value. It offers the best representation of both parameter accuracy (parameter covariance or correlation) and precision (parameter variance) at the same time.

    
    The first and foremost step is to establish the parameterization for loading paths so they can be optimized. This is not a straightforward choice, as the number of loading path required is not yet established. Another issue is that the number of data points is discrete, thus the gradient of the D-optimality with respect to the control parameters is not smooth. For the sake of time spent for optimization, the number of data points should be kept as small as possible, thus exacerbating the issue of differentiability. Even worse is perhaps that the objective function is essentially flat when not near the optimum, making gradient algorithms not practical. Monte Carlo, random search or divide and conquer strategies are needed, which are much more time consuming. The search space also increases exponentially with the number of loading paths, making a fully exhaustive search difficult to implement. Thus, a simple parameterization for loading paths is ideal. 
    

%%%%%%%%%%%%%%%%%%%%%%%%%%%%%%%%%%%%%%%%%%%%%%%%%%%%%%%%%%%%
%-------------------	begin FIGURE 	-------------------%
\begin{figure}
\centering
\includegraphics[width=2.5in]{Images/chapter5/optimaldesign}
\caption{Our approach for optimizing for the optimal loading paths for parameter estimation.}
\label{fig:optimaldesign}
\end{figure}
%-------------------	 end FIGURE 	-------------------%
%%%%%%%%%%%%%%%%%%%%%%%%%%%%%%%%%%%%%%%%%%%%%%%%%%%%%%%%%%%%

    
    We define loading paths based on the following conditions: 
\begin{enumerate}
\item The number of loading paths is as small as possible
\item The number of variables needed to define a loading path is as small as possible
\item Possible application to mechanical testing of tissues.
\end{enumerate} 
    Starting with planar extensions only, we chose a loading path as data points which shares the same stretch ratio, $\lambda_1/\lambda_2$, which is typically the same definition used for biaxial mechanical testing. This only requires one constant to be defined for each loading path and the resulting fan shape covers the largest range of deformation with the least number of data points. To determine the optimal loading paths, 1) the total number of loading paths was set, then 2) each combination of the stretch ratios was evaluated for the highst D-optimality (Fig. \ref{fig:optimaldesign}). For a total number of loading paths ranging from 1 to 6, we found the the point when the D-optimality stops increasing significantly, and choose this as the optimal set. Next the shear component, $\kappa_1$, is added. For this study, we constrained the shear to be $0<\kappa_1<0.2$. Only the positive values of $\kappa_1$ is allowed due to the material symmetry. Furthermore, the optimal planar extensions loading paths are always included as a part of the data set.














%%%%%%%%%%%%%%%%%%%%%%%%%%%%%%%%%%%%%%%%%%%%%%%%%%%%%%%%%%%%%
%%  Model Applications										%
%%%%%%%%%%%%%%%%%%%%%%%%%%%%%%%%%%%%%%%%%%%%%%%%%%%%%%%%%%%%%

%-----------------------------------------------------------
%	Use of structural constitutive models
%-----------------------------------------------------------
\subsection{Example application for planar soft tissues}

	The exogenously cross-linked structural model presented in Zhang and Sacks \cite{zhang_modeling_2017} is used to test our approach (Fig. \ref{fig:simulationframework}) and see if $\Psi_{eff}$ (Eqn. \ref{eqn:finalexponentialmodelformscaled}) can completely reproduce its response. This meso-scale structural model is computationally expensive due to integration over the collagen fiber architecture but have been shown to be able to accurately reproduce the mechanical response of a variety of soft tissues, such as mitral valve leaflets \cite{zhang_meso_2016}, ovine pulmonary artery \cite{fata_insights_2014}, myocardium \cite{avazmohammadi_novel_2016}, and exogenously cross-linked bovine pericaridium \cite{sacks_novel_2015}. To briefly summarize, this model is composed of 3 components: collagen, $\Psi_\mathrm{col}$, matrix, $\Psi_\mathrm{mat}$, and interactions, $\Psi_\mathrm{int}$. 
%==========================================================%
%-------------------    begin EQUATION     -------------------%
\begin{equation}
\Psi     = \Psi_\mathrm{col} + \Psi_\mathrm{mat} + \Psi_\mathrm{int} \label{eqn:structuralmodelcomponents}
\end{equation}
%-------------------     end EQUATION     -------------------%
%==========================================================%
    The matrix term, $\Psi_\mathrm{mat}$, is a modified version of the Yeoh model that is more linear when expressed in 2nd Piola Kirchhoff stress versus stretch, 
%==========================================================%
%-------------------    begin EQUATION     -------------------%
\begin{equation}\label{eqn:matrixmodel}
\begin{aligned}
\Psi_\mathrm{mat} = &\frac{\eta_M}{2} \left[ \frac{1}{a}\left( I_1 -3\right)^{a} + \frac{r}{b} \left( I_1 -3\right)^{b} \right], \\
&\text{with } 1<a<b, ab <2, 0 \leq r.
\end{aligned}
\end{equation}
%-------------------     end EQUATION     -------------------%
%==========================================================%
    This model contains four parameters: $\eta_M$ is the modulus parameter corresponding to the same parameter in the Neo Hookean model, $a$, $b$, and $r$ are the shape parameters, where $a$ and $b$ control the shape of the two terms, while $r$ is the weight between the two terms. In general, $a \approx 1$, $b \approx 1.87$ and $r \approx 15$ can be treated as constants. 


    The response of collagen fibers is given by the integration over the collagen fibers architecture, their orientation and crimp. The fiber orientations is described by a beta distribution function and fiber crimp is described by another beta distribution function of the stretches needed to straighten the fibers, the slack stretch $\lambda_s$. These are referred to as the orientation distribution function (ODF), $\Gamma$, and the recruitment distribution function (RDF), $D$, respectively, with the forms given in Zhang and Sacks \cite{zhang_meso_2016, zhang_modeling_2017, sacks_novel_2015}. The form of the strain energy function is 
%==========================================================%
%-------------------    begin EQUATION     -------------------%
\begin{equation} \label{eqn:collagen}
\begin{aligned}
\Psi_\mathrm{col} =& \phi_\mathrm{col} \eta_C \int\displaylimits_\theta \Gamma(\theta) 
\int\displaylimits_1^{\lambda_\theta} D\left(\lambda_s \right) \left( \frac{\lambda_\theta}{\lambda_s} - 1\right)^2 \mathrm{d}\lambda_s \mathrm{d}\theta,
\end{aligned}
\end{equation}
%-------------------     end equation     -------------------%
%----------------------------------------------------------%
    where $\eta_C$ is the modulus of the collagen fibers, $\lambda_\theta = \sqrt{\mathbf{n}_\theta \cdot \mathbf{C}\mathbf{n}_\theta}$ is the stretch of the ensemble of collagen fiber with the same orientation, and $\lambda_\mathrm{\theta}/\lambda_s$ is the true stretch of the collagen fibers after they are straightened \cite{zhang_meso_2016}. Similarly, the response of the interaction term is given by integration over pairs of fibers based on their orientation and crimp, which contains a quadruple integral. 
%==========================================================%
%-------------------    begin EQUATION     -------------------%
\begin{equation}
\Psi_\mathrm{int} = \frac{\eta_I}{2} \int\displaylimits_\alpha \int\displaylimits_\beta \Gamma(\alpha) \Gamma(\beta) \int\displaylimits_1^{\lambda_\alpha} \int\displaylimits_1^{\lambda_\beta} D\left( x_\alpha \right) D\left( x_\beta \right) \left( \frac{\lambda_\alpha \lambda_\beta}{x_\alpha x_\beta} - 1\right)^2 \,\mathrm{d}x_\alpha \,\mathrm{d}x_\beta \,\mathrm{d}\alpha \,\mathrm{d}\beta.
\end{equation}
%-------------------     end EQUATION     -------------------%
%==========================================================%
For clarity of presentation, the slack stretches, $\lambda_s$, are replaced by $x_\alpha$ and $x_\beta$ for fiber oriented along the angle $\alpha$ and $\beta$ respectively. Only one parameter, the modulus $\eta_I$, is used to account for all interactions, with the same $\Gamma$ and $D$ already given above. 

The second Piola Kirchhoff stress, $\mathbf{S}=2\frac{\partial\Psi}{\partial\mathbf{C}}$, is 
%==========================================================%
%-------------------    begin EQUATION     -------------------%
\begin{equation} \label{eqn:fullcollagen}
\begin{aligned}
\mathbf{S} = & \phi_\mathrm{col} \eta_C \int\displaylimits_\theta \Gamma(\theta)\left\lbrace 
\int\displaylimits_1^{\lambda_\theta} \frac{D\left( x \right)}{x} \left( \frac{1}{x}- \frac{1}{\lambda_\theta}\right) \mathrm{d}x \right\rbrace \mathbf{n}_\theta\otimes\mathbf{n}_\theta \mathrm{d}\theta \\
+ & \phi_\mathrm{col} \eta_I \int\displaylimits_\alpha \int\displaylimits_\beta \Gamma \left(\alpha \right) \Gamma \left( \beta \right) \\
& \times \left[ \left\lbrace 
\int\displaylimits_1^{\lambda_\alpha} \int\displaylimits_1^{\lambda_\beta} 
\frac{2 \lambda_\beta D(x_\alpha) D(x_\beta)}{x_\alpha x_\beta} 
\left( \frac{\lambda_\alpha}{x_\alpha} \frac{\lambda_\beta}{x_\beta} - 1\right) \mathrm{d}x_\alpha \, \mathrm{d}x_\beta\right.\right.   \\
&\quad\left.\left.+\int\displaylimits_1^{\lambda_\beta} D(x_\beta) \left( \frac{\lambda_\beta}{x_\beta} -1  \right)^2 \mathrm{d}x_\beta \right\rbrace \right.  \frac{\mathbf{n}_\alpha \otimes \mathbf{n}_\alpha}{\lambda_\alpha}  \\
& + \left. \left\lbrace
\int\displaylimits_1^{\lambda_\alpha} \int\displaylimits_1^{\lambda_\beta} 
\frac{2 \lambda_\alpha D(x_\alpha) D(x_\beta)}{x_\alpha x_\beta} 
\left( \frac{\lambda_\alpha}{x_\alpha} \frac{\lambda_\beta}{x_\beta} - 1\right) \mathrm{d}x_\alpha \, \mathrm{d}x_\beta\right.\right.   \\
&\quad \left.\left. + \int\displaylimits_1^{\lambda_\alpha} D(x_\alpha) \left( \frac{\lambda_\alpha}{x_\alpha} -1  \right)^2 \mathrm{d}x_\alpha \right\rbrace \frac{\mathbf{n}_\beta \otimes \mathbf{n}_\beta}{\lambda_\beta}  \right] \mathrm{d}\alpha \, \mathrm{d}\beta\\
+ & \phi_\mathrm{mat} \eta_M \left[\left(\left( I_1 - 3\right)^{a - 1} + r \left( I_1 - 3\right)^{b - 1}\right) \left( \mathbf{I} - C_{33}\mathbf{C}^{-1}\right)  \right],\\
\end{aligned}
\end{equation}
%-------------------     end EQUATION     -------------------%
%==========================================================%
    where $\phi_\mathrm{col}$ and $\phi_\mathrm{mat}$ are the mass fraction of collagen and matrix respectively. 
    
    Although this model is very accurate and predictive, it is also computationally expensive. Moreover, numerical integration results in a significant decrease in numerical precision when the number of quadrature point is insufficient. This can create a number of issues for convergence during parameter optimization. Thus, the implementation of this model is complicated by a constant balance between computational cost and numerical robustness during optimization. Using $\Psi_{eff}$ to reproduce its response is a solution to these issues. %Furthermore, some modifications can be made to reduce the form of $\psi_{eff}$ based on how well each term captures the behavior of the respective material (see Appendix \ref{sec:specificform}).
    
    
    



    









%-----------------------------------------------------------
%    Parameter estimation
%-----------------------------------------------------------
\subsection{Parameter estimation}

    The objective function we use is 
%==========================================================%
%-------------------    begin EQUATION     -------------------%
\begin{equation}\label{eqn:objectivefunction}
\begin{aligned}
S_m =& \dpd{\Psi}{E_m} = \mathbf{m}_0\cdot\mathbf{S}\mathbf{m}_0,
    \  S_n = \dpd{\Psi}{E_n} = \mathbf{n}_0\cdot\mathbf{S}\mathbf{n}_0, 
    \  S_\phi = \dpd{\Psi}{E_\phi} = 2\mathbf{m}_0\cdot\mathbf{S}\mathbf{n}_0 \\
\mathcal{F} =& \sum_i^n \left(S_m(\hat{E}_M^i, \hat{E}_S^i, \hat{E}_\phi^i) - \hat{S}_M^i \right)^2 + \left(S_n(\hat{E}_M^i, \hat{E}_S^i, \hat{E}_\phi^i) - \hat{S}_S^i \right)^2     \\
    &+ \left(S_\phi(\hat{E}_M^i, \hat{E}_S^i, \hat{E}_\phi^i) - \hat{S}_\phi^i \right)^2
\end{aligned}
\end{equation}
%-------------------     end EQUATION     -------------------%
%==========================================================%
    This removes rigid body rotation and puts the stresses along the material axes, giving the response functions and minimizes covariance between the parameters during optimization. Other possible options include the log of the L2-norm, the log-norm presented by Aggarwal \cite{aggarwal_improved_2017}, and L2-norm of the strain energy and other stresses.


    For optimal speed, gradient methods are ideal. Because we require some non-linear constraints to enforce convexity, we utilized the interior point algorithm provided by the IPOPT library \cite{waechter_implementation_2005}. The initial guess is easily derived for the model scaling method, with the parameter $c_0$ being the maximum strain energy in the available data. The exponent parameters $b_i$ are generally very consistent in value, with the quadratic parameters being $b_1 \approx 10$, $b_2 \approx 10$, and $b_4 \approx -10$, the quartic parameters being $b_5 \approx 2000$, $b_6 \approx 500$, $b_9 \approx 200$, $b_{10} \approx 200$, $b_{11} \approx 200$, $b_{12} \approx 200$. We find this setup to work extremely well, and no further modifications are necessary. 




\subsection{Reproducing the response of soft tissues using the effective constitutive model and optimal loading paths}\label{sec:reproducefung}

    For our approach (Fig. \ref{fig:simulationframework}), we need to overcome the limitation of common phenomenological approaches in predicting the mechanical response of micro-models outside of the data set used for parameter estimation \cite{sun_biaxial_2003}. To verify this, we will first reproduced the results of Sun \textit{et al.} \cite{sun_biaxial_2003} using the generalized Fung model, then repeat the process using $\Psi_{eff}$ (Eqn. \ref{eqn:finalexponentialmodelformscaled}) with optimal loading paths. We will focus on glutaraldehyde cross-linked bovine pericardium as the tissue. Bovine pericardium is the most common material used to fabricate bioprosthetic heart valves. It is extremely dense in collagenous fibers, having broad fiber splays with approximately $30\deg$ in standard deviation. The resulting mechanical behavior has strong coupling between the axial stretches. Thus, the mechanical response of some bovine pericardium specimens was fitted to the structural model (Eqn. \ref{eqn:fullcollagen}). The structural model is then used to generate stress-strain data along loading paths with stress ratios ($S_{11}/S_{22}$) of $0.1:1$, $0.5:1$, $0.75:1$, $1:1$, $1:0.75$, $1:0.5$, and $1:0.1$. Following Sun \textit{et al.} \cite{sun_biaxial_2003}, the generalized Fung model (Eqn. \ref{eqn:generalizedfungmodela}) was fitted to the five loading paths in the physiologic range ($0.5:1$, $0.75:1$, $1:1$, $1:0.75$, $1:0.5$), then the remaining two loading paths were predicted and compared to the data. Next, the reverse scenario was done, where the generalized Fung model was fitted to the non-physiologic loading paths ($0.1:1$ and $1:0.1$), while the remaining five loading paths were predicted. Following the same step, $\Psi_{eff}$ was fitted to the three optimal loading paths ($0.1:1$, $1:1$, and $1:0.1$) while the remaining loading paths were predicted. 
    
    We also considered some alternative loading paths. In the original paper by Fung et al. on the constitutive modeling of arteries \cite{fung_pseudoelasticity_1979}, they discussed the use of 'physiologic protocols' as the optimal data set for parameter estimation. These 'physiologic protocols' are loading paths that cover a range past some predetermined lower bounds for $E_{11}$ and $E_{22}$. Conceptually, this is meant to correspond to the range after accounting for the strain between the zero stress configuration and the \textit{in vivo} unloaded (not stress-free) configuration. For clarity, to distinguish between this and the physiologic range (the range where the physiologic loading path is likely to reside) in Sun \textit{et al.}, we will refer to this as the prestrained range. We reproduced the prestrained protocols in figure 5 of Fung \textit{et al.} \cite{fung_pseudoelasticity_1979}, and compared the results to those with optimal loading paths. 








%---    Results
\section{Results}

%%%%%%%%%%%%%%%%%%%%%%%%%%%%%%%%%%%%%%%%%%%%%%%%%%%%%%%%%%%%%
%%  Optimal loading paths for parameter estimation  %%%%%%%%%

\subsection{Optimal \textit{in silico} loading paths for parameter estimation}

	The optimal loading paths for reproducing the response of dense collagenous soft tissues such as bovine pericardium and porcine aortic valve consist of 8 loading paths (Fig. \ref{fig:doptimality}C): 5 loading paths consisting of only inplane extensions (Fig. \ref{fig:doptimality}D), and 3 with the addition of the shear component. The increase in D-optimality is significantly less after three loading paths for the in-plane extensions (Fig. \ref{fig:doptimality}A). Same is true for the loading paths with shear. These three loading paths are the equibiaxial stress (Fig. \ref{fig:doptimality}B, Blue), and two uniaxial loading paths (Black, Green). The type of stress is consistent with the stress used for parameter estimation. The loading paths with shear simply iterates on these three loading paths by adding the shear component, we specified the maximum strain to be applied to be $0.2$. We consider this to be the minimal number of loading paths necessary for parameter estimation. In practice, it's better to add the intermediate inplane tension loading paths (Red, Orange) as a precaution, forming the full set of 8 loading paths (Fig. \ref{fig:doptimality}C). 
	
	
	The equi-biaxial stress loading path is the most important loading path. The D-optimality is $10^{-17}$ for the equi-biaxial loading path versus less than $10^{-300}$ (using \textit{Mathematica}'s extended precision) for other loading paths. The set of optimal loading paths always contains the equi-biaxial stress loading path if the total number is odd, whereas the optimal set always constains two loading paths just besides the equibiaxial loading paths if it is even. The other loading paths complements this path by spanning the range being searched. More detail on the optimal loading path results is presented in appendix \ref{sec:optimalpaths}.
	
%-------------------	begin FIGURE 	-------------------%
\begin{figure}[hbtp]
\centering
\includegraphics[width=5.5in]{Images/chapter5/doptimality}
\caption{A) The best D-optimality value for a given number of loading paths used to generate the data, which stops increasing significantly after three. B) The stress-strain curve of the optimal set of five loading paths with no shear. C) The full set of optimal loading paths including the shear component are shown. The same colored loading paths are built upon the corresponding D) planar stretch loading paths by adding a shear component. }
\label{fig:doptimality}
\end{figure} 
%-------------------	 end FIGURE 	-------------------%









%%%%%%%%%%%%%%%%%%%%%%%%%%%%%%%%%%%%%%%%%%%%%%%%%%%%%%%%%%%%%
%%  Parameter estimation and the quality of fit             %

\subsection{Parameter estimation and the quality of fit}

    The time taken for parameter estimation (5-10 seconds) is significantly lower in comparison to meso-scale structural approaches, such for the mitral valve \cite{zhang_meso_2016} (10-40 minutes) and exogenously crosslinked tissues \cite{zhang_modeling_2017}(30 min - 4 hours). In addition, we found that the model scaling method significantly improves the consistency of convergence. Parameters converges in approximately 40-60 iterations regardless of starting point, whereas it can vary between 40-120 iterations without using scaling. The additional iterations occur within the valley like region in the objective function surface (Fig. \ref{fig:objfunctionsurfaces}), where the gradient and thus the step size is very small. Of course, with sufficiently good initial guess, we found both methods to be essentially equivalent. We note that the model scaling method does not improve the correlations between the exponent parameters $b_1-b_{13}$ in $Q$. With that being said, the correlations between the exponent parameters $b_1-b_{13}$ are significantly better than the correlations between these exponents and modulus $c_0$ (Fig. \ref{fig:gvsecorrelation} and Appendix \ref{sec:parametercorrelation} Table \ref{tb:correlationE} \& \ref{tb:correlationG} vs. Table \ref{tb:ABcorrelation}), which is not much of a problem for parameter estimation. It is difficult to further improve the parameter correlation of $Q$ without changing the form of the model, but, for our purpose, this is already sufficient. 
    
    
    Qualitatively, $\Psi_{eff}$ (Eqn. \ref{eqn:finalexponentialmodelformscaled}) matches the response of collagenous soft tissues reproduced using the structural model (Eqn. \ref{eqn:fullcollagen}). It is able to follow all the characteristics of the response function (derivatives of the strain energy density function), including the symmetry with respect to shear (Fig. \ref{fig:modelfit}). The average $R^2$ is 0.958 (n = 6) for the bovine pericardium specimens tested. We found similar values for porcine aortic valve leaflets. The main improvements are in the near uniaxial strain regions. 
%-------------------	begin FIGURE 	-------------------%
\begin{figure}[hptb]
\centering
\includegraphics[width=\textwidth]{Images/chapter5/modelfit}
\caption{Parameter estimation results showing that $\Psi_{eff}$ is able to replicate the response function of the structural model (Eqn. \ref{eqn:objectivefunction}) (Top) $S_m = \partial\Psi/\dif E_m$, (Middle) $S_n = \partial\Psi/\dif E_n$, and (Bottom) $S_\phi = \partial\Psi/\dif E_\phi$ for each pair of Green Lagrange strain components.}
\label{fig:modelfit}
\end{figure} 
%-------------------	 end FIGURE 	-------------------%


    For a more detailed comparison, we replicated the result of Sun \textit{et al.} \cite{sun_biaxial_2003}. Similarly, we found that the generalized Fung model (Eqn. \ref{eqn:generalizedfungmodela}) fitted the five loading paths in the physiologic range very well (Fig. \ref{fig:fungphyfit}), but predicted the remaining unfitted loading paths poorly (Fig. \ref{fig:fungphypred}). When the non-physiologic loading paths are fit ((Fig. \ref{fig:fungphyfit})), the remaining protocols are still predicted poorly. However, we do note here that the generalized Fung model cannot fit the non-physiologic protocols very well, illustrating the limitation of the generalized Fung model at fitting the response of soft tissue in a wide range of deformations (Section \ref{sec:possibleforms}). 

%%%%%%%%%%%%%%%%%%%%%%%%%%%%%%%%%%%%%%%%%%%%%%%%%%%%%%%%%%%%
%-------------------	begin FIGURE 	-------------------%
\begin{figure}[hptb]
\centering
\includegraphics[width=\textwidth]{Images/chapter5/fungphyfit}
\caption{Reproducing the results of Sun \textit{et al.} \cite{sun_biaxial_2003} showing that the generalized Fung model is able to fit the loading paths in the physiologic range very well. A) The $S_{11}$ surface fitted to the data points. B) The $S_{22}$ surface fitted to the data points. C) Best fit of the $S_{11}$ component of the loading paths. D) Best fit of the $S_{22}$ component of the loading paths.}
\label{fig:fungphyfit}
\end{figure} 
%-------------------	 end FIGURE 	-------------------%
%%%%%%%%%%%%%%%%%%%%%%%%%%%%%%%%%%%%%%%%%%%%%%%%%%%%%%%%%%%%

%%%%%%%%%%%%%%%%%%%%%%%%%%%%%%%%%%%%%%%%%%%%%%%%%%%%%%%%%%%%
%-------------------	begin FIGURE 	-------------------%
\begin{figure}[hptb]
\centering
\includegraphics[width=\textwidth]{Images/chapter5/fungphypred}
\caption{Reproducing the results of Sun \textit{et al.} \cite{sun_biaxial_2003} showing the A) $S_{11}$ component and B) $S_{22}$ component of the remaining unfitted loading paths are predicted poorly from fit (Fig. \ref{fig:fungphyfit}). The inset in A shows the corresponding loading paths.}
\label{fig:fungphypred}
\end{figure} 
%-------------------	 end FIGURE 	-------------------%
%%%%%%%%%%%%%%%%%%%%%%%%%%%%%%%%%%%%%%%%%%%%%%%%%%%%%%%%%%%%


%%%%%%%%%%%%%%%%%%%%%%%%%%%%%%%%%%%%%%%%%%%%%%%%%%%%%%%%%%%%
%-------------------	begin FIGURE 	-------------------%
\begin{figure}[hptb]
\centering
\includegraphics[width=\textwidth]{Images/chapter5/fungoutfit}
\caption{Reproducing the results of Sun \textit{et al.} \cite{sun_biaxial_2003} showing the best fit of the generalized Fung model to the loading paths in the non-physiologic range is poor. A) The $S_{11}$ surface fitted to the data points. B) The $S_{22}$ surface fitted to the data points. C) Best fit of the $S_{11}$ component of the loading paths. D) Best fit of the $S_{22}$ component of the loading paths.}
\label{fig:fungoutfit}
\end{figure} 
%-------------------	 end FIGURE 	-------------------%
%%%%%%%%%%%%%%%%%%%%%%%%%%%%%%%%%%%%%%%%%%%%%%%%%%%%%%%%%%%%

%%%%%%%%%%%%%%%%%%%%%%%%%%%%%%%%%%%%%%%%%%%%%%%%%%%%%%%%%%%%
%-------------------	begin FIGURE 	-------------------%
\begin{figure}[hptb]
\centering
\includegraphics[width=\textwidth]{Images/chapter5/fungoutpred}
\caption{Reproducing the results of Sun \textit{et al.} \cite{sun_biaxial_2003} showing the A) $S_{11}$ component and B) $S_{22}$ component of the equi-biaxial stress loading path are predicted poorly from fit (Fig. \ref{fig:fungoutfit}). The inset in A shows the corresponding loading paths.}
\label{fig:fungoutpred}
\end{figure} 
%-------------------	 end FIGURE 	-------------------%
%%%%%%%%%%%%%%%%%%%%%%%%%%%%%%%%%%%%%%%%%%%%%%%%%%%%%%%%%%%%
    

	Using $\Psi_{eff}$ (Eqn. \ref{eqn:finalexponentialmodelformscaled}) (Fig. \ref{fig:effphyfit}) improves these results, but using non-optimal loading paths, such as based on Fung \textit{et al.}'s prestrained protocols \cite{fung_pseudoelasticity_1979}, lead to poor predictions for other loading paths (Fig. \ref{fig:effphypred}). Although not obvious at first, $\Psi_{eff}$ severe underestimates the response of the material in the low stress region. The D-optimality with two protocols in this prestrained range is only $1.35$, which improves to $1.98\times 10^4$ with six protocols. This pales in comparison to in comparison to with $9.7 \times 10^2$ for the two optimal protocols and $2.2 \times 10^7$ with three optimal protocols. When both $\Psi_{eff}$ and three optimal loading paths are utilized, we found that the loading paths are both fitted (Fig. \ref{fig:effoptfit}) and predicted very well (Fig. \ref{fig:effoptpred}). We also tested other non-optimal loading paths with modifications to the form of $\Psi_{eff}$ (Appendix \ref{sec:otherresults}). To briefly summarize these results, with an optimal set of loading paths, $\Psi_{eff}$ is able to fully reproduce the response of collagenous soft tissue for a wide range of deformation. However, without optimal loading paths, the form of $\Psi_{eff}$ can have unpredictable impact on the predicted response, even though the quality of fit is very similar. 


%%%%%%%%%%%%%%%%%%%%%%%%%%%%%%%%%%%%%%%%%%%%%%%%%%%%%%%%%%%%
%-------------------	begin FIGURE 	-------------------%
\begin{figure}[hptb]
\centering
\includegraphics[width=\textwidth]{Images/chapter5/effphyfit}
\caption{The fit of $\Psi_{eff}$ to the prestrained loading paths is very good. A) The $S_{11}$ surface fitted to the data points. B) The $S_{22}$ surface fitted to the data points. C) Best fit of the $S_{11}$ component of the loading paths. D) Best fit of the $S_{22}$ component of the loading paths.}
\label{fig:effphyfit}
\end{figure} 
%-------------------	 end FIGURE 	-------------------%
%%%%%%%%%%%%%%%%%%%%%%%%%%%%%%%%%%%%%%%%%%%%%%%%%%%%%%%%%%%%

%%%%%%%%%%%%%%%%%%%%%%%%%%%%%%%%%%%%%%%%%%%%%%%%%%%%%%%%%%%%
%-------------------	begin FIGURE 	-------------------%
\begin{figure}[hptb]
\centering
\includegraphics[width=\textwidth]{Images/chapter5/effphypred}
\caption{$\Psi_{eff}$ predicts the A) $S_{11}$ component and B) $S_{22}$ component of the unfitted loading paths very poorly even though the fit to the prestrained range is very good (Fig. \ref{fig:effphyfit}). The inset in A shows the corresponding loading paths.}
\label{fig:effphypred}
\end{figure} 
%-------------------	 end FIGURE 	-------------------%
%%%%%%%%%%%%%%%%%%%%%%%%%%%%%%%%%%%%%%%%%%%%%%%%%%%%%%%%%%%%


	

%%%%%%%%%%%%%%%%%%%%%%%%%%%%%%%%%%%%%%%%%%%%%%%%%%%%%%%%%%%%
%-------------------	begin FIGURE 	-------------------%
\begin{figure}[hptb]
\centering
\includegraphics[width=\textwidth]{Images/chapter5/effoptfit}
\caption{$\Psi_{eff}$ fit optimal loading paths very well. A) The $S_{11}$ surface fitted to the data points. B) The $S_{22}$ surface fitted to the data points. C) Best fit of the $S_{11}$ component of the loading paths. D) Best fit of the $S_{22}$ component of the loading paths.}
\label{fig:effoptfit}
\end{figure} 
%-------------------	 end FIGURE 	-------------------%
%%%%%%%%%%%%%%%%%%%%%%%%%%%%%%%%%%%%%%%%%%%%%%%%%%%%%%%%%%%%

%%%%%%%%%%%%%%%%%%%%%%%%%%%%%%%%%%%%%%%%%%%%%%%%%%%%%%%%%%%%
%-------------------	begin FIGURE 	-------------------%
\begin{figure}[hptb]
\centering
\includegraphics[width=\textwidth]{Images/chapter5/effoptpred}
\caption{Combining $\Psi_{eff}$ with optimal loading paths to predicts the A) $S_{11}$ component and B) $S_{22}$ component of the remaining unfitted loading paths very well from fit (Fig. \ref{fig:effoptfit}). The inset in B shows the corresponding loading predicted paths.}
\label{fig:effoptpred}
\end{figure} 
%-------------------	 end FIGURE 	-------------------%
%%%%%%%%%%%%%%%%%%%%%%%%%%%%%%%%%%%%%%%%%%%%%%%%%%%%%%%%%%%%





	



%%%%%%%%%%%%%%%%%%%%%%%%%%%%%%%%%%%%%%%%%%%%%%%%%%%%%%%%%%%%%
%%  Application to numerical simulations of BHVs under      %
%   quasistatic loading                                     %

\subsection{Numerical simulation of equibiaxial tension process and simulated bioprosthetic heart valve deformation}
	
    Planar biaxial test simulations were conducted to ensure that $\Psi_{eff}$ (Eqn. \ref{eqn:finalexponentialmodelformscaled}) and the elasticity tensor (Appendix \ref{sec:elasticitytensor}, Eqn. \ref{eqn:greenelasticityform}) were properly implemented in the finite element simulation framework.  We compared the computation time for both $\Psi_{eff}$ and Holzapfel-Gasser-Ogden model for biaxial simulation of bioprosthetic heart valve tissues and expectedly found no significant increase in computational cost. The total elapsed time for $\Psi_{eff}$ is 7.58 seconds in comparison to 6.40 seconds for the Holzapfel-Gasser-Ogden model, much faster than any micro-models can achieve.  

	Next we simulated tri-leaflet valves with model parameters derived from bovine pericardium, porcine aortic valve leaflet, and an idealized isotropic case. This is a simple demonstration of the use of the $\Psi_{eff}$ for the upscaling and homogenizing of micro-models. The model parameters for the bovine pericardium case were derived from the simplified structural model and model parameter of Aggarwal and Sacks \cite{aggarwal_inverse_2015}, and the resulting response matched very well qualitatively. Due to a lack of fiber mapping in the quasi-static simulation software used, some minor difference are still expected. We found no difficulty when simulating the pericardium, aortic, or isotropic valves. Suggesting that $\Psi_{eff}$ is quite robust numerically.
	
	
\begin{figure}
\centering
\includegraphics[width=\textwidth]{Images/chapter5/valvesimulations}
\caption{Simulations of intact tri-leaflet valves using A) the porcine aortic valve properties with an uniform fiber orientation distribution, B) exogenously cross-linked bovine pericardium properties with the most homogenous stress distribution, and C) the porcine aortic valve properties properties which results in a very heterogeneous stress distribution and the belly region caving in. The top row shows the side view of the valves at 80 mmHg and the bottom row shows the top-down view of the valves at the same transvalvular pressure.}
\label{fig:valvesimulations}
\end{figure}
    
    The material property has significant effects on the mechanical behaviors of the leaflets (Fig. \ref{fig:valvesimulations}). The strain distributions within the leaflets were obtained for the pressure-loaded, fully-closed configurations of the valve, and then plotted with the maximum in plane Green-Lagrange strain (MIPE). When comparing the three different material, we can see that the native aortic valve properties result in significant heterogeneities in the deformation of the leaflets (Fig. \ref{fig:valvesimulations}C). Specifically, the belly region of the leaflets significantly protrudes out, increasing the load in the surrounding regions, especially near the commissures. This results in some stress concentrations that are not conducive to heart valve durability and health in general. The bovine pericardium valve (Fig. \ref{fig:valvesimulations}B) and the isotropic valve (Fig. \ref{fig:valvesimulations}A) on the other hand have significantly more homogeneous leaflet deformations, especially from the top-down view. Both of these undergo approximately the same deformation of 0.2 in MIPE. The largest difference between the two is near the commissure regions of the valve. Where the isotropic case is under significantly higher strain. Functionally, the material properties of the exogenously cross-linked bovine pericardium are the most suitable for the valve leaflets, where it more evenly distribute the stresses. 
    
    Much of the reasons behind these differences are likely to be due to the differences between the apparent mechanical properties \textit{in vivo} and the measured mechanical response in the laboratory setting. This is especially true for the aortic valve, which is extremely anisotropic with very high compliance in the radial direction of the leaflets. This difference is most likely due to the mismatch of referential configuration between the two states. Residual strain or residual stress has significant impact on the functional properties of the leaflets, specifically the apparent anisotropy and stiffness. Collagen fiber directions and varying regional properties can also have significant impact on the functional properties of the leaflets, and thus the results of the simulation. The valve leaflet shape, root geometry and properties, the arterial or ventricular geometry and loading conditions, can all be significant factors affecting the functions of the valves and in distributing the stress in the leaflets. Furthermore, how these factors affect the fluid dynamics of the valves is also an interesting question, suitable for further study. All in all, this is meant to be a demonstration and proof of concept for using $\Psi_{eff}$ to handle a wide range of soft tissue behaviors and anisotropy for the simulation of biological organs, in this case heart valves. Further and more detailed studies will be reserved for the future.  
  

    
    



%---    Discussion
%%%%%%%%%%%%%%%%%%%%%%%%%%%%%%%%%%%%%%%%%%%%%%%%%%%%%%%%%%%%%
%%	Discussion												%
%%%%%%%%%%%%%%%%%%%%%%%%%%%%%%%%%%%%%%%%%%%%%%%%%%%%%%%%%%%%%

\section{Discussion}

%-----------------------------------------------------------
%	Model performance
%-----------------------------------------------------------
\subsection{Using the effective constitutive model for homogenization in numerical simulations}

	The most fundamental issue with using phenomenological models for soft tissue and organ numerical simulations are that they 1) cannot simulate deformation beyond the range of data used for parameter estimation, and 2) cannot be widely used for tissues other than the ones they are specifically formulated for. Without being able to fully reproduce the response of micro-models, the resulting response may become inconsistent with the mechanisms of these micro-models, impacting their ability to simulate soft tissue responses, particular when modeling time-dependent processes. In the present work we found that using $\Psi_{eff}$ (Eqn. \ref{eqn:finalexponentialmodelformscaled}) along with optimally selected loading paths reconciles this issue. $\Psi_{eff}$ demonstrates much better capabilities at fitting the mechanical response of soft tissues in general. Admittedly, this may not be especially important for simulations of soft tissues in the normal physiological range as most models can fit the response of tissues if the range of deformation is small, as demonstrated with the generalized Fung model. However, for simulating abnormal conditions such as those that will drastically alters the deformation of the tissue, then using $\Psi_{eff}$ will be much more accurate. 
    
    
    The second and equally important part is the need for optimal data to determine the model parameters. Admittedly, the amount of data needed is not necessarily extensive. For example, we have shown that just three carefully selected loading paths can greatly improve the predictive capability of $\Psi_{eff}$ over the entire range of deformations. However, when the loading paths are selected poorly, $\Psi_{eff}$ still has some issue when predicting protocol beyond the range used to fit the model. Examples of this are when only using a single protocol under equi-biaxial stress (Appendix \ref{sec:otherresults}, Fig. \ref{fig:effequifit}D), or only using protocols in the prestrained range (Fig. \ref{fig:effphypred}). Mechanisms are still the major factor limiting the predictive capability in these cases. However, the intended role of $\Psi_{eff}$ is only to homogenize the response of mechanisms-based micro-models, not to help to better understand soft tissue function. The loading paths can be simulated by choice, thus should not be a major factor affecting $\Psi_{eff}$ in numerical simulations. 
    
    
    As we have shown, $\Psi_{eff}$ is able to handle a wide range of soft tissue behavior with no change in model form. This greatly simplifies the need of implementing a different constitutive models for every tissue type, especially when the Jacobian or the elasticity tensor must be implemented separately for computational efficiency, which can be quite complex, i.e. in ABAQUS UMAT. $\Psi_{eff}$ alone is capable of fully reproducing their mechanical response for simulations without significant loss in accuracy. Thus, the use of effective constitutive models can greatly facilitate in not only the computation speed of numerical simulations, but also the speed of implementing constitutive models of different soft tissue for simulations. In these cases, only the parameters of $\Psi_{eff}$ and organ geometry needs to be changed, as we demonstrated for the simulaiton of bovine pericardiu, porcine aortic and isotropic heart valves. $\Psi_{eff}$ is smooth, easily differentiable, and easy to implement. With optimal loading path and model scaling, the process of converting micro-model response to $\Psi_{eff}$ (Eqn. \ref{eqn:finalexponentialmodelformscaled}) should take not more than a few seconds, while saving a significant amount of time during numerical simulations. 
    
    
    On the other hand, micro-models are very useful for reproducing the response of tissue to which the full microstructure are known. This avoids the need for extensive mechanical data and parameter estimation, saving a time consuming step for evaluating different material designs. Some structural and geometry information may also be measurable \textit{in vivo} due to advances in techniques such as 3D ultrasound \cite{steiner_diagnostic_1994, yang_3d_2008, fenster_3_1996} and DT-MRI \cite{basser_vivo_2000, basser_microstructural_2011}, and can be directly incorporated into meso-scale structural models. However, these techniques are not yet sufficient to determine the mechanical properties of tissues. As such, micro-models are still a necessary and important part of any predictive simulation. Not surprisingly, even most traditional invariant based models, such as the Holzapfel-Gasser-Ogden model \cite{holzapfel_new_2000}, are being extended to incorporate the microstructures of the tissue \cite{holzapfel_modelling_2015}. 
    
    

%-------	growth and remodeling	-------%
\subsection{Effective constitutive model applications}

    One application of effective constitutive models is for simulating time dependent processes, such as growth and remodeling. Growth and remodeling have been a long-time interest of the biomechanics community, and has important role in predictive simulations. Theories for growth and remodeling have been well studied, from Rodriguez in 1994 to Lanir and others in the current time \cite{lanir_mechanistic_2014, gleason_mixture_2004, rodriguez_stress_1994, humphrey_constrained_2002, cowin_tissue_2004, taber_biomechanics_1995}. The general theories for growth and remodeling involve the mechanisms for the changes in the reference configurations and a constrained mixture model involving the combined response of old original materials and new generated materials. This again multiplies the computational cost of the material models and the summation of many individual responses can significantly reduce numerical precision. Here homogenization using $\Psi_{eff}$ (Eqn. \ref{eqn:finalexponentialmodelformscaled}) can be useful. 
    

%-------	inverse modeling	-------%
    Another important application is for inverse modeling, which is important for patient specific modeling. Outside of \textit{in vitro} studies, performing the experiments necessary to determine the mechanical response of soft tissues is extremely difficult. Here, inverse modeling approaches are a solution to this problem \cite{lee_inverse_2014, aggarwal_inverse_2015, aggarwal_patient_2013, kim_inverse_2009, liu_inverse_2013}. In inverse modeling, the model parameters and the errors between the simulated and measured strains are simultaneously optimized. However, the available data that can be obtained \textit{in vivo} is limited, and is not always sufficient to accurately determine the model parameters. In these cases, the tissue microstructure can be used along with meso- and multi-scale models to narrow down the range of possible parameters. However, this multiplies the already hefty costs of the these constitutive models. Here, the approach we proposed (Fig. \ref{fig:simulationframework}B) can be used to reduce computational cost.

 


%-----------------------------------------------------------
%	Model scaling method
%-----------------------------------------------------------

\subsection{Model scaling method in other applications}
    
    Although not introduced as such, the model scaling method, or a similar technique to this, was briefly described by Fung \textit{et al.} in their original work on the mathematical modeling of arteries \cite{fung_pseudoelasticity_1979}. The paper introduced the strain energy density function as 
%==========================================================%
%-------------------	begin EQUATION 	-------------------%
\begin{equation}\label{eqn:fungarterymodel}
\rho_0 W^{(2)} = \frac{C^\prime}{2}\operatorname{exp}\left[\alpha_1 \left(E_{\theta\theta}^2 - E_{\theta\theta}^{*2} \right) + \alpha_2 \left(E_{zz}^2 - E_{zz}^{*2} \right) + \alpha_4 \left(E_{\theta\theta}E_{zz} - E_{\theta\theta}^*E_{zz}^* \right) \right]
\end{equation}
%-------------------	 end EQUATION 	-------------------%
%==========================================================%    
    in equation 2 of the said work ($C$ is changed to $C^\prime$ to consistency in notation with the present work). $E_{\theta\theta}^*$ and $E_{zz}^*$ are introduced as strains corresponding to some fixed stresses of $S_{\theta\theta}^*$ and $S_{zz}^*$, usually taken in the physiologic range. Similarly, this "scaling" can be absorbed into the parameter $C^\prime$ like in the present work. This idea was not greatly expanded upon, but \textit{Fung et al.} notes that:
\begin{quotation}
"But in practice it is very helpful to introduce $E_{\theta\theta}^*$ and $E_{zz}^*$. Not only are the values corresponding to $S_{\theta\theta}^*$ and $S_{zz}^*$ very important information, but also their use makes the constants [$C^\prime$], $\alpha_1$, $\alpha_2$, and $\alpha_4$ much more stable for each set of specimen." \cite{fung_pseudoelasticity_1979}
\end{quotation} 
    and that 
\begin{quotation}
"[$E_{\theta\theta}^*$ and $E_{zz}^*$] are indexes of compliance of the vessel. Using $E_{\theta\theta}^*$ and $E_{zz}^*$, the variations of the constants $C$, $\alpha_1$, $\alpha_2$, and $\alpha_4$, which determines the shape of the stress-strain curve, are greatly reduced. The assignment of $S_{\theta\theta}^*$ and $S_{zz}^*$ is arbitrary, but hopefully standard values will be adopted by the biomechanics community." \cite{fung_pseudoelasticity_1979}
\end{quotation}

	In truth, we did not find that the model scaling method necessarily makes $\alpha_1$, $\alpha_2$, and $\alpha_4$ more consistent, but rather that they are exactly the same values with or without this method, assuming parameter estimation was not trapped in some local minimum. The model scaling method does make reaching the values of these parameter more consistent. The biggest benefit remains the significant improvement in the correlation between the parameters $C^\prime$ and $\alpha_1$, $\alpha_2$, and $\alpha_4$, improving the conditioning of the objective function surface during parameter estimation. It also imparts some physical meaning to the value of $C^\prime$, or for $A_s$ and $c_0^\prime$ in present work. For Fung \textit{et al.}, this is some arbitrary physiologic stresses, for us, this is exactly 'maximum' (with respect to the objective function) value of strain energy within the data used for parameter estimation. This does bestow some consistency to the value of $C^\prime$, as it is exactly the total strain energy density at the stresses of $S_{\theta\theta}^*$ and $S_{zz}^*$, which will likely be similar between specimens taken from the same arteries from health subjects. However, the choice of $E_{\theta\theta}^*$ and $E_{zz}^*$, or $E_m^\mathrm{max}$, $E_n^\mathrm{max}$, and $E_\phi^\mathrm{max}$ for $\Psi_{eff}$ (Eqn. \ref{eqn:finalexponentialmodelformscaled}), should not be arbitrary. The model scaling method works due to altering the functional effect of $c_0$ and $b_i$, or $C$ and $\alpha$. $E_m^\mathrm{max}$, $E_n^\mathrm{max}$, and $E_\phi^\mathrm{max}$ should be chosen deliberately so that area under the constitutive model, based on the objective function, remains approximately the same, thus decoupling changes in modulus and changes in curvature from the exponential parameters. 


	Perhaps, the biggest advantage of the model scaling method is that it is applicable to nearly any constitutive model with an exponential function, such as models like the Holzapfel-Gasser-Ogden, Humphrey, Vito, or even the meso-scale structural model with simplified ensemble response such as in Fan and Sacks \cite{fan_simulation_2014}, Lee \textit{et al.}\cite{lee_effects_2015}, and Aggarwal and Sacks \cite{aggarwal_inverse_2015}. Even polynomial model forms with a power law, $\Psi=A\epsilon^B$, such as the generalized Ogden model, or the elastin model for the mitral valve in Zhang et al. \cite{zhang_meso_2016} can see benefits from the model scaling method. In this case, the scaling term becomes $A = \bar{A} e^{-B \log(\epsilon_{max})}$. In summary, this model scaling method should have significantly implications in improving the speed and consistency of parameter estimation for any model with an exponential like form.
    
    
    
    
%-----------------------------------------------------------
%	Optimal
%-----------------------------------------------------------	
\subsection{The equibiaxial stress protocol in optimal loading paths}

    One highlight from the optimal loading path study is that the equibiaxial stress loading path is extremely important. The equibiaxial stress loading path is always the one shown in figures for most paper, as it gives most intuitively understandable information on the mechanical properties of the tissue. It gives insights into the general form, anisotropy and stiffness of the material at a glance, and is not surprisingly also the best loading path for parameter estimation. However, it is surprising just how little the equibiaxial stress loading path can provide alone. The difference in magnitude between the D-optimality values for one vs. two loading paths is almost 20. The equibiaxial stress alone simply is not enough to determine the material parameters using phenomenological approaches. However, with the addition of one or two more protocols, even if they are along a similar loading paths can significantly improve the predictive capabilities. However, this may be partially overcome by meso-scale structural approaches, given the information on the microstructure of the tissue.


\subsection{Alternative options for optimal \textit{in silico} loading paths}

	The limitation on predicting outside of the loading paths used for parameter estimation can be somewhat remedied by densely sampling the response of the micro-model over a larger range of deformations. However, this is not entirely ideal for computational speed during parameter estimation, and choosing the sampling points for parameter estimation is not a trivial task itself. Points with high stresses tend to weight heavily during parameter estimation, proper care needs to be taken to capture both the high stress and low stress response. Given that the number of data points scales cubically with the distance between data points, this approach is still limited. 

    
    Having said this, our investigation of optimal loading paths is restricted to constant strain ratios or constant stress ratios. In reality, there are many ways to define loading paths, some can be quite creative. We do not deny the possibility of other forms of loading paths that are more optimal. However, the current approach with three, or at most five protocols, is already sufficient. We did test some alternative loading paths, such as Fung \textit{et al.}'s post pre-strain loading paths \cite{fung_pseudoelasticity_1979}. They cover much of the physiological range, but are still insufficient for parameter estimation. Increasing the number of loading paths in this case has minor improvements, but pales in comparison to just picking better types of loading paths. The poor predictive capabilities for the low stress region can have significant impacts on underestimating the mechanical properties of matrix and elastin, and their properties can be important to the functions of micro-models. The mechanical properties of the matrix for example, having significant implications for simulating the process of permanent set in exogenously cross-linked soft tissues \cite{zhang_modeling_2017}. Failing to properly reproduce this response, can affect the predictive capabilities of the associated micro-models, causing the whole framework of using $\Psi_{eff}$ to facilitate numerical simulations (Fig. \ref{fig:simulationframework}B) to fall apart. 
    
    










%---    Conclusion
%%%%%%%%%%%%%%%%%%%%%%%%%%%%%%%%%%%%%%%%%%%%%%%%%%%%%%%%%%%%%
%%	Conclusion												%
%%%%%%%%%%%%%%%%%%%%%%%%%%%%%%%%%%%%%%%%%%%%%%%%%%%%%%%%%%%%%

% \section{Limitations and conclusion}

%-----------------------------------------------------------
%	Limitations
%-----------------------------------------------------------
\section{Limitations} 
	
    One major limitation of $\Psi_{eff}$ (Eqn. \ref{eqn:finalexponentialmodelformscaled}) is the larger number of parameters, 14 in the fully generalized form. This is not very favorable, where the time complexity for most optimization algorithms scales nonlinearly with the number of parameters. However, $\Psi_{eff}$ has very low computational cost, and reasonably low parameter covariance, thus this should not be a major problem. Alternatives are also less favorable, as they either require more parameters or cannot sufficiently capture the response of soft tissues in a wide range or reproduce the response of multiple tissue types. 
    
    Another limitation, which also applies to all phenomenological models, is that $\Psi_{eff}$ has no intrinsic mechanisms built in. Without sufficient mechanical data to derive the model parameters, phenomenological models have limited predictive capabilities. Specifically, the phenomenological models do poorly when extrapolating outside of the range of available data. This also means that phenomenological models can only provide limited information on how the tissue functions. It can reproduce the mechanical response of soft tissues very well, but it is also harder to infer more about the structure and function of the tissue. This is not a major concern for us. For the mechanisms, or the structure to function relationship of soft tissues, micro-models already fulfills the need. $\Psi_{eff}$ is intended as a fit all model for fulfilling the gap between predictive micro-models and computationally efficient simulations. With carefully selected of loading paths for parameter estimation (Section \ref{sec:optimaldesign}), $\Psi_{eff}$ can accurately reproduce the mechanical response of soft tissue within the expected range. In other words, $\Psi_{eff}$ does not have to be able to predict the mechanical response of soft tissues under unmeasured and extrapolated deformations, it only has to be able to fully reproduce the entire range of responses predicted by the micro-models, which is its main purpose. 



%-----------------------------------------------------------
%	Conclusion
%-----------------------------------------------------------
\section{Conclusion and Future Directions} 

	In this work, we developed a constitutive model form for the effective response of planar soft tissues. This effective constitutive model (Eqn. \ref{eqn:finalexponentialmodelformscaled}) is applicable to a wide range of soft tissue responses, while being as computationally efficient as most common phenomenological approaches, such as Holzapfel-Gasser-Ogden or the generalized Fung model. This model utilizes the modeling scaling method for minimizing covariance between parameter, which shows significant improvements in the speed and accuracy of parameter estimation. We have shown that our effective constitutive model along with optimal loading paths is able to fully replicate the response of complex meso-scale structural models for the entire range of deformations, whereas most phenomenological models have difficulties when predicting unfitted loading paths. The effective constitutive model is robust enough to be able to handle a wide range of soft tissue behaviors and anisotropy for accurate numerical simulations, such as in simulations of heart valves. Thus, the effective constitutive model can play an important role for the upscaling and homogenization of the response of complex micro-models for improving the efficiency for organ-level simulations. 
    

	One nature extension to this effective modeling approach is for 3 dimensional soft tissues. The extension to 3 dimensions doubles the number of inputs in comparison to planar models, with the additional inputs being $E_{13}$, $E_{23}$, and $E_{33}$. This means that the initial most generalized form for the 3-D soft tissue models (Eqn. \ref{eqn:exponentialmodelform}) has 209 terms before reduction. This is unmanageable for establishing the initial approach, where using a planar soft tissue model is more suitable. However, applying the same restrictions to the model form (section \ref{sec:finalform}) reduces this to 48 terms. This is possible due to symmetry by expressing the components of the Green-Lagrange strain tensor with respect to the material axis (Eqn. \ref{eqn:greenstrain}). This is not so easy to do in 3-dimensions, requiring a third vector corresponding to the direction of least stiffness, which in turn requires the 3 dimensional fiber orientation distribution. 
    
    
    
    
    
    
    
    




\newpage

%---    Bioliography
\bibliographystyle{plainnat}
\bibliography{phd}

\chapter{Numerical simulation of intact bioprosthetic heart valves}

\section*{Preface}
\addcontentsline{toc}{section}{Preface}%

    Bioprosthetic heart valves (BHVs), fabricated from exogenously crosslinked collagenous tissues, remain the most popular heart valve replacement design. However, the life span of BHVs remains limited to 10–15 years, in part because the mechanisms that underlie BHV failure remain poorly understood. The current process for evaluating BHV designs is an expensive and time consuming three-stage process: 1) accelerated wear testing(AWT), 2) large animal studies, and 3) clinical trials. AWT is currently the only way to evaluate BHV durability in a feasible amount of time (months instead of years). However, the loading conditions and environment during AWT are not physiological and the only durability information currently used is the presence of visible damage. Only the clinical evaluation stage can provide true indications of the \textit{in vivo} performance of BHV designs. However, this is the final, most difficult, most expensive and most time consuming stage. Clearly, current methods of evaluating BHV designs are not feasible for advancing the BHV technology in a timely manner. In this chapter, we implement our constitutive model for the time evolving properties of BHVs in response to permanent set developed in a previous chapter in a numerical simulation framework, and attempt to predict the geometric and structural change that occurs with the onset of permanent set. 




%---    INTRODUCTION
\section{Introduction}

\subsection{Background}
    The most popular replacement heart valves continue to be bioprosthetic heart valves (BHV) fabricated from xenograft biomaterials, for both established surgical and more novel percutaneous valve designs \cite{schoen_evolving_2008,schoen_new_2006,schoen_cardiac_2005,schoen_calcification_2005,schoen_pathology_2001}. While these devices have provided great benefits for many patients, device failure continues to be the result of leaflet structural deterioration mediated by fatigue and/or tissue mineralization \cite{vesely_tissue_2001,sacks_collagen_2002}. As a result, the life span of BHVs remains limited to 10–15 years, and the mechanisms that underlie such failures remain poorly understood. Thus, improved durability remains an important clinical goal and represents a unique cardiovascular engineering challenge resulting from the extreme valvular mechanical demands. Yet, current BHV assessments rely exclusively on device-level evaluations, which are confounded by simultaneous and highly coupled biomaterial mechanical fatigue, valve design, hemodynamics, and calcification. Thus, despite decades of clinical BHV usage and growing popularity, there exists no acceptable method for simulating BHV durability in any design context. While efforts to mitigate tissue mineralization have progressed well, efforts to minimize structural degradation remain limited. For example, in vitro accelerated wear tests (AWT) are used only for the most basic device level durability evaluations. Yet no systematic attempt has been made to glean important durability information from these required tests to inform mechanisms and means to simulate BHV durability. There is thus a profound need for the development of a novel simulation technologies for the prediction of the changes in the properties of BHVs to long-term cyclic loading.
    
    
\subsection{Mechanisms of BHV failure}

	The causes of BHV failure can be divided into two broad categories, mineralization and structural damage, with both processes occurring in parallel or independently \cite{sacks_collagen_2002}. Mineralization is the accumulation of mineral deposits, mainly calcium phosphate, within the BHV leaflets \cite{schoen_calcification_2005}. This disrupts the underlying microstructure preventing the proper mechanical function of BHVs, increasing the likelihood of tearing, and reducing flexibility (preventing normal opening and closing motions, and induce valve stenosis). The causes of calcification and associated anti-calcification treatments are extensively studied in literature \cite{park_novel_1997, isenburg_tannic_2005, vyavahare_prevention_1997}. On the other hand, structural damage includes the collagen fiber damage and breakdown of the non-fibrous part of extracellular matrix (ECM), \emph{which we will refer as simply the matrix}. Fourier transform infrared spectroscopy(FTIR) results have shown changes in the collagen fiber molecular structure after 50 million cycles \cite{sun_response_2004}, which suggests that some collagen fiber damage has occurred during this period. However, its effect on the mechanical response of BHVs was not detectable. Nevertheless, it is important to maintain the structural integrity of the BHV, as this will help to improve BHV durability.
	

\subsection{Constitutive model of BHV biomaterial under permanent set}

    In previous chapters, we developed a constitutive model for the underlying mechanical properties of collagenous soft tissues used to fabricate BHVs (Chapter 2, 3) as well as the extensions for the evolution of the material properties in response to permanent set (Chapter 4). The constitutive model utilizes the mechanisms at the meso-scale (i.e. at the level of the fiber, 10-100 $\mu$m in length scale), takes into account the layered structure of these tissue and the contributions from the distinct collagen and elastin fiber networks within each tissue layer to predict the mechanical response at the tissue scale. This approach was further extend for the effect of crosslinking on the fiber matrix interactions to simulate exogenously crosslinked materials. Requisite collagen and elastin fibrous structural information can be obtained using microscopy and conventional histology, and the approach was validated using a novel fibril-level (0.1 to 1 $\mu$m) approach to compare the predicted collagen fiber/fibril mechanical behavior with extant MV small angle X-ray scattering data. 
    
    
    Permanent set is the significant changes in geometry of BHVs observed in experimental studies \cite{smith_fatigue_1999}\cite{smith_high_1997} with in vivo operation. This can lead to stress concentrations that can have significant impact on structural damage. Structural damage was not detected in the time frame when permanent set is the most dominant. We hypothesize that the scission-healing reaction of glutaraldehyde is the underlying mechanism responsible for permanent set in exogenously crosslinked soft tissues. The continuous scission-healing process of glutaraldehyde allows a portion of the exogenously crosslinked matrix, which is considered to be the non-fibrous part of the extra-cellular matrix, to be re-crosslinked in the loaded state. We model the permanent set effect in chapter 4 by assuming that the exogenously crosslinked matrix undergoes changes in reference configurations over time. The changes in the collagen fiber architecture due to dimensional changes allow us to predict subsequent changes in mechanical response. A key finding we have is that the collagen fiber architecture has a limiting effect on the maximum changes in geometry that the permanent set effect can induce. This is due to the recruitment of collagen fibers as the changes in geometry due to permanent set increase. This means we can potentially optimize the BHV geometry based on the predicted the final BHV geometry after permanent set has largely ceased. 
    

\subsection{Numerical simulation framework}

    However, numerical simulations using our constitutive model for permanent set is quite costly. Due to the incorporation of the quadruple integral to calculate fiber interactions, the resulting computational cost is five magnitudes higher than conventional phenomenological approaches. It is for this reason that we developed an effective constitutive model for planar soft tissues, and a parameter estimation approach for rapidly determining the model parameters from the micro-model. This approach takes advantage of the computationally efficient forms of phenomenological models and the mechanisms and predictive capabilities of micro-models (meso-scale, multi-scale, or other complex constitutive models) (Fig. \ref{fig:simulationframework}A). For each iteration of the simulation, whether forward, inverse, or time-evolving, the effective constitutive model is first fit to the micro-model(s) to determine the model parameters, then it is used to performed the actual numerical simulation. Subsequent updates to the evolving material properties, geometry and boundary conditions are then performed (Fig. \ref{fig:simulationframework}B). This approach can greatly improve the computational efficiency of numerical simulations and make it possible for a finite element implementation of the permanent set simulation (Chapter 4). In this chapter, we develop and numerical simulation framework for the time evolving properties of BHVs in response to permanent set, and attempt to predict the geometric and structural change that occurs with long term cyclic loading. 
    
%%%%%%%%%%%%%%%%%%%%%%%%%%%%%%%%%%%%%%%%%%%%%%%%%%%%%%%%%%%%
%-------------------	begin FIGURE 	-------------------%
\begin{figure}
\centering
\includegraphics[width=\textwidth]{Images/chapter5/simulationframework}
\caption{Proposed frame work for using an effective constitutive model to improve the efficiency of using complex meso- or multi-scale models (micro\Hyphdash models) in numerical simulations. Here, A) effective constitutive models act as an intermediate step between micro-models and numerical simulations, where micro-models inform the changes to the effective constitutive model while the effective constitutive model for the simulation. B) An example of how this may be implemented for time\Hyphdash evolving is shown.}
\label{c6:fig:simulationframework}
\end{figure}
%-------------------	 end FIGURE 	-------------------%
%%%%%%%%%%%%%%%%%%%%%%%%%%%%%%%%%%%%%%%%%%%%%%%%%%%%%%%%%%%%
    
    

%---    METHODS
\section{Methods}

\subsection{Overall framework}

    To implement the permanent set model (chapter 4) for numerical simulations, we will utilize the effective constitutive model approach (chapter 5). This implementation has 4 main components (Fig. \ref{c6:fig:pssimoverview}): 1) initial state model, 2) quasistatic simulation, 3) updates to the material properties in response to permanent set and 4) updates to the finite element model geometry in response to permanent set. 
    
    
%%%%%%%%%%%%%%%%%%%%%%%%%%%%%%%%%%%%%%%%%%%%%%%%%%%%%%%%%%%%
%-------------------	begin FIGURE 	-------------------%
\begin{figure}
\centering
\includegraphics[width=5.5in]{Images/chapter6/pssimoverview.pdf}
\caption{Proposed frame work for simulation the evolving properties of BHVs in response to permanent set.}
\label{c6:fig:pssimoverview}
\end{figure}
%-------------------	 end FIGURE 	-------------------%
%%%%%%%%%%%%%%%%%%%%%%%%%%%%%%%%%%%%%%%%%%%%%%%%%%%%%%%%%%%%

\subsection{Initial state model}

    In this part the initial parameters of the simulation is established. The 3 main components are: 1) finite element mesh for BHV geometry, 2) leaflet material properties, and 3) mapped collagen fiber architecture. For the BHV geometry, our group developed a pipline using micro-CT to measure the 3-dimensional geometry of the BHV and fitting the atrial surface points to a NURBS mesh using the grash hopper plugin in Rhino (Copyright Robert McNeel \& Associates) (Fig. \ref{c6:fig:fegeometry}). We also previously developed an approach for mapping the collagen fiber architecture to the finite element mesh \cite{aggarwal_patient_2013,aggarwal_inverse_2015}. However, due to the lack of available experimental data, we used the Edwards valve geometry from \cite{aggarwal_inverse_2015}, bovine pericardium material properties from \cite{sacks_novel_2016}, and circumferential aligned collagen fiber orientation distributions.
    
%%%%%%%%%%%%%%%%%%%%%%%%%%%%%%%%%%%%%%%%%%%%%%%%%%%%%%%%%%%%
%-------------------	begin FIGURE 	-------------------%
\begin{figure}
\centering
\includegraphics[width=5.5in]{Images/chapter6/fegeometry.pdf}
\caption{Pipeline for converting micro-CT data for each reference and loading states to finite element meshs and geometry.}
\label{c6:fig:fegeometry}
\end{figure}
%-------------------	 end FIGURE 	-------------------%
%%%%%%%%%%%%%%%%%%%%%%%%%%%%%%%%%%%%%%%%%%%%%%%%%%%%%%%%%%%%

\subsection{Quasistatic simulation of bioprosthetic heart valves}

	For the quasistatic simulation, which is necessary to determine the current loaded state, we will utilized the effective model approach (Fig. \ref{c6:fig:simulationframework}) established in chapter 5. For finite element model, we utilized the custom finite element simulation software developed by Hsu \textit{et al.} \cite{hsu_dynamic_2015, kamensky_immersogeometric_2015, kiendl_isogeometric_2015, wu_anisotropic_2018}. Briefly, the finite element code was developed for isogeometric fluid solid dynamics simulation of heart valves, focusing mostly on the tri-leaflet atrioventricular valves. The tri-leaflet geometry is based on the commonly used Edwards Pericardial Heart Valve with Kirchhoff-Love shells for the leaflets \cite{kiendl_isogeometric_2015} and finite element solver developed in by Kamensky \textit{et al.} \cite {kamensky_immersogeometric_2015}. We utilized this code to simulate leaflet deformation under physiological quasi--static transvalvular pressure. 
    A total was 484 B\'ezier elements was used for each leaflets, with a leaflet density of 1.0 g/cm$^3$ and a uniform leaflet thickness of 0.386mm thick \cite{hsu_dynamic_2015}. Contact between leaflets is handled by a penalty-based approach and imposed at quadrature points of the shell structure, and clamped boundary condition is applied to the leaflet attachment edge. 
    For simplicity and consistency, the collagen fiber direction was assume to be aligned to the circumferential direction of each leaflet. No root, atrial chamber or the surrounding artery was used. The bioprothetic heart valve stent was made rigid and undeformable, serving a stationary reference for the leaflets. A similar validation process to the one presented by Wu \textit{et al.} \cite{wu_anisotropic_2018} to verify the implementation.
	
	
	For the constitutive model for the leaflet material, we used the effective material model (chapter 5)
	%==========================================================%
%-------------------	begin EQUATION 	-------------------%
\begin{equation} \label{c6:eqn:finalexponentialmodelformscaled}
\begin{aligned}
\Psi_{eff} 	=& c_0 \left(e^{Q} - 1\right) = c_0^\prime e^{-Q_{max}}\left(e^{Q} - 1\right)    \\
Q		=& b_1 E_m^2 + b_2 E_n^2 + b_3 E_\phi^2 + b_4 E_m E_n + b_5 E_m^4 + b_6 E_n^4 + b_7 E_m^3 E_n + b_8 E_m^2 E_n^2 \\ 
&+ b_9 E_m E_n^3 + b_{10} E_\phi^4 + b_{11} E_m^2E_\phi^2 + b_{12} E_n^2 E_\phi^2 + b_{13} E_m E_n E_\phi^2 \\
Q		=& b_1 (E_m^{max})^2 + b_2 (E_n^{max})^2 + b_3 (E_\phi^{max})^2 + b_4 (E_m^{max}) (E_n^{max}) + b_5 (E_m^{max})^4   \\
    &+ b_6 (E_n^{max})^4 + b_7 (E_m^{max})^3 (E_n^{max}) + b_8 (E_m^{max})^2 (E_n^{max})^2 + b_9 (E_m^{max}) (E_n^{max})^3	\\
	&+ b_{10} (E_\phi^{max})^4 + b_{11} (E_m^{max})^2(E_\phi^{max})^2 + b_{12} (E_n^{max})^2 (E_\phi^{max})^2    \\ 
	&+ b_{13} (E_m^{max}) (E_n^{max}) (E_\phi^{max})^2,
\end{aligned}
\end{equation}
%-------------------	 end EQUATION 	-------------------%
%==========================================================%
	to homogenize the mechanical response of the permanent set model. The resulting elasticity tensor for can be found in Appendix \ref{sec:elasticitytensor}, Eqn. \ref{eqn:greenelasticityform}). To test the finite element implementation, and the effective constitutive model approaches, we performed several quasi-static simulations of BHVs with different leaflet materials properties. The main properties considered are bovine pericardium (most commonly used for bioprothetic heart valve leaflets), porcine aortic valve (highly anisotropy response), and bovine pericardium with an uniform fiber ODF (isotropic). We replicated the response of these tissues using the static part of the permanent set constitutive model for collagenous soft tissues (Eqn. \ref{c6:eq:fullEXLmodel}) based on their microstructure. We then fit $\Psi_{eff}$ (Eqn. \ref{c6:eqn:finalexponentialmodelformscaled}) their response by sampling along optimal loading paths. Next, we evaluated the computational cost and numerical robustness of $\Psi_{eff}$ and its ability to handle a wide range of material properties and the complex \textit{in vivo} deformations in numerical simulations.
    

\subsection{Constitutive model for the evolving material properties under permanent set}

    The detailed theory for the constitutive model for permanent set is presented in chapter 4. Briefly, the constitutive model consists of three parts: collagen, matrix, and interactions,
%==========================================================%
%-------------------	begin EQUATION 	-------------------%
\begin{equation}
\Psi 	= \Psi_\mathrm{col} + \Psi_\mathrm{mat} + \Psi_\mathrm{int} \label{c6:eqn:structuralmodelcomponents}. 
\end{equation}
%-------------------	 end EQUATION 	-------------------%
%==========================================================%
    The time evolving mechanism is based on the work by Rajagopal and Wineman \cite{rajagopal_constitutive_1992}, where we assume that the response of the EXL matrix is a constrained mixture model consisting of the origin fraction being continuously converted to new fractions with a reference state in the current loaded configuration. 
\begin{equation} \label{c6:eq:wineman}
\phi_m \mathbf{S}_m = b(s)\mathbf{\bar{S}}_m^\mathrm{existing} + \int\displaylimits_0^s a(s,\hat{s})\mathbf{\bar{S}}_m^\mathrm{new} \mathrm{d}\hat{s},
\end{equation}

    The final model form as a function of the permanent set rate constant $k $, the permanent set deformation $\mathbf{F}_\mathrm{PS}$, the strain history $\mathbf{A}(s)$, and the material parameters of the constitutive model in the uncycled state. The input of the model is the applied deformation $\mathbf{C}$ referenced to the current unloaded state $\Omega_\mathrm{PS}$, given by the deformation $\mathbf{F}_\mathrm{PS}$ from $\Omega_0$. The full form is
\begin{equation}\label{c6:eq:fullEXLmodel}
\mathbf{S} = \mathbf{S}\left(k , \mathbf{F}_\mathrm{PS}, \mathbf{A}(\hat{s}), \mathbf{C}\right) = \phi_\mathrm{col} \left[ \mathbf{S}_\mathrm{col} + \mathbf{S}_\mathrm{int}\right] + \phi_m \mathbf{S}_\mathrm{m},
\end{equation}
where the collagen contribution is 
\begin{equation} \label{c6:eq:fullcollagen}
\begin{split}
\phi_\mathrm{col}\mathbf{S}_\mathrm{col}&\left(k , \mathbf{F}_\mathrm{PS}, \mathbf{A}(\hat{s}), \mathbf{C}\right) \\
&= \phi_\mathrm{col} \eta_C \int\displaylimits_\theta \Gamma_1(\mathbf{F}_{\mathrm{PS}}, \theta)\left\lbrace 
\int\displaylimits_1^{\lambda_\theta} \frac{D_1\left( \mathbf{F}_{\mathrm{PS}}, x \right)}{x} \left( \frac{1}{x}- \frac{1}{\lambda_\theta}\right) \mathrm{d}x \right\rbrace \mathbf{n}_\theta\otimes\mathbf{n}_\theta \mathrm{d}\theta,
\end{split}
\end{equation}
where $\lambda_\theta = \sqrt{\mathbf{n}_\theta \cdot \mathbf{C}\mathbf{n}_\theta}$ is the stretch of the fiber ensemble oriented along $\theta$, the fiber ensemble interactions is 
\begin{equation} \label{c6:eq:fullinteractions}
\begin{split}
\phi_\mathrm{int}\mathbf{S}_\mathrm{int}&\left(k , \mathbf{F}_\mathrm{PS}, \mathbf{A}(\hat{s}), \mathbf{C}\right) \\
=& \phi_\mathrm{col} \eta_\mathrm{int} \int\displaylimits_\alpha \int\displaylimits_\beta \Gamma_1 \left(\mathbf{F}_\mathrm{PS}, \alpha \right) \Gamma_1 \left(\mathbf{F}_\mathrm{PS},  \beta \right) \\
&\times\left[ \left\lbrace 
\int\displaylimits_1^{\lambda_\alpha} \int\displaylimits_1^{\lambda_\beta} 
\frac{2 \lambda_\beta D_1(\mathbf{F}_\mathrm{PS}, x_\alpha) D_1(\mathbf{F}_\mathrm{PS}, x_\beta)}{x_\alpha x_\beta} 
\left( \frac{\lambda_\alpha}{x_\alpha} \frac{\lambda_\beta}{x_\beta} - 1\right) \mathrm{d}x_\alpha \, \mathrm{d}x_\beta \right.\right. \\
&+ \left. \left. \int\displaylimits_1^{\lambda_\beta} D_1(\mathbf{F}_\mathrm{PS}, x_\beta) \left( \frac{\lambda_\beta}{x_\beta} -1  \right)^2 \mathrm{d}x_\beta \right\rbrace \right.  \frac{\mathbf{n}_\alpha \otimes \mathbf{n}_\alpha}{\lambda_\alpha}  \\
&+ \left. \left\lbrace
\int\displaylimits_1^{\lambda_\alpha} \int\displaylimits_1^{\lambda_\alpha} 
\frac{2 \lambda_\beta D_1(\mathbf{F}_\mathrm{PS}, x_\alpha) D_1(\mathbf{F}_\mathrm{PS}, x_\beta)}{x_\alpha x_\beta} 
\left( \frac{\lambda_\alpha}{x_\alpha} \frac{\lambda_\beta}{x_\beta} - 1\right) \mathrm{d}x_\alpha \, \mathrm{d}x_\beta 
\right. \right. \\
&+\left. \left. \int\displaylimits_1^{\lambda_\alpha} D_1(\mathbf{F}_\mathrm{PS}, x_\alpha) \left( \frac{\lambda_\alpha}{x_\alpha} -1  \right)^2 \mathrm{d}x_\alpha \right\rbrace \frac{\mathbf{n}_\beta \otimes \mathbf{n}_\beta}{\lambda_\beta}  \right] \mathrm{d}\alpha \, \mathrm{d}\beta,
\end{split}
\end{equation}
and the EXL matrix is
\begin{equation} \label{c6:eq:fullmatrix}
\begin{split}
\phi_m \mathbf{S}_\mathrm{m}&\left(k , \mathbf{F}_\mathrm{PS}, \mathbf{A}(\hat{s}), \mathbf{C}\right) \\
&= \phi_m \eta_m \left[ \vphantom{\int\displaylimits_0^s} \mathrm{Exp}\left[-k  \cdot s\right]  \left(\left( \bar{I_1} (\mathbf{F}_\mathrm{PS}, \mathbf{A}(0)) - 3\right)^{\alpha - 1} + r \left( \bar{I_1} (\mathbf{F}_\mathrm{PS}, \mathbf{A}(0)) - 3\right)^{\beta - 1}\right)  \right.\\
&\times \left( \mathbf{\tilde{B}}(\mathbf{F}_\mathrm{PS}, \mathbf{A}(0))^{-1} - \tilde{B}_{33}^{-1}(\mathbf{F}_\mathrm{PS}, \mathbf{A}(0))C_{33}\mathbf{C}^{-1}\right) \\
&+ \int\displaylimits_0^s k \cdot \mathrm{Exp}\left[-k (s - \hat{s})\right] \left(\left( \bar{I_1} (\mathbf{F}_\mathrm{PS}, \mathbf{A}(\hat{s})) - 3\right)^{\alpha - 1} + r \left( \bar{I_1} (\mathbf{F}_\mathrm{PS}, \mathbf{A}(\hat{s})) - 3\right)^{\beta - 1}\right) \\
&\times \left. \vphantom{\int\displaylimits_-^s} \left( \mathbf{\tilde{B}}(\mathbf{F}_\mathrm{PS}, \mathbf{A}(\hat{s}))^{-1} - \tilde{B}_{33}^{-1}(\mathbf{F}_\mathrm{PS}, \mathbf{A}(\hat{s}))C_{33}\mathbf{C}^{-1}\right) \mathrm{d}\hat{s}\right].
\end{split}
\end{equation}

\subsection{Geometry update in response to permanent set}

    Although the permanent set model can find the local change in reference geometry, ($\mathbf{F}_\mathrm{PS}$), can be determined from equation \ref{c6:eq:fullmatrix} using
\begin{equation}\label{c6:eq:optimization}
\begin{gathered}
\mathbf{F}_\mathrm{PS} = \operatorname*{arg\,min}_\mathbf{F} \left\Vert \mathbf{S}\left(k , \mathbf{I}, \mathbf{A}(\hat{s}), \mathbf{C}=\mathbf{F}^\mathsf{T}\mathbf{F}\right) - 0 \right\Vert.
\end{gathered}
\end{equation}
    This is only local and will not be sufficient for generating a compatible mesh. To generate the updated reference configuration, we will do this by simulation. First the local change in geometry is found (Eqn. \ref{c6:eq:optimization}). Next, we will compute an equivalent stress using equation \ref{c6:eq:fullEXLmodel}, by
\begin{equation}\label{c6:eq:psstress}
\begin{gathered}
\mathbf{S}_\mathrm{PS} = \mathbf{S}(\mathbf{F}_\mathrm{PS}).
\end{gathered}
\end{equation}
    This permanent set stress, $\mathbf{S}_\mathrm{PS}$, is equivalent of a growth stress, which is then added to the weak form presented in equation 1 of Wu et al. \cite{wu_anisotropic_2018} as 
\begin{equation}\label{c6:eq:weakform}
\begin{aligned}
\int_{\Gamma_0} \mathbf{w}\cdot\rho h_{th} \left.\dpd[2]{y}{t}\right|_\mathbf{X} \dif\Gamma + 
\int_{\Gamma_0} \int_{-h_{th}/2}^{h_{th}/2}\delta\mathbf{E}:(\mathbf{S} + \mathbf{S}_\mathrm{PS}) \dif \xi^3\dif\Gamma&  \\
- \int_{\gamma_0} \mathbf{w}\cdot\rho h_{th}\mathbf{f}\dif\Gamma -\int_{\Gamma_t} \mathbf{w}\cdot\mathbf{h}\dif\Gamma = 0&
\end{aligned}
\end{equation}   
    The resulting control point displacements are then added to the finite element mesh from the previous time step to generate the new mesh in the new reference configuration. 
    

\subsection{Complete implementations with Python wrapper}

    The complete implementation of the time evolving simulation is done using a Python3 wrapper (Fig. \ref{c6:fig:pythonimplementation}) of the different components: 
    \begin{enumerate}[label=\Alph*]
        \item Quasi-static (QS) simulation code to obtain the NURBS control point displacement data for different loading conditions
        \item A post processor for translate displacement data to strain data at each gauss point
        \item The constitutive model for determine the change in material properties due to permanent set gauss point
        \item Parameter estimation code for determining the effective model parameters at each time step and gauss point
        \item Python code for updating the NURBS finite element mesh after each time step
        \item Python code for updating the collagen fiber orientation after each time step
    \end{enumerate}
    To start, we will use the Edwards valve geometry from \cite{aggarwal_inverse_2015}, bovine pericardium material properties from \cite{sacks_novel_2016}, and circumferential aligned collagen fiber orientation distributions. Next, the following process are iterated:
    \begin{enumerate}
        \item Perform QS simulation using the finite element code to determine the loading state
        \item Using the post processor to convert control point displacement data to local strain data, $\mathbf{A}(\Hat{s})$, at the current time $\Hat{s}$
        \item Append the strain in the loaded configuration, $\mathbf{A}(\Hat{s})$, to the full strain history $\mathbf{A}(s)$
        \item Update the permanent set constitutive model (Eqn. \ref{c6:eq:fullEXLmodel})
        \item Use equation \ref{c6:eq:fullEXLmodel} to compute the local $\mathbf{F}_\mathrm{PS}$, 
        $\mathbf{S}_\mathrm{PS}$, and mechanical data along optimal loading paths (chapter \ref{sec:optimaldesign})
        \item Use the local permanent set deformation to convect the collagen fiber orientation distribution $\Gamma_i$ in the current state to $\Gamma_{i+1}$ in the post permanent set state using 
        %-------------------	begin EQUATION 	-------------------%
        \begin{equation}\label{c6:eqn:45}
        \begin{aligned}
        \Gamma_{i+1}[\mu_\Gamma,\sigma_\Gamma, \theta_{i+1}] = \Gamma_i[\mu_\Gamma, \sigma_\Gamma, \theta_i(\prescript{i+1}{i}{\mathbf{F}},\theta_{1+1})\frac{\prescript{i+1}{i}{\lambda}_{\theta_i}^2}{\prescript{i+1}{i}{J_\mathrm{2D}}}].
        \end{aligned}
        \end{equation}
        %-------------------	 end EQUATION 	-------------------%
        Note that the angle $\theta_1$ of a fibre originally oriented at $\theta_0$ can be determined using
        %-------------------	begin EQUATION 	-------------------%
        \begin{equation}\label{c6:eqn:46}
        \begin{aligned}
        \theta_{i+1}(\prescript{i+1}{0i}{\mathbf{F}},\theta_i) = \tan^{-1}\left(\frac{\prescript{i+1}{i}{F}_{21}\cos{\theta_i} + \prescript{i+1}{i}{F}_{22}\sin{\theta_i}}{\prescript{i+1}{i}{F}_{11}\cos{\theta_i} + \prescript{i+1}{i}{F}_{12}\sin{\theta_i}}\right)
        \end{aligned}
        \end{equation}
        %-------------------	 end EQUATION 	-------------------%
        \item[7a] Perform QS simulation using the finite element code with $\mathbf{S}_\mathrm{PS}$ as the loading condition
        \item[7b] Added the control point displacement data to the current mesh geometry
        \item[8] Determine the new effective constitutive model parameters by performing parameter estimation on the optimal loading path data using equation \ref{c6:eqn:finalexponentialmodelformscaled}
        \item[9] Go to step 1 and rerun the QS simulation using the new mesh geometry, effective constitutive model parameters, and collagen fiber architecture
    \end{enumerate}
    This loop is run for the desired number of cycles. 
%%%%%%%%%%%%%%%%%%%%%%%%%%%%%%%%%%%%%%%%%%%%%%%%%%%%%%%%%%%%
%-------------------	begin FIGURE 	-------------------%
\begin{sidewaysfigure}
\centering
\includegraphics[width=8.2in]{Images/chapter6/pythonimplementation.pdf}
\caption{The python wrapper for communicating and execution of the different component of the time dependent simulation framework.}
\label{c6:fig:pythonimplementation}
\end{sidewaysfigure}
%-------------------	 end FIGURE 	-------------------%
%%%%%%%%%%%%%%%%%%%%%%%%%%%%%%%%%%%%%%%%%%%%%%%%%%%%%%%%%%%%

%---    Results
\section{Results and discussion}


%%%%%%%%%%%%%%%%%%%%%%%%%%%%%%%%%%%%%%%%%%%%%%%%%%%%%%%%%%%%%
%%  Application to numerical simulations of BHVs under      %
%   quasistatic loading                                     %

\subsection{Numerical simulation of bioprosthetic heart valve deformation}
	
    Planar biaxial test simulations were conducted to ensure that $\Psi_{eff}$ (Eqn. \ref{c6:eqn:finalexponentialmodelformscaled}) and the elasticity tensor (Appendix \ref{sec:elasticitytensor}, Eqn. \ref{eqn:greenelasticityform}) were properly implemented in the finite element simulation framework.  We compared the computation time for both $\Psi_{eff}$ and Holzapfel-Gasser-Ogden model for biaxial simulation of bioprosthetic heart valve tissues and expectedly found no significant increase in computational cost. The total elapsed time for $\Psi_{eff}$ is 7.58 seconds in comparison to 6.40 seconds for the Holzapfel-Gasser-Ogden model, much faster than any micro-models can achieve.  

	Next we simulated tri-leaflet valves with model parameters derived from bovine pericardium, porcine aortic valve leaflet, and an idealized isotropic case. This is a simple demonstration of the use of the $\Psi_{eff}$ for the upscaling and homogenizing of micro-models. The model parameters for the bovine pericardium case were derived from the simplified structural model and model parameter of Aggarwal and Sacks \cite{aggarwal_inverse_2015}, and the resulting response matched very well qualitatively. Due to a lack of fiber mapping in the quasi-static simulation software used, some minor difference are still expected. We found no difficulty when simulating the pericardium, aortic, or isotropic valves. Suggesting that $\Psi_{eff}$ is quite robust numerically.
	
	
\begin{figure}
\centering
\includegraphics[width=\textwidth]{Images/chapter5/valvesimulations}
\caption{Simulations of intact tri-leaflet valves using A) the porcine aortic valve properties with an uniform fiber orientation distribution, B) exogenously cross-linked bovine pericardium properties with the most homogenous stress distribution, and C) the porcine aortic valve properties properties which results in a very heterogeneous stress distribution and the belly region caving in. The top row shows the side view of the valves at 80 mmHg and the bottom row shows the top-down view of the valves at the same transvalvular pressure.}
\label{fig:valvesimulations}
\end{figure}
    
    The material properties have significant effects on the mechanical behaviors of the leaflets (Fig. \ref{fig:valvesimulations}). The strain distributions within the leaflets were obtained for the pressure-loaded, fully-closed configurations of the valve, and then plotted with the maximum in plane Green-Lagrange strain (MIPE). When comparing the three different material, we can see that the native aortic valve properties result in significant heterogeneities in the deformation of the leaflets (Fig. \ref{fig:valvesimulations}C). Specifically, the belly region of the leaflets significantly protrudes out, increasing the load in the surrounding regions, especially near the commissures. This results in some stress concentrations that are not conducive to heart valve durability and health in general. The bovine pericardium valve (Fig. \ref{fig:valvesimulations}B) and the isotropic valve (Fig. \ref{fig:valvesimulations}A) on the other hand have significantly more homogeneous leaflet deformations, especially from the top-down view. Both of these undergo approximately the same deformation of 0.2 in MIPE. The largest difference between the two is near the commissure regions of the valve. Where the isotropic case is under significantly higher strain. Functionally, the material properties of the exogenously cross-linked bovine pericardium are the most suitable for heart valve leaflets, which distributes the stresses more evenly. 
    
    Much of the reasons behind these differences are likely to be due to the differences between the apparent mechanical properties \textit{in vivo} and the measured mechanical response in the laboratory setting. This is especially true for the aortic valve, which is extremely anisotropic with very high compliance in the radial direction of the leaflets. This difference is most likely due to the mismatch of referential configuration between the two states. Residual strain or residual stress has significant impact on the functional properties of the leaflets, specifically the apparent anisotropy and stiffness. Collagen fiber directions and varying regional properties can also have significant impact on the functional properties of the leaflets, and thus the results of the simulation. The valve leaflet shape, root geometry and properties, the arterial or ventricular geometry and loading conditions, can all be significant factors affecting the functions and stress distribution of the valve leaflets. Furthermore, how these factors affect the fluid dynamics of the valves is also an interesting question, suitable for further study. All in all, this is meant to be a demonstration and proof of concept for using $\Psi_{eff}$ to handle a wide range of soft tissue behaviors and anisotropy for the simulation of biological organs, in this case heart valves. Further and more detailed studies will be reserved for the future.  



\subsection{Permanent set simulation of bioprosthetic heart valve}
    This initial was done with 1) homogenous material parameters simular to bovine pericardium from Sacks and Zhang \cite{sacks_novel_2016}, 2) the Edward valve geometry from \cite{aggarwal_inverse_2015}, and collagen fiber distributions aligned with the circumference direction and a standard deviation of $30^\circ$. The key result is that permanent set slow done after around 20 million cycles and nearly completely seizes after 30 million cycles (Fig. \ref{c6:fig:psdeformation} \&\ref{c6:fig:psnoi}). This effect is closely linked the microstructural of the collagen fiber network. The gradual slow down is correlated with the straightening and recruitment of collagen fibers. This was predicted from the constitutive model (Eqn. \ref{c6:eq:fullEXLmodel}) (Fig. \ref{fig:parametric}). This is an important structural response, which allows us to predict the final reference geometry of the BHV. Since 30 million cycles corresponds to only 1 year after being surgically implanted, the BHV will remain in this configuration for the rest of its 9-14 year life span. By optimizing the initial BHV design so that the peak stress is minimal in the configuration after permanent set has seized, we can potentially improve the durability of BHVs by minimizing the load on the collagen fibers. Because collagen fibers have high rates of failure after being extended by 7-8\% \cite{lanir_structural_1979}\cite{buehler_atomistic_2006}, more evenly distributing the stresses can reduce this mode of failure. 
    
    
    Here we also note that the regions that undergo most permanent set are: the belly region, center of the free edge, and the regions near the commissures, where the leaflets initial makes contact (Fig. \ref{c6:fig:psdeformation}). These regions are also the most common regions of failure in BHVs. Due to the change in reference configuration, the collagen fibers in these regions recruit more quickly and may even held in a constant extended state. This can have dramatic consequences on the likelihood of failure of these collagen fibers due to their low extensibility, and could be a major mechanism for the fiber level damage in BHVs. 
    
    
    We have also shown that we can predict the change in the collagen fiber architecture (Fig. \ref{c6:fig:psnoi}). Here we plotted the normalized orientation index (NOI)
    \begin{equation}
        NOI = \frac{\sigma_iso - \sigma(s)}{\sigma_iso},
    \end{equation}
    which is based on how spread apart (standard deviation $\sigma$) is collagen fiber orientation is. Here, $NOI = 1$ indicate fully aligned distributions (delta distributions), which $NOI = 0$ indication that the fiber distribution is an uniform distribution. We can see that the belly region and the free edge undergoes the greatest degree of realignment. This is most likely due to the fact that these two regions receives the least amount of support from the neighboring leaflets. The most important aspect of being able to predict the structural changes is that we can use it to compute the degree of recruitment of collagen fibers in a similar method to chapter \ref{c2:sec:233}. This allows us to compute the distribution of collagen fiber by their stretches after being straightened. This mechanism can be used to potentially develop a strain-level dependent model for the likelihood of collagen fiber damage in a future extension. 

    
%%%%%%%%%%%%%%%%%%%%%%%%%%%%%%%%%%%%%%%%%%%%%%%%%%%%%%%%%%%%
%-------------------	begin FIGURE 	-------------------%
\begin{figure}
\centering
\includegraphics[width=5in]{Images/chapter6/psdeformation.pdf}
\caption{The simulation of the evolution of the referential configuration with the maximum principal in\Hyphdash plane Green-Lagrange strain (MIPE) overlayed on top at different cycle levels. Colors indicate the magnitude of MIPE, and the lines indicate the principal direction of the permanent set deformation.}
\label{c6:fig:psdeformation}
\end{figure}
%-------------------	 end FIGURE 	-------------------%
%%%%%%%%%%%%%%%%%%%%%%%%%%%%%%%%%%%%%%%%%%%%%%%%%%%%%%%%%%%%

%%%%%%%%%%%%%%%%%%%%%%%%%%%%%%%%%%%%%%%%%%%%%%%%%%%%%%%%%%%%
%-------------------	begin FIGURE 	-------------------%
\begin{figure}
\centering
\includegraphics[width=5in]{Images/chapter6/permanentsetnoi.pdf}
\caption{The simulation of the evolution of the referential configuration with the normalized orientation index (NOI) overlayed, showing the changes in the degree of alignment of collagen fibers at different cycle levels. Higher NOI indicate higher aligned fiber orientation distributions.}
\label{c6:fig:psnoi}
\end{figure}
%-------------------	 end FIGURE 	-------------------%
%%%%%%%%%%%%%%%%%%%%%%%%%%%%%%%%%%%%%%%%%%%%%%%%%%%%%%%%%%%%

%---    Discussion
\section{Summary and Future Directions}
	We have developed the a complete time-dependent framework for the simulation of BHVs under long term cyclic loading. This simulation utilizes the predictive mechanism based constitutive model for the permanent set effect in exogenously crosslinked soft tissues that we previously developed. We have shown that we can use this simulation to predict the evolving geometry, microstructural and material property changes. These results can then be used to predict regions of increase likelihood of structural damage, and can be used to optimal the initial design of BHVs based on these factors. Most important of these effects is that the collagen fiber architecture can play a role in limiting the permanent set effect, where the straightening of collagen fibers prevents further changes in geometry. Thus, accounting for the permanent set effect is especially important in the design of BHVs to better improve their performance and durability. 
	
	
	The two main future extensions of this constitutive model and simulation is for: 1) structural damage and 2) growth and remodeling. Structural damage is difficult to quantify as it only gradually accumulates over long periods of time. Due to the exponential nature and large structural reserves of soft tissues, small decreases in the number or modulus of collagen fiber is very difficult to detect. This is also complicated by the fact that strain is difficult to quantify in the first place. In accelerated wear testing or other similar environments, this process is further complicate by the heterogeneity of the resulting response. However, by simulation and removing of the permanent set effect on the change in geometry, we can more accurately determine the remaining changes due to structural damage, This opens the doors to the development of mechanism-based structural damage models which is lack in literature. Growth and remodeling is a similar important future area, as this have important implications in prediction of the outcomes of diseases, injuries, and surgical interventions. The permanent set model and simulation framework developed herein is a simplification of the growth and remodeling framework by removing the growth component. This can have important potential implications is devices such as tissue engineered valves which have the possibility of growing and adaption to the surrounding environment if seeded with interstitial cells. 




%---    Bioliography
\bibliographystyle{plainnat}
\bibliography{phd}


\appendices

% \chapter{Improved method for the analysis planar biaxial mechanical data}

\section*{Preface}
\addcontentsline{toc}{section}{Preface}%

Simulation of the mechanical behavior of soft tissues is critical for many physiological and medical device applications. Accurate mechanical test data is crucial for both obtaining the form and robust parameter determination of the constitutive model. For incompressible soft tissues that are either membranes or thin sections, planar biaxial mechanical testing configurations can provide much information about the anisotropic stress–strain behavior. However, the analysis of soft biological tissue planar biaxial mechanical test data can be complicated by in-plane shear, tissue heterogeneities, and inelastic changes in specimen geometry that commonly occur during testing. These inelastic effects, without appropriate corrections, alter the stress-traction mapping and violates equilibrium so that the stress tensor is incorrectly determined. To overcome these problems, we presented an analytical method to determine the Cauchy stress tensor from the experimentally derived tractions for tethered testing configurations. We accounted for the measured testing geometry and compensate for run-time inelastic effects by enforcing equilibrium using small rigid body rotations. To evaluate the effectiveness of our method, we simulated complete planar biaxial test configurations that incorporated actual device mechanisms, specimen geometry, and heterogeneous tissue fibrous structure using a finite element (FE) model. We determined that our method corrected the errors in the equilibrium of momentum and correctly estimated the Cauchy stress tensor. We also noted that since stress is applied primarily over a subregion bounded by the tethers, an adjustment to the effective specimen dimensions is required to correct the magnitude of the stresses. Simulations of various tether placements demonstrated that typical tether placements used in the current experimental setups will produce accurate stress tensor estimates. Overall, our method provides an improved and relatively straightforward method of calculating the resulting stresses for planar biaxial experiments for tethered configurations, which is especially useful for specimens that undergo large shear and exhibit substantial inelastic effects.



%---    INTRODUCTION
\section{Introduction}

A central tenet of the biosolid continuum mechanics of biological tissues is the development of the constitutive model. Such models are critical to the insight into the development of accurate computational simulations of such structures as the heart and its valves, arteries, cartilaginous structures, and their engineered tissue equivalents. While formulation of the theoretical framework is always the first step, rigorous experimentation must be performed in parallel to explore all relevant deformations to both obtain the necessary constitutive model parameters and to evaluate its predictive capabilities {Sacks, 2000 #33069}. Thus there is an increasing need for multi-axial mechanical data to fully explore and understand the complex structures of biological tissues.
For incompressible planar membrane or thin soft tissue sections, a planar biaxial mechanical testing configuration can provide much information about the stress-strain behavior {Sacks, 2000 #33069}. Planar biaxial tests can be performed with either extensional deformations only, or in combination with in-plane shear {Sun, 2003 #4637}. However, an ongoing problem in soft tissue mechanics is that they are not truly elastic. Soft tissues have been shown to exhibit elastic behavior under physiological conditions, yet also exhibit permanent set-like changes in configuration from preconditioning {Stella, 2007 #28656;Sacks, 2000 #33069}. In addition, due to their very low stiffness in the zero stress state, even mounting and handling can alter the shape of the test specimen. This may result in the stress-free reference state of the specimen changing drastically from the one measured prior to mounting.
The situation becomes more complex when shear strains are involved. In our original method to induce shear strains in soft tissues. In our first studies of this aspect {Sacks, 1999 #1151}, components of the first Piola-Kirchhoff stress tensor P were determined from the experimentally measured axial forces using P11=f1(1)/A(1), P22=f2(2)/A(2), P12=P21=0, where f(i) and A(i) are the axial forces and initial cross-sectional areas respectively, with i=1,2 (Fig. 1). The resulting second Piola-Kirchhoff stress tensor S was computed using  . We later noted that this mapping produced incorrect results in the presence of shear and an initial alternative method was developed {Freed, 2010 #30830}. Although theoretically correct, that work did not offer a generalized solution. Specifically, the method did not accounted for changes in geometry of the unloaded state as a result of preconditioning and other related inelastic dimensional effects. As a result, the run-time specimen configuration will be a quadrilateral due to both shear and extensional strains that occurred during preconditioning. While others have developed various methods of deriving the stress under biaxial testing (e.g. {Fomovsky, 2010 #33811}), no method to date address actual testing geometry or compensates for changes in specimen geometry during the experiment. Moreover, inherent heterogeneities in tissue structure will always affect the accuracy of the resultant stress analysis.  No systematic study to date has incorporates these effects on accuracy of stress measures from planar biaxial tests.
	In approaching a solution to this problem, we first recognize that there are certain key considerations and limitations inherent in utilizing and interpreting biaxial mechanical data. As noted in Sun et al. {Sun, 2003 #6507}, no biaxial experiment can produce the ideal homogeneous strain and stress state. Subsequent errors in the computation of stress components will propagate to the estimates of the material constants, ultimately limiting our ability to accurately simulate soft tissue structures. Moreover, we need to first determine the constitutive model form and material parameters before undertaking the complex task of simulation of complete organ systems and inverse modeling. Thus, there remains a need for an improved method to derive the stress-strain relation from biaxial tests, within the assumption of specimen strain and stress field homogeneity. 
The current work presents a straightforward, generalized numerical technique that accounts for changes in specimen geometry and testing configuration for planar biaxial testing. It utilizes only the (1) initial dimensions, (2) fiducial markers and (3) axial forces, and works within the assumption of strain and stress field homogeneity within the specimen. Specifically, we focus on compensating for the following aspects: 
1.	Changes from the initial, never loaded, directly measured specimen geometry to the post-mounting/preconditioning state. 
2.	Accounting for heterogenous effects using rigid body minimization.
3.	Accounting for the actual specimen geometry and tether testing configuration during run time.
We also present a full simulation of the biaxial device geometry with real tissue properties, anisotropy and heterogeneities, using methodologies  presented by Lee et al.{Lee, 2014 #34319}, to evaluate the accuracy of the method. 


%---    METHODS
\section{Methods}

\subsection{Kinematics of a Planar Biaxial Test}
    
    Assuming a homogenous deformation, the kinematical description of the planar biaxial test is
        %-------------------	begin EQUATION 	-------------------%
        \begin{equation}
        \begin{aligned}
        x_1 = \lambda_1X_1+\gamma_1X_2, \quad x_2 = \lambda_2X_2+\gamma_2X_1, \quad x_3 = \lambda_3X_3
        \end{aligned}\label{A:eqn:1}
        \end{equation}
        %-------------------	 end EQUATION 	-------------------%
    where $X_k$ and $x_k$ are coordinates for material particles in the reference and current configurations, respectively, $\lambda_k$ are the stretches and $\gamma_k$ are the shears. The shear relative to the third axis is 0, with resulting deformation gradient tensor $\mathbf{F}$
        %-------------------	begin EQUATION 	-------------------%
        \begin{equation}
        \begin{aligned}
        \mathbf{F} = 
        \begin{bmatrix}
        \dpd{x_1}{X_1} & \dpd{x_1}{X_2} & \dpd{x_1}{X_3} \\
        \dpd{x_2}{X_1} & \dpd{x_2}{X_2} & \dpd{x_2}{X_3} \\
        \dpd{x_3}{X_1} & \dpd{x_3}{X_2} & \dpd{x_3}{X_3}
        \end{bmatrix}
        = 
        \begin{bmatrix}
        \lambda_1   & \gamma_1  & 0 \\
        \gamma_2    & \lambda_2 & 0 \\
        0           & 0         & \frac{1}{\lambda_1\lambda_2 - \gamma_1\gamma_2}
        \end{bmatrix}
        \end{aligned}\label{A:eqn:2}
        \end{equation}
        %-------------------	 end EQUATION 	-------------------%
    Note that $F_{33} = (\lambda_1\lambda_2 - \gamma_1\gamma_2)^{-1}$ is computed by the incompressibility constraint $det(\mathbf{F}) = 1$. All specimen deformations are assumed to be completely quantified from the interior of the specimen (approximately the inner third by linear dimension or area) using fiducial markers or texture mapping techniques \cite{sacks_biaxial_2000}\cite{jor_estimating_2010}.
    
    
\subsection{Analysis of Stress}

\subsubsection{Planar Biaxial Testing Experimental Configuration}

    Planar biaxial devices follow a typical design overall, but vary in the specific boundary conditions utilized. We start with a rectangular specimen with known side lengths $L_1$ and $L_2$, and an initial thickness $L_3$, and then mounted with resultant forces $f^{(1)}$ and $f^{(2)}$ (Fig. (Fig.1).1). Previously, we determined the stresses using
        %-------------------	begin EQUATION 	-------------------%
        \begin{equation}
        \begin{gathered}
        P_{11} = \frac{f_1^{(1)}}{A^{(1)}}, \quad P_{22} = \frac{f_2^{(2)}}{A^{(2)}}, \quad P_{11} = P_{22} = 0 \\
        \mathbf{P}
        =   \begin{bmatrix}
            P_{11} & P_{12} \\
            P_{21} & P_{22} 
            \end{bmatrix}
        \end{gathered}\label{A:eqn:3}
        \end{equation}
        %-------------------	 end EQUATION 	-------------------%
    where $\mathbf{f^{(i)}}$ and $A^{(i)}$ are the axial forces and initial cross-sectional areas, respectively, with $i = {1,2}$ \cite{sacks_method_1999}. The Cauchy stress $\mathbf{t}$ and second Piola–Kirchhof stress tensor $\mathbf{S}$ are computed using standard formulations $\mathbf{t} = \mathbf{P} \cdot \mathbf{F}^\mathsf{T}$ and $\mathbf{S} = \mathbf{F}^{-1}\mathbf{P}$.
    
    
    It should be noted that the methods of attachment are generally separated into tethered \cite{sacks_biaxial_2000}\cite{bellini_biaxial_2011}\cite{azadani_comparison_2012}\cite{kamenskiy_passive_2014}\cite{gregory_comparison_2011} and clamped boundaries \cite{sun_effects_2005}\cite{oconnell_human_2011}\cite{sommer_multiaxial_2013}\cite{hu_influence_2013}\cite{simon-allue_unraveling_2014}. In the present work, we focus on tethered boundary configurations. We do not intend to convey that any particular method is optimal for any application, but rather to show that a tethered attachment system, with its ability to allow free lateral displacements and to apply relatively uniform distribution of boundary forces, can be used to accurately obtain the stress–strain relation directly from the experimentally obtained data. We feel this is important as the first step in any tissue mechanical analysis in order to establish the form of the strain energy function, which is best done using directly determined S and F whenever possible. As such, this method will only require direct measurement of (1) initial specimen dimensions, (2) fiducial marker positions, and (3) the measured axial forces. Due to the intrinsic differences in the stress state induced, this method will not be directly applicable to clamped boundaries. Additionally, the following assumptions are made throughout the present work:
        \begin{enumerate}
        \item The tissue is at all times in quasi-static equilibrium.
        \item The deformations are homogeneous, and consequentially:
            \begin{enumerate}
                \item The specimen is located at the center of the apparatus and does not translate.
                \item The testing system is symmetric.
            \end{enumerate}
        \item The applied tractions are evenly distributed per side, given by the average applied by the four attached tethers.
        \end{enumerate}
    
    
\subsubsection{Mounting and Preconditioning Effects}

    Distortions will occur to some extent during mounting and testing due to the high compliance of soft tissue specimen in the low stress range. Although the mechanisms for preconditioning effects remain unknown, it is known that the effect is not strictly viscoelastic \cite{sacks_biaxial_2000}\cite{lanir_structural_1979}\cite{lanir_two_1974}. Instead, it is similar to permanent-set like effects, but is reversible over time \cite{sacks_biaxial_2000}. For example, it has been shown that the process itself reverts over the course of 24 hrs in chemically treated pericardium tissue \cite{sacks_biaxial_2000}. Thus, the effect only lasts for the current test and is utilized to induce a stable, repeatable response \cite{sacks_biaxial_2000}\cite{lanir_structural_1979}\cite{lanir_two_1974}. The new unloaded configuration can be quite different from the initial rectangular state.
    
    
    To compensate for these effects, we first establish the following configurations (Fig. (Fig.2).2). The initial free floating state $\Omega_0$ is defined to be the initial state of the specimen immediately after being cut to size, and is well defined and rectangular. After mounting, preconditioning, and other inelastic run-time effects, the specimen is then fully unloaded and the new unloaded geometry is defined as $\Omega_1$. We define a deformation $\prescript{1}{0}{\mathbf{F}}$ which maps $\Omega_0$ to $\Omega_1$. $\Omega_1$ is the reference configuration used for all stress and strain calculations, with the associated deformation $\prescript{t}{0}{\mathbf{F}}$.
    
    
    Direct dimensional measurements of $\Omega_1$ requires removing the specimen from the device, which must be done carefully to avoid damaging the tissue and inducing additional distortions. An alternative is to image the specimen in situ, which poses its own sets of challenges. Moreover, stress is only induced in the region bounded by the tethers \cite{sun_effects_2005}\cite{hu_influence_2013}\cite{simon-allue_unraveling_2014}, with the surrounding tissue deforming minimally. Therefore, the tether bounded area is best used for the specimen geometry. As a result, the preconditioning effects are accrued in the region of interest (ROI, region bounded by the markers) may not be represented by the visible edges of the specimen. However, $\Omega_1$ can be easily estimated from the deformation in the inner region of the specimen via the fiducial markers, assuming the overall specimen deformation is approximately homogenous. This simplifies the approach and also provides an easy way of determining the thickness, all without physically removing the specimen from the device. We will thus assume that $\Omega_1$ is known precisely and that the specimen undergoes a homogenous deformation quantified by the strain measurement. Note that the magnitude of preconditioning effects can vary considerably in different tissues. For example, it can be modest for a heart valve leaflet (Fig. 3(a)) or very significant for a murine right ventricle (RV) free wall tissue specimen (Fig. 3(b)).
    
    


\subsubsection{Equilibrium}

    In direct analysis of biaxial experimental data, we have observed that the shear components of $\mathbf{t}$ derived from the previous methods \cite{freed_hypoelastic_2010} will not be equal, violating equilibrium. In addition to preconditioning and inelastic effects, the overall geometry and orientation of the specimen are not exactly predicted by the deformation at the center region of the specimen due to real tissue heterogeneities. These will induce the specimen to rotate slightly (i.e., undergo rigid body rotation) with respect to the applied forces, leading to no net moment on the specimen as a whole. This difference between the rigid body angle calculated with respect to the center of the specimen and the real rigid body angle produces a small angular moment in the derived stress.
    
    
    To account for this, we assume the body forces are negligible, and the rigid body moment $\mathbf{M}$ is given by
        %-------------------	begin EQUATION 	-------------------%
        \begin{equation}\label{A:eqn:4}
        \begin{aligned}
        \mathbf{M} = \int_S \mathbf{r}\times\mathbf{T}\dif S
        \end{aligned}
        \end{equation}
        %-------------------	 end EQUATION 	-------------------%
    where $\mathbf{r}$ is the position vector and $\mathbf{T}$ is the boundary traction vector. We parameterize $\mathbf{r}$ as $\mathbf{r}(s,\theta) = s \mathbf{x}^2 + (1-s)\mathbf{x}^1, \, s\in[0,1]$, where $\mathbf{x}^1 = \prescript{t}{0}{\mathbf{F}}(\theta)\cdot\mathbf{X}^1$ and $\mathbf{x}^2 = \prescript{t}{0}{\mathbf{F}}(\theta)\cdot\mathbf{X}^2$ are the corner points bounding the sides in the current state, $\theta$ is the rigid body angle of the deformation gradient, and $l_3$ is the current thickness. Thus, we are left with the sum of the following integral for all four sides:
        %-------------------	begin EQUATION 	-------------------%
        \begin{equation}\label{A:eqn:5}
        \begin{aligned}
        M_3(\theta) = l_3 \int_0^1(\mathbf{r}(\theta,s)\times\mathbf{T}|\mathbf{r}'(\theta,s)|\dif s.
        \end{aligned}
        \end{equation}
        %-------------------	 end EQUATION 	-------------------%
    
    
    We note that in the present system (Fig. 1(a)) the orientation of the traction $\mathbf{T}$ changes as the specimen deforms, and we thus employ the following approach to enforce momentum balance by adjusting $\theta$ (Fig. (Fig.4).4). The initial estimate of the rigid body angle, $\theta$, is derived from the deformation gradient $\prescript{t}{0}{F}$. Based on $\theta$, the quantities describing the current geometry (e.g., tether orientations) are derived. The first moment can be calculated according to equation \ref{A:eqn:5} and can serve as a tolerance check. If the moment does not converge to zero, a new rigid body angle $\theta$ is proposed and the process is repeated until equilibrium is satisfied. Once the best $\theta$ is found, the current geometry of the specimen can be determined from the deformation gradient $\prescript{t}{0}{\mathbf{F}}$. This, when paired with the known tractions $\mathbf{T}$, allows us to determine the Cauchy stress from $\mathbf{T} = \mathbf{t}\cdot\mathbf{n}$. In practice, the rigid body rotation is well within the experimentally measured rigid body rotations, typically $<3$ deg. Note that all calculations were implemented in a custom written Mathematica 10 program.
    
    
\subsubsection{Derivation of Traction Vectors for Self-Equilibrating Tethered Systems}

    \paragraph{Device Geometry} The traction vector should be determined based on the testing system. For devices such as BioTester (Cell Scale), where one end of the tethers is fixed, the orientation of the tethers is easier to determine. Self-equilibrating systems can be more complicated. Typical self-equilibrating tether systems involve wrapping tethers around a pulley with the ends attached to the specimen. For two-point attachment, only one pulley is involved. For four-point attachment, two pulleys are joined by a bar that can rotate about its midpoint \cite{sacks_biaxial_2000}. The number of pulleys can be doubled for eight-point attachment, 16-point attachment, etc. In the case of two tethers, the system is constrained by the total length of the tether around the pulley which can be used to determine the displacement of the actuator. For every additional pulley added, a degree of freedom must be added representing the orientation of the bar joining it to the rest of the system. In all cases, the number of the constraints is equal to the number of degrees of freedom. We shall use the four-point attachment as an example, as it is the most commonly used number of tethers. We will assume the tethers are evenly spread (Fig. (Fig.11).
    
    
    \paragraph{Traction Orientation Vectors} The general process for determining the orientation of the traction vectors is in three steps. (1) Determine the locations of orientation of the tethers in the initial unloaded state; this can be measured directly. (2) Determine the locations of the end of the tethers on the tissue using the deformation gradient. (3) Determine the remaining end of the tethers based on the constraints and mechanisms unique to that system. For devices such as BioTester (Cell Scale), step 3 is simple as the ends are fixed. The midpoint of the tethers on the specimen can be produced by the deformation gradient. The midpoint of the tether at the actuator only displaces with the actuator. The displacement of the actuator $\delta$ can be measured directly, or determined using optimization by assuming the distance from the actuator and specimen remains constant, rather like how $\delta$ is determined below. For our self-equilibrating example, we will denote the four tethers attached to the tissue using the vectors $\mathbf{v}^1$, $\mathbf{v}^2$, $\mathbf{v}^3$, and $\mathbf{v}^4$. The tether vectors $\mathbf{v}^i$ are given by the difference between the tangent points on the shafts and the attachment points on the tissue. Let $\mathbf{o}_Y$ be the pivot point of the lever system, which is moved along the experimental axis by the linear actuators. The position of $\mathbf{o}_Y$ is indeterminate during the experiment. $\mathbf{Y}^i$ are the position of tangent points on the shafts of the lever system relative to $\mathbf{o}_Y$ in the initial free floating configuration. It is clear that the current coordinates of the tangent points on the shafts is simply $\mathbf{y}^i = R(\phi)\mathbf{Y}^i + \mathbf{o}_Y$, where $R(\phi)$ is a rotation matrix about $\mathbf{o}_Y$ and $\phi$ is the angle rotated to equilibrate the tension for all four tethers (Fig. (Fig.11).11). Furthermore, let $\mathbf{X}^i$ be the four tether attachment points on the tissue in the initial free floating configuration, The current coordinates of the attachment points are simply determined using the overall deformation gradient $\prescript{t}{0}{\mathbf{F}} = \prescript{t}{1}{\mathbf{F}}\prescript{1}{0}{\mathbf{F}}$, $\mathbf{x} = \prescript{t}{0}{\mathbf{F}}\mathbf{X}$ x=t0F·X. Thus the tether vectors, $\mathbf{v}^i$, are given by the difference

%---    Results
\input{Chapters/appendixA-content/appendixA.3.tex}

%---    Discussion
\input{Chapters/appendixA-content/appendixA.4.tex}

%---    Conclusion
\input{Chapters/appendixA-content/appendixA.5.tex}




%---    Bioliography
\bibliographystyle{plainnat}
\bibliography{phd}

\chapter{More on effective constitutive modeling of soft tissues, kinematics, parameter correlation, and form.}

\section*{Preface}
\addcontentsline{toc}{section}{Preface}%

In this appendix, we cover some of the lose ends for the effective constitutive models (chapter 5) we developed and as well as some of the considerations and tests we have done that were not included in the main text. First we will look at the choice of kinematic variable and how they affect the constitutive model. They we will look at how this choice will affect the parameter correlations, stress and elasticity tensor forms, and thus how we arrived at the choice of Green-Lagrange strain for our effective constitutive model. We will also go into more detail about the optimal loading paths, how they develop as you are additional loading paths. Finally, we will go into a deeper look at the effect of changing the effective model form and loading paths of the ability of the model to fit and predict mechanical responses. 



%---    INTRODUCTION
%%%%%%%%%%%%%%%%%%%%%%%%%%%%%%%%%%%%%%%%%%%%%%%%%%%%%%%%%%%%%
%%	Constitutive model form									%
%%%%%%%%%%%%%%%%%%%%%%%%%%%%%%%%%%%%%%%%%%%%%%%%%%%%%%%%%%%%%

\section{Effective constitutive model formulation}\label{sec:greenvshencky}

%-----------------------------------------------------------
%	Kinematics
%-----------------------------------------------------------
% \subsection{Kinematic considerations}


% 	The invariants and pseudo-invariants of the right or left Cauchy Green tensor is the most popular for constitutive models of soft tissues. Indeed, we use them often with our structural models \cite{fata_insights_2014, zhang_meso_2016, Avazmohammadi2017b, sacks_novel_2016, zhang_modeling_2017}. There is a large number of invariants, each describes a facet of deformation: isotropic, volumetric strain, anisotropic, or interactions between them. The breadth of choices, allows for more freedom in selecting a combination for constitutive models that best describes the mechanisms in the tissue. Each invariant describes a facet of deformation: isotropic, volumetric strain, anisotropic, or interactions between them. The three isotropic invariants of the right Cauchy Green tensor are $I_1 = Tr(\mathbf{C})$, $I_2 = \frac{1}{2}\left( Tr(\mathbf{C})^2 - Tr(\mathbf{C}^2)\right)$, $I_3 = \det(\mathbf{C})$. The extension for anisotropic materials is summarized by Holzapfel \cite{holzapfel_nonlinear_2000}. The usual starting point is $I_4 = \mathbf{m}\cdot \mathbf{C} \mathbf{m}$, where $\mathbf{m}$ is a unit vector for material axis, or the preferred direction, most commonly that of the constituent fibers, such as collagen and elastin. Other pseudo invariants include $I_5 = \mathbf{m}\cdot \mathbf{C}^2 \mathbf{m}$, $I_6$ and $I_7$ which are equivalent of $I_4$ and $I_5$ for another family of fibers along some $\mathbf{n}$, $I_8 = \mathbf{m}\cdot\mathbf{C}\mathbf{n}$ and $I_9 = (\mathbf{m}\cdot\mathbf{n})^2$ can be used to describe the interaction between these two fiber families \cite{sacks_novel_2016}\cite{avazmohammadi_novel_2017}, etc. 
	
	
% 	In a generalized model, where no assumptions can be made on the behaviors and mechanisms of the tissue,  the use of invariants do not lend itself to creating a single generalized form. One example is when modeling the effects of fiber rotation in tissues with widely distributed fiber orientations. This is one of the most difficult behavior to model using phenomenological approaches. Fiber rotation manifests as coupling interactions between axial stretches in the mechanical response of soft tissues. Invariant based models cannot easily reproduce this effect. One approach is to use the Driessen model \cite{driessen_structural_2005}, where additional fiber families are added until it can match the response of the soft tissue. However, this significantly increases computational costs and requires two additional parameters for each additional fiber family. Further extensions to this approach is to incorporate the fiber orientation distribution directly, such as in meso-scale structural approaches \cite{sacks_incorporation_2003a, fata_insights_2014, zhang_meso_2016}. However, this also increases the complexity and computational cost of the model, which defeats the point of phenomenological approaches entirely. Adding coupling invariants such as $I_8$ and $I_9$ are also an option. However, the inherent similarities between these invariants makes it a difficult choice to incorporate in a generalized constitutive model form. 

    
%     For the most part, it is easy to convert strains to other types. Even 
    
    
% 	The components of strain tensors, such as the Green-Lagrange, $\mathbf{E}$, right Cauchy-Green, $\mathbf{C}$, left Cauchy-Green, $\mathbf{B}$, or the right stretch, $\mathbf{U}$ can be used directly in the constitutive model. Although it is easy to convert them to each other, One example using strain tensor components is the commonly used generalized Fung model \cite{fung_biomechanics_1993}. The effects of choice of strain tensor is not entirely clear. Specifically, $\mathbf{U}$ scales linear with the deformation, whereas $\mathbf{E}$, $\mathbf{C}$, and $\mathbf{B}$ scales quadratically with the deformation. Functionally, these tensors behave very similarly. For example, $\mathbf{E}$ is just $\mathbf{C}$ scaled by $1/2$. More often than not, the choice is to simply default to $\mathbf{E}$, which leads to a simple mathematical form for the stress-strain relationship and the elasticity tensor. 

%-------	Hencky strains		-------%

    One additional strain basis we have considered is the Hencky strains, which is the logarithmic strain calculated from the upper triangular decomposition of the deformation gradient tensor with respect to the material axis as described in Criscione \textit{et al.} \cite{criscione_experimentally_2003a} and then further expanded on by Srinivasa \cite{srinivasa_use_2012} and Freed \cite{freed_logarithmic_2015, freed_conjugate_2017, erel_stress/strain_2017}. This has the following number of advantages: (1) The decomposed strains are easy to interpret physically, (2) it results in an extremely simple mathematical form the Cauchy stress, (3) the logarithmic strains are beneficial when dealing with experimental errors in reference configurations, and (4) It is expected to be slightly less correlated in comparison to the Green Lagrange strains. The Hencky strains have the greatest in difference compared to the Green-Lagrange strains, and are thus a good choice for comparing the strain bases.
    
    
%%%%%%%%%%%%%%%%%%%%%%%%%%%%%%%%%%%%%%%%%%%%%%%%%%%%%%%%%%%%
%-------------------	begin FIGURE 	-------------------%
\begin{figure}
\centering
\includegraphics[width=5in]{Images/chapter5/henckykinematics}
\caption{The upper triangular decomposition of the deformation gradient tensor with respect the preferred material axis of soft tissues, whose components are used to define the Hencky strains.}
\label{fig:henckykinematics}
\end{figure}
%-------------------	 end FIGURE 	-------------------%
%%%%%%%%%%%%%%%%%%%%%%%%%%%%%%%%%%%%%%%%%%%%%%%%%%%%%%%%%%%%
    
    
    The Hencky strains can be formulated as followed. Briefly, the upper triangular decomposition, $\mathbf{f}$, when expressed with respect to the material axis of the tissue is given by
%==========================================================%
%-------------------	begin EQUATION 	-------------------%
\begin{equation}
\begin{aligned}
\left[\mathbf{f}\right]_{\mathbf{m}_0,\mathbf{n}_0} = \begin{bmatrix}
\lambda_m 	& \lambda_m\phi \\
0			& \lambda_n
\end{bmatrix}.
\end{aligned}\label{eqn:uppertriangulardecomposition}
\end{equation}
%-------------------	 end EQUATION 	-------------------%
%==========================================================%
    Here, $\mathbf{m}_0$ is the material axis in the referential configuration, which is generally the preferred direction of the fibers embedded in the tissue, $\mathbf{n}_0$ is the direction perpendicular to $\mathbf{m}_0$, $\mathbf{m}_t$ is the material axis in the deformed configuration, $\mathbf{n}_t$ is the direction perpendicular to $\mathbf{m}_t$, $\lambda_m$ and $\lambda_n$ are the stretches along these axes respectively, and $\phi$ is the angle of shear between $\mathbf{m}_0$ and $\mathbf{n}_0$. The corresponding deformation gradient tensor can thus be expressed as 
%==========================================================%
%-------------------	begin EQUATION 	-------------------%
\begin{equation}
\begin{aligned}
\mathbf{F} = \lambda_m\mathbf{m}_t\otimes\mathbf{m}_0 + \lambda_m\phi\mathbf{m}_t\otimes\mathbf{n}_0 + \lambda_n\mathbf{n}_t\otimes\mathbf{n}_0.
\end{aligned}
\end{equation}
%-------------------	 end EQUATION 	-------------------%
%==========================================================%
    The Hencky strains ($\gamma_1, \gamma_2, \gamma_3$), which are functions of the components of $\mathbf{f}$, can be determined using, 
%==========================================================%
%-------------------	begin EQUATION 	-------------------%
\begin{subequations}\label{eqn:henckystrains}
\begin{align}
\gamma_1 &= \log(\lambda_m), &	\gamma_2 &= \log(\lambda_n), 	& \gamma_3 &= \phi	\\
\lambda_m &= \mathbf{m}_t\cdot\mathbf{F}\mathbf{m}_0, &	\lambda_n &= \mathbf{n}_t\cdot\mathbf{F}\mathbf{n}_0,	&	\phi &= \left(\mathbf{m}_t\cdot\mathbf{F}\mathbf{m}_0\right)^{-1}\mathbf{m}_t\cdot\mathbf{F}\mathbf{n}_0.
\end{align}
\end{subequations}
%-------------------	 end EQUATION 	-------------------%
%==========================================================%
    
    
%-------	Hencky vs. Green-Lagrange strain	-------%

	We are most interested in the difference between Green-Lagrange and the Hencky strains for the formulation of the effective constitutive model (Summary Table \ref{tb:greenvshencky}). Firstly, as stated above , Hencky strains lead to a very convenient form for the Cauchy stresses,
%==========================================================%
%-------------------	begin EQUATION 	-------------------%
\begin{equation}\label{eqn:cauchystressform}
\mathbf{T}	= \frac{1}{J} \dpd{\Psi}{\gamma_1} \mathbf{m}_t\otimes\mathbf{m}_t 
			+ \frac{1}{J} \dpd{\Psi}{\gamma_2} \mathbf{n}_t\otimes\mathbf{n}_t 
			+ \frac{\lambda_n}{J\lambda_m} \dpd{\Psi}{\gamma_3} \left(\mathbf{m}_t\otimes\mathbf{n}_t + \mathbf{n}_t\otimes\mathbf{m}_t \right).
%\\
%\dpd{\Psi}{\gamma_1} 	= J \mathbf{m}\cdot\mathbf{T}\mathbf{m}, \quad 
%\dpd{\Psi}{\gamma_2} 	= J \mathbf{s}\cdot\mathbf{T}\mathbf{s}, \quad 
%\dpd{\Psi}{\gamma_3} 	= J \frac{\lambda_m}{\lambda_n}\mathbf{m}\cdot\mathbf{T}\mathbf{s}.
\end{equation}
%-------------------	 end EQUATION 	-------------------%
%==========================================================%
    where $\Psi$ is the strain energy density function. In comparison, the 2nd Piola Kirchhoff stress with Green-Lagrange strains is,
%==========================================================%
%-------------------	begin EQUATION 	-------------------%
\begin{equation} \label{eqn:2ndpkstressform}
\mathbf{S} = \dpd{\Psi}{E_m}\mathbf{m}_0\otimes\mathbf{m}_0
+ \dpd{\Psi}{E_n}\mathbf{n}_0\otimes\mathbf{n}_0 
+ \frac{1}{2}\dpd{\Psi}{E_{\phi}}\left(\mathbf{m}_0\otimes\mathbf{n}_0 + \mathbf{n}_0\otimes\mathbf{m}_0\right),
\end{equation}
%-------------------	 end EQUATION 	-------------------%
%==========================================================%
    with the Cauchy stress obtained from push forward, $\mathbf{T} = (1/J) \mathbf{F}\mathbf{S}\mathbf{F}^\mathsf{T}$. Expressing the Green-Lagrange strain with respect to the material axis, ${E_m, E_n, E_\phi}$, is preferred, so that the model parameters are invariant with respect to rigid body motion and changes in the reference coordinate system. There isn't a significant advantage to either strain measure here. However, we do note that the partial derivative of Green-Lagrange strain, $\pd{\mathbf{E}}{\mathbf{C}}$ (Eqn. \ref{eqn:partialgreens}), is much simpler than that of the Hencky strains, $\pd{\gamma}{\mathbf{C}}$ (Eqn. \ref{eqn:henckyderivatives}). As a result, the elasticity tensor, $\mathbb{C} = C_{ijkl}$, for the Hencky strain is much more complex (see Appendix \ref{sec:elasticitytensor}). 

	One other aspect is the difference in correlation between model parameters when using Green-Lagrange strain, $\{E_m, E_n, E_\phi\}$ vs. using Hencky strains $\{\gamma_1, \gamma_2, \gamma_3 \}$. For example, using the for form of the generalized Fung model,
%==========================================================%
%-------------------	begin EQUATION 	-------------------%
\begin{subequations} \label{eqn:generalizedfungmodel}
\begin{align}
\Psi 	&= c_0 \left(e^{Q} - 1\right),\notag \\
\mathrm{with}	\qquad	Q	&= b_1 E_m^2 + b_2 E_n^2 + b_3 E_\phi^2 + 2 b_4 E_m E_n + 2 b_5 E_mE_\phi + 2 b_6 E_nE_\phi	\label{eqn:generalizedfungmodela} \\ 
\mathrm{or}		\qquad	Q	&= b_1 \gamma_1^2 + b_2 \gamma_2^2 + b_3 \gamma_3^2 + 2 b_4 \gamma_1\gamma_2 + 2 b_5 \gamma_1\gamma_3 + 2 b_6 \gamma_2\gamma_3	\label{eqn:generalizedfungmodelb} 
\end{align}
\end{subequations}
%-------------------	 end EQUATION 	-------------------%
%==========================================================%
    we compared these two strain basis by computing the correlation matrix between the parameters $b_1$ to $b_6$ for the mechanical data acquired from an exogenously cross-linked bovine pericardium specimen \cite{sun_response_2004} (Appendix \ref{sec:parametercorrelation}). The only difference is that we replaced Green Lagrange strain, $E_m$, $E_n$, $E_\phi$ (Eqn. \ref{eqn:generalizedfungmodela}), with the corresponding Hencky strains $\gamma_1$, $\gamma_2$, $\gamma_3$ (Eqn. \ref{eqn:generalizedfungmodelb}) in the model. The correlation between the parameters are mostly similar in both cases, but the Hencky strains come on top with slightly lower correlations, and the determinant of the correlation matrix is higher, $9.38\times10^{-3}$, in comparison to using the Green Lagrange strains, $7.13\times10^{-3}$, (Fig. \ref{fig:gvsecorrelation}, Appendix \ref{sec:parametercorrelation} Table \ref{tb:correlationE} \& \ref{tb:correlationG}). 



%%%%%%%%%%%%%%%%%%%%%%%%%%%%%%%%%%%%%%%%%%%%%%%%%%%%%%%%%%%%
%-------------------	begin FIGURE 	-------------------%
\begin{figure}
\centering
\includegraphics[width=\textwidth]{Images/chapter5/gvsecorrelation}
\caption{(A) The correlation between parameters pairs in the generalized Fung model when using Green-Lagrange vs Hencky strains. (B) The difference in correlation between each pair of parameters, which is very small but with the Hencky strains being lower overall.}
\label{fig:gvsecorrelation}
\end{figure}
%-------------------	 end FIGURE 	-------------------%
%%%%%%%%%%%%%%%%%%%%%%%%%%%%%%%%%%%%%%%%%%%%%%%%%%%%%%%%%%%%

    
    The last and most important difference between Green-Lagrange and Hencky strains is that they handle compression very differently. With the same model form, stresses increase exponentially under compression with Hencky strains, but only mildly with Green-Lagrange strains. We illustrated this again using the generalized Fung model (Eqn. \ref{eqn:generalizedfungmodel}) in the physiologically relevant range, with Green-Lagrange or Hencky strain as the input variable (Fig. \ref{fig:gvsecompression}). The reason for this is actually fairly simple, Green-Lagrange strain maps deformations from $\mathbf{E}: [0,\infty] \rightarrow [-1/2,\infty]$, while Hencky strains maps deformations from $\mathbf{\gamma}: [0,\infty] \rightarrow [-\infty,\infty]$, drastically increasing the magnitude of the same strain value under compression. This behavior very convenient for modeling collageneous tissues, where collagen fibers crimps unload compression but do not increase the stress \cite{soares_mathematical_2017}. Due to all factors considered (Table \ref{tb:greenvshencky}), we proceed to use the Green-Lagrange strain tensor as the best kinematic basis to formulate the effective constitutive model. 
    
    
%%%%%%%%%%%%%%%%%%%%%%%%%%%%%%%%%%%%%%%%%%%%%%%%%%%%%%%%%%%%
%-------------------	begin FIGURE 	-------------------%
\begin{figure}
\centering
\includegraphics[width=3.25in]{Images/chapter5/gvsecompression}
\caption{The response of a generalized Fung model when using Green-Lagrange (Black) vs Hencky strains(Red). Both models are able to match extensional response nearly perfectly with respect to each other, but drastically differ in response under compression.}
\label{fig:gvsecompression}
\end{figure}
%-------------------	 end FIGURE 	-------------------%
%%%%%%%%%%%%%%%%%%%%%%%%%%%%%%%%%%%%%%%%%%%%%%%%%%%%%%%%%%%%


	





%----------------------------------------------------------%
%-------------------	begin TABLE 	-------------------%
\begin{table}
\caption{The difference between using the Green-Lagrange strain tensor versus the Hencky strains to formulate constitutive models. \textbf{Bold} text indicates key advantages and \textit{Italic} text indicate key disadvantages.}
\begin{center}
\label{tb:greenvshencky}
\begin{tabular}{|L{0.9in}|L{2.25in}|L{2.25in}|}
\hline
\rowcolor{Gray}
\multicolumn{1}{|c|}{Attributes} 
	& \multicolumn{1}{c|}{\textbf{Green-Lagrange strain}} 
    & \multicolumn{1}{c|}{\textbf{Hencky strain}}\\
\hline
General & Most commonly used in modeling	& \textbf{Easy to interpret physically} \\
\hline
Stress 	& Even simpler form for the 2nd Piola Kirchhoff stress \footnotesize
\begin{equation*}
\begin{aligned}
\mathbf{S} =& \dpd{\Psi}{E_m}\mathbf{m}_0\otimes\mathbf{m}_0 + \dpd{\Psi}{E_n}\mathbf{n}_0\otimes\mathbf{n}_0 \\&+ \frac{1}{2}\dpd{\Psi}{E_{\phi}}\left(\mathbf{m}_0\otimes\mathbf{n}_0+\mathbf{n}_0\otimes\mathbf{m}_0\right)
\end{aligned}
\end{equation*}
\normalsize
	& Simple form for the Cauchy stress	\footnotesize
\begin{equation*}
\begin{aligned}
\mathbf{T} =& \dpd{\Psi}{\gamma_{1}}\mathbf{m}_t\otimes\mathbf{m}_t + \dpd{\Psi}{\gamma_{2}}\mathbf{n}_t\otimes\mathbf{n}_t \\ &+ \frac{\lambda_n}{\lambda_m}\dpd{\Psi}{\gamma_{3}}\left(\mathbf{m}_t\otimes\mathbf{n}_t+\mathbf{n}_t\otimes\mathbf{m}_t\right)
\end{aligned}
\end{equation*}
\normalsize
\\
\hline
Elasticity tensor & \textbf{Much simpler form for the elasticity tensor}
	& \textit{The equations for the elasticity tensor is extremely long} \\
\hline
Parameter covariance 	& 	& \textbf{Modestly less correlation between parameters} \\
\hline
Response under compression	&	\textbf{Modest changes in stress under compression, behaves much more similar to soft tissues due to collagen fiber crimp}
	& \textit{Behaves badly under large compression} \begin{itemize}
	\item Log scaling cause the strain energy to increase exponentially with compression
	\end{itemize}\\
\hline
\end{tabular}
\end{center}
\end{table}
%-------------------	 end TABLE 		-------------------%
%----------------------------------------------------------%

%---    METHODS
\section{Elasticity tensor}\label{sec:elasticitytensor}

	An analytical form for the elasticity tensor is extremely important for fast and convergent numerical simulations. A constitutive model without a smooth, continuous, and convex elasticity tensor can pose significant problems for simulation of nonlinear materials to converge quickly, or even to converge at all. The strain basis used for the model can have significant impact on the form of the elasticity tensor. Here, we derived the generalized form of the elasticity tensor for using the Green-Lagrange basis (Eqn. \ref{eqn:greenstrain}) and the Hencky strain basis (Eqn. \ref{eqn:henckystrains}). Note that this is the generalized form, which does not depend on the explicit form of the constitutive model, it is only a function of the strain basis and the response functions (derivatives of the strain energy function), whose form doesn't not have to be explicitly stated. The 9 response functions are:
%=======	BEGIN Equation		=======%
\begin{equation}
\dpd{\Psi}{E_m}, \quad \dpd{\Psi}{E_n}, \quad \dpd{\Psi}{E_\phi}, \quad \frac{\partial^2\Psi}{\partial E_m^2}, \quad \frac{\partial^2\Psi}{\partial E_n^2}, \quad \frac{\partial^2\Psi}{\partial E_\phi^2}, \quad \frac{\partial^2\Psi}{\partial E_m\partial E_n}, \quad \frac{\partial^2\Psi}{\partial E_m\partial E_\phi}, \quad \frac{\partial^2\Psi}{\partial E_n\partial E_\phi}.
\end{equation}
%=======	END Equation		=======%
    
\subsection{Derivation with Green-Lagrange strain}
	
    The elasticity tensor is given by the second derivative of the strain energy function with respect to the right Cauchy strain. Using the chain rules, fully expanding all terms, and enforcing symmetry of partial derivatives, $\md{\Psi}{2}{E_m}{}{E_n}{} = \md{\Psi}{2}{E_n}{}{E_m}{}$, the generalized form for the elasticity tensor is given by 
%=======	BEGIN Equation		=======%
\begin{equation} \label{eqn:generalizedelasticityform}
\begin{aligned}
\dod[2]{\Psi}{\mathbf{C}} =& 
	\dpd{\Psi}{E_m} \dod[2]{E_m}{\mathbf{C}} 
    + \dpd{\Psi}{E_n} \dod[2]{E_n}{\mathbf{C}}
    + \dpd{\Psi}{E_\phi} \dod[2]{E_\phi}{\mathbf{C}} \\
+& \dpd[2]{\Psi}{E_m} \dod{E_m}{\mathbf{C}}\dod{E_m}{\mathbf{C}} 
	+ \dpd[2]{\Psi}{E_n} \dod{E_n}{\mathbf{C}} \dod{E_n}{\mathbf{C}} 
    + \dpd{\Psi}{E_\phi} \dod{E_\phi}{\mathbf{C}} \dod{E_\phi}{\mathbf{C}} \\
+& \dmd{\Psi}{2}{E_m}{}{E_n}{} \left(\dod{E_n}{\mathbf{C}} \dod{E_m}{\mathbf{C}} + \dod{E_m}{\mathbf{C}} \dod{E_n}{\mathbf{C}}\right)   \\
    +& \dmd{\Psi}{2}{E_m}{}{E_\phi}{} \left(\dod{E_\phi}{\mathbf{C}} \dod{E_m}{\mathbf{C}} + \dod{E_m}{\mathbf{C}} \dod{E_\phi}{\mathbf{C}}\right)  \\
    +& \dmd{\Psi}{2}{E_n}{}{E_\phi}{} \left(\dod{E_n}{\mathbf{C}} \dod{E_\phi}{\mathbf{C}} + \dod{E_\phi}{\mathbf{C}} \dod{E_n}{\mathbf{C}}\right).  \\
\end{aligned}
\end{equation}
%=======	END Equation		=======%
To break this down, we begin with the derivatives of the Green-Lagrange strains, which are given by,
%==========================================================%
%-------------------	begin EQUATION 	-------------------%
\begin{equation}\label{eqn:partialgreens}
\begin{aligned}
\dod{E_m}{\mathbf{C}} &= \frac{1}{2} \mathbf{m}\otimes\mathbf{m}	\\
\dod{E_n}{\mathbf{C}} &= \frac{1}{2} \mathbf{n}\otimes\mathbf{n} \\
\dod{E_\phi}{\mathbf{C}} &= \frac{1}{4} \left(\mathbf{m}\otimes\mathbf{n} + \mathbf{n}\otimes\mathbf{m} \right),
\end{aligned}
\end{equation}
%-------------------	 end EQUATION 	-------------------%
%==========================================================%
and 
%==========================================================%
%-------------------	begin EQUATION 	-------------------%
\begin{equation}
\dod[2]{E_m}{\mathbf{C}} = \dod[2]{E_n}{\mathbf{C}} = \dod[2]{E_\phi}{\mathbf{C}} = \mathbf{0}.
\end{equation}
%-------------------	 end EQUATION 	-------------------%
%==========================================================%
Right away, the Green-Lagrange strains have the benefit of the second derivatives being zero, reducing the elasticity tensor (Eqn. \ref{eqn:generalizedelasticityform}) from 9 to 6 terms. Substituting with the partial derivatives (Eqn. \ref{eqn:partialgreens}) gives
%==========================================================%
%-------------------	begin EQUATION 	-------------------%
\begin{equation}\label{eqn:greenelasticityform}
\begin{aligned}
\dod[2]{\Psi}{\mathbf{C}} =
	& \frac{1}{4}\dpd[2]{\Psi}{E_m} \mathbf{m}\otimes\mathbf{m}\otimes\mathbf{m}\otimes\mathbf{m}	\\
    &+ \frac{1}{8}\dmd{\Psi}{2}{E_m}{}{E_\phi}{} 
    	\left(
        	\mathbf{m}\otimes\mathbf{m}\otimes\mathbf{m}\otimes\mathbf{n}
            +\mathbf{m}\otimes\mathbf{m}\otimes\mathbf{n}\otimes\mathbf{m} \right.\\
            &\quad +\left.\mathbf{m}\otimes\mathbf{n}\otimes\mathbf{m}\otimes\mathbf{m}
            +\mathbf{n}\otimes\mathbf{m}\otimes\mathbf{m}\otimes\mathbf{m}
        \right)	\\
    &+ \frac{1}{4}\dmd{\Psi}{2}{E_m}{}{E_n}{} 
    	\left(
        	\mathbf{m}\otimes\mathbf{m}\otimes\mathbf{n}\otimes\mathbf{n}
            +\mathbf{n}\otimes\mathbf{n}\otimes\mathbf{m}\otimes\mathbf{m}
        \right)	\\
    &+ \frac{1}{16}\dpd[2]{\Psi}{E_\phi} 
    	\left(
        	\mathbf{m}\otimes\mathbf{n}\otimes\mathbf{m}\otimes\mathbf{n}
            +\mathbf{m}\otimes\mathbf{n}\otimes\mathbf{n}\otimes\mathbf{m} \right. \\
            &\quad \left.+\mathbf{n}\otimes\mathbf{m}\otimes\mathbf{m}\otimes\mathbf{n}
            +\mathbf{n}\otimes\mathbf{m}\otimes\mathbf{n}\otimes\mathbf{m}
        \right)	\\
    &+ \frac{1}{8}\dmd{\Psi}{2}{E_n}{}{E_\phi}{} 
    	\left(
        	\mathbf{m}\otimes\mathbf{n}\otimes\mathbf{n}\otimes\mathbf{n}
            +\mathbf{n}\otimes\mathbf{m}\otimes\mathbf{n}\otimes\mathbf{n} \right. \\
            &\quad +\left. \mathbf{n}\otimes\mathbf{n}\otimes\mathbf{m}\otimes\mathbf{n}
            +\mathbf{n}\otimes\mathbf{n}\otimes\mathbf{n}\otimes\mathbf{m}
        \right)	\\
    &+ \frac{1}{4} \dpd[2]{\Psi}{E_n} \mathbf{n}\otimes\mathbf{n}\otimes\mathbf{n}\otimes\mathbf{n}	\\
\end{aligned}
\end{equation}
%-------------------	 end EQUATION 	-------------------%
%==========================================================%


\subsection{Derivation with Hencky strains}

	The elasticity tensor when using the Hencky strains is much more complex. For start, the second derivatives of the Hencky strains are non-zero. The derivatives themselves are complex both in form and conceptually. The derivation of the derivatives is not straight forward. To make this simpler, we start with an alternative definition for the Hencky strains, which relates the 4 variables, $\gamma_1$, $\gamma_2$, $\gamma_3$, and $\mathbf{C}$.  
%==========================================================%
%-------------------	begin EQUATION 	-------------------%
\begin{equation}\label{eqn:invariantset}
\begin{aligned}
    \gamma_1 &= \ln \left( \lambda_m \right), &  \lambda_m^2 &= \mathbf{m}\cdot\mathbf{C}\mathbf{m}  \\
    \gamma_2 &= \ln \left( \lambda_n \right), &  \lambda_n^2 &= \mathbf{n}\cdot\mathbf{C}\mathbf{n} 
                    - \lambda_m^2 \phi^2   \\
    \gamma_3 &= \phi, & \phi &= \left( \lambda_m\right)^{-2}\mathbf{m}\cdot\mathbf{C}\mathbf{n}.
\end{aligned}
\end{equation}
%-------------------	 end EQUATION 	-------------------%
%==========================================================%
It is important here that with this definition, the vector basis, $\mathbf{m}$ and $\mathbf{n}$, are defined on the reference coordinate system, which does not change with deformation. By chain rule, the first derivatives are as followed,
%==========================================================%
%-------------------	begin EQUATION 	-------------------%
\begin{subequations}
\begin{align}
\dod{\gamma_1}{\mathbf{C}} =& \dpd{\gamma_1}{\lambda_m}\dod{\lambda_m}{\mathbf{C}},	
	& \dod{\lambda_m^2}{\mathbf{C}} =& \dpd{\lambda_m^2}{\mathbf{C}}:\dod{\mathbf{C}}{\mathbf{C}} 
	+ \dpd{\lambda_m^2}{\lambda_n}\dod{\lambda_n}{\mathbf{C}}
    + \dpd{\lambda_m^2}{\phi}\dod{\phi}{\mathbf{C}} \\
\dod{\gamma_2}{\mathbf{C}} =& \dpd{\gamma_2}{\lambda_n}\dod{\lambda_n}{\mathbf{C}},	
	& \dod{\lambda_n^2}{\mathbf{C}} =& \dpd{\lambda_n^2}{\mathbf{C}}:\dod{\mathbf{C}}{\mathbf{C}} 
	+ \dpd{\lambda_n^2}{\lambda_m}\dod{\lambda_m}{\mathbf{C}}
    + \dpd{\lambda_n^2}{\phi}\dod{\phi}{\mathbf{C}} \\
\dod{\gamma_3}{\mathbf{C}} =& \dod{\phi}{\mathbf{C}},
	& \dod{\phi}{\mathbf{C}} =& \dpd{\phi}{\mathbf{C}}:\dod{\mathbf{C}}{\mathbf{C}} 
	+ \dpd{\phi}{\lambda_m}\dod{\lambda_m}{\mathbf{C}}
    + \dpd{\phi}{\lambda_n}\dod{\lambda_n}{\mathbf{C}}. 
\end{align}
\end{subequations}
%-------------------	 end EQUATION 	-------------------%
%==========================================================%
First, note that all the partial derivatives are functions of each other. This is indeed problematic, but also note from equation \ref{eqn:invariantset} that $\lambda_m$ does not depend on $\lambda_n$ and $\phi$, and $\phi$ does not depend on $\lambda_n$. This means that the following partial derivatives are zero,
%==========================================================%
%-------------------	begin EQUATION 	-------------------%
\begin{equation}\label{eqn:henckystraindependence}
\begin{aligned}
\dpd{\lambda_m}{\lambda_n} = \dpd{\lambda_m}{\phi} = \dpd{\phi}{\lambda_n} = 0,
\end{aligned}
\end{equation}
%-------------------	 end EQUATION 	-------------------%
%==========================================================%
allowing the equations to be solved. 

	For the second derivatives, note from the definition we have above, the Hencky strains are only linear function of $\mathbf{C}$, thus their 2nd \emph{partial derivatives} are zero with respect to $\mathbf{C}$ only,
%==========================================================%
%-------------------	begin EQUATION 	-------------------%
\begin{equation}
\begin{aligned}
\dpd{}{\mathbf{C}}\left(\dod{\gamma_1}{\mathbf{C}}\right) 
	=\dpd{}{\mathbf{C}}\left(\dod{\gamma_2}{\mathbf{C}}\right)
    =\dpd{}{\mathbf{C}}\left(\dod{\gamma_3}{\mathbf{C}}\right)
    =0.
\end{aligned}
\end{equation}
%-------------------	 end EQUATION 	-------------------%
%==========================================================%
The second derivatives are thus defined to be,
%==========================================================%
%-------------------	begin EQUATION 	-------------------%
\begin{subequations}
\begin{align}
\dod{}{\mathbf{C}}\left(\dod{\gamma_1}{\mathbf{C}}\right) =&
    \dpd{}{\lambda_m}\left(\dod{\gamma_1}{\mathbf{C}}\right)\dod{\lambda_m}{\mathbf{C}}
    + \dpd{}{\lambda_n}\left(\dod{\gamma_1}{\mathbf{C}}\right)\dod{\lambda_n}{\mathbf{C}}
    + \dpd{}{\phi}\left(\dod{\gamma_1}{\mathbf{C}}\right)\dod{\phi}{\mathbf{C}}	\\
\dod{}{\mathbf{C}}\left(\dod{\gamma_2}{\mathbf{C}}\right) =&
    \dpd{}{\lambda_m}\left(\dod{\gamma_2}{\mathbf{C}}\right)\dod{\lambda_m}{\mathbf{C}}
    + \dpd{}{\lambda_n}\left(\dod{\gamma_2}{\mathbf{C}}\right)\dod{\lambda_n}{\mathbf{C}}
    + \dpd{}{\phi}\left(\dod{\gamma_2}{\mathbf{C}}\right)\dod{\phi}{\mathbf{C}}	\\
\dod{}{\mathbf{C}}\left(\dod{\phi}{\mathbf{C}}\right) =&
    \dpd{}{\lambda_m}\left(\dod{\phi}{\mathbf{C}}\right)\dod{\lambda_m}{\mathbf{C}}
    + \dpd{}{\lambda_n}\left(\dod{\phi}{\mathbf{C}}\right)\dod{\lambda_n}{\mathbf{C}}
    + \dpd{}{\phi}\left(\dod{\phi}{\mathbf{C}}\right)\dod{\phi}{\mathbf{C}} 
\end{align}
\end{subequations}
%-------------------	 end EQUATION 	-------------------%
%==========================================================%
Taking advantage of equation \ref{eqn:henckystraindependence}, the first and second derivatives of the Hencky strains are presented as followed:
%==========================================================%
%-------------------	begin EQUATION 	-------------------%
\begin{subequations} \label{eqn:henckyderivatives}
\begin{align}
\dod{\gamma_1}{\mathbf{C}} =& \frac{1}{2\lambda_m^2} \mathbf{m}\otimes\mathbf{m}	\\
\dod{\gamma_2}{\mathbf{C}} =& \frac{1}{2\lambda_n^2} \left(\mathbf{n}\otimes\mathbf{n} - \phi \left( \mathbf{m}\otimes\mathbf{n} + \mathbf{n}\otimes\mathbf{m}\right) + \phi^2 \mathbf{m}\otimes\mathbf{m} \right)	\\
\dod{\gamma_3}{\mathbf{C}} =& \frac{1}{2\lambda_m^2} \left( \mathbf{m}\otimes\mathbf{n} + \mathbf{n}\otimes\mathbf{m} - 2\phi \mathbf{m}\otimes\mathbf{m}\right) \\
\dod[2]{\gamma_1}{\mathbf{C}} =& -\frac{1}{2}\frac{1}{\lambda_m^4}\mathbf{m}\otimes\mathbf{m} \otimes \mathbf{m}\otimes\mathbf{m}	\\
\begin{split}
\dod[2]{\gamma_2}{\mathbf{C}} =& -\frac{1}{2}\frac{1}{\lambda_n^4} 
    \left[
    \mathbf{n}\otimes\mathbf{n}\otimes\mathbf{n}\otimes\mathbf{n}
    -\phi \mathbf{m}\otimes\mathbf{n}\otimes\mathbf{n}\otimes\mathbf{n}
    -\phi \mathbf{n}\otimes\mathbf{m}\otimes\mathbf{n}\otimes\mathbf{n}\right.   \\
    &-\phi \mathbf{n}\otimes\mathbf{n}\otimes\mathbf{m}\otimes\mathbf{n}
    -\phi \mathbf{n}\otimes\mathbf{n}\otimes\mathbf{n}\otimes\mathbf{m}   
    +\phi^2 \mathbf{m}\otimes\mathbf{m}\otimes\mathbf{n}\otimes\mathbf{n} \\
    &\left.+\phi^2 \mathbf{n}\otimes\mathbf{n}\otimes\mathbf{m}\otimes\mathbf{m}
    \right]\\
    &-\left(\frac{1}{4}\frac{1}{\lambda_m^2\lambda_n^2} + \frac{1}{2}\phi^2\frac{1}{\lambda_n^4} \right)
    \left[
    \mathbf{m}\otimes\mathbf{n}\otimes\mathbf{m}\otimes\mathbf{n}
    +\mathbf{m}\otimes\mathbf{n}\otimes\mathbf{n}\otimes\mathbf{m} \right. \\
    &+\left. \mathbf{n}\otimes\mathbf{m}\otimes\mathbf{m}\otimes\mathbf{n}
    +\mathbf{n}\otimes\mathbf{m}\otimes\mathbf{n}\otimes\mathbf{m}
    \right]\\
    &+\left(\frac{1}{2}\phi\frac{1}{\lambda_m^2\lambda_n^2} + \frac{1}{2}\phi^3\frac{1}{\lambda_n^4} \right)
    \left[
    \mathbf{m}\otimes\mathbf{m}\otimes\mathbf{m}\otimes\mathbf{n}
    +\mathbf{m}\otimes\mathbf{m}\otimes\mathbf{n}\otimes\mathbf{m} \right. \\
    &+\left. \mathbf{m}\otimes\mathbf{n}\otimes\mathbf{m}\otimes\mathbf{m}
    +\mathbf{n}\otimes\mathbf{m}\otimes\mathbf{m}\otimes\mathbf{m}
    \right]\\
    &-\left(\phi^2\frac{1}{\lambda_m^2\lambda_n^2} + \frac{1}{2}\phi^4\frac{1}{\lambda_n^4} \right)
    \left[
    \mathbf{m}\otimes\mathbf{m}\otimes\mathbf{m}\otimes\mathbf{m}
    \right]
\end{split}\\
\begin{split}
\dod[2]{\gamma_3}{\mathbf{C}} =& -\frac{1}{2}\frac{1}{\lambda_m^4} \times \left(
        \mathbf{n}\otimes\mathbf{m}\otimes\mathbf{m}\otimes\mathbf{m} + \mathbf{m}\otimes\mathbf{n}\otimes\mathbf{m}\otimes\mathbf{m}\right. \\
        &+ \left.\mathbf{m}\otimes\mathbf{m}\otimes\mathbf{n}\otimes\mathbf{m} +
        \mathbf{m}\otimes\mathbf{m}\otimes\mathbf{m}\otimes\mathbf{n} - 
        4\phi\mathbf{m}\otimes\mathbf{m} \otimes \mathbf{m}\otimes\mathbf{m} \right)
\end{split}
\end{align}
\end{subequations}
%-------------------	 end EQUATION 	-------------------%
%==========================================================%

	Saving everyone from the algebra, without further ado, the most elegant form of the elasticity tensor for constitutive models based on the Hencky strains is, 
%==========================================================%
%-------------------	begin EQUATION 	-------------------%
\begin{subequations} \label{eqn:elasticityhencky}
\begin{align}
\begin{split}
\dod[2]{\Psi}{\mathbf{C}} =&
	\psi_1 \mathbf{m}\otimes\mathbf{m}\otimes\mathbf{m}\otimes\mathbf{m}	\\
    &+ \psi_2
    	\left(
        	\mathbf{m}\otimes\mathbf{m}\otimes\mathbf{m}\otimes\mathbf{n}
            +\mathbf{m}\otimes\mathbf{m}\otimes\mathbf{n}\otimes\mathbf{m} \right. \\
            &\quad +\left. \mathbf{m}\otimes\mathbf{n}\otimes\mathbf{m}\otimes\mathbf{m}
            +\mathbf{n}\otimes\mathbf{m}\otimes\mathbf{m}\otimes\mathbf{m}
        \right)	\\
    &+ \psi_3
    	\left(
        	\mathbf{m}\otimes\mathbf{m}\otimes\mathbf{n}\otimes\mathbf{n}
            +\mathbf{n}\otimes\mathbf{n}\otimes\mathbf{m}\otimes\mathbf{m}
        \right)	\\
    &+ \psi_4
    	\left(
        	\mathbf{m}\otimes\mathbf{n}\otimes\mathbf{m}\otimes\mathbf{n}
            +\mathbf{m}\otimes\mathbf{n}\otimes\mathbf{n}\otimes\mathbf{m} \right. \\
            &\quad +\left.\mathbf{n}\otimes\mathbf{m}\otimes\mathbf{m}\otimes\mathbf{n}
            +\mathbf{n}\otimes\mathbf{m}\otimes\mathbf{n}\otimes\mathbf{m}
        \right)	\\
    &+ \psi_5
    	\left(
        	\mathbf{m}\otimes\mathbf{n}\otimes\mathbf{n}\otimes\mathbf{n}
            +\mathbf{n}\otimes\mathbf{m}\otimes\mathbf{n}\otimes\mathbf{n}
            +\mathbf{n}\otimes\mathbf{n}\otimes\mathbf{m}\otimes\mathbf{n}
            +\mathbf{n}\otimes\mathbf{n}\otimes\mathbf{n}\otimes\mathbf{m}
        \right)	\\
    &+ \psi_6 \mathbf{n}\otimes\mathbf{n}\otimes\mathbf{n}\otimes\mathbf{n}
\end{split}	\\
\begin{split}
\psi_1 =&
        -\frac{1}{2}\frac{1}{\lambda_m^4}W^{(1)}
        -\left(\frac{\phi^2}{\lambda_m^2\lambda_n^2}+\frac{1}{2}\frac{\phi^4}{\lambda_n^4}\right)W^{(2)}
        +2\frac{\phi}{\lambda_m^4}W^{(3)}
        +\frac{1}{4}\frac{1}{\lambda_m^4}W^{(11)}   \\
        &+\frac{1}{4}\frac{\phi^4}{\lambda_n^4}W^{(22)}
        +\frac{\phi^2}{\lambda_m^4}W^{(33)}
        +\frac{1}{2}\frac{\phi^2}{\lambda_m^2\lambda_n^2}W^{(12)}
        -\frac{\phi}{\lambda_m^4}W^{(13)}
        -\frac{\phi^3}{\lambda_m^2 \lambda_n^2}W^{(23)}
\end{split}	\\
\begin{split}
\psi_2 =&
        \frac{1}{2}\left(\frac{\phi}{\lambda_m^2\lambda_n^2} + \frac{\phi^3}{\lambda_n^4}\right)W^{(2)}
        -\frac{1}{2}\frac{1}{\lambda_m^4}W^{(3)}
        -\frac{1}{4}\frac{\phi^3}{\lambda_n^4}W^{(22)}
        -\frac{1}{2}\frac{\phi}{\lambda_m^4}W^{(33)}    \\
        &-\frac{1}{4}\frac{\phi}{\lambda_m^2 \lambda_n^2}W^{(12)}
        +\frac{1}{4}\frac{1}{\lambda_m^4}W^{(13)}
        +\frac{3}{4}\frac{\phi^2}{\lambda_m^2 \lambda_n^2}W^{(23)}
\end{split}	\\
\psi_3 =&
        -\frac{1}{2}\frac{\phi^2}{\lambda_n^4}W^{(2)}
        +\frac{1}{4}\frac{\phi^2}{\lambda_n^4}W^{(22)}
        +\frac{1}{4}\frac{1}{\lambda_m^2 \lambda_n^2}W^{(12)}
        -\frac{1}{2}\frac{\phi}{\lambda_m^2 \lambda_n^2}W^{(23)}	\\
\psi_4 =&
        -\left(\frac{1}{4}\frac{1}{\lambda_m^2\lambda_n^2}+\frac{1}{2}\frac{\phi^2}{\lambda_n^4}\right)W^{(2)}
        +\frac{1}{4}\frac{\phi^2}{\lambda_n^4}W^{(22)}
        +\frac{1}{4}\frac{1}{\lambda_m^4}W^{(33)}
        -\frac{1}{2}\frac{\phi}{\lambda_m^2 \lambda_n^2}W^{(23)}	\\
\psi_5 =&
        \frac{1}{2}\frac{\phi}{\lambda_n^4}W^{(2)}
        -\frac{1}{4}\frac{\phi}{\lambda_n^4}W^{(22)}
        +\frac{1}{4}\frac{1}{\lambda_m^2 \lambda_n^2}W^{(23)}	\\
\psi_6 =&
        -\frac{1}{2}\frac{1}{\lambda_n^4}W^{(2)}
        +\frac{1}{4}\frac{1}{\lambda_n^4}W^{(22)}
\end{align}
\end{subequations}
%-------------------	 end EQUATION 	-------------------%
%==========================================================%
where $W^{ij} = \md{\Psi}{2}{\gamma_i}{}{\gamma_j}{}$.

\paragraph{Some final remarks} 
	The elasticity tensor for the Hencky strains (Eqn. \ref{eqn:elasticityhencky}) are somewhat complicated, but most terms are weighted by $\gamma_3$, which is the shear angle $\phi$. Under no shear, the form for the elasticity tensor reduces significantly, to similar in form to the Green-Lagrange strains (Eqn. \ref{eqn:greenelasticityform}). The shear angle is typically a small value as well, making most terms in the elasticity tensor small as well. However, they are still necessary to accurately compute the elasticity tensor and are representative of the coupling between the strains components. 
    
    More interesting is perhaps that the Hencky strains overall leads to less covariance in model parameters overall, despite there being many more coupling terms than the Green-Lagrange strain. This is indicative of the covariance within the model form and of the tissue, which in turn is necessary to reproduce soft tissue responses. Due to this natural covariance of the tissues themselves, truly non-covariant models are not feasibly attainable. The level of covariance demonstrated by $\Psi_{eff}$ (Eqn. \ref{eqn:finalexponentialmodelformscaled}) within are likely close to optimal without sacrificing for addition model complexity. 



















% loss of benefits for the correlation between parameters from polynomial to exponential model. 

%---    Results
\section{Parameter correlation}\label{sec:parametercorrelation}

	The Fisher's information matrix for a maximum likelihood problem described by a multivariate normally distributed statistical model is defined to be the negative of the second partial derivatives of the log-likelihood function with respect to the parameters, $\xi_i$, evaluated at the maximum likelihood estimates. When converting to least squares minimization, which is equivalent of a negative log-likelihood minimization, the information matrix is defined to be
%==========================================================%
%-------------------	begin EQUATION 	-------------------%
\begin{equation}\label{eqn:informationmatrix}
\begin{aligned}
\mathcal{I}_{jk} = \dmd{\mathcal{F}}{2}{\xi_j}{}{\xi_k}{} = \mathcal{H}_{jk},
\end{aligned}
\end{equation}
%-------------------	 end EQUATION 	-------------------%
%==========================================================%
which is also exactly the Hessian matrix, $\mathbfcal{H}$, at the best fit value. The D-optimality is the determinant of this information matrix, whereas other optimality measures are mostly similar functions of this information matrix. The main reason is because, for a maximum likelihood problem governed by a normally distributed statistical model, the covariance matrix is exactly proportional to the inverse of the information matrix,
%==========================================================%
%-------------------	begin EQUATION 	-------------------%
\begin{equation}\label{eqn:covariancematrix}
\begin{aligned}
\mathbf{Cov} = \sigma^2 \mathbfcal{I}^{-1},
\end{aligned}
\end{equation}
%-------------------	 end EQUATION 	-------------------%
%==========================================================%
where $\sigma^2$ is the scalar variance of the statistical model, and can be estimated by the mean squared error weighted by the degree of freedom. Maximizing the D-optimality is equivalent of minimizing the variance of each parameter along with minimizing the covariance between parameters. Defining the objective function as
%==========================================================%
%-------------------	begin EQUATION 	-------------------%
\begin{equation}
\begin{aligned}
\mathcal{F} = \sum_i \left(f(x_i) - f_i\right)^2,
\end{aligned}
\end{equation}
%-------------------	 end EQUATION 	-------------------%
%----------------------------------------------------------%
where $f(x_i)$ is the model and $f_i$ are date points, the information matrix, or Hessian matrix, is given by 
%==========================================================%
%-------------------	begin EQUATION 	-------------------%
\begin{equation}\label{eqn:hessianmatrix}
\begin{aligned}
\dpd{\mathcal{F}}{\xi_j} =& \sum_i 2\left(f(x_i) - f_i\right)\dpd{f(x_i)}{\xi_j}	\\
\mathcal{H}_{jk} = \dmd{\mathcal{F}}{2}{\xi_j}{}{\xi_k}{} =& \sum_i 2\dpd{f(x_i)}{\xi_k}\dpd{f(x_i)}{\xi_j}	+ 2\left(f(x_i) - f_i\right)\dmd{f(x_i)}{2}{\xi_j}{}{\xi_k}{}
\end{aligned}
\end{equation}
%-------------------	 end EQUATION 	-------------------%
%----------------------------------------------------------%
Note here that $\mathcal{J}_{ij} = \dpd{f(x_i)}{\xi_j}$ is the Jacobian matrix for the non-linear least squares problem, and when the errors are small, i.e. when the model fits the data, in other words such that $\left(f(x_i) - f_i\right)$ is small or approximately 0, then the information matrix can by approximated by  
\begin{equation}
\mathcal{I}_{jk} \approx \mathcal{J}_{ji}\mathcal{J}_{ik} \quad \mathrm{or} \quad
\mathbfcal{I} \approx \mathbfcal{J}^\mathsf{T}\mathbfcal{J},
\end{equation}
a form much more familiar to most doing nonlinear least squares parameter estimation. 


	For $\Psi_{eff}$ (Eqn. \ref{eqn:finalexponentialmodelformscaled}), let $f = c_0Q^\prime e^{Q}$ be the stress, where $Q^\prime$ is one of $\pd{Q}{E_m}$, $\pd{Q}{E_n}$, or $\pd{Q}{E_\phi}$. The Jacobian matrix is thus, 
\begin{equation}	
\mathcal{J}_{ij} = Q^\prime e^Q \delta_{0j} + c_0 e^Q \dpd{Q^\prime}{\xi_j}+ c_0Q^\prime e^Q \dpd{Q}{\xi_j}
\end{equation}
Note that since $Q$ is a sum of polynomials, the second partial derivatives of $Q$ and $Q^\prime$ with respect to $\mathbf{\xi}$ is precisely 0,
\begin{equation}		
\md{Q}{2}{\xi_j}{}{\xi_k}{}  = \md{Q^\prime}{2}{\xi_j}{}{\xi_k}{} = 0.
\end{equation}
Thus,
\begin{equation}	
\begin{aligned}
\dmd{f}{2}{\xi_j}{}{\xi_k}{} =& e^Q \delta_{0j}\dpd{Q^\prime}{\xi_k} + Q^\prime e^Q \delta_{0j}\dpd{Q}{\xi_k} 
+  e^Q\dpd{Q^\prime}{\xi_j}\delta_{0k} + c_0 e^Q\dpd{Q^\prime}{\xi_j}\dpd{Q}{\xi_k} 
+ Q^\prime e^Q \dpd{Q}{\xi_j}\delta_{0k} 	\\
&+ c_0e^Q \dpd{Q}{\xi_j}\dpd{Q^\prime}{\xi_k}
+ c_0Q^\prime e^Q \dpd{Q}{\xi_j}\dpd{Q}{\xi_k} 
\end{aligned}
\end{equation}
and the Hessian matrix is given by
% \begin{equation}	
% \begin{aligned}
% \mathcal{H}_{jk} &= 2\sum_i\left[ (Q^\prime e^Q)^2  \delta_{0j}\delta_{0k} + c_0Q^\prime e^{2Q} \delta_{0j}\dpd{Q^\prime}{\xi_k}+ c_0 (Q^\prime e^Q)^2 \delta_{0j}\dpd{Q}{\xi_k}\right.	\\
% &+c_0Q^\prime e^{2Q} \dpd{Q^\prime}{\xi_j} \delta_{0k} + (c_0 e^Q)^2 \dpd{Q^\prime}{\xi_j}\dpd{Q^\prime}{\xi_k}+ (c_0e^Q)^2Q^\prime  \dpd{Q^\prime}{\xi_j}\dpd{Q}{\xi_k}	\\
% &+ c_0(Q^\prime e^Q)^2 \dpd{Q}{\xi_j}\delta_{0k} + Q^\prime (c_0 e^Q)^2 \dpd{Q}{\xi_j}\dpd{Q^\prime}{\xi_k}+ (c_0Q^\prime e^Q)^2 \dpd{Q}{\xi_j}\dpd{Q}{\xi_k}	\\
% &+ (c_0Q^\prime e^Q - f_i)\left( e^Q \delta_{0j}\dpd{Q^\prime}{\xi_k} + Q^\prime e^Q \delta_{0j}\dpd{Q}{\xi_k} 
% +  e^Q\dpd{Q^\prime}{\xi_j}\delta_{0k} 
% \right.	\\
% &\left.\left.+ c_0 e^Q\dpd{Q^\prime}{\xi_j}\dpd{Q}{\xi_k} + Q^\prime e^Q \dpd{Q}{\xi_j}\delta_{0k}  + c_0e^Q \dpd{Q}{\xi_j}\dpd{Q^\prime}{\xi_k}
% + c_0Q^\prime e^Q \dpd{Q}{\xi_j}\dpd{Q}{\xi_k} \right)\right]
% \end{aligned}
% \end{equation}

\begin{equation}	
\begin{aligned}
\mathcal{H}_{jk} &= 2\sum_i\left[ (Q^\prime e^Q)^2  \delta_{0j}\delta_{0k} + c_0Q^\prime e^{2Q} \left(\delta_{0j}\dpd{Q^\prime}{\xi_k} + \dpd{Q^\prime}{\xi_j} \delta_{0k}\right) \right.	\\
&+ c_0 (Q^\prime e^Q)^2 \left(\delta_{0j}\dpd{Q}{\xi_k} + \dpd{Q}{\xi_j}\delta_{0k}\right) + (c_0 e^Q)^2 \dpd{Q^\prime}{\xi_j}\dpd{Q^\prime}{\xi_k} \\
&+  (c_0e^Q)^2Q^\prime \left( \dpd{Q^\prime}{\xi_j}\dpd{Q}{\xi_k}	+ \dpd{Q}{\xi_j}\dpd{Q^\prime}{\xi_k}\right) + (c_0Q^\prime e^Q)^2 \dpd{Q}{\xi_j}\dpd{Q}{\xi_k} \\
&+ (c_0Q^\prime e^Q - f_i)\left.\left[ e^Q \left(\delta_{0j}\dpd{Q^\prime}{\xi_k} + \dpd{Q^\prime}{\xi_j}\delta_{0k}\right) + Q^\prime e^Q \left(\delta_{0j}\dpd{Q}{\xi_k}  + \dpd{Q}{\xi_j}\delta_{0k}\right)
\right.\right.	\\
&\left.\left.+ c_0 e^Q\left(\dpd{Q^\prime}{\xi_j}\dpd{Q}{\xi_k}  +  \dpd{Q}{\xi_j}\dpd{Q^\prime}{\xi_k}\right)
+ c_0Q^\prime e^Q \dpd{Q}{\xi_j}\dpd{Q}{\xi_k} \right]\right].
\end{aligned}
\end{equation}
Here, the summation is over each data point $i$ and then sums for each component of stress $S_m$, $S_n$, and $S_\phi$ for $Q^\prime = \pd{Q}{E_m}$, $\pd{Q}{E_n}$ and $\pd{Q}{E_\phi}$ respectively. 

	Using this approach, we compared the covariance of the model parameters for with Green-Lagrange strains and with Hencky strains for the Fung model (Table \ref{tb:correlationE}\&\ref{tb:correlationG}, Fig. \ref{fig:gvsecorrelation}), $\Psi_{eff}$ (Eqn. \ref{eqn:finalexponentialmodelformscaled}) (Fig. \ref{fig:gvsecorrelationeff}), and the polynomial series type model with $i,j,k \leq 6$ (Fig. \ref{fig:gvsecorrelationpoly}). The Hencky strains have lower parameter correlation as shown, but this appears to be minimal for $\Psi_{eff}$. The main benefits are seen with higher order coupling terms, and may still be beneficial for other constitutive models forms. 



%----------------------------------------------------------%
%-------------------	begin TABLE 	-------------------%
\begin{table}[ht]
\caption{The correlation between model parameter when using Green Lagrange strains}
\begin{center}
\label{tb:correlationE}
\small 
\begin{tabular}{|l|cccccc|}
\hline
		 & $b_1$ & $b_2$& $b_3$& $b_4$& $b_5$& $b_6$ \\
\hline  
$b_1$	 & 1. & 0.0439916 & 0.0195981 & -0.571707 & -0.545413 & 0.314503 \\
$b_2$	 & 0.0439916 & 1. & 0.0195981 & -0.571707 & 0.314503 & -0.545413 \\
$b_3$	 & 0.0195981 & 0.0195981 & 1. & 0.306268 & -0.555437 & -0.555437 \\
$b_4$	 & -0.571707 & -0.571707 & 0.306268 & 1. & -0.167068 & -0.167068 \\
$b_5$	 & -0.545413 & 0.314503 & -0.555437 & -0.167068 & 1. & -0.179471 \\
$b_6$	 & 0.314503 & -0.545413 & -0.555437 & -0.167068 & -0.179471 & 1. \\
\hline
\end{tabular}
\normalsize
\end{center}
\end{table}
%-------------------	 end TABLE 		-------------------%
%----------------------------------------------------------%


%----------------------------------------------------------%
%-------------------	begin TABLE 	-------------------%
\begin{table}[ht]
\caption{The correlation between model parameter when using Hencky strains}
\begin{center}
\label{tb:correlationG}
\small
\begin{tabular}{|l|cccccc|}
\hline
		& $b_1$ & $b_2$& $b_3$& $b_4$& $b_5$& $b_6$ \\
\hline
$b_1$	& 1. & 0.0709169 & 0.016636 & -0.590101 & -0.576084 & 0.322457 \\
$b_2$	& 0.0709169 & 1. & -0.0173651 & -0.601126 & 0.282699 & -0.49857 \\
$b_3$	& 0.016636 & -0.0173651 & 1. & 0.294546 & -0.514879 & -0.524943 \\
$b_4$	& -0.590101 & -0.601126 & 0.294546 & 1. & -0.100413 & -0.171661 \\
$b_5$	& -0.576084 & 0.282699 & -0.514879 & -0.100413 & 1. & -0.222942 \\
$b_6$	& 0.322457 & -0.49857 & -0.524943 & -0.171661 & -0.222942 & 1. \\
\hline
\end{tabular}
\normalsize
\end{center}
\end{table}
%-------------------	 end TABLE 		-------------------%
%----------------------------------------------------------%


%%%%%%%%%%%%%%%%%%%%%%%%%%%%%%%%%%%%%%%%%%%%%%%%%%%%%%%%%%%%
%-------------------	begin FIGURE 	-------------------%
\begin{figure}
\centering
\includegraphics[width=6.0in]{Images/chapter5/gvsecorrelationeff}
\caption{(A) The correlation between parameters pairs in $\Psi_{eff}$ (Eqn. \ref{eqn:finalexponentialmodelformscaled}) when using Green-Lagrange vs Hencky strains. (B) The difference in correlation between each pair of parameters is minimal.}
\label{fig:gvsecorrelationeff}
\end{figure}
%-------------------	 end FIGURE 	-------------------%
%%%%%%%%%%%%%%%%%%%%%%%%%%%%%%%%%%%%%%%%%%%%%%%%%%%%%%%%%%%%


%%%%%%%%%%%%%%%%%%%%%%%%%%%%%%%%%%%%%%%%%%%%%%%%%%%%%%%%%%%%
%-------------------	begin FIGURE 	-------------------%
\begin{figure}
\centering
\includegraphics[width=\textwidth]{Images/chapter5/gvsecorrelationpoly}
\caption{(A) The correlation between parameters pairs in a polynomial series type model with powers up to 6 when using Green-Lagrange vs (B) Hencky strains. (C) The difference in correlation between each pair of parameters. The red bracketed terms are benefited from using Hencky strains whereas the significantly fewer blue bracketed terms has better correlations with Green-Lagrange strain.}
\label{fig:gvsecorrelationpoly}
\end{figure}
%-------------------	 end FIGURE 	-------------------%
%%%%%%%%%%%%%%%%%%%%%%%%%%%%%%%%%%%%%%%%%%%%%%%%%%%%%%%%%%%%
















%---    Discussion
\section{Optimal \textit{in silico} loading paths} \label{sec:optimalpaths}

	The most optimal loading path is the equibiaxial stress loading paths. This is not surprising as the equibiaxial stress response generally given very intuitive information on the mechanical response of soft tissues. Any odd number of loading paths will include the equibiaxial stress loading paths (Fig. \ref{fig:oddpaths}). Even when the number is even, we can observe that at least a few protocols are nearly equibiaxial in stress (Fig. \ref{fig:evenpaths}). For this reason, using an odd number of loading paths is highly recommended, as they will always include the minimal necessary set of loading paths, with the rest being average ratios in between (Fig. \ref{fig:oddpaths}). Even number of loading paths are generally very in consistent, and change slightly with each additional pairs of loading paths.  

%%%%%%%%%%%%%%%%%%%%%%%%%%%%%%%%%%%%%%%%%%%%%%%%%%%%%%%%%%%%
%-------------------	begin FIGURE 	-------------------%   
\begin{figure}
\centering
\includegraphics[width=5.5in]{Images/chapter5/oddpaths}
\caption{The optimal loading paths for generating data for a given total odd number of paths as well as the associated mechanical response, which is consistently containing the equi-biaxial stress loading path and the loading paths at the boundary.}
\label{fig:oddpaths}
\end{figure} 
%-------------------	 end FIGURE 	-------------------%
%%%%%%%%%%%%%%%%%%%%%%%%%%%%%%%%%%%%%%%%%%%%%%%%%%%%%%%%%%%%


%%%%%%%%%%%%%%%%%%%%%%%%%%%%%%%%%%%%%%%%%%%%%%%%%%%%%%%%%%%%
%-------------------	begin FIGURE 	-------------------%   
\begin{figure}
\centering
\includegraphics[width=5.5in]{Images/chapter5/evenpaths}
\caption{The optimal loading paths for generating data for a given even total number as well as the associated mechanical response, which is much more unpredictable than the odd (Fig. \ref{fig:oddpaths}) but gravitates towards the boundaries and the equi-biaxial loading paths as the number increases.}
\label{fig:evenpaths}
\end{figure} 
%-------------------	 end FIGURE 	-------------------%
%%%%%%%%%%%%%%%%%%%%%%%%%%%%%%%%%%%%%%%%%%%%%%%%%%%%%%%%%%%%

%---    Conclusion
\section{Additional results for other tissue types and effective constitutive model forms} \label{sec:otherresults}

	Bovine pericardium is a good soft tissue to start with due high axes stretch coupling from their broad fiber splays. However, soft tissue behavior is drastically different with greater degree and anisotropy. For example aortic valve leaflets have larger elastin content resulting in larger toe region and extremely narrow ODFs, which can cause contraction along the material axis under equi-biaxial tension \cite{billiar_biaxial_2000b}. This behavior is hard for most constitutive models to replicate. However, $\Psi_{eff}$ (Eqn. \ref{eqn:finalexponentialmodelformscaled}) has no problem replicating this behavior (Fig. \ref{fig:aorticfit}). 

%%%%%%%%%%%%%%%%%%%%%%%%%%%%%%%%%%%%%%%%%%%%%%%%%%%%%%%%%%%%
%-------------------	begin FIGURE 	-------------------%
\begin{figure}
\centering
\includegraphics[width=\textwidth]{Images/chapter5/aorticfit}
\caption{Parameter estimation results for porcine aortic valve specimen with highly align collagen fibers, $\sigma_{ODF} =10\deg$. The response function A) $S_m$ and b) $S_n$ are shown. C) This specimen contracts in the preferred fiber direction under equi-biaxial tension, a trait of soft tissues with highly aligned collagen fibers. D) $\Psi_{eff}$ (Eqn. \ref{eqn:finalexponentialmodelformscaled}) is able to reproduce this effect.}
\label{fig:aorticfit}
\end{figure} 
%-------------------	 end FIGURE 	-------------------%
%%%%%%%%%%%%%%%%%%%%%%%%%%%%%%%%%%%%%%%%%%%%%%%%%%%%%%%%%%%%
    
    Based on D-optimality, we determined the optimal loading paths and shown its importance for model predictability. The equi-biaxial stress loading path (near equi-biaxial) came out as especially important, as it is included in all sets of optimal loading paths except for two and four (Fig. \ref{fig:oddpaths}\&\ref{fig:evenpaths}). Commonly, when reproducing results from older publication, only the equi-biaxial stress loading path may be included in the figures as it is the most informative. Indeed, using the equi-biaxial stress loading paths simply results in much higher D-optimality than other choices. However, using this loading path alone is not sufficient for parameter estimation. Although the quality of fit is very good (Fig. \ref{fig:effequifit}A\&B), it cannot predict other loading paths (Fig. \ref{fig:effequifit}C\&D) 
    
%%%%%%%%%%%%%%%%%%%%%%%%%%%%%%%%%%%%%%%%%%%%%%%%%%%%%%%%%%%%
%-------------------	begin FIGURE 	-------------------%
\begin{figure}[!hbtp]
\centering
\includegraphics[width=\textwidth]{Images/chapter5/effequifit}
\caption{$\Psi_{eff}$ fitting to a single equi-biaxial loading path, showing the A) $S_{11}$ and B) $S_{22}$ component. The prediction for the C) $S_{11}$ component and D) $S_{22}$ component of the unfitted loading paths are poor. The inset in C shows the corresponding loading paths.}
\label{fig:effequifit}
\end{figure} 
%-------------------	 end FIGURE 	-------------------%
%%%%%%%%%%%%%%%%%%%%%%%%%%%%%%%%%%%%%%%%%%%%%%%%%%%%%%%%%%%%

	
%%%%%%%%%%%%%%%%%%%%%%%%%%%%%%%%%%%%%%%%%%%%%%%%%%%%%%%%%%%%
%-------------------	begin FIGURE 	-------------------%
\begin{figure}[!hbtp]
\centering
\includegraphics[width=\textwidth]{Images/chapter5/modelsfit}
\caption{All three models, A\&B) $\Psi_{eff}$ (Eqn. \ref{eqn:finalexponentialmodelformscaled}), C\&D) extended Fung (\ref{eqn:fullsunmodel}), and E\&F) the Sun model (\ref{eqn:extendedfung}), can fit the data equally as well.}
\label{fig:modelsfit}
\end{figure} 
%-------------------	 end FIGURE 	-------------------%
%%%%%%%%%%%%%%%%%%%%%%%%%%%%%%%%%%%%%%%%%%%%%%%%%%%%%%%%%%%%

%%%%%%%%%%%%%%%%%%%%%%%%%%%%%%%%%%%%%%%%%%%%%%%%%%%%%%%%%%%%
%-------------------	begin FIGURE 	-------------------%
\begin{figure}[!hbtp]
\centering
\includegraphics[width=\textwidth]{Images/chapter5/modelspred}
\caption{The predictions for the unfitted loading paths from Fig. \ref{fig:modelsfit}. The blue and red arrows points out the poorly predicted paths, and which path they are on the inset figures.}
\label{fig:modelspred}
\end{figure} 
%-------------------	 end FIGURE 	-------------------%
%%%%%%%%%%%%%%%%%%%%%%%%%%%%%%%%%%%%%%%%%%%%%%%%%%%%%%%%%%%%


    With the addition of other loading paths, even non-optimal, this can significantly improve the predictive capabilities (Fig. \ref{fig:modelspred}A\&B). However, we can clearly see that because the loading paths are not optimal, the $0.1/1$ loading path is not predicted very well (Fig. \ref{fig:modelspred}B). We tested to see if this can be improved through more specific forms of $\Psi_{eff}$ (Eqn. \ref{eqn:finalexponentialmodelformscaled}). For this, we looked at an extension of the generalized Fung model to quadratic terms presented by Sun \textit{et al.} \cite{sun_biaxial_2003} to better fit the response of the glutaraldehyde cross-linked bovine pericardium. The additional terms are only of cubic and quartic powers, $B_{ijkl}E_{ij}^2E_{kl}^2$. In comparison to $\Psi_{eff}$, this is only missing $E_m^3E_n$ and $E_m^3E_n$,
%==========================================================%
%-------------------	begin EQUATION 	-------------------%
\begin{equation}\label{eqn:fullsunmodel}
\begin{aligned}
\Psi	=& c_0 \left(e^{Q} - 1\right) \\
Q		=& A_1 E_{11}^2 + A_2 E_{22}^2 + 2A_3E_{11}E_{22} + A_4 E_{12}^2 + 2A_5E_{12}E_{11}	\\
	&+ 2A_6E_{12}E_{22} + B_1 E_{11}^4 + B_2 E_{22}^4 + 2B_3E_{11}^2E_{22}^2 + B_4 E_{12}^4	\\
    &+ 2B_5E_{12}^2E_{11}^2 + 2B_6E_{12}^2E_{22}^2 
\end{aligned}\tag{Sun et al. \cite{sun_biaxial_2003} Eqn. 4}
\end{equation}
%-------------------	 end EQUATION 	-------------------%
%==========================================================%   
We will call this the extended Fung model. In addition, Sun \textit{et al.} \cite{sun_biaxial_2003} also recommended a more minimalistic form, where only the coupling term $2B_3E_{11}^2E_{22}^2$ is added as , 
%==========================================================%
%-------------------	begin EQUATION 	-------------------%
\begin{equation}\label{eqn:extendedfung}
\begin{aligned}
\Psi	=& c_0 \left(e^{Q} - 1\right) \\
Q		=& A_1 E_{11}^2 + A_2 E_{22}^2 + 2A_3E_{11}E_{22} + A_4 E_{12}^2 + 2A_5E_{12}E_{11}	\\
	&+ 2A_6E_{12}E_{22} + 2B_3E_{11}^2E_{22}^2 + B_4 E_{12}^4.
\end{aligned}
\end{equation}
%-------------------	 end EQUATION 	-------------------%
%==========================================================% 
    We shall call this the Sun model. 

	Although the equality of fit are all equally as good \ref{fig:modelsfit}, the extended Fung model (Eqn. \ref{eqn:fullsunmodel}) predicts $S_{22}$ component of the 0.1/1 loading path much better, but predicts the wrong sign for $S_{11}$ for the same loading path, as well as predicting the 1/0.1 loading path worse (Fig. \ref{fig:modelspred}C\&D). The Sun model (Eqn. \ref{eqn:extendedfung}) on the other hand is similar to $\Psi_{eff}$ but worse at predicting $S_{11}$ of the 1/0.1 loading path (Fig. \ref{fig:modelspred}E\&F). It's hard to predict how these constitutive models will behave when the loading paths are not optimal. This is especially true when only a single protocol is used, where predictions for all other protocols can be very poor (Fig. \ref{fig:effequifit}). However, as little as three loading paths are needed to fully reproduce the mechanical response of soft tissue over the entire range of deformations (Fig. \ref{fig:effoptpred}).





%---    Bioliography
\bibliographystyle{plainnat}
\bibliography{phd}


%\bibliographystyle{plain}  % Here the bibliography 		     %
%\bibliography{phd}        % is inserted.			     %
%\index{Bibliography@\emph{Bibliography}}%			     %


%%%%%%%%%%%%%%%%%%%%%%%%%%%%%%%%%%%%%%%%%%%%%%%%%%%%%%%%%%%%%%%%%%%%%%
% Generate the index.						     %
%%%%%%%%%%%%%%%%%%%%%%%%%%%%%%%%%%%%%%%%%%%%%%%%%%%%%%%%%%%%%%%%%%%%%%
%								     %
% NOTE: For master's theses and reports, NOTHING is permitted to     %
%	come between the bibliography and the vita. This section     %
%	to generate the index (if used) MUST be moved to before      %
%	the bibliography section.				     %
%								     %
%\printindex%    % Include the index here. Comment out this line      %
%		% with a percent sign if you do not want an index.   %
%%%%%%%%%%%%%%%%%%%%%%%%%%%%%%%%%%%%%%%%%%%%%%%%%%%%%%%%%%%%%%%%%%%%%%

%%%%%%%%%%%%%%%%%%%%%%%%%%%%%%%%%%%%%%%%%%%%%%%%%%%%%%%%%%%%%%%%%%%%%%
% Vita page.							     %
%%%%%%%%%%%%%%%%%%%%%%%%%%%%%%%%%%%%%%%%%%%%%%%%%%%%%%%%%%%%%%%%%%%%%%

\begin{vita}
Will Zhang, 
born in Linhe, Inner Mongolia, China, on 10 August 1989, 
the son of Dr. Wenxiu Zhang and Guiping Tian.  He received the Bachelor
of Science in Honours Biophysics from University of British Columbia.
Continuing his academic career, he applied to the University of Texas at Austin for enrollment in their graduate program in Biomedical Engineering. He was accepted and started graduate studies in May, 2018.

\end{vita}


\end{document}



\section{Valvular diseases and prevalence}

    Cardiovascular diseases are the number one killer in the United States and around the world. Heart valve treatment is a common cardiovascular surgical procedure with over 100,800 done annually in the U.S. alone \cite{mozaffarian_heart_2015} and 275,000 to 370,000 in developed nations \cite{manji_future_2012}. Calcification is the primary cause of AV failure and no proven therapy currently exists for halting this progression. Calcified aortic valve disease is a slow, progressive, multi-factorial disorder that is more common with age, without being an inevitable consequence of aging \cite{towler_molecular_2013,freeman_management_2002,freeman_spectrum_2005,kurtz_aortic_2010,beckmann_insights_2010}. The disease is characterized by a thickening and calcification of the leaflets and is diagnosed in two stages: aortic sclerosis and aortic stenosis. aortic sclerosis, present in more than 25\% of patients over the age of 65 \cite{obrien_pathogenesis_2006}, represents the early onset of calcified aortic valve disease absent of physical obstruction to the left ventricular outflow. Aortic stenosis exists in 2-5\% of the elderly population \cite{obrien_pathogenesis_2006}, which is characterized by late-stage obstruction, impaired leaflet motion, valve tissue adaptation, and resistance to blood flow \cite{poggio_noggin_2013,grau_analysis_2012,gharacholou_aortic_2011,pflederer_aortic_2010}. Although aortic sclerosis causes significant thickening of the AV leaflets, there is little to no change in the mechanical properties of the valve, making the disease relatively asymptomatic. Recent statistics have shown that within 10 years of their initial diagnosis, 10\% of aortic sclerotic patients reach a state of severe calcified aortic valve disease that requires immediate AV replacement once symptoms emerge \cite{gharacholou_aortic_2011}. 
    
    
    Aortic stenosis has been identified as the end-stage of calcified aortic valve disease that progresses from the microscopic early changes of aortic sclerosis to, in a subset of patients, asymptomatic and then symptomatic Aortic stenosis \cite{kurtz_aortic_2010,otto_calcific_2010,aikawa_look_2011}. Once aortic sclerosis is detected, there is an increased risk of cardiovascular events. In early Aortic stenosis, when mild symptoms begin to present, survival rates deviate much more than expected and decline dramatically with the onset of severe symptomatic Aortic stenosis. Over the last decade, several clinical trials, mostly extensions of atherosclerosis-related studies, have been performed to halt the progression of calcified aortic valve disease with randomized studies showing substantial equivalence between treatments and placebo \cite{parolari_do_2011,moura_rosuvastatin_2007,cowell_randomized_2005,benton_statins_2009,rossebo_intensive_2008}. However, there are currently no pharmacological therapies available to treat calcified aortic valve disease symptomatic patients that are considered superior to full valve replacement surgery. 



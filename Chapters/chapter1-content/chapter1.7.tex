\section{Motivation, rationale, and specific aims}

\subsection{Motivation and innovation}

\subsubsection{The Lack of Established Constitutive Models for Planar Soft Tissues}
    Although finite element analysis is an effective method for stress analysis, truly predictable computational simulations cannot be done without an accurate constitutative model for the underlying biomechanical response. There is currently a lack of a well establish constitutive model for collagenous soft tissues. Currently each research group uses their own model, often simple forms of the Fung model or linear functions of the invariants. These models are often not informative, offering no insights into the structural functional relationship. Unfortunately, this also means these models are also lack predictive capabilities as a result, and can only interpolated the experimental data. This is a problem for simulating the effect of fatigue in which the geometry and the mechanical properties of BHVs can change significantly, necessating extrapolation of the data into unmeasured regime. Structural models theories exists [27-29] but is under utilized and not well validated for the wide range of tissue that may be applicable for BHV devices. Thus, the establishment of a generalized and well-validated constitutive model is a critical step the in process of developing accurate simulations of BHV devices.

\subsubsection{The Lack of Fundamental Modeling Approach for EXLs, PS, and Collagen Fiber-Level Damage in Soft Tissue-Derived Biomaterials}

    There is no accepted constitutive model for the effect of EXLs on the mechanical properties of soft tissues. Without understanding the mechanical properties of the subcomponents of the bioprosthetic biomaterial, we cannot properly predict how they will change over time. Although calcification plays an important role in the failure of BHVs, addressing this concern alone will not alleviate the loss of functional due to mechanical fatigue. Although this proposal will not simulate the full process to failure, as previously indicated (1.B), we will for the first time examine the early stages of fatigue in PS and collagen fiber-level damage. This will nevertheless give us a firsthand estimate of BHV durability and design, as well as build the foundation for future extensions of the model.

\subsubsection{The Lack of A Framework for Simulating The Response of BHV Devices in Response to Cyclic Loading}

    Our ability to evaluate the design of BHVs devices are directly linked to our ability to understand and rigorously experiment on the in vivo durability of these devices. However, clinical and large anime testing remains costly and reserved for latter stages of the design analysis. Furthermore, it is difficult for such studies to paint a complete picture of the underlying mechanisms due bio-variability and direct ways of controlling the biological system. This has led to a stagnation of BHV device development \cite{schoen_cardiac_2005}. Computational simulations can be an exceptional complement to in vitro and animal experiment in understanding how biomechanical properties evolve and drive mechanical function and performance. Indeed, the consensus in the field is the increasing role of computational simulation in the development of cardiovascular devices [30]. Finite element analysis is unknown for its ability to determine the stress at a device-level and can be used to reiterate on conceptual designs to optimize the functional properties of prototypes [31-33]. Thus, computational implementation of the tissue structural PS-fatigue model is the best way to guide our understanding of BHV device performance.

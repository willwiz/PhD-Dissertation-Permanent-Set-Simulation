Heart valves are complex multi-layered structures that prevent backflow by opening and closing depending on the direction of flow. There are four valves in the heart, consisting of two atrioventricular preventing backflow between the atriums and ventricles, the mitral (MV) and tricuspid valves (TV), and two semilunar preventing backflows from the aorta and to the vena cava, the aortic (AV) and pulmonary valves (PV) (Fig. 1). The coordinated movement of the four heart valves enables them to maintain unidirectional blood flow during the cardiac cycle. When healthy, heart valves are incredibly resilient, opening and closing approximately 3 - 4 billion times throughout an average life-span [1]. The pressure changes during the cardiac cycle expose the heart valves to constant changes in forces and hemodynamics. This physiological demand is especially harsh on the mitral and aortic valves, needing to withstand average pressures of 80mmHg for the aortic and 120mmHg for the mitral valve to sustain circulation throughout the rest of the body. The biomechanical properties of heart valves must be able to withstand and function efficiently in this complex mechanical environment. Thus, the heart valve leaflets develop and maintain an intricate, highly organized, and multi-scale connective tissue system that allows them to do so [10]. 

Heart valves are multi-layered structures with specialized extracellular matrix structure. The aortic valve, for example, consists of three histologically distinct layers, whereas the mitral valve has four (Fig. 1.2 and 1.3). The fibrosa layer, which is located on the ventricular side of atrioventricular valvesand the atrial side of semilunar valves, is composed of circumferentially aligned collagen fibers that provide the leaflets with the necessary tensile strength to open and transmit forces during coaptation while closed. The spongiosa layer is situated adjacent to the fibrosa and though it contains some collagen, its main constituents are the hydrophilic glycosaminoglycans and proteoglycans, which give the valve its compressive properties and allow it to absorb high forces during coaptation. The ventricularis and atrialis are the layers that are adjacent to blood flow in atrioventricular valvesand semilunar valves, respectively. These layers are rich in radially oriented elastin fibers and facilitate the closure movement by extending the valve leaflet as it opens and recoils when it closes. The annulus and chordae tendineae of the atrioventricular valves and the connection between the leaflets and the surrounding myocardium in the SLV provide additional support. 

\subsection{Bioprosthetic heart valve replacements}
     
    Since initially introduced in 1960, valve replacement surgery has significantly reduced the mortality of patients with heart valve diseases \cite{braunwald_complete_1960,pibarot_valvular_2009,ross_replacement_1967}. As reported by The Society of Thoracic Surgeons 59,555 Americans underwent valve replacement surgery in 2014 (48,060 AV replacement, 9595 MV replacement, 1900 replacing both the AV and MV).  Of the four heart valves, the AV has been studied most, followed next by the MV, whereas fewer studies have been performed. This is primarily due to the fact that the AV \cite{christie_age_1995,merryman_correlation_2006,adamczyk_characteristics_2002,sacks_aortic_1998,billiar_biaxial_2000,billiar_biaxial_2000b,vesely_comparison_1998,vesely_micromechanics_1992,vesely_role_1998} and MV \cite{stella_biaxial_2007,sacks_surface_2002,gorman_dynamic_1996,gorman_effect_2004} are more suspectible to diseases and more frequently warrant replacement surgery. While important differences exist in valve geometries and function, the mechanics of the AV are primarily used as a baseline for the development of models of heart valve function. Thus, this dissertation will focus on the aortic valve as an example as it is one of the most common valves for heart valve replacement surgeries. The same framework proposed in this dissertation is otherwise applicable to the mitral and other valves. 
    
    
    Current clinically surgical valve replacement utilize either mechanical valves (usually made of pyrolytic carbon or titanium) or valves constructed from biologically\Hyphdash derived soft tissues.  Mechanical valves have evolved significantly since the first ball-and cage design, spanning a variety of shapes sizes and materials and rarely experiencing mechanical failure. However, they non-physiologic blood flow patterns and results in a substantial thrombogenic response, requiring lifelong post-operative anticoagulation therapy and its inherent risks \cite{pibarot_valvular_2009}.
    
    
    On the other hand, bioprosthetic heart valves (BHVs) are comprised of decellularized bovine pericardium or porcine aortic valve tissues (most commonly the pericardium) and offer a higher degree of functionality including improved hemodynamics and a higher resistance to thrombosis. Although they are chemically fixed, these tissues are still prone to valve calcification, structural deterioration, and eventual failure \cite{sacks_collagen_2002,vyavahare_mechanisms_1999}. Durability is the major limitation of current BHV technology—the 15-year durability of heterograft BHVs in the aortic position is less than 50\% for middle-aged patients and slightly better for older patients, however BHVs continue to be the preferred replacement valve \cite{siddiqui_bioprosthetic_2009}. Regardless of its shortcomings, heart valve replacement has had a substantial impact on cardiac surgery with a consistently increasing number of surgeries per year \cite{starr_artificial_2007}.

    
    Currently, bovine pericardium (BP) and porcine AV tissues are still the only clinically approved xenograft biomaterial for BHVs. Other tissue sources have been pursued, but none has reached clinical widespread application. Collagenous soft tissues used for bioprosthetic devices are EXL, often using glutaraldehyde (GLUT), to suppress antigenic properties of the tissue and to stablized the CFA \cite{khor_methods_1997}. GLUT is popular due to its affordability and solubility in an aqueous solution \cite{jayakrishnan_glutaraldehyde_1996}. However, GLUT is also implicated in the calcification of BHVs \cite{golomb_role_1987}, which is further accelerated by mechanical stress \cite{schoen_calcification_2005}. There exists some alcohol based pretreatments for GLUT EXL BHV to reduce calcification \cite{vyavahare_prevention_1997}, but there is little that we understand about the mechanical properties of EXL tissue. Specifically, there have been no attempt to constitutively model exogenous cross-linking and how it modulate the stresses. 
    
    
    GLUT is observed to significally stiffen the ECM but not that of the collagen fibers \cite{gentleman_mechanical_2003,yang_mechanical_2008,yang_micromechanical_2007}. It is also a Schiff-base, which means it reversibly reacts with aldehydes. Over time, this scission-healing behavior allows for a gradual change in the reference state, resulting in PS. This has yet to be properly modeled in biological tissues. Although there have not been any introduction of new cross-linking methods from a industrial stand point, research in alternative technologies such as carbodiimide is ongoing \cite{sung_crosslinking_2003,billiar_effects_2001,kemp_effects_1995}, including a new method developed by our collaborator to stablized elastin and GAGs in addition to collagen \cite{tam_novel_2015}. As GLUT and these other exogenous cross-linking methods remain necessary, it is important that we understand their role in predicting the fatigue and durability of bioprothetic devices.
    
    
    The most recent development in BHV technology is the percutaneous (or transcatheter) valve replacement. This technology is a minimally invasive procedure which delivers the replacement valves with catheterization from a large blood vessel, most commonly the femoral artery. This procedure makes valve replacement more feasible for those who cannot tolerate full surgical interventions. However, this new technology also presents additional design challenges, including complex folding and compression during delivery. As a result, the leaflets are required to be significantly thinner than in traditional BHVs, which increases the leaflet stress and potentially the rate of failure. Existing data on transcatheter aortic valve interventions suggest a 2-year mortality rate of 33.9\% overall \cite{mozaffarian_heart_2016} and over 68\% when specifically replacing stenotic aortic valves \cite{makkar_transcatheter_2012}. As such, this further accentuates the need to develop an approach to improve BHV durability. 
    
    
    BHVs have become the preferred replacement valve. The sustained growth of AV and MV replacement surgeries, the lack of substantial technological breakthroughs in the field over the last decades \cite{soares_biomechanical_2016}, and the continuous need to improve BHV durability, promote a strong necessity to develop a higher understanding of the mechanisms involved in BHV function and failure.  Critical BHV engineering aims to ensure valve functionality for its clinical performance with a combination of hemodynamic, biomechanical and biological aspects, and to predict and extend as much as possible valve durability. The major hurdle of the technology remains the lifespan of the BHV implant. Improvements to BHV design and durability have substantial clinical impact, even if marginally (i.e. 3–5 years) \cite{jamieson_carpentier_1998}.
    
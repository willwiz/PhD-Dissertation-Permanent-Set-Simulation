    Soft-tissue-derived exogenously cross-linked (EXL) biomaterials continue to play an important role in surgical repair and medical devices. This is especially true for bioprosthetic heart valves (BHV), by having advantages in immunogenic and mechanical behaviors [1]. Despite ongoing research, our understanding of these materials and of the mechanisms leading to their failure remains at an empirical level. The need for advancements in predicting the material behavior is further underscored by the development of percutaneously-delivered BHV devices. While these devices reduce surgical risk, they also present additional challenges for the design of the BHVs due to limitations in thickness, and folding and compression during delivery. A significant challenge prohibiting accurate and predictive simulations of BHVs are a lack of understanding of the effect of exogenous cross-linkers, such as glutaraldehyde (GLUT), and an associated predictive material model in response to cyclic loading. GLUT EXLs form polymeric chains through the cross-linking process which more tightly bond the collagen fibers to the non-fibrous matrix, increasing the non-fibrous matrix stiffness and fiber-fiber interactions. However, GLUT EXLs also undergo Schiff-base reactions that are hypothesized to lead to scission-healing behaviors that result in the permanent set (PS) phenomena. PS continuously changes in the reference geometry of the BHV and can induce extra-physiological stress concentrations in the BHV leaflets. Microstructural-based constitutive models for tissues and their use in valve-level simulations can lead to insights into the underlying mechanisms and more accurate prediction of their long-term performance. Thus, we seek to develop a framework for modeling and simulating biologically-derived EXL soft collagenous tissues for BHV, accounting for the effects of EXL, PS, and collagen fiber-level damage, taking the following approach:


    \subsubsection*{Specific Aim 1: Establish and validate a generalized nonlinear hyperelastic meso-scale structural constitutive model (MSSCM) for native collagenous soft tissues.} As a first step towards modeling EXL biomaterials, we will take a generalized meso-scale (at the level of the constituent fibers) structural modeling approach to accurately model the mechanical response of soft tissues. Firstly, we will develop an improved analytical method for processing extant mechanical data under generalized 2D deformations, to more accurately characterize the tissue. Next, after posing the model form, we will extensively validate critical assumptions such as affine fiber deformation, mechanical response of the constituent fibers, and direct integration of physical measurements of the fiber microstructure. The validation will be done through extensive mechanical characterization through different testing methods and microstructural characterization at the 1) micro-scale (fibril) and 2) meso-scale using optical techniques such as multiphoton microscopy and x-ray scattering. The two tissues we will examine for application and validation of the model are the ovine pulmonary artery and the porcine mitral valve leaflets, which offer a diverse selection structural composition to span applicable realms.
    
    \addcontentsline{toc}{subsubsection}{Specific Aim 1: Establish and validate a generalized nonlinear hyperelastic meso-scale structural constitutive model for native collagenous soft tissues.}%


    \subsubsection*{Specific Aim 2: Extend the MSSCM to account for of the presence of EXLs and subsequent response to continuous cyclic loading.} Using GLUT treated bovine pericardium, we aim to separately model the mechanical response of the collagen fibers, matrix, and fiber-fiber interactions due to cross-linking. For this, we will develop a method to map the collagen fiber architecture characterized from the native state to the EXL state. Next, we will develop a PS model based on a constantly evolving referential configuration that occurs due to scission-healing when the tissue is held in an extended state. We will validate the model for orientation and strain level dependence. The model will be tested and validated using constant cyclic strain, as well as stress control experimental data. Finally, we will extend the model for collagen fiber-level damage that may occur during cyclic loading. 
    
    \addcontentsline{toc}{subsubsection}{Specific Aim 2: Extend the meso-scale structural constitutive model to account for of the presence of EXLs and subsequent response to continuous cyclic loading.}%
    
    
    \subsubsection*{Specific Aim 3: Application to organ/device level.} In this final aim, we will develop a full 3D finite element implementation utilizing the open source FEniCS framework for greater modularity and extensibility. The model will utilize real BHV geometries and fiber microstructure mapped from experimental measurements, and will be validated by simulating of the PS experiments from SA 2. We will further perform organ-level study of PS and collagen fiber-level damage using accelerated wear testing of specialized ovine mitral transcutaneous BHVs. We will use an inverse modeling approach with micro-CT measurements [2] to quantify changes in mechanical behaviors. The validated model and rate constants will then be used to parametrically examine the changes in BHV geometry and stress distribution overtime, where change in resting and loaded geometry of the valve as well as the development of stress concentrations are the early indicators of BHV failure. We will also explore initial geometries and material properties which may minimize the risks of these effects. With this, we aim to develop a better understanding of the underlying process that occurs during long-term cyclic loading using our constitutive modeling approach and device level applications, and translate the insights gained to improving BHV design and durability. 

    \addcontentsline{toc}{subsubsection}{Specific Aim 3: Application to organ/device level}%
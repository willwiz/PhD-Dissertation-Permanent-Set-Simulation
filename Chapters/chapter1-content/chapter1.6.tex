\section{Computational approach to studying bioprosthetic heart valve failure}

    Computational modeling is a valuable tool for studying the mechanisms underlying the heart valves functions \cite{soares_biomechanical_2016}. Due to complex valve anatomy and dynamic environment, there is no in vitro system that can truly completely replicate and controllably perturb the complex in vivo environment. Computational simulations have been used for determining the stress distribution on the leaflets, correlate regions of stress concentration with regions of leaflet calcification and/or leaflet tear in implanted valves, and guide the design of biomedical device using these information. For example, Sun et al. \cite{sun_simulated_2005} presented a study of prosthetic valve deformation under quasi-static loading. In this study, quasi-static leaflet deformations under 40, 80, and 120 mmHg transvalvular pressures were simulated in a pericardial BHV. A Fung-elastic material model utilizing material parameters derived from actual leaflet biaxial tests and measured leaflet collagen fiber structure axes obtained from physical leaflets were used \cite{sun_finite_2005,sun_effects_2005}. However, central to any predictive simulation is a predictive and mechanism based constitutive model. There has been no true mechanism based models for predicting the evolution of material properties, geometry and microstruture for soft tissue or soft tissue derived material. Thus, the goal of this current dissertation is the constitutive modeling and simulation of evolving properties of BHVs under long term cyclic loading, utilizing the underlying mechanisms of soft tissue biomechanics and scission-healing of glutaraldehyde to predict geometrical and mechanical properties. 



\chapter{Constitutive modeling of soft tissues}

\section{Major structural bearing components of soft tissues}
\subsection{Collagen}
\subsection{Elastin}
\subsection{Proteoglycans and Glycosaminoglycans}
\subsection{Ground matrix}

\section{Characterizing the mechanical responses of soft tissues}
\subsection{Current techniques for acquiring mechanical responses of soft tissues}
\subsection{Methods for the analysis of mechanical data}
\subsection{Improved method for analyzing the biaxial mechanical response of soft tissues*}

\section{Material models for soft tissues}
\subsection{Phenomenological approaches}
\subsection{Multi-scale and other structural-mechanism approaches}
\subsection{Meso-scale structural approaches for collagenous soft tissues}

\section{Key mechanisms of meso-scale structural constitutive models}
\subsection{Fiber-level response of collagen and elastin fibers}
\subsubsection{Mechanical response of single collagen fibers}
\subsubsection{The elastica effect of collagen fibers}
\begin{itemize}
\item Write of the work on improved elastica model based on Garikipati's paper.
\item Summarizing the elastica portion of "Large strain stimulation promotes extracellular matrix production and stiffness in an elastomeric scaffold model" by Joao. 
\end{itemize}

\subsection{Modeling the tissue-level response}
\subsubsection{Response of collagen fiber ensembles}
\subsubsection{Affine kinematics in dense collagenous soft tissues}
Summarized the results of the affine kinematics paper by Chung-hao Lee here




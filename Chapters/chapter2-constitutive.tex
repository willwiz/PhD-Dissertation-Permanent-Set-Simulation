\chapter{Structural constitutive models for planar collagenous soft tissues}


\section*{Preface}
\addcontentsline{toc}{section}{Preface}%

    Fundamental to developing a deeper understanding of soft tissue function and pathology is the development of an accurate tissue-level constitutive model. In the present work, we developed a novel meso-scale (i.e. at the level of the fiber, 10-100 $\mu$m in length scale) structural constitutive model (MSSCM) with application to MV leaflet and ovine pulmonary arterial tissues. This model takes into account the layered structure of these tissue and the contributions from the distinct collagen and elastin fiber networks within each tissue layer. Requisite collagen and elastin fibrous structural information for each layer were quantified using second harmonic generation microscopy and conventional histology. A comprehensive mechanical data set was also used to guide model formulation and parameter estimation. Furthermore, novel to tissue-level structural constitutive modeling approaches, we allowed the collagen fiber recruitment function to vary with orientation. Finally, a novel fibril-level (0.1 to 1 $\mu$m) validation approach was used to compare the predicted collagen fiber/fibril mechanical behavior with extant MV small angle X-ray scattering data. Results demonstrated excellent agreement, indicating that the MSSCM fully captures the tissue-level function. Future utilization of the MSSCM in computational models of the MV will aid in producing highly accurate simulations in non-physiological loading states that can occur in repair situations, as well as guide the form of simplified models for real-time simulation tools.

\textbf{The work contained in this chapter was published as}:  %\bibentry*{}




\section{Major structural bearing components of soft tissues}
\subsection{Collagen}
\subsection{Elastin}
\subsection{Proteoglycans and Glycosaminoglycans}
\subsection{Ground matrix}

\section{Characterizing the mechanical responses of soft tissues}
\subsection{Current techniques for acquiring mechanical responses of soft tissues}
\subsection{Methods for the analysis of mechanical data}
\subsection{Improved method for analyzing the biaxial mechanical response of soft tissues*}

\section{Material models for soft tissues}
\subsection{Phenomenological approaches}
\subsection{Multi-scale and other structural-mechanism approaches}
\subsection{Meso-scale structural approaches for collagenous soft tissues}

\section{Key mechanisms of meso-scale structural constitutive models}
\subsection{Fiber-level response of collagen and elastin fibers}
\subsubsection{Mechanical response of single collagen fibers}
\subsubsection{The elastica effect of collagen fibers}
\begin{itemize}
\item Write of the work on improved elastica model based on Garikipati's paper.
\item Summarizing the elastica portion of "Large strain stimulation promotes extracellular matrix production and stiffness in an elastomeric scaffold model" by Joao. 
\end{itemize}

\subsection{Modeling the tissue-level response}
\subsubsection{Response of collagen fiber ensembles}
\subsubsection{Affine kinematics in dense collagenous soft tissues}
Summarized the results of the affine kinematics paper by Chung-hao Lee here




%%%%%%%%%%%%%%%%%%%%%%%%%%%%%%%%%%%%%%%%%%%%%%%%%%%%%%%%%%%%%
%%	Conclusion												%
%%%%%%%%%%%%%%%%%%%%%%%%%%%%%%%%%%%%%%%%%%%%%%%%%%%%%%%%%%%%%

% \section{Limitations and conclusion}

%-----------------------------------------------------------
%	Limitations
%-----------------------------------------------------------
\section{Limitations} 
	
    One major limitation of $\Psi_{eff}$ (Eqn. \ref{eqn:finalexponentialmodelformscaled}) is the larger number of parameters, 14 in the fully generalized form and 10 for most specific soft tissue types (Appendix \ref{sec:specificform}). This is not very favorable, where the time complexity for most optimization algorithms scales nonlinearly with the number of parameters. However, $\Psi_{eff}$ has very low computational cost, and reasonably low parameter covariance, thus this should not be a major problem. Alternatives are also less favorable, as they either require more parameters or cannot sufficiently capture the response of soft tissues in a wide range or reproduce the response of multiple tissue types. 
    
    Another limitation, which also applies to all phenomenological models, is that $\Psi_{eff}$ has no intrinsic mechanisms built in. Without sufficient mechanical data to derive the model parameters, phenomenological models have limited predictive capabilities. Specifically, the phenomenological models do poorly when extrapolating outside of the range of available data. This also means that phenomenological models can only provide limited information on how the tissue functions. It can reproduce the mechanical response of soft tissues very well, but it is also harder to infer more about the structure and function of the tissue. This is not a major concern for us. For the mechanisms, or the structure to function relationship of soft tissues, micro-models already fulfills the need. $\Psi_{eff}$ is intended as a fit all model for fulfilling the gap between predictive micro-models and computationally efficient simulations. With carefully selected of loading paths for parameter estimation (Section \ref{sec:optimaldesign}), $\Psi_{eff}$ can accurately reproduce the mechanical response of soft tissue within the expected range. In other words, $\Psi_{eff}$ does not have to be able to predict the mechanical response of soft tissues under unmeasured and extrapolated deformations, it only has to be able to fully reproduce the entire range of responses predicted by the micro-models, which is its main purpose. 



%-----------------------------------------------------------
%	Conclusion
%-----------------------------------------------------------
\section{Conclusion and Future Directions} 

	In this work, we developed a constitutive model form for the effective response of planar soft tissues. This effective constitutive model (Eqn. \ref{eqn:finalexponentialmodelformscaled}) is applicable to a wide range of soft tissue responses, while being as computationally efficient as most common phenomenological approaches, such as Holzapfel-Gasser-Ogden or the generalized Fung model. This model utilizes the modeling scaling method for minimizing covariance between parameter, which shows significant improvements in the speed and accuracy of parameter estimation. We have shown that our effective constitutive model along with optimal loading paths is able to fully replicate the response of complex meso-scale structural models for the entire range of deformations, whereas most phenomenological models have difficulties when predicting unfitted loading paths. The effective constitutive model is robust enough to be able to handle a wide range of soft tissue behaviors and anisotropy for accurate numerical simulations, such as in simulations of heart valves. Thus, the effective constitutive model can play an important role for the upscaling and homogenization of the response of complex micro-models for improving the efficiency for organ-level simulations. 
    

	One nature extension to this effective modeling approach is for 3 dimensional soft tissues. The extension to 3 dimensions doubles the number of inputs in comparison to planar models, with the additional inputs being $E_{13}$, $E_{23}$, and $E_{33}$. This means that the initial most generalized form for the 3-D soft tissue models (Eqn. \ref{eqn:exponentialmodelform}) has 209 terms before reduction. This is unmanageable for establishing the initial approach, where using a planar soft tissue model is more suitable. However, applying the same restrictions to the model form (section \ref{sec:finalform}) reduces this to 48 terms. This is possible due to symmetry by expressing the components of the Green-Lagrange strain tensor with respect to the material axis (Eqn. \ref{eqn:greenstrain}). This is not so easy to do in 3-dimensions, requiring a third vector corresponding to the direction of least stiffness, which in turn requires the 3 dimensional fiber orientation distribution. 
    
    
    
    
    
    
    
    

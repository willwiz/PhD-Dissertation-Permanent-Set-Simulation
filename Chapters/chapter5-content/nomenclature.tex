%%%%%%%%%%%%%%%%%%%%%%%%%%%%%%%%%%%%%%%%%%%%%%%%%%%%%%%%%%%%%
%%  nomenclature											%
%%%%%%%%%%%%%%%%%%%%%%%%%%%%%%%%%%%%%%%%%%%%%%%%%%%%%%%%%%%%%
\section*{Nomenclature} \label{c3:sec:nomenclature}
\begin{mynom}
\textbf{Key Terms}  \\
{\it Phenomenological model}\>\>\>\>\>\>\>\tabfill{Constitutive models that reproduces the mechanical response of materials without taking into considerations any underlying structure or mechanisms}\\
{\it Micro-models}\>\>\>\>\tabfill{Constitutive models utilizing structures and mechanism at a lower scale to predict the mechanical response at a higher scale}  \\
{\it Structural model}\>\>\>\>\>\tabfill{Constitutive models that utilize the meso-scale microstructures to predict the mechanical response of soft tissues}   \\
{\it Effective constitutive model}\>\>\>\>\>\>\>\tabfill{A computationally phenomenological model used to reproduce the response of micro material models in numerical simulations} \\
{\it Polynomial series family}\>\>\>\>\>\>\>\tabfill{Effective model forms composed primarily of sums of polynomial functions}    \\
{\it Separated exponential family}\>\>\>\>\>\>\>\>\tabfill{Effective model forms composed primarily of sums of exponential functions}  \\
{\it Single exponential family}\>\>\>\>\>\>\>\tabfill{Effective model forms composed primarily of a single exponential function of sums of polynomial functions}   \\
{\it ODF}\>\tabfill{Orientation distribution function}    \\
{\it RDF}\>\tabfill{Recruitment distribution function, the distribution of stretched needed to straighten the collagen fibers}    \\
{\it Physiologic range}\>\>\>\>\>\tabfill{The range of deformations most likely to contain the physiologic loading path}   \\
\textbf{Symbols}    \\
%-----------------------------------------------------------
%	Symbols
%-----------------------------------------------------------
{$s$}\>\tabfill{Scalar variables} \\
{$\mathbf{v}$}\>\tabfill{Vector variables, bold lower case} \\
{$\mathbf{M}$}\>\tabfill{Matrix variables, bold upper case} \\
{$\hat{\Psi},\hat{S},\hat{f}$}\>\tabfill{Data} \\
{$\mathbf{I}$}\>\tabfill{Identity tensor} \\
{$J$}\>\tabfill{The Jacobian for volume change due to deformation} \\
{$\mathbf{F}$}\>\tabfill{The deformation gradient tensor} \\
{$\mathbf{f}$}\>\tabfill{The upper triangular decomposition of the deformation gradient tensor} \\
{$\mathbf{C}$}\>\tabfill{Right Cauchy-Green strain tensor} \\
{$\mathbf{B}$}\>\tabfill{Left Cauchy-Green strain tensor} \\
{$\mathbf{E}$}\>\tabfill{Green Lagrange strain} \\
{$\mathbf{U}$}\>\tabfill{Right stretch tensor} \\
{$\mathbf{T}$}\>\tabfill{The Cauchy stress tensor} \\
{$\mathbf{S}$}\>\tabfill{Second Piola Kirchhoff tensor} \\
{$I_1$}\>\tabfill{First invariant of the right Cauchy strain tensor} \\
{$I_2$}\>\tabfill{Second invariant of the right Cauchy strain tensor} \\
{$I_3$}\>\tabfill{Third invariant of the right Cauchy strain tensor} \\
{$I_4$}\>\tabfill{Fourth pseudo-invariant of the right Cauchy strain tensor describing the stretch along an axis} \\
{$I_8$}\>\tabfill{Eighth pseudo invariant, describing the relative stretch along two axes} \\
{$I_8^{ext}$}\>\tabfill{The extensional component of $I_8$ } \\
{$\mathbf{m}_0$}\>\tabfill{The material axis in the reference configuration} \\
{$\mathbf{n}_0$}\>\tabfill{The perpendicular axis to the material axis in the reference configuration} \\
{$\mathbf{m}_t$}\>\tabfill{The material axis in the deformed configuration} \\
{$\mathbf{n}_t$}\>\tabfill{The perpendicular axis to the material axis in the deformed configuration} \\
{$\lambda_m$}\>\tabfill{The stretch along the material axis} \\
{$\lambda_n$}\>\tabfill{The stretch perpendicular to the material axis} \\
{$\phi$}\>\tabfill{The shear angle between $\mathbf{m}_0$ and $\mathbf{n}_0$} \\
{$E_m, E_n, E_\phi$}\>\>\tabfill{The Green Lagrange strain along the respective axes and to shearing} \\
{$S_m, S_n, S_\phi$}\>\>\tabfill{The 2nd Piola Kirchhoff stress along the respective axes and to shearing, which are also response functions, gradients of the strain energy} \\
{$\gamma_1,\gamma_2,\gamma_3$}\>\>\tabfill{The Hencky strains} \\
{$\Psi$}\>\tabfill{The strain energy} \\
{$\Psi_\mathrm{col}, \Psi_\mathrm{int}, \Psi_\mathrm{mat}$}\>\>\tabfill{Strain energy of the collagen fiber, ensemble-ensemble interactions, and matrix components respectively} \\
{$\eta_C$, $\eta_M$, $\eta_I$}\>\>\tabfill{The modulus of collagen, matrix and fiber-fiber interactions} \\
{$\phi_\mathrm{col}$, $\phi_\mathrm{mat}$, $\phi_\mathrm{int}$}\>\>\tabfill{The mass fractions of collagen, matrix and fiber-fiber interactions} \\
{$D$}\>\tabfill{Collagen fiber recruitment distribution function} \\
{$\Gamma$}\>\tabfill{Collagen fiber orientation distribution function} \\
{$\lambda$}\>\tabfill{Stretch} \\
{$\lambda_s$}\>\tabfill{The slack stretch, the stretch needed to straighten the collagen fiber crimp} \\
{$\mathcal{F}$}\>\tabfill{Objective function for parameter estimation} \\
{$\mathbfcal{I}$}\>\tabfill{The information matrix} \\
{$\mathbf{J}$}\>\tabfill{The Jacobian of the objective function} \\
{$\mathbfcal{H}$}\>\tabfill{The Hessian of the objective function} \\
{$\mathbf{\xi}$}\>\tabfill{Vector of material parameters}
\end{mynom}
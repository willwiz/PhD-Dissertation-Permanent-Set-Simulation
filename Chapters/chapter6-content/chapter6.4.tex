\section{Summary and Future Directions}
	We have developed the a complete time-dependent framework for the simulation of BHVs under long term cyclic loading. This simulation utilizes the predictive mechanism based constitutive model for the permanent set effect in exogenously crosslinked soft tissues that we previously developed. We have shown that we can use this simulation to predict the evolving geometry, microstructural and material property changes. These results can then be used to predict regions of increase likelihood of structural damage, and can be used to optimal the initial design of BHVs based on these factors. Most important of these effects is that the collagen fiber architecture can play a role in limiting the permanent set effect, where the straightening of collagen fibers prevents further changes in geometry. Thus, accounting for the permanent set effect is especially important in the design of BHVs to better improve their performance and durability. 
	
	
	The two main future extensions of this constitutive model and simulation is for: 1) structural damage and 2) growth and remodeling. Structural damage is difficult to quantify as it only gradually accumulates over long periods of time. Due to the exponential nature and large structural reserves of soft tissues, small decreases in the number or modulus of collagen fiber is very difficult to detect. This is also complicated by the fact that strain is difficult to quantify in the first place. In accelerated wear testing or other similar environments, this process is further complicate by the heterogeneity of the resulting response. However, by simulation and removing of the permanent set effect on the change in geometry, we can more accurately determine the remaining changes due to structural damage, This opens the doors to the development of mechanism-based structural damage models which is lack in literature. Growth and remodeling is a similar important future area, as this have important implications in prediction of the outcomes of diseases, injuries, and surgical interventions. The permanent set model and simulation framework developed herein is a simplification of the growth and remodeling framework by removing the growth component. This can have important potential implications is devices such as tissue engineered valves which have the possibility of growing and adaption to the surrounding environment if seeded with interstitial cells. 
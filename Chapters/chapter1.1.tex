\subsection{Prevalence of heart valve diseases}
    Cardiovascular diseases are the number one killer in the United States and around the world. 
Heart valve treatment is a common cardiovascular surgical procedure with over 100,800 done annually in the U.S. alone \cite{mozaffarian_heart_2016} and 275,000 to 370,000 in developed nations \cite{manji_future_2012}. Almost all contemporary heart valve replacement designs use exogenously crosslinked (EXL) collagenous soft tissues (e.g. bovine pericardium) to manufacture leaflets for bioprosthetic valves (BHVs) \cite{starr_artificial_2007, soares_biomechanical_2016}. BHVs have advantages in immunogenicity and hemodynamics over other designs, but also have a limited life span of 10-15 years. As a recent development in BHV technology, transcatheter valve interventions \cite{bonow_accaha_2006, guidoin_marvel_2010} reduce surgical risk and make valve replacement more feasible for those who cannot tolerate full surgical interventions. However, this new technology also presents additional design challenges, including complex folding and compression during delivery. As a result, the leaflets are required to be significantly thinner than in traditional BHVs, which increases the leaflet stress and potentially the rate of failure. Existing data on transcatheter aortic valve interventions suggest a 2-year mortality rate of 33.9\% overall \cite{mozaffarian_heart_2016} and over 68\% when specifically replacing stenotic aortic valves \cite{makkar_transcatheter_2012}. As such, this further accentuates the need to develop an approach to improve BHV durability. 

    This is for checking on whether the next line is auto indented. Normally only the first line is indented but the lines afterwards are not. However, we can clearly see that this is not true for Overleaf, where the following lines are also indented. This is not normal, but can be quite good for enhancing readability. This way, the indented blocks can be viewed as belonging to the re
\begin{enumerate}
\item Cardiovascular diseases are the number one killer in the United State. 
\item Valvular diseases have the highest mortality among cardiovascular diseases. 
\item Heart valve diseases generally falls in to two categories: stenosis and regurgitation. 
\item Stenosis is the hardening of the leaflets, more often than not, this is due to calcification of the leaflets. 
\item This can have significant impacts on the fluid dynamics through the valve, impacting the efficiency, stroke volume and other complications. 
\item Regurgitation on the other hand is the improper closure of the leaflets during systole. The are many possible cause. heart valve stenosis, heart valve regurgitation, Endocarditis, and genetic valve diseases (Bicuspid valve disease, quadricuspid valve diseases, etc.). 
\item 
\item Current treatments for Valvular diseases are highly invasive and have high mortality rates. 
\item Repair procedures are highly difficult but are still the preferred surgical procedure
\item Repair procedures are generally composed of annuloplasty 
\item The are many questions include the shape of the annulus and impact on the surrounding myocardium, root and arteries. 
\item Replacement procedures faces even more questions.




\end{enumerate}
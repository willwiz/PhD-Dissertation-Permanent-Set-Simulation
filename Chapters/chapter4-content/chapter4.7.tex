\section{Discussion}

\subsection{Permanent set is sufficient to describe early stages of BHV cycling}
	The most important result from this study is that permanent set alone is sufficient to explain all responses due to cyclic loading in the range of 0 to 65 million cycles. 
	This strongly suggests that there is no detectable damage to the collagen fiber architecture, and that the overall collagen fiber architecture stays intact and convected under affine kinematics. 
	This is not unexpected, as we previously found dense collagenous tissues to behave affinely when deforming in the physiological range \cite{lee_presence_2015}. 
	Although the permanent set effect is very noticeable in the early stages of the cyclic loading, it still takes millions of cycles; in other words, years. 
	Due to the time scale difference between the opening and closing of heart valves versus permanent set, BHVs always deforms quasi-statically, which means it always follows affine kinematics. 
	It follows that any structural changes in the collagen fiber architecture is affine as well. 
	This is all extremely important, as this allows us to completely separate permanent set from other cyclic loading effects and independently determine the permanent set rate constant just from the cyclic loading data in the early stage. 

\subsection{Lack of detectable structural damage}
	We observed in multiple instances \cite{sun_response_2004, sellaro_effects_2007} that there are molecular conformation changes in the collagen fiber during this early stage. 
	This suggests that while effects of collagen fiber damage are not detectable, it remains an ongoing but much slower process in comparison to permanent set. 
	Significant tearing and delamination are observed by Sacks and Smith \cite{sacks_effects_1998} after 500 million cycles, but this corresponds to the late stage of cyclic loading (Fig. \ref{fig:hypothesis}) during which we do not have extant mechanical data for constitutive modeling. 
	There are no other existing experimental data quantifying the intrinsic structural damage in BHVs during cyclic loading. 
	This is not surprising as actual structural damage is difficult to distinguish from other processes such as permanent set. 
	The relation between molecular changes and mechanical response is not well understood. 
	This requires multi-scale modeling for exogenously crosslinked tissue, which is also an extremely challenging task with no published studies currently. 
	The most promising way of quantifying structural damage is through constitutive modeling and simulations. 
	Structural models can separate structural damage and permanent set, but we do not have sufficiently data at high cycle numbers where structural damage is detectable. 
	This remains an important extension for the model in the future. 

\subsection{Permanent set is driven by the scission-healing of the matrix}
	One important assumption in our model is that permanent set only occurs in the matrix due to scission-healing. 
	Based on our theory for permanent set, the process is driven entirely by the first order kinetics of the crosslinking reactions of GLUT leading to scission-healing, as well as the kinematics involved in the reference state evolution. 
	Our results indicate that these mechanisms and permanent set can indeed explain the response to cyclic loading.
	This highlights the importance of understanding the effects of GLUT crosslinks and its role in the cyclic loading response of BHVs. 
	The use of GLUT was originally intended for suppressing immunogenicity by crosslinking antigen within BP xenographs, but also has the fortunate consequence of stiffening the mechanical response of the BHV. 
	Unfortunately, the scission-healing behavior of GLUT also playing a major role in the cyclic loading of BHVs. 
	By severely changing the geometry of the BHV in the first 1-2 years, it strongly influences the cyclic loading response of BHVs at latter stages. 
	It is difficult to properly design BHVs without accounting for the permanent set deformations caused by GLUT's scission-healing. 
	Alternative exogenous crosslinking chemistry \cite{tam_fixation_2017, tam_novel_2015} may be an important area for technological advancement of BHVs. 
	Protecting the optimal mechanical response of the BHVs by reducing the impact permanent set can significantly limit the peak stress on BHVs and protect the underlying tissue microstructure. 

\subsection{Recruitment of collagen fibers limit permanent set}
	Our parametric study of permanent set suggests that collagen fibers may play a significant role in limiting the effect of permanent set on the geometry of BHVs. 
	The stiffness of collagen fibers are over three magnitudes higher than the exogenously crosslinked matrix. 
	As collagen fibers are recruited, residual stress starts accumulating between collagen fibers and the matrix. 
	However, due to the significant difference in stiffness, the exogenously crosslinked matrix cannot extend the collagen fibers by a significant amount. 
	In addition, we generally found significant collagen fiber structural reserves in collagenous tissue \cite{zhang_meso_2016}. From the parametric study, no more than 2-4\% of collagen fibers are straightened under physiological loading levels (up to 1 MPa). 
	Also, taking into account the rapid recruitment of native collagen fibers (collagen fibers do not extend by more than 5-8\% strain before breaking \cite{buehler_atomistic_2006}), the collagen fiber architecture serves as a barrier in limiting the changes in geometry due to permanent set. 
	This potentially gives us a way of predicting and designing BHVs based on the expected maximum permanent set deformation. 
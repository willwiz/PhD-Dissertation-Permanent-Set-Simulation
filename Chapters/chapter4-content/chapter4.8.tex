%%%%%%%%%%%%%%%%%%%%%%%%%%%%%%%%%%%%%%%%%%%%%%%%%%%%%%%%%%%%%%%%%%%%%%%%%%%%%%%%
%%  Conclusions
%%%%%%%%%%%%%%%%%%%%%%%%%%%%%%%%%%%%%%%%%%%%%%%%%%%%%%%%%%%%%%%%%%%%%%%%%%%%%%%%


\section{Summary and Future Directions}
	We have developed the first structural-based constitutive model for the time evolving properties of exogenously crosslinked collagenous soft tissues under cyclic loading. We focused on permanent set as the mechanism for the geometry changes in the early stage of cycling and developed our constitutive model based on the underlying scission-healing reaction of the GLUT crosslinked matrix. Permanent set allows the reference configuration of the exogenously crosslinked matrix to evolve over time and convect the collagen fiber architecture through the change in geometry. The results show that permanent set alone is sufficient to explain all changes in the early stage of BHV cycling, and more importantly predict how the shape and reference configuration evolve during this stage. Moreover, structural damage does not play a detectable role up to 65 million cycles. Our model also indicates that the collagen fiber architecture can play a role in limiting the permanent set effect, where the straightening of collagen fibers prevents further changes in geometry. Thus, accounting for the permanent set effect is especially important in the design of BHVs to better improve their performance and durability. 
	
	In addition to the exogenously crosslinked tissue applications addressed herein, we have observed permanent set like phenomenon in mitral valve tissue during pregnancy \cite{rego_mitral_2016}. In that study, our results suggested that much of the growth and remodeling in the MV leaflet does not begin immediately, but rather undergoes mostly passive leaflet enlargement until these parameters reach a critically low level, at which point growth and remodeling are triggered. This initial tissue distension process is very similar in behavior to the permanent set mechanism outlined in the present work. Thus, the current approach could be applied to these types of the early phases soft tissue remodeling, where non-failure mechanisms occur before the onset of growth of tissue growth and remodeling. In addition, although the permanent set model we described only include the remodeling of the matrix due to scission-healing, the same concept can be extended by separating the rate constant into growth and resorption to simulate growth and remodeling of the matrix. Furthermore, the frame work outlined in section 5 can also be extended for the remodeling of the collagen fiber architecture, given further studies on mechanisms for how the collagen fiber architecture grows once the critical level observed in Rego \textit{et al} \cite{rego_mitral_2016} is exceeded. This is the advantage for the structural-based approach to modeling permanent set, which allows us to describe the mechanical response based on real physically measure-able quantities. We can further extend the more toward effects such as structural damage are the fiber-level, proteolytic degradation, and growth based on how this effects affect the components of the permanent set model layed out herein.
\section{Limitations and conclusions}

\subparagraph{Limitations} 
	Our available cyclic loading data for exogenously crosslinked BP is limited in terms of cycled duration, multitude of strain levels, the rate of cyclic loading and number of specimens. 
	However, It is important to note that our goal was to develop the constitutive model form for exogenously crosslinked tissue under cyclic loading, not to obtain a population of material parameters for statistical testing and simulations. 
	The extant experimental data we used for parameter estimation was sufficient for this purpose, which have specimens under both constant strain level and time evolving strain level, and with loading along the preferred collagen fiber direction and more rigorously loading orthogonal to preferred collagen fiber direction.
	We note that some modification may be necessary, such as extending to model for the rate constant as a function of strain level and rate of cycling. 
	However, the permanent set mechanism described was able to explain the effects of cyclic loading in the early stage. 
	The lack of cyclic loading data beyond 65 million cycles, especially up to 100 or 200 million cycles, means that we are not able to incorporate structural damage into our constitutive model at this point. 
	Although, based on what we observed, structural damage is not an important factor up to 50-65 million cycles. 

\subparagraph{Summary and conclusions}
	We have developed the first structural-based constitutive model for the evolving properties of exogenously crosslinked tissues under cyclic loading. 
	We focused on permanent set as the mechanism for the geometry changes in the early stage of cycling and developed our constitutive model based on the underlying scission-healing reaction of the GLUT crosslinked matrix. 
	Permanent set allows the reference configuration of the exogenously crosslinked matrix to evolve over time and convect the collagen fiber architecture through the change in geometry.
	The results show that permanent set alone is sufficient to explain all changes in the early stage of BHV cycling, and more importantly predict how the shape and reference configuration evolve during this stage.
	Moreover, structural damage does not play a detectable role up to 65 million cycles. 
	Our model also indicates that the collagen fiber architecture can play a role in limiting the permanent set effect, where the straightening of collagen fibers prevent further changes in geometry. 
	Thus, accounting for the permanent set effect is especially important in the design of BHVs to better improve their performance and durability. 
\chapter{Summary, conclusions, and future directions}

\section{Summary}
    
    Soft-tissue-derived exogenously cross-linked (EXL) biomaterials continue to play an important role in the fabrications of  bioprosthetic heart valves (BHV), by having advantages in immunogenic and mechanical behaviors \cite{starr_artificial_2007}. Despite ongoing research, our understanding of these materials and of the mechanisms leading to their failure remains at an empirical level. A significant challenge prohibiting accurate and predictive simulations of BHVs are a lack of understanding of the mechanism of BHV failure. One significant mechanical change in response to cyclic loading is the change in geometry of the BHV leaflets. This process is induced by permanent set, which changes the referential configuration of the leaflet material and thus how they deform in response to transvavular loading. Further analysis shown that significant structure damage occurred within the same region that undergo significantly permanent set \cite{smith_fatigue_1999}. Permanent set occurs only within the early stages after surgical implantation. Although no structural damage occurs, this has important implications in geometrical change and durability of BHVs. Using permanent set we can predict change in geometry and optimal the initial design of BHVs for increase durability. To develop the constitutive model and framework for numerical simulation, we have taken the following steps:
    
    \begin{enumerate}
        \item We developed a meso-scale structural constitutive model for soft tissues. Constitutive models are fundamental to developing a deeper understanding of and predicting soft tissue function and pathology. To develop this constitutive model, we utilized the mitral valve leaflets as an example. This model takes into account the layered structure of these tissue and the contributions from the distinct collagen and elastin fiber networks within each tissue layer. Requisite collagen and elastin fibrous structural information for each layer were quantified using second harmonic generation microscopy and conventional histology. A comprehensive mechanical data set was also used to guide model formulation and parameter estimation. Furthermore, novel to tissue-level structural constitutive modeling approaches, we allowed the collagen fiber recruitment function to vary with orientation. Finally, a novel fibril-level (0.1 to 1 $\mu$m) validation approach was used to compare the predicted collagen fiber/fibril mechanical behavior with extant MV small angle X-ray scattering data. Results demonstrated excellent agreement, indicating that the MSSCM fully captures the tissue-level function. Future utilization of the MSSCM in computational models of the MV will aid in producing highly accurate simulations in non-physiological loading states that can occur in repair situations, as well as guide the form of simplified models for real-time simulation tools.
        \item We extended the meso-scale structural modeling approach for the effect of exogenous crosslinking, which is a crucial part of the fabrication of BHVs to suppress immunogenicity. Despite decades of research, development and clinical use, no such model previously existed. In this study, we develop the first rigorous full structural model (i.e. explicitly incorporating various features of the collagen fibre architecture) for exogenously cross-linked soft tissues. This was made possible, in-part, with the use of native to cross-linked matched experimental datasets and an extension to the collagenous structural constitutive model so that the uncross-linked collagen fibre responses could be mapped to the cross-linked configuration. This allowed us to separate the effects of cross-linking from kinematic changes induced in the cross-linking process, which in turn allowed the non-fibrous tissue matrix component and the interaction effects to be identified.  The most novel findings of this study were that: (i) the effective collagen fibre modulus was unaffected by cross-linking and (ii) fibre-ensemble interactions played a large role in stress development, often dominating the total tissue response (depending on the stress component and loading path considered).
        \item We extended this constitutive model further for time-dependent effects like permanent set. Permanent set is an effect which cause BHVs to undergo significant changes in geometry with in vivo operation, which lead to stress concentrations that can have significant impact on structural damage. These changes do not appear to be due to plastic deformation, as the leaflets only deform in the elastic regime. Moreover, structural damage was not detected by the 65 million cycle time point. We hypothesize that the scission-healing reaction of glutaraldehyde is the underlying mechanism responsible for permanent set in exogenously crosslinked soft tissues. The continuous scission-healing process of glutaraldehyde allows a portion of the exogenously crosslinked matrix, which is considered to be the non-fibrous part of the extra-cellular matrix, to be re-crosslinked in the loaded state. To model the permanent set effect, we assume that the exogenously crosslinked matrix undergoes changes in reference configurations over time. The changes in the collagen fiber architecture due to dimensional changes allow us to predict subsequent changes in mechanical response. Results show that permanent set alone can explain and, more importantly, predict how the mechanical response of the biomaterial change with time. An important finding we have is that the collagen fiber architecture has a limiting effect on the maximum changes in geometry that the permanent set effect can induce. This is due to the recruitment of collagen fibers as the changes in geometry due to permanent set increase. This means we can potentially optimize the BHV geometry based on the predicted the final BHV geometry after permanent set has largely ceased. 
        \item We developed a framework for the use of effective constitutive models to homogenize and facilitate numerical simulations. Effective which only reproduces the essential responses of soft tissue but not the underlying mechanisms. Current soft tissue constitutive modeling approaches have become increasingly complex, often utilizing meso- and multi-scale methods for greater predictive capability and linking to the underlying mechanisms. However, such modeling approaches are associated with substantial computational costs, making the use of effective constitutive model approaches an important part of efficient numerical simulations. We evaluated this approach and demonstrated that it is able to handle materials of widely varying degrees of anisotropy, such as exogenously cross-linked bovine pericardium and aortic valve leaflet. This effective constitutive model approach has shown significant potential for improving the computational efficiency and numerical robustness of multi-scale and meso-scale modeling approaches, facilitating the application of inverse modeling and simulations of growth and remodeling of soft tissues and organs.
        \item Finally, we completed this process by developing a Python based framework for the implementation and simulation of permanent set. This simulation utilizes the predictive mechanism based constitutive model for the permanent set effect in exogenously crosslinked soft tissues that we previously developed. We have shown that we can use this simulation to predict the evolving geometry, microstructural and material property changes. These results can then be used to predict regions of increase likelihood of structural damage, and can be used to optimal the initial design of BHVs based on these factors. Most important of these effects is that the collagen fiber architecture can play a role in limiting the permanent set effect, where the straightening of collagen fibers prevents further changes in geometry. Thus, accounting for the permanent set effect is especially important in the design of BHVs to better improve their performance and durability. 
    \end{enumerate}
    
\section{Conclusions}

    In conclusion, we have developed a novel predictive and mechanism based constitutive modeling and numerical simulation framework for the time dependent simulation of bioprosthetic heart valves. This work has important implication for predicting BHV durability. By being able to predict the evolving geometry of BHVs, we utilize this approach for optimizing the initial design of BHVs for more homogeneous distrubuted loading to the consitutive collagen fibers as well as better fluid dynamics. With future extensions to this model and framework, this can open the door to improving the process for BHV design and thus more durable BHVs. 

\section{Future directions}
    
    The next sequential step for this work is the inclusion of structural damage to the constitutive model. Part of the reason preventing the include of structural damage to this current work is the lack of sufficient experimental data. Structural damage is difficult to quantify as it only gradually accumulates over long periods of time. Due to the exponential nature and large structural reserves of soft tissues, small decreases in the number or modulus of collagen fiber is very difficult to detect. This is also complicated by the fact that strain is difficult to quantify in the first place. In accelerated wear testing or other similar environments, this process is further complicate by the heterogeneity of the resulting response. However, by simulation and removing of the permanent set effect on the change in geometry, we can more accurately determine the remaining changes due to structural damage, This opens the doors to the development of mechanism-based structural damage models which is lack in literature. 
    
    
    Another part of this ongoing project is the simulation of the fluid dynamic of the BHVs in response to the complex environment of it tissues. Fluid solid interactions (FSI) have long been a popular field of study \cite{kamensky_immersogeometric_2015}\cite{hsu_dynamic_2015}\cite{wu_anisotropic_2018}\cite{kiendl_isogeometric_2015}. However, this permanent set framework opens the doors to evaluating how the fluid dynamics change over time in response to cyclic loading. The FSI finite element soft is directive included in the permanent set simulations to facilitate this extension, and more parametric studies to evaluated and optimize BHV design designs. This also lends itself to help us better understand the results of accelerated wear testing environments, which as FDA required but drastically differs from the in vivo environments. This will provide a better analysis of fatigue testing results, better prepare these designs for clinical trials and surgical implantation. 
    
    
    Growth and remodeling is a similar important future area, as this have important implications in prediction of the outcomes of diseases, injuries, and surgical interventions. The permanent set model and simulation framework developed herein is a simplification of the growth and remodeling framework by removing the growth component. This can have important potential implications is devices such as tissue engineered valves which have the possibility of growing and adaption to the surrounding environment if seeded with interstitial cells. In addition to the exogenously crosslinked tissue applications addressed herein, we have observed permanent set like phenomenon in mitral valve tissue during pregnancy \cite{rego_mitral_2016}. In that study, our results suggested that much of the growth and remodeling in the MV leaflet does not begin immediately, but rather undergoes mostly passive leaflet enlargement until these parameters reach a critically low level, at which point growth and remodeling are triggered. This initial tissue distension process is very similar in behavior to the permanent set mechanism outlined in the present work. Thus, the current approach could be applied to these types of the early phases soft tissue remodeling, where non-failure mechanisms occur before the onset of growth of tissue growth and remodeling. This is the advantage for the structural-based approach to modeling permanent set, which allows us to describe the mechanical response based on real physically measure-able quantities. 
    
    
%---    Bioliography
\bibliographystyle{plainnat}
\bibliography{phd}
    
 
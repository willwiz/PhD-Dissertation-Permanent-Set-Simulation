\section{Primary results}
    
    From the five specimens used, the model was able to successfully fit all data quite well (total fit $r^2 > 0.97$). Moreover, the final parameter values were quite consistent, with generally low standard errors (table 3). The mean collagen fibre modulus (approx. 279 MPa) and fibre splay (approx. $38^\circ$) were comparable to previous studies \cite{fan_simulation_2014}. Interestingly, the lower bound stretch was small (1.01 or approx. 1\% strain), 
    so it is likely that it could be set to 1 (i.e. zero strain). The native collagen fibre recruitment parameters were also consistent (table 3), and indicated a very rapid recruitment at stretch of approximately 1.18–1.2 (figure 9). This is a more complete picture of the entire fibre recruitment than in our previous work \cite{sun_finite_2005}\cite{fan_simulation_2014}, and suggests that the collagen fibres are effectively well ordered with a small deviation in crimp amplitude and wavelength.
    

\begin{table}
\centering
\caption{Equibiaxial strain testing results.}\label{c3:tab:3}
\begin{tabular}{L{.5in}R{.7in}R{0.5in}R{0.5in}R{0.6in}R{0.5in}R{0.6in}R{0.5in}}
\hline
& \multicolumn{1}{c}{\textbf{modulus}} & \multicolumn{2}{c}{\textbf{ODF}} & \multicolumn{4}{c}{\textbf{recruitment}}\\
\cline{2-8}
& $\eta_c(MPa)$ & $\mu_\Gamma({}^\circ)$ & $\sigma_\Gamma({}^\circ)$ & $\mu_0$ & $\sigma_0$ & $\prescript{}{0}{\lambda}_{lb}$
& $\prescript{}{0}{\lambda}_{ub}$\\
mean & 278.94 & 6.513 & 38.430 & 1.185 & 0.014 & 1.011 & 1.197  \\
s.e.m. & 22.38 & 1.645 & 0.922 & 0.032 & 0.001 & 0.007 & 0.035  \\
\hline
\hline
& \multicolumn{2}{c}{\textbf{Interactions}} & & \multicolumn{4}{c}{\textbf{Matrix}} \\
\cline{2-8}
& $d_0(kPa)$ & $d_1$ & & $\mu_a(kPa)$ & $a$ & $\mu_b(kPa)$ & $b$\\
mean & 1.040 & 42.267 & & 56.74 & 1.068 & 1294.38 & 1.873  \\
s.e.m. & 0.255 & 9.772 & & 10.69 & 0.004 & 340.71 & 0.007 \\
\hline
\end{tabular}
\end{table}


    One advantage of our approach is that the various contributions to the total stress can be separated (figure 8). To better reveal the present findings, it is useful to examine the effects on the individual stress components under various loading paths. Following the trends of the ensemble results (figure 9), we noted that the interactions produced substantial contributions to the total stress (figure 10). Interestingly, for $S_{11}$ the interactions actually produced the largest contributions, followed by the matrix and collagen fibres. By contrast, for $S_{22}$ the contributions were much more dependent on the particular loading path, with the collagen phase dominating when $\lambda_2>\lambda_1$. When $\lambda_1>\lambda_2$, the matrix phase dominated $S_22$. We further note here that the contribution of the matrix was much less loading path sensitive, due to its near-linear, isotropic behaviour.
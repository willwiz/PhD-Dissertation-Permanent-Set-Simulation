\section{Delineation and modelling of the tissue-level mechanical effects of exogenous cross-links} \label{c3:sec:3}

\subsection{Rationale}

    While extensive work has been done on the characterization of the biomechanical effects of EXL formation on soft tissues \cite{sun_biaxial_2003,sacks_bioprosthetic_2006,sun_simulated_2005,sellaro_effects_2007,alferiev_prevention_2003,wells_cyclic_2005,wells_effects_2002,wells_effects_2000,wells_thermomechanical_1998,lee_high_1994,naimark_correlation_1992,lee_effect_2001,barber_mechanics_1999,mirnajafi_effects_2005}, there has been surprisingly little work done known to the authors on the development of formal constitutive models (other than \cite{sacks_structural_2000}). Related work on proteoglycan and related collagen fibril sub\Hyphdash forms have revealed complex micromechanical interactions (e.g. \cite{cavalcante_mechanical_2005,coughlin_dynamic_1996}), but micromechanical interactions modified by EXLs on the macroscale tissue responses remain largely unknown. Thus, prior to developing the constitutive model form, we first carefully examined the effects of EXL formation on the measured tissue\Hyphdash level biomechanical behaviours in the present data set.


\begin{table}
\centering
\caption{Equibiaxial strain testing results.}\label{c3:tab:2}
\begin{tabular}{L{.5in}R{.48in}R{.48in}R{.48in}R{.48in}R{.48in}R{.48in}R{.48in}R{.48in}}
\hline
\multicolumn{1}{l}{\textbf{ID}} &
\multicolumn{1}{c}{\textbf{12}} &
\multicolumn{1}{c}{\textbf{21}} &
\multicolumn{1}{c}{\textbf{41}} &
\multicolumn{1}{c}{\textbf{61}} &
\multicolumn{1}{c}{\textbf{68}} &
\multicolumn{1}{c}{\textbf{mean}} &
\multicolumn{1}{c}{\textbf{s.e.m.}} &
\multicolumn{1}{c}{\textbf{p-value}}\\
\hline
\multicolumn{9}{c}{initial tangent modulus (kPa)} \\
\hline
native & 73.9 & 65.6 & 25.5 & 110.6 & 107.3 & 76.6 & 17.4 & \\
\hline
GLUT & 1059.3 & 248.4 & 152.4 & 746.0 & 272.1 & 495.6 & 195.2 & 0.037   \\
\hline
\multicolumn{9}{c}{MTM (kPa)} \\
\hline
native & 54208 & 49014 & 53091 & 72460 & 77121 & 61179 & 6341.3 &   \\
\hline
GLUT & 67771 & 28615 & 58025 & 62319 & 70718 & 57490 & 8433.8 & 0.567 \\
\hline
\multicolumn{9}{c}{upper bound stress (kPa)} \\
\hline
native & 1164.0 & 885.4 & 956.2 & 1048.4 & 1289.4 & 1068.7 & 80.80 &  \\
\hline
GLUT & 1081.7 & 925.8 & 1204.2 & 1138.2 & 1353.8 & 1140.8 & 78.69 & 0.245   \\
\hline
\end{tabular}
\end{table}





\subsection{Effects of exogenous cross-link formation on collagen fibre ensembles}

    All specimens exhibited anisotropic dimensional changes due to preconditioning and cross-linking (figure \ref{c3:fig:3}b). Interestingly, we found that about 6\% shrinkage occurred in the preferred direction, and approximately 7\% expansion in the cross-preferred directions. Such changes can alter both the angular dependence on collagen recruitment and the collagen fibre orientation distribution. We determined basic characteristics of the collagen fibre stress–strain relations directly from the data (no modelling), including the lower and upper bound-associated stresses, the initial tangent modulus, and the MTM from a running 15-point window. From this analysis, we were able to determine a number of important mechanical characteristics (table 2 and figure \ref{c3:fig:5}). These include:
        \begin{enumerate}
            \item All specimens exhibited an approximately 6.5-fold increase in the initial tangent modulus (table 2), very similar to values and native/EXL ratios reported in \cite{mirnajafi_effects_2005}, which were conducted under flex conditions.
            \item The upper bound stress and MTM were found to be unaffected by EXL formation (table 2).
            \item EXL formation induced a reduction in achieved strain levels compared with the native state (figure \ref{c3:fig:5}a).
            \item The effective collagen modulus was unaffected by EXL formation (figure \ref{c3:fig:5}c).
        \end{enumerate}
    Collectively, these results reveal some important features of the effects of EXLs on collagen tissues. First, as noted in our previous studies \cite{sacks_structural_2000,mirnajafi_effects_2005} EXLs produce a substantial increase in the low-strain modulus. Next, equation \ref{c3:eqn:23} indicates that the MTM is proportional to the collagen fibre modulus and the recruitment function D. Use of equation \ref{c3:eqn:23} compensates for the effect of the changes in tissue dimensions due to cross-linking on the fibre recruitment, allowing separation of changes in fibre architecture from the modulus on the ensemble stress–strain curve. Thus, the lack of changes in effective modulus are independent of any effects of changes resulting from tissue dimensions and represent an accurate modulus estimate.
    
    
    
\section{Discussion}

\subsection{Modeling approach and major findings}

    In the present study, we developed a novel layer-specific meso-scale structural model of the MV leaflets starting from the fiber-level. The mechanical response at the layer- and tissue-level is driven by structural measurements and observations obtained from SHG and histology. We then utilized an extensive experimental mechanical database that included uniaxial, planar EB strain, and a comprehensive set planar biaxial stress test protocols. This approach allowed us to properly evaluate the mechanical properties of MV tissues and to form a complete dataset for parameter estimation. All information was then integrated to form a predictive model for the MV leaflet tissues.
    
    
\subsection{Model validation}

    Novel to this study was the use of the MV SAXS data \cite{liao_relation_2007} to validate the predicted collagen fiber modulus and fiber recruitment (Section \ref{c2:sec:26}). SALS techniques have been previously used to obtain the orientation of fibrous structures in biological tissues \cite{sacks_small_1997}. However, the wavelength of light (in the hundreds of nm) is much too large in comparison the size of the fibrils, which have a D-period of approximately 64 nm \cite{kastelic_structural_1980}\cite{hodge_recent_1963}\cite{chapman_electron_1984}. SAXS, on the other hand, utilizes a wavelength of 0.1–0.2 nm, and provides a direct way of measuring the collagen fibril D-period, and thus the fibril strains \cite{sasaki_elongation_1996}\cite{sasaki_stress_1996}\cite{liao_relation_2007}. In response to tissue-level deformation, the fibril strain depends on both its orientation and slack strain (Section \ref{c2:sec:26}). Therefore, the MV SAXS simulation can provide an independent way of validating the MSSCM collagen model component. Both the good agreement and sensitivity to accurate measures (Section \ref{c2:sec:233}, Fig. \ref{c2:fig:7}) provides additional confidence in our approach beyond basic measures such as goodness of fit. In particular, this lends confidence to the fiber/fibril kinematics used in the present form of the MSSCM, and suggests that the fibrils do indeed form contiguous, tightly bounded fibers and undergo minimal slipping.
    
    
    It is also important to note that it is difficult to directly combine measurements at multiple scales, such as fibril strain as measured by SAXS and the applied tissue-level membrane stress, and interpret them directly. In this case, we cannot approximate the modulus of collagen fibrils using the slope of the tissue-level stress–fibril strain curve. Although we assume every fibril has a similar modulus, the apparent modulus of collagen fibrils at the tissue-level decreases in response to increase in the slack stretch (Eqn. \ref{eqn:collagenfiberlaw}). Thus interpretation of such results requires careful analysis of the structure and kinematically related mechanisms of the tissue. Finally, we also validated the fiber $\Gamma(\theta)$ against optical SHG measurements, which demonstrated very good agreement (Fig. \ref{c2:fig:8}). To the authors’ knowledge, this is the first time macro/micro validations of this kind have been performed for any soft tissue
    
    


\subsection{Collagen fiber modulus}

    As the most important load bearing component of soft tissues, the structure and properties of collagen is of major interest. One specific property is the modulus of the collagen fiber, which when combined with the level of fiber undulation, determines the macroscopic behavior of the collagen network at the tissue-level. Given that type I collagen fibers forms the common building blocks for the bulk of the MV tissue (Fig. \ref{c2:fig:2}), the modulus serves both as an important validation for the results of the model and predicting the correct behavior of the tissue under complex loading conditions. There have been many attempts to measure the stiffness of collagen fibers in the literature \cite{shen_stress_2008}\cite{gentleman_mechanical_2003}\cite{eppell_nano_2006}\cite{yang_mechanical_2008}\cite{yang_micromechanical_2007}\cite{wenger_mechanical_2007}. These measurements come in two forms: (1) models at the tissue-level and (2) direct measurements at the fibril-level. Measurements at the tissue-level can be inaccurate, as they demand an extensive understanding of biomechanics of the tissue, physically accurate models of the mechanical behavior and detailed measurements of the structure and composition. On the other hand, there is currently no method for the real time estimation of the instantaneous cross sectional area of collagen fiber or fibrils. Current measurements are typically done a priori \cite{gentleman_mechanical_2003} or a posteriori \cite{eppell_nano_2006}, and can have significant impact on the results. These measurements also require highly accurate and noise insensitive instruments operating at the micrometer and nanometer scale. As such, these experiments are typically done using techniques such as atomic force microscopy (AFM) and have a large variability for the resulting numbers. Nonetheless, these measurements serve as an important reference for any results measure at the tissue-level, and the current literature paints a distinct picture.


    For the most part, these findings fall into two distinct categories: (1) the dried properties at the 2–10 GPa range and (2) hydrated properties at the 200–900 MPa range. This difference in stiffness is due to water acting as a plasticizer in collagen fibrils [62]. Using AFM, Yang et al. \cite{yang_micromechanical_2007} found a stiffness of 1.4 GPa for non-cross linked fibrils and 3.4 GPa for cross-linked collagen fibrils in the bovine Achilles tendon. Similarly, Wenger et al. \cite{wenger_mechanical_2007} found collagen fibril stiffness of 5–11.5 GPa in rat tail tendon. However, in both papers the fibrils were dried under ambient conditions and nitrogen, respectively. On the other side, Shen et al. \cite{shen_stress_2008} found a modulus of $860\pm450$ MPa for collagen fibrils obtained from sea cucumber dermis. In the study by Gentleman et al. [23], they found a modulus of $269.7\pm11.9$ to $484.7\pm76.3$ MPa for collagen fibers in the bovine Achilles tendon, varying with the fiber diameter, but only for the cross-linked fibers. Eppell et al. \cite{eppell_nano_2006}, found a modulus of 0.5–0.4 GPa at low strain (0.05–0.30) but can increase to up to 12 GPa at high strain. In these cases, the fibrils were hydrated before testing. This effect has also been studied using electrospun collagen scaffolds, where Yang et al. found that the fibril modulus decrease from 1.3–7.8 GPa to 0.07–0.26 GPa when hydrated [25].


    In present study we found the collagen fiber modulus to be between 132.5 and 167.3 MPa. In our estimation of the collagen volume fraction, we assumed the tissue is entirely composed of collagen, elastin and proteoglycans. While this serves as a good estimation of the relative composition of the MV, fluid constitutes 81.8\% of the total mass of the MV anterior leaflet \cite{lis_biochemical_1987}. Additionally, using the dry mass is not a viable alternative as the fibers may have shrunk when they are dried \cite{leikin_raman_1997}. As such, we chose the current approach to provide an unbiased effective collagen modulus for the MV. This value is not applicable to other tissues with significantly different structural composition and may explain minor differences in the modulus estimate for the anterior and posterior leaflet. When the residual volume is taken into account, it is not difficult to imagine the estimated modulus to be 2–4 times higher than the value reported, allowing it to fall in line with the modulus of the hydrated collagen fibrils in other studies. This suggests that the model is reasonably accurate in estimating the mechanical properties of the collagen fibers. Furthermore, there were no statistical significance between the fiber modulus for either leaflet, which is reasonable given that the majority of collagen fibers are type I. Finally, that the assumption of minimal fiber–fiber and fiber–matrix interaction appears to be essentially correct. This is consistent with the ability for the fibers to rotate and extend freely with applied forces, keeping the effective tissue modulus low so the leaflets can coapt as necessary for valve function.
    
    
    
    
\subsection{Collagen fiber recruitment}

    While type I collagen naturally occurs in a crimped state, the bending stiffness as the collagen fiber unravels is typically small enough to allow us to use the recruitment approach \cite{lanir_constitutive_1983}\cite{fata_insights_2014}\cite{sacks_incorporation_2003}\cite{lanir_structural_1979}\cite{kastelic_structural_1980}\cite{hansen_recruitment_2002}\cite{cacho_constitutive_2007}\cite{grytz_constitutive_2009}. Novel to recruitment approach, we propose an angle variant recruitment distribution based on the hypothesis that the ensemble stress will be limited to a small range due to homeostasis. Current experimental techniques do not easily allow for direct measurement, and distinguishing individual fibers is difficult and sometimes required to be specified manually \cite{hill_theoretical_2012}. Determining whether a fiber is straightened is also problematic. For best accuracy, the fiber must lay in the plane of the image. The planar projection of a fiber at an angle with the plane of the image creates a misrepresentation of the tortuosity of the fiber. Additionally, the tissue must be loaded under EB strain with no shearing. Failure to do so will result in different strain for each fiber ensemble, thus the tortuosity of fibers at different angle cannot be referenced to the same strain. These issues are further magnified when quantifying angular variations. In the end, direct measurement methods are best served as a preliminary estimate for modeling purposes.


    Assuming that the ensemble stress of collagen fibers is constant with orientation due to homeostasis, both the current model (Eqn. \ref{eqn:recruitmentasafunctionofangle}) and the orientation-variant model (Eqn. \ref{eqn:recruitmentasafunctionnormal}) produce similar results for the MV leaflets; the quality of the best fits ($R^2 = 0.988$ vs $0.996$) are very comparable for both models. We hypothesize that this is due to the narrow splay of the collagen fiber ODF in the MV and thus the ensemble stress was effectively constant over this range for both models. This simplifies the recruitment distribution function for implementation in MV simulation and measurement via experimental means. Also, by assuming orientation-variance in Eqn. \ref{eqn:recruitmentasafunctionofangle}, the parameters are expressed as a percentage of the maximum ensemble strain, which is not an intuitive quantity. Thus, the parameters determined from Eqn. \ref{eqn:recruitmentasafunctionofangle} are best served as an approximation, rather than any physically accurate quantity. Therefore, we refocused the results as if the recruitment strains are invariant with angle (Eqn. \ref{eqn:recruitmentasafunctionnormal}), which has a more interpretive physical meaning. In either case, the theory then dictates that the tissue should exhibit a linear response once all fibers are straightened under EB strain loading. Indeed, this is observed in the MV for both the anterior (Fig. \ref{c2:fig:3}) and posterior leaflet.


    For each leaflet tested under EB strain and uniaxial extension, the stress strain curve transitions to linear in P–F \cite{sasaki_elongation_1996}\cite{sasaki_stress_1996}\cite{liao_relation_2007}. For the anterior leaflet this transition point occurs at $782.1\pm121.5$ kPa in 2nd Piola Kirchhoff or $1243.8\pm220.8$ kPa in Cauchy stress. Our inverse model using a full collagen mapped transverse model shows that the peak circumferential stress is at $241.4\pm40.5$ kPa and the peak radial stress is at $432.6\pm46.5$ kPa \cite{lee_inverse_2014}. While it is not possible to produce the equivalent ensemble stress for the in vivo data, it is estimated that only 20–40\% of fibers are recruited under physiological stress. This substantial structural reserve is likely an adaption to maintain structural integrity for when the body is under extraneous physiological stress or attempt to maintain basic function in cases of heart diseases.
    



\subsection{Elastin fiber network response}

    There is no established constitutive model form for individual elastin fibers in valvular tissue. We have taken a similar approach to modeling the fiber ensembles in the ovine pulmonary arteries \cite{fata_insights_2014}. However, in the MV, we noticed a slight nonlinearity in the pre collagen fiber recruitment response contrary to that of the pulmonary arteries \cite{fata_insights_2014}. It is not clear what the reason behind this is. Likely the cross-linking, both between and within a fiber in the elastin network, is different in order to fulfill the different functions for each tissue. To fit the low stress region, an exponential or power law is necessary. We chose the power law so we can constrain the order of the nonlinearity. This may result in some error when predicting and extrapolating past the range of the data. However, the stress contribution of elastin peaks at around 5–15 kPa at maximum stretch, which is much less that of the collagen (Fig. \ref{c2:fig:4}a), so this should not be a significant effect. The collagen also essentially functions as a stopper, putting a constraint on the maximum extension of the leaflets (especially in the circumferential direction). For this reason, the material model of the elastin this should not be a problem for the predictive capabilities of the model in most applications.


    Another interesting result is that the exponents of the elastin model differed between the layers and between the leaflets. This may be evidence of residual stress or strain between the layers. For the aortic valve, Stella and Sacks \cite{stella_biaxial_2007} noted that the elastin dominant ventricularis of the leaflet contracted by 10.9\% and 8.2\% in the radial and circumferential direction while the collagen dominant fibrosa elongated by 28.2\% and 4.8\% in the radial and circumferential direction after layer separation. Similar effect likely exists in the MV and the additional strain will result in a higher apparent stiffness due to the nonlinearity in the stress strain response. Thus the difference in the atrialis and ventricular of each leaflet may, at least in part, be due to this reason. It is interesting that the elastin in the anterior leaflet is significantly stiffer in the circumferential direction compared to that of the posterior leaflet, while in the radial direction it was the opposite. This corroborates with physical observations where the ventricularis in the posterior leaflet was much smaller compared to the anterior leaflet; sometimes nearly nonexistent. Again, this may be a product of the motion of the leaflets when closing.
    
    
    
    
\subsection{Layer structure and function}

    In the MV, the anterior leaflet extends to meet the posterior leaflet, while the posterior expands much less in comparison \cite{amini_vivo_2012}\cite{rausch_vivo_2011}. In order to accomplish this, the radial direction of the anterior leaflet needs to be able to extend significantly, while the circumferential direction needs to be able to maintain its integrity as it meets the posterior leaflet. There are a number of ways in which collagenous tissue have adapted to increase extensibility. The most common way is through adjustment in the crimping of the collagen fibers. However, once fiber recruitment is initiated, the tangent modulus increases rapidly, limiting further extension. A second method is through the rotation of fiber ensembles. As the ratio of radial to circumferential stretch increases, this induces a rotation of all fibers toward the radial direction. The overall effect is similar to recruitment due to crimping, but the tangent modulus increases more gradually in comparison. Intriguingly, this adaptation is seen in the MV leaflets. For both leaflets, collagen is mostly present in the fibrosa (Fig. \ref{c2:fig:2}). This results in the majority of collagen fibers being circumferentially aligned. Rather than having independent family of fibers each dictating the mechanical response of each direction, this is highly coupled (Fig. \ref{c2:fig:11}), with the bulk of the stress in the radial direction coming from the circumferential splay. However, the collagen fibers in atrialis are still needed for additional support. This makes the width of the fibers splay dictate the resulting response. In the anterior leaflet, the elastin and collagen fiber splays are narrower (Fig. \ref{c2:fig:9}), thus increasing the radial extensibility. Comparatively, the wider splay in the posterior leaflet produce a slightly more isotropic response resulting in smaller radial extension. Additionally, collagen fibers are also recruited faster in the anterior leaflet to further constrain the circumferential extension. Overall, collagen primarily functions to limit extension, whereas elastin determines the motion of the valve in the low stress region. This is perhaps why the radial direction of the anterior leaflet has the lowest stiffness, which allows for the fastest rate of extension. Interestingly, elastin is also the stiffest in circumferential direction of the anterior leaflet, following a similar trend as the collagen. The elastin in the posterior leaflet is likewise more isotropic than that of the anterior.
    
    
    
    
\subsection{Model predictive capabilities}

    Overall the model demonstrated very good ability to predict the extra-physiological protocols (Fig. \ref{c2:fig:4}c \& d). Using the correct mass fractions, for the ratio of collagen in the fibrosa vs the atrialis in particular, was especially important for accurately predicting these extra-physiological protocols. For predictive purposes, more leeway can be given for the mass fraction of elastin. Since the layers are treated completely independent of each other (not sharing a single modulus), errors in the mass fraction is absorbed by the modulus.
    
    
    
    
\subsection{Limitations}

    While every effort was made to ensure tissue viability, all studies were performed under in vitro conditions. While there appears to be differences between the in vivo properties of MV anterior leaflet \cite{krishnamurthy_material_2008} and the passive properties measured in an in vitro setting \cite{grashow_planar_2006}\cite{may-newman_biaxial_1995}. The mechanism behind these estimated differences remain unknown. It is unlikely to be due to the contractile forces exerted by the MV interstitial cells on the surrounding ECM, which appears to be small \cite{buchanan_interlayer_2013}. Residual stress may be involved, as studies have shown that the MV leaflets are also under significant residual strain \cite{amini_vivo_2012}. Additionally, the viscoelastic properties of the MV were not considered in this model, however we have found that the leaflet tissues are essentially functionally elastic \cite{grashow_biaxial_2006}\cite{grashow_planar_2006}. Direct validation of the individual layer stress contributions is not currently available, as it requires the separation of the individual layers, which in itself is a complex task. Additionally, a more sophisticated model is also necessary to take into account the effect of residual strain on the mechanical response. Nevertheless, we have shown that our models have good predictive capabilities for the mechanical response and fiber orientations distributions, and are validated using SAXS studies. Overall, this model is a faithful representation of the structural function relationship of MV tissues.
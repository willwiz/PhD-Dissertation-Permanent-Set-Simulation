\section{Conclusions and future directions}

In this study, we developed a novel fiber- and layer-specific structurally-driven constitutive model of the MV leaflets. The model incorporated fiber ODFs for collagen and elastin, collagen fiber recruitment, and related fiber-level mechanical phenomena. The model was validated by simulating the SAXS experiments and compared against the measured response. The results were consistent and show that the model correctly predicted the collagen fibril deformations measured by SAXS. Furthermore, the material parameters estimated were also consistent under EB strain testing and uniaxial testing. Thus the model is validated both via microscopic and meso-scale measurement. For the MV leaflets, we determined an effect modulus of 132.5–167.3 MPa for the collagen fibers which matches well with existing literature. This result suggests the tissue structure is an important predictor of leaflet function, where the fiber ODF and recruitment couples to determine the mechanical response of each leaflet. Moreover, fiber ODF tends to be narrower and the recruitment tend to happen over smaller strain range to allow for greater extension in the anterior leaflets while maintaining tensile strength. Overall, the model shows very good predictive ability and hopes to help in producing more accurate simulations of MV behavior in vivo, with the ultimate goal in improving long-term durability of MV repair.

This model may also serve as a baseline for the study of disease and aging valvular tissues. Pathological change to biological tissue is a very complex topic. In addition to the variety of pathological conditions, the changes due to a common disorder will be different person by person. There is not yet sufficient data to fully understand and predict the mechanical changes. However, this model can serve as a starting point for analyzing the underlying changes due to pathological conditions, when the requisite data become available.

We should note that the utility of the present model lies in the accuracy and details of how the various aspects of the MV tissue coordinate to produce the bulk-level response. This is not only important to ECM mechanics but also coupling to the interstitial cell (MVIC) population. We have recently shown that simulated MVIC moduli for the four layers were found to be all within a narrow range of 4.71–5.35 kPa, suggesting that MVIC deformation is primarily controlled by each tissue layer’s respective structure and mechanical behavior rather than the intrinsic MVIC stiffness \cite{lee_effects_2015}. This novel result further suggests that while the MVICs may be phenotypically similar throughout the leaflet, they experience layer-specific mechanical stimulatory inputs due to distinct extracellular matrix architecture and mechanical behaviors of the four MV leaflet tissue layers. This also suggests that MVICs may behave in a layer-specific manner in response to mechanical stimuli in both normal and surgically modified MVs. Development of detailed layer-specific models, such as the one presented herein, will clearly aid in further our understanding of these phenomena. However, for computational implementation, a multi-scale approach \cite{lee_mitral_2015} will likely be needed to make the current implementation computationally tractable.
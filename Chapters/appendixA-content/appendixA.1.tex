\section{Introduction}

A central tenet of the biosolid continuum mechanics of biological tissues is the development of the constitutive model. Such models are critical to the insight into the development of accurate computational simulations of such structures as the heart and its valves, arteries, cartilaginous structures, and their engineered tissue equivalents. While formulation of the theoretical framework is always the first step, rigorous experimentation must be performed in parallel to explore all relevant deformations to both obtain the necessary constitutive model parameters and to evaluate its predictive capabilities {Sacks, 2000 #33069}. Thus there is an increasing need for multi-axial mechanical data to fully explore and understand the complex structures of biological tissues.
For incompressible planar membrane or thin soft tissue sections, a planar biaxial mechanical testing configuration can provide much information about the stress-strain behavior {Sacks, 2000 #33069}. Planar biaxial tests can be performed with either extensional deformations only, or in combination with in-plane shear {Sun, 2003 #4637}. However, an ongoing problem in soft tissue mechanics is that they are not truly elastic. Soft tissues have been shown to exhibit elastic behavior under physiological conditions, yet also exhibit permanent set-like changes in configuration from preconditioning {Stella, 2007 #28656;Sacks, 2000 #33069}. In addition, due to their very low stiffness in the zero stress state, even mounting and handling can alter the shape of the test specimen. This may result in the stress-free reference state of the specimen changing drastically from the one measured prior to mounting.
The situation becomes more complex when shear strains are involved. In our original method to induce shear strains in soft tissues. In our first studies of this aspect {Sacks, 1999 #1151}, components of the first Piola-Kirchhoff stress tensor P were determined from the experimentally measured axial forces using P11=f1(1)/A(1), P22=f2(2)/A(2), P12=P21=0, where f(i) and A(i) are the axial forces and initial cross-sectional areas respectively, with i=1,2 (Fig. 1). The resulting second Piola-Kirchhoff stress tensor S was computed using  . We later noted that this mapping produced incorrect results in the presence of shear and an initial alternative method was developed {Freed, 2010 #30830}. Although theoretically correct, that work did not offer a generalized solution. Specifically, the method did not accounted for changes in geometry of the unloaded state as a result of preconditioning and other related inelastic dimensional effects. As a result, the run-time specimen configuration will be a quadrilateral due to both shear and extensional strains that occurred during preconditioning. While others have developed various methods of deriving the stress under biaxial testing (e.g. {Fomovsky, 2010 #33811}), no method to date address actual testing geometry or compensates for changes in specimen geometry during the experiment. Moreover, inherent heterogeneities in tissue structure will always affect the accuracy of the resultant stress analysis.  No systematic study to date has incorporates these effects on accuracy of stress measures from planar biaxial tests.
	In approaching a solution to this problem, we first recognize that there are certain key considerations and limitations inherent in utilizing and interpreting biaxial mechanical data. As noted in Sun et al. {Sun, 2003 #6507}, no biaxial experiment can produce the ideal homogeneous strain and stress state. Subsequent errors in the computation of stress components will propagate to the estimates of the material constants, ultimately limiting our ability to accurately simulate soft tissue structures. Moreover, we need to first determine the constitutive model form and material parameters before undertaking the complex task of simulation of complete organ systems and inverse modeling. Thus, there remains a need for an improved method to derive the stress-strain relation from biaxial tests, within the assumption of specimen strain and stress field homogeneity. 
The current work presents a straightforward, generalized numerical technique that accounts for changes in specimen geometry and testing configuration for planar biaxial testing. It utilizes only the (1) initial dimensions, (2) fiducial markers and (3) axial forces, and works within the assumption of strain and stress field homogeneity within the specimen. Specifically, we focus on compensating for the following aspects: 
1.	Changes from the initial, never loaded, directly measured specimen geometry to the post-mounting/preconditioning state. 
2.	Accounting for heterogenous effects using rigid body minimization.
3.	Accounting for the actual specimen geometry and tether testing configuration during run time.
We also present a full simulation of the biaxial device geometry with real tissue properties, anisotropy and heterogeneities, using methodologies  presented by Lee et al.{Lee, 2014 #34319}, to evaluate the accuracy of the method. 

\section{Introduction}
    A central need in the application of continuum mechanics to biological tissues is the development of the constitutive models. Such models are critical for insights into the development of accurate computational simulations of the heart and its valves, arteries, cartilaginous structures, and engineered tissue equivalents. While the formulation of the theoretical framework is always the first step, rigorous experimentation must be performed in parallel to explore all relevant deformations to both obtain the necessary constitutive model parameters and evaluate its predictive capabilities \cite{sacks_biaxial_2000}. Thus, there is an increasing need for multi-axial mechanical data to fully explore and understand the complex structures of biological tissues. For incompressible planar membrane or thin soft tissue sections, a planar biaxial mechanical testing configuration can provide much information about the stress–strain behavior \cite{sacks_biaxial_2000}. Planar biaxial tests can be performed with either extensional deformations only, or in combination with in-plane shear \cite{sun_biaxial_2003}.


    However, an ongoing problem in soft tissue mechanics is that they are not truly elastic. Soft tissues have been shown to exhibit elastic behavior under physiological conditions, yet also exhibit permanent setlike changes in configuration from preconditioning \cite{sacks_biaxial_2000}\cite{sun_biaxial_2003}\cite{stella_time_2007}. In addition, due to their very low stiffness in the zero stress state, even mounting and handling can alter the shape of the test specimen. This may result in a drastic change in the stress-free reference state of the specimen from the one measured prior to mounting. The situation becomes more complex when shear strains are involved. In our first studies \cite{sacks_method_1999}, a simplified method was used to determine the components of the first Piola–Kirchhoff stress tensor $P$ from the initial dimensions and experimentally measured axial forces, with the second Piola–Kirchhoff stress tensor $S$ determined using $S = F^{-1}P$. We later noted that this mapping did not produce fully accurate results in cases where the shear strain was substantial. An initial alternative method was developed \cite{freed_hypoelastic_2010}, but was not a generalized solution. Specifically, the method did not accounted for changes in geometry of the unloaded state as a result of preconditioning and other related inelastic dimensional effects. As a result, the run-time specimen configuration will be a quadrilateral due to both shear and extensional strains that occurred during preconditioning. While others have developed various methods of deriving the stress under biaxial testing (e.g. \cite{fomovsky_evolution_2010}), no method to date addresses the actual testing geometry or compensates for changes in specimen geometry during the experiment. Moreover, inherent heterogeneities in tissue structure will always affect the accuracy of the resultant stress analysis. No systematic study to date has incorporated these effects nor determined their influence on the accuracy of stress tensor components from planar biaxial tests.
    
    
    In approaching a solution to this problem, we first recognized certain key considerations and limitations inherent in interpreting any biaxial mechanical data. As noted in Sun et al. \cite{sun_numerical_2003}, no biaxial experimental configuration can produce an ideal homogeneous strain and stress state. This will result in subsequent errors in the computation of stress tensor components, which will propagate into the estimates of the material parameters, ultimately limiting our ability to accurately simulate soft tissues. Moreover, we believe one should first determine the constitutive model form and material parameters before undertaking the complex task of simulating complete organ systems. Thus, there remains a need for an improved method to derive the stress–strain relation directly from biaxial mechanical test data, with the assumption of homogeneous strain and stress fields.
    
    
    The current work presents a straightforward, generalized approach for computing the effective Cauchy stress tensor for planar biaxial mechanical experiments that utilize a tethered mounting configuration. This method utilizes only the (1) the initial specimen dimensions, (2) the measured fiducial markers positions, and (3) the measured axial forces. It works under the assumption of homogeneous strain and stress fields within the specimen and compensates for the following attributes:
        \begin{enumerate}
            \item changes from the unloaded, initially measured specimen geometry to the postmounted, preconditioning state
            \item the effects of structural heterogeneities in tissues, which can result in rigid body rotations
            \item actual specimen geometry and tether attachment configurations during run time
        \end{enumerate}
    To assist with validation and provide additional insights, we also present a comprehensive simulation of the entire biaxial testing geometry using simulated tissue properties, anisotropy, and structural heterogeneities.